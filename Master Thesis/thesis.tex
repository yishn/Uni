\documentclass[11pt,a4paper,openany]{memoir}
\usepackage{fontspec}
\usepackage[english]{babel}
\usepackage{amsmath, amsfonts, amssymb, amsthm}
\usepackage{paralist}
\usepackage{tikz-cd}
\linespread{1.1}
\usepackage{hyperref}
\hypersetup{
	bookmarksopen=true,
	pdfstartview=FitH
}
\usepackage[inner=4cm,outer=3cm,top=3cm,bottom=3cm]{geometry}
\author{Yichuan Shen}
\title{Title of the Master Thesis}
\makeindex

\begin{document}

\theoremstyle{plain}
\theoremstyle{definition}
\newtheorem{theorem}{Theorem}[chapter]
\newtheorem{lemma}[theorem]{Lemma}
\newtheorem{proposition}[theorem]{Proposition}
\newtheorem{corollary}[theorem]{Corollary}
\theoremstyle{definition}
\newtheorem*{definition}{Definition}
\newtheorem*{example}{Example}
\theoremstyle{remark}
\newtheorem*{remark}{Remark}

\frontmatter
\pagenumbering{gobble} 

\makeatletter
\begin{center}
\vspace*{0cm}
\begin{large}
Universität Heidelberg\\
Fakultät für Mathematik und Informatik\\
Studiengang Mathematik\\
\vspace{8mm}
\end{large}
\vfill 
\begin{large}
\textbf{MASTER THESIS}\\
\end{large}
\vspace{10mm}
\begin{huge}\@title \end{huge}\\
\vspace{10mm}
by \textsc{yichuan shen}\\
\vspace{3cm}
\vfill
Supervisor: \textsc{prof. dr. alexander schmidt}\\
\today
\end{center}
\makeatother

\clearpage
\ 
\clearpage

\vspace*{0cm}
\vfill
\paragraph{Declaration.} I declare that I carried out this master thesis independently, and only with the cited sources, literature and other professional sources. The master thesis was not used in the same or in a similar version to achieve an academic grading or is being published elsewhere.\\
\vspace{7mm}\\
\rule{7cm}{0.4pt}
\clearpage

\vspace*{0cm}
\vfill
\renewcommand{\abstractname}{Abstract}
\begin{abstract}
Lorem ipsum dolor sit amet, consectetur adipiscing elit. Curabitur interdum, dui non molestie tincidunt, lectus enim aliquet arcu, suscipit pulvinar dui tortor vitae mi. Nam cursus mi sem, ac gravida tellus congue vitae. Nam elementum eleifend lorem dignissim rhoncus. Quisque ac nulla dui. Duis rutrum eros ornare faucibus congue.
\end{abstract}

\vspace{1cm}

\renewcommand{\abstractname}{Zusammenfassung}
\begin{abstract}
Integer congue euismod justo, quis venenatis tortor. Proin sem leo, accumsan eget pulvinar vitae, tincidunt ut elit. Morbi ut purus volutpat, efficitur quam id, consequat augue. Cras ullamcorper lacus eget massa vehicula, a elementum diam lacinia. Morbi efficitur elementum malesuada. Cras a aliquam tortor, et interdum neque.
\end{abstract}
\vfill
\clearpage

\mainmatter
\pagenumbering{arabic} 
\setcounter{page}{5}
\tableofcontents

\chapter{Introduction}

\clearpage

\section{Notation}

$\begin{array}{ll}
\mathbb{N} & \text{Natural numbers} \\
\mathbb{Z} & \text{Integers}\\
\mathbb{Q} & \text{Rational numbers}\\
\mathbb{R} & \text{Real numbers}\\
\mathbb{C} & \text{Complex numbers}\\
\mathbb{Z}_p & \text{$p$-adic integers}\\
\mathbb{Q}_p & \text{$p$-adic numbers}
\end{array}$

\chapter{First chapter}

\begin{thebibliography}{99999999}

\bibitem[DDMS99]{DDMS99}
	J. D. Dixon, M. P. F. du Sautoy, A. Mann, D. Segal (1999)
	\textit{Analytic Pro-$p$ Groups}.
	Cambridge Studies in Advanced Mathematics,
	2nd Edition.

\end{thebibliography}

%\printindex
\end{document}
