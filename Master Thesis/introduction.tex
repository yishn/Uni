\chapter{Introduction}

Given an algebraic variety $X$, we can ask ourselves how much information about $X$ is contained in its étale fundamental group $\pi_1^{\text{ét}}(X)$. Especially interesting are schemes which are reconstructible solely from its étale fundamental group. If $X$ is the spectrum of a field $K$, the étale fundamental group of $X$ is just its absolute Galois group $G_K$. 

Finite fields have an abelian absolute Galois group. In fact, they all have the same absolute Galois group $\widehat{\mathbb{Z}}$, therefore not much information can be retrieved from these groups. The study of these types of questions belong to the area of \textit{anabelian geometry}. This term refers to the fact that the less abelian $\pi_1^\text{ét}(X)$ is, the more information it carries about $X$.

Number fields stand in stark contrast to finite fields. Two number fields $k$, $k'$ with isomorphic Galois groups $G_k$, $G_{k'}$ are isomorphic as fields as well. It was J. Neukirch who investigated this question for the first time in the end of the 1960s (see \cite{Ne69}, \cite{Uc76}, \cite{Ne77}, \cite{Uc77}). Together with K. Uchida, their studies resulted in the following theorem for global fields:

\begin{theorem}[\textit{Neukirch-Uchida}]\label{thm:neukirch-uchida}
Let $k,k'$ be two global fields. Then any isomorphism $\sigma: G_k\to G_{k'}$ between their absolute Galois groups is induced by a unique field isomorphism $\varphi: k'^{\text{s}}\to k^\text{s}$ between their separable closures, i.e: 
\[\sigma(g) =\varphi^{-1}g\varphi \quad \text{for all }g\in G_k\] 
In particular $\varphi$ induces a field isomorphism $k'\to k$.
\end{theorem}

This means that global fields should be reconstructible from their Galois groups. Furthermore, the Galois group $G_k$ also contains important arithmetic information from $k$. 

It is believed that spectra of finitely generated fields are the building blocks of the anabelian world, which led Grothendieck to the conjecture that Theorem \ref{thm:neukirch-uchida} should also hold for arbitrary finitely generated fields $k$ and $k'$ of characteristic $0$.

This has been proven by F. Pop (see \cite{Po90}, \cite{Po94}, \cite{Po95}) using model theory. In this thesis, we want to follow M. Spiess \cite{Sp96} to prove the statement for finitely generated fields with transcendence degree $1$ over $\mathbb{Q}$, using a purely arithmetical approach and a direct generalization of the original proof of Theorem \ref{thm:neukirch-uchida}:

\pagebreak

\begin{theorem}[\textit{Pop}]\label{thm:main-result}
Let $K,K'$ be two finite field extensions of $\mathbb{Q}(T)$. Then any isomorphism $\sigma:G_K\to G_{K'}$ between their absolute Galois groups is induced by a unique field isomorphism $\varphi:\overline{K'}\to\overline{K}$ between their algebraic closures, i.e: 
\[\sigma(g) =\varphi^{-1}g\varphi \quad \text{for all }g\in G_K\]
In particular $\varphi$ induces a field isomorphism $K'\to K$.
\end{theorem}

\section{Overview}

In the first chapter, we want to recap some results from valuation theory. In the second part, we will introduce some lemmata in order to characterize henselian fields over $K$ of valuations of rational rank $2$ in terms of their Galois cohomology groups of dimension $3$.

In the second chapter, we will show that an isomorphism $\sigma: G_K\to G_{K'}$ induces bijections on the equivalence classes of defectless valuations of $K$ and $K'$ using the cohomological characterization of henselian fields. Using the Theorem of Neukirch-Uchida \ref{thm:neukirch-uchida}, we will also see that $\sigma$ induces a field isomorphism between the residue fields of corresponding defectless valuations of rational rank $1$. We will proceed to show that $\sigma$ induces group isomorphisms $K'^\times/n\to K^\times/n$ for all $n\in\mathbb{N}$.

In the third chapter, we want to use the correspondences established in the second chapter to prove Theorem \ref{thm:main-result}. Before we are able to do that, we need to discuss free-by-finite groups. Using the existence of a so-called admissible set, we can construct a group isomorphism $K'^\times\to K^\times$. Finally, we will show that the group isomorphism can be extended to a field isomorphism $K\to K'$. By passing to the limits, we obtain a field isomorphism $\varphi: \overline{K'}\to\overline{K}$ which induces $\sigma$.

\section{Acknowledgments}

\clearpage

\section{Notation}

$\begin{array}{ll}
\mathbb{N} & \text{natural numbers starting from $1$} \\
\mathbb{Z} & \text{integers}\\
\mathbb{Q} & \text{rational numbers}\\
\mathbb{R} & \text{real numbers}\\
\mathbb{C} & \text{complex numbers}\\
\mathbb{F}_q & \text{finite field with $q$ elements}\\
\mathbb{Z}_p & \text{$p$-adic integers}\\
\widehat{\mathbb{Z}} & \text{profinite completion of $\mathbb{Z}$}\\
\\
\overline{K} & \text{algebraic closure of $K$}\\
K^\text{s} & \text{separable closure of $K$}\\
K^\text{t} & \text{absolute inertia field of a henselian $K$}\\
K_v & \text{henselization of $K$ with respect to $v$}\\
\mu_n & \text{$n$-th roots of unity in a separable closure}\\
\\
G_K & \text{absolute Galois group of $K$}\\
\operatorname{Gal}(L/K) & \text{Galois group of $L/K$}\\
\operatorname{char}(K) & \text{characteristic of $K$}\\
\operatorname{trdeg}(L/K) & \text{transcendence degree of $L/K$}\\
\end{array}$
