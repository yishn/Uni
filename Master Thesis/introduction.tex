\chapter{Introduction}

Given an algebraic variety $X$, we can ask ourselves how much information about $X$ is contained in its étale fundamental group $\pi_1^{\text{ét}}(X)$. Especially interesting are schemes which are reconstructible solely from their étale fundamental groups. If $X$ is the spectrum of a field $K$, the étale fundamental group of $X$ is just its absolute Galois group $G_K$. 

Finite fields have an abelian absolute Galois group. In fact, they all have the same absolute Galois group $\widehat{\mathbb{Z}}$, therefore not much information can be retrieved from these groups. The study of these types of questions belong to the area of \textit{anabelian geometry}. This term refers to the fact that the less abelian $\pi_1^\text{ét}(X)$ is, the more information it carries about $X$.

Number fields stand in stark contrast to finite fields. Two number fields $k$, $k'$ with isomorphic absolute Galois groups $G_k$, $G_{k'}$ are isomorphic as fields as well. This result is due to J. Neukirch who investigated these types of questions for the first time in the end of the 1960s (see \cite{Ne69}, \cite{Ne77}). With K. Uchida's studies (see \cite{Uc77}), it resulted in the following theorem for global fields:

\begin{theorem}[\textit{Neukirch-Uchida}]\label{thm:neukirch-uchida}
Let $k,k'$ be two global fields. Then any isomorphism $\sigma: G_k\to G_{k'}$ between their absolute Galois groups is induced by a unique field isomorphism $\varphi: k'^{\text{s}}\to k^\text{s}$ between their separable closures, i.e: 
\[\sigma(g) =\varphi^{-1}g\varphi \quad \text{for all }g\in G_k\] 
In particular $\varphi$ induces a field isomorphism $k'\to k$.
\end{theorem}

This means that global fields should be reconstructible from their absolute Galois groups. Furthermore, the Galois group $G_k$ also contains important arithmetic information from $k$. 

It is believed that spectra of finitely generated fields are the building blocks of the anabelian world, which led to the conjecture that Theorem \ref{thm:neukirch-uchida} should also hold for arbitrary infinite fields $k$ and $k'$ which are finitely generated over their prime fields. This has been proven by F. Pop using model theory (see \cite{Po90}, \cite{Po94}, \cite{Po95}). 

In this thesis, we want to follow M. Spieß \cite{Sp96} to prove the statement for finitely generated fields with transcendence degree $1$ over $\mathbb{Q}$, using a purely arithmetical approach, which is a direct generalization of the original proof of Theorem \ref{thm:neukirch-uchida}:

\pagebreak

\begin{theorem}[\textit{Pop}]\label{thm:main-result}
Let $K,K'$ be two finite field extensions of $\mathbb{Q}(T)$. Then any isomorphism $\sigma:G_K\to G_{K'}$ between their absolute Galois groups is induced by a unique field isomorphism $\varphi:\overline{K'}\to\overline{K}$ between their algebraic closures, i.e: 
\[\sigma(g) =\varphi^{-1}g\varphi \quad \text{for all }g\in G_K\]
In particular $\varphi$ induces a field isomorphism $K'\to K$.
\end{theorem}

\section{Overview}

In the first chapter, we want to recap some results from valuation theory. Then we will introduce some lemmata in order to be able to characterize henselian fields over $K$ of valuations of rational rank $2$ in terms of their Galois cohomology groups of dimension $3$.

In the second chapter, we will show that an isomorphism $\sigma: G_K\to G_{K'}$ induces bijections on the equivalence classes of defectless valuations of $K$ and $K'$ using the cohomological characterization of henselian fields developed in the first chapter. Using the theorem of Neukirch-Uchida \ref{thm:neukirch-uchida}, we will also see that $\sigma$ induces a field isomorphism between the residue fields of corresponding discrete valuations. 

In the third chapter, we will establish a field isomorphism between the constant fields of $K$ and $K'$ respectively. We will proceed to show that $\sigma$ induces group isomorphisms $K'^\times/n\to K^\times/n$ for all $n\in\mathbb{N}$.

In the fourth chapter, we want to use the correspondences established in the previous chapters to prove Theorem \ref{thm:main-result}. Before we are able to do that, we need to discuss free-by-finite groups. Using the existence of a so-called admissible set, we can construct a group isomorphism $K'^\times\to K^\times$. Finally, we will show that this group isomorphism can be extended to a field isomorphism $K'\to K$. By passing to the limits, we obtain a field isomorphism $\varphi: \overline{K'}\to\overline{K}$ which induces $\sigma$.

\section{Acknowledgments}

I would like to thank my supervisor, Prof. Dr. Alexander Schmidt, for giving me such an interesting topic to pursue as my master thesis, and for all the insights and patient answers to all the countless questions I had. 

I also wish to offer my sincerest thanks to all my friends, who supported me in this endeavor and encouraged me to go on. Last but not least, I am grateful for my parents who never ran out of patience with me. 

This thesis was hard, arduous work, but, in the end, also very satisfying. And I couldn't have done it without all of you. Thank you.

\clearpage

\section{Notation}

$\begin{array}{ll}
\mathbb{N} & \text{natural numbers starting from $1$} \\
\mathbb{Z} & \text{integers}\\
\mathbb{Q} & \text{rational numbers}\\
\mathbb{R} & \text{real numbers}\\
\mathbb{C} & \text{complex numbers}\\
\mathbb{F}_q & \text{finite field with $q$ elements}\\
\mathbb{Z}_p & \text{$p$-adic integers}\\
\widehat{\mathbb{Z}} & \text{profinite completion of $\mathbb{Z}$}\\
\\
\overline{K} & \text{algebraic closure of $K$}\\
K^\text{s} & \text{separable closure of $K$}\\
K^\text{t} & \text{absolute inertia field of a henselian $K$}\\
K_v & \text{henselization of $K$ with respect to $v$}\\
\mu_n & \text{$n$-th roots of unity in a separable closure}\\
\\
G_K & \text{absolute Galois group of $K$}\\
\operatorname{Gal}(L/K) & \text{Galois group of $L/K$}\\
\operatorname{char}(K) & \text{characteristic of $K$}\\
\operatorname{trdeg}(L/K) & \text{transcendence degree of $L/K$}\\
\end{array}$
