\chapter{Synopsis}

\section{Some results on $\operatorname{Spv}(K)$}

\begin{definition}
Let $K$ be a field and $P$ its prime field. The \textit{(Kronecker) dimension}\index{dimension!Kronecker} $\dim(K)$ of $K$ is defined as:
\[\dim(K) = \begin{cases}
\operatorname{trdeg}(K/P), & \operatorname{char}(K)>0\\
\operatorname{trdeg}(K/P)+1, & \text{otherwise}
\end{cases} \]
\end{definition}

\begin{definition}
Let $K$ be a field of finite dimension. We denote the set of equivalence classes of non-archimedean valuations of $K$ by $\operatorname{Spv}(K)$. We often do not distinguish between a valuation and its equivalence class. 

We denote the valuation ring of $v\in\operatorname{Spv}(K)$ by $(\mathcal{O}_v,\mathfrak{m}_v)$ and set $U_v=\mathcal{O}_v^\times$ and $k(v)=\mathcal{O}_v/\mathfrak{m}_v$. The \textit{rank}\index{valuation!rank} of $v$ is defined as the Krull dimension of its valuation ring $\mathcal{O}_v$. The \textit{rational rank}\index{valuation!rational rank} of $v:K\to\Gamma_v$ is defined as the rank of its valuation group $\Gamma_v$ as a $\mathbb{Z}$-module:
\[\operatorname{rrank}(v) = \dim_\mathbb{Q}(\Gamma_v\otimes_\mathbb{Z}\mathbb{Q}) \]
\end{definition}

\begin{lemma}
Let $v\in\operatorname{Spv}(K)$. Then we have:
\[\operatorname{rank}(v)\leq \operatorname{rrank}(v)\quad\text{and}\quad\operatorname{rrank}(v)+\dim k(v)\leq\dim(K) \]
\end{lemma}

\begin{lemma}
Let $K$ be a finitely generated field and $v\in\operatorname{Spv}(K)$ with:
\[\operatorname{rrank}(v)+\dim k(v)= \dim(K) \]
Then $\Gamma_v$ is a finitely generated $\mathbb{Z}$-module and $k(v)$ is a finitely generated field.
\end{lemma}

\begin{definition}
We define:
\begin{align*}
\operatorname{Spv}_\text{d}(K) &= \{v\in\operatorname{Spv}(K)\mid\operatorname{rank}(v)=\operatorname{rrank}(v) \} \\
\operatorname{Spv}_\text{rd}(K) &= \{v\in\operatorname{Spv}_\text{d}(K)\mid \operatorname{rrank}(v)+\dim k(v)=\dim(K) \}
\end{align*}
Valuations in $\operatorname{Spv}_\text{d}$ are called \textit{defectless}\index{valuation!defectless}, those in $\operatorname{Spv}_\text{rd}$ are called \textit{rank-defectless}\index{valuation!rank-defectless}. For $\ast\in\{\text{d},\text{rd}\}$ we set:
\[\operatorname{Spv}_\ast^{(i)}(K)=\{v\in\operatorname{Spv}_\ast(K)\mid \operatorname{rrank}(v)=i \} \]
\end{definition}

\begin{definition}
We can endow $\operatorname{Spv}(K)$ with a natural topology by declaring following sets as a basis of open sets:
\[\{v\in\operatorname{Spv}(K)\mid v(x)\geq 0\text{ for all }x\in S\} \]
where $S$ ranges over all finite subsets of $K$. 
\end{definition}

\begin{lemma}
$\operatorname{Spv}(K)$ with the topology defined above is a quasicompact $T_0$-space. Every closed subset $A\subset\operatorname{Spv}(K)$ is the closure of its closed points. In particular the set of closed points $|\operatorname{Spv}(K)|$ of $\operatorname{Spv}(K)$ is a quasicompact $T_1$-space, endowed with the induced topology.
\end{lemma}

\begin{lemma}
Let $v,w\in\operatorname{Spv}(K)$. We have:
\[\mathcal{O}_w\subset\mathcal{O}_v \iff w\in\overline{\{v\}} \]
Thus $v$ is a closed point if and only if $\dim k(v)=0$.
\end{lemma}

\begin{lemma}
Let $v\in\operatorname{Spv}(K)$. There is a canonical bijection:
\[\{\text{Specializations of }v \} \to \operatorname{Spv}k(v),\ w\mapsto w\mod v\]
If $w\in\operatorname{Spv}_\text{rd}^{(i)}(K)$ for some $i\geq 1$ then there exists a unique $v(w)\in\operatorname{Spv}_\text{rd}^{(i-1)}(K)$ such that $\mathcal{O}_w\subset\mathcal{O}_v$. $w$ is henselian if and only if $v$ and $w\mod v$ are henselian.
\end{lemma}

\paragraph{} Let $K$ be a field of characteristic $0$ and dimension $2$.

\begin{lemma}
Let $K$ be henselian with respect to two different valuations $w_1,w_2\in |\operatorname{Spv}(K)|$. Then $K$ is algebraically closed or $w_1,w_2\in\operatorname{Spv}_\text{rd}^{(2)}(K),\ v(w_1)=v(w_2)$ and $k(v(w_1))$ is separably closed. In the last case we have $\operatorname{cd}(G_K)\leq 2$.
\end{lemma}

\begin{definition}
Let $K$ be finitely generated. A \textit{model} of $K$ is a regular, proper scheme $\mathfrak{X}$ over $\mathbb{Z}$ with function field $K$. A \textit{morphism of models} is a morphism of schemes inducing the identity on $K$. We denote the category of models of $K$ by $\mathbf{M}(K)$; it is filtered. 

For every model $\mathfrak{X}$ there is a canonical map $\pi_\mathfrak{X}:\operatorname{Spv}(K)\to\mathfrak{X}$. $\pi_\mathfrak{X}(v)$  is the unique point of $\mathfrak{X}$ such that $\mathcal{O}_{\mathfrak{X}, \pi_\mathfrak{X}(v)}\subset\mathcal{O}_v$ and the inclusion $\mathcal{O}_{\mathfrak{X}, \pi_\mathfrak{X}(v)}\hookrightarrow \mathcal{O}_v$ is a homomorphism of local rings.
\end{definition}

\begin{lemma}
\begin{enumerate}[(i)]
\item $\operatorname{Spv}(K)$ together with the maps $\pi_{\mathfrak{X}}$ is the projective limit of topological spaces: 
\[\operatorname{Spv}(K) =\varprojlim_{\mathfrak{X}\in\mathbf{M}(K)}\mathfrak{X}\]
In particular we have:
\[ |\operatorname{Spv}(K)| =\varprojlim_{\mathfrak{X}\in\mathbf{M}(K)}|\mathfrak{X}| \]
\item For all $v\in\operatorname{Spv}(K)$ we have:
\[ \mathcal{O}_v = \varinjlim_{\mathfrak{X}\in\mathbf{M}(K)}\mathcal{O}_{\mathfrak{X},\pi_\mathfrak{X}(v)} \]
\item For $v\in\operatorname{Spv}(K)$ let $K_v$ be the quotient field of a henselization of $\mathcal{O}_v$ and for $\mathfrak{X}\in\mathbf{M}(K),\ x\in\mathfrak{X}$ let $K_\mathfrak{X}$ be the quotient field of a henselization of $\mathcal{O}_{\mathfrak{X},x}$. Then:
\[ K_v=\varinjlim_{\mathfrak{X}\in\mathbf{M}(K)}K_{\pi_\mathfrak{X}(v)} \]
\end{enumerate}
\end{lemma}

\begin{lemma}\label{lemma:main-result}
Let $\ell\neq 2$ be a prime. The following map, induced by the restrictions, is injective:
\[ \mathrm{H}^3(K,\mathbb{Z}/\ell\mathbb{Z})\to\prod_{w\in |\operatorname{Spv}(K)| }\mathrm{H}^3(K_w,\mathbb{Z}/\ell\mathbb{Z}) \]
\end{lemma}

\begin{lemma}
Let $K$ be henselian with respect to $w\in|\operatorname{Spv}(K)|$. Set $k=k(w)$ and $\Gamma = \Gamma_w$. Then $k$ is an algebraic extension of a finite field $\mathbb{F}_p$. Let $\ell\neq p$ be a prime and assume $\mu_\ell\subset K$. Then the following conditions are equivalent:
\begin{enumerate}[(i)]
\item $\mathrm{H}^3(K,\mathbb{Z}/\ell\mathbb{Z})=\mathbb{Z}/\ell\mathbb{Z}$
\item $\ell^\infty\nmid[k:\mathbb{F}_p]$ and $\dim_{\mathbb{F}_\ell}(\Gamma/\ell\Gamma) = 2$
\end{enumerate}
Otherwise we have $\mathrm{H}^3(K,\mathbb{Z}/\ell\mathbb{Z})=0$. If $L/K$ is an algebraic extension and (i) holds, we have:
\[ \mathrm{H}^3(L,\mathbb{Z}/\ell\mathbb{Z}) = \begin{cases}
\mathbb{Z}/\ell\mathbb{Z}, & \ell^\infty\nmid [L:K]\\
0, & \text{otherwise}
\end{cases} \]
\end{lemma}

\begin{proposition}
Let $K_0/K$ be an algebraic extension.
\begin{enumerate}[(i)]
\item Suppose there exists a prime number $\ell$ with $\ell^\infty\nmid[K_0:K]$ and such that for every algebraic extension $L/K_0$ we have: 
\[ \mathrm{H}^3(L,\mathbb{Z}/\ell\mathbb{Z}) = \begin{cases}
\mathbb{Z}/\ell\mathbb{Z}, & \ell^\infty\nmid [L:K_0]\\
0, & \text{otherwise}
\end{cases} \]
Then there exists a unique henselian $w\in|\operatorname{Spv}(K)|$.
\item Assume that the hypothesis of (i) is satisfied for two different prime numbers. Then $w$ is defectless.
\end{enumerate}
\end{proposition}

\section{Local correspondence}

Let $K,K'$ be two finitely generated fields of characteristic $0$ and dimension $2$ and $\sigma:G_K\to G_{K'}$ be an isomorphism of profinite groups. For intermediate fields $L$ of $\overline{K}/K$ the corresponding intermediate field of $\overline{K'}/K'$ will be denoted by $L'$. In other words $\sigma G_L=G_{L'}$.

\begin{lemma}
Let $w\in\operatorname{Spv}_\text{d}^{(2)}(K)$ and let $E=K_w$ be a henselization of $w$. Then $E'$ is a henselization of $K'$ with respect to a unique $w'\in\operatorname{Spv}_\text{d}^{(2)}(K')$. The map $w\mapsto w'$ does not depend on the choice of the henselization.
\end{lemma}

\begin{lemma}
$\sigma$ induces a bijection $\sigma:\operatorname{Spv}_\text{d}^{(1)}(K)\to\operatorname{Spv}_\text{d}^{(1)}(K')$. If $E=K_v\subset\overline{K}$ is a henselization of $K$ with respect to $v\in\operatorname{Spv}_\text{d}^{(1)}(K)$ then $E'$ is a henselization of $K'$ with respect to $v'=\sigma(v)$. Furthermore, if $E^{\text{t}}$ and $E'^{\text{t}}$ denotes the inertial fields of $E$ and $E'$ respectively, then $\sigma$ induces an isomorphism:
\[ G_{k(v)}\cong\operatorname{Gal}(E^\text{t}/E)\to\operatorname{Gal}(E'^{\text{t}}/E')\cong G_{k(v')} \]
which is induced by a unique isomorphism $\varphi_v^\text{s}:k(v')^\text{s}\to k(v)^\text{s}$. In particular $\varphi_v=\varphi_v^\text{s}|_{k(v')}:k(v')\to k(v)$ is an isomorphism.
\end{lemma}
