\chapter{Synopsis}

\section{Some Results on $\operatorname{Spv}(K)$}

\begin{definition}
Let $K$ be a field and $P$ its prime field. The \textit{(Kronecker) dimension}\index{dimension!Kronecker} $\dim(K)$ of $K$ is defined as:
\[\dim(K) = \begin{cases}
\operatorname{trdeg}(K/P), & \operatorname{char}(K)>0\\
\operatorname{trdeg}(K/P)+1, & \text{otherwise}
\end{cases} \]
\end{definition}

\begin{definition}\label{2.1}
Let $K$ be a field of finite dimension. We denote the set of equivalence classes of non-archimedean valuations of $K$ by $\operatorname{Spv}(K)$. We often do not distinguish between a valuation and its equivalence class. 

We denote the valuation ring of $v\in\operatorname{Spv}(K)$ by $(\mathcal{O}_v,\mathfrak{m}_v)$ and set $U_v=\mathcal{O}_v^\times$ and $k(v)=\mathcal{O}_v/\mathfrak{m}_v$. The \textit{rank}\index{valuation!rank} of $v$ is defined as the Krull dimension of its valuation ring $\mathcal{O}_v$. The \textit{rational rank}\index{valuation!rational rank} of $v:K\to\Gamma_v$ is defined as the rank of its valuation group $\Gamma_v$ as a $\mathbb{Z}$-module:
\[\operatorname{rrank}(v) = \dim_\mathbb{Q}(\Gamma_v\otimes_\mathbb{Z}\mathbb{Q}) \]
\end{definition}

\begin{lemma}
Let $v\in\operatorname{Spv}(K)$. Then we have:
\[\operatorname{rank}(v)\leq \operatorname{rrank}(v)\quad\text{and}\quad\operatorname{rrank}(v)+\dim k(v)\leq\dim(K) \]
\end{lemma}

\begin{lemma}
Let $K$ be a finitely generated field and $v\in\operatorname{Spv}(K)$ with:
\[\operatorname{rrank}(v)+\dim k(v)= \dim(K) \]
Then $\Gamma_v$ is a finitely generated $\mathbb{Z}$-module and $k(v)$ is a finitely generated field.
\end{lemma}

\begin{definition}
We define:
\begin{align*}
\operatorname{Spv}_\text{d}(K) &= \{v\in\operatorname{Spv}(K)\mid\operatorname{rank}(v)=\operatorname{rrank}(v) \} \\
\operatorname{Spv}_\text{rd}(K) &= \{v\in\operatorname{Spv}_\text{d}(K)\mid \operatorname{rrank}(v)+\dim k(v)=\dim(K) \}
\end{align*}
Valuations in $\operatorname{Spv}_\text{d}$ are called \textit{defectless}\index{valuation!defectless}, those in $\operatorname{Spv}_\text{rd}$ are called \textit{rank-defectless}\index{valuation!rank-defectless}. For $\ast\in\{\text{d},\text{rd}\}$ we set:
\[\operatorname{Spv}_\ast^{(i)}(K)=\{v\in\operatorname{Spv}_\ast(K)\mid \operatorname{rrank}(v)=i \} \]
\end{definition}

\begin{definition}
We can endow $\operatorname{Spv}(K)$ with a natural topology by declaring following sets as a basis of open sets:
\[\{v\in\operatorname{Spv}(K)\mid v(x)\geq 0\text{ for all }x\in S\} \]
where $S$ ranges over all finite subsets of $K$. 
\end{definition}

\begin{lemma}
$\operatorname{Spv}(K)$ with the topology defined above is a quasicompact $T_0$-space. Every closed subset $A\subset\operatorname{Spv}(K)$ is the closure of its closed points. In particular the set of closed points $|\operatorname{Spv}(K)|$ of $\operatorname{Spv}(K)$ is a quasicompact $T_1$-space, endowed with the induced topology.
\end{lemma}

\begin{lemma}
Let $v,w\in\operatorname{Spv}(K)$. We have:
\[\mathcal{O}_w\subset\mathcal{O}_v \iff w\in\overline{\{v\}} \]
Thus $v$ is a closed point if and only if $\dim k(v)=0$.
\end{lemma}

\begin{lemma}
Let $v\in\operatorname{Spv}(K)$. There is a canonical bijection:
\[\{ w\in\operatorname{Spv}(K)\mid \mathcal{O}_w\subset\mathcal{O}_v \} \to \operatorname{Spv}k(v),\ w\mapsto w\mod v\]
If $w\in\operatorname{Spv}_\text{rd}^{(i)}(K)$ for some $i\geq 1$ then there exists a unique $v(w)\in\operatorname{Spv}_\text{rd}^{(i-1)}(K)$ such that $\mathcal{O}_w\subset\mathcal{O}_v$. $w$ is henselian if and only if $v$ and $w\mod v$ are henselian.
\end{lemma}

\paragraph{} Let $K$ be a finitely generated field of characteristic $0$ and dimension $2$.

\begin{lemma}\label{2.2}
Let $K$ be henselian with respect to two different valuations $w_1,w_2\in |\operatorname{Spv}(K)|$. Then $K$ is algebraically closed or $w_1,w_2\in\operatorname{Spv}_\text{rd}^{(2)}(K),\ v(w_1)=v(w_2)$ and $k(v(w_1))$ is separably closed. In the last case we have $\operatorname{cd}(G_K)\leq 2$.
\end{lemma}

\begin{definition}
Let $K$ be finitely generated. A \textit{model} of $K$ is a regular, proper scheme $\mathfrak{X}$ over $\mathbb{Z}$ with function field $K$. A \textit{morphism of models} is a morphism of schemes inducing the identity on $K$. We denote the category of models of $K$ by $\mathfrak{M}(K)$; it is filtered. 

For every model $\mathfrak{X}$ there is a canonical map $\pi_\mathfrak{X}:\operatorname{Spv}(K)\to\mathfrak{X}$. $\pi_\mathfrak{X}(v)$  is the unique point of $\mathfrak{X}$ such that $\mathcal{O}_{\mathfrak{X}, \pi_\mathfrak{X}(v)}\subset\mathcal{O}_v$ and the inclusion $\mathcal{O}_{\mathfrak{X}, \pi_\mathfrak{X}(v)}\hookrightarrow \mathcal{O}_v$ is a homomorphism of local rings.
\end{definition}

\begin{lemma}\phantomsection\label{2.3}
\begin{enumerate}[(i)]
\item $\operatorname{Spv}(K)$, together with the maps $\pi_{\mathfrak{X}}$, is the projective limit of topological spaces: 
\[\operatorname{Spv}(K) =\varprojlim_{\mathfrak{X}\in\mathfrak{M}(K)}\mathfrak{X}\]
In particular we have:
\[ |\operatorname{Spv}(K)| =\varprojlim_{\mathfrak{X}\in\mathfrak{M}(K)}|\mathfrak{X}| \]
\item For all $v\in\operatorname{Spv}(K)$ we have:
\[ \mathcal{O}_v = \varinjlim_{\mathfrak{X}\in\mathfrak{M}(K)}\mathcal{O}_{\mathfrak{X},\pi_\mathfrak{X}(v)} \]
\item For $v\in\operatorname{Spv}(K)$ let $K_v$ be the quotient field of a henselization of $\mathcal{O}_v$ and for $\mathfrak{X}\in\mathfrak{M}(K),\ x\in\mathfrak{X}$ let $K_\mathfrak{X}$ be the quotient field of a henselization of $\mathcal{O}_{\mathfrak{X},x}$. Then:
\[ K_v=\varinjlim_{\mathfrak{X}\in\mathfrak{M}(K)}K_{\pi_\mathfrak{X}(v)} \]
\end{enumerate}
\end{lemma}

\begin{lemma}\label{2.4}
Let $\ell\neq 2$ be a prime. The following map, induced by the restrictions, is injective:
\[ \mathrm{H}^3(K,\mathbb{Z}/\ell\mathbb{Z})\to\prod_{w\in |\operatorname{Spv}(K)| }\mathrm{H}^3(K_w,\mathbb{Z}/\ell\mathbb{Z}) \]
\end{lemma}

\begin{lemma}\label{2.5}
Let $K$ be henselian with respect to $w\in|\operatorname{Spv}(K)|$. Set $k=k(w)$ and $\Gamma = \Gamma_w$. Then $k$ is an algebraic extension of a finite field $\mathbb{F}_p$. Let $\ell\neq p$ be a prime and assume $\mu_\ell\subset K$. Then the following conditions are equivalent:
\begin{enumerate}[(i)]
\item $\mathrm{H}^3(K,\mathbb{Z}/\ell\mathbb{Z})=\mathbb{Z}/\ell\mathbb{Z}$
\item $\ell^\infty\nmid[k:\mathbb{F}_p]$ and $\dim_{\mathbb{F}_\ell}(\Gamma/\ell\Gamma) = 2$
\end{enumerate}
Otherwise we have $\mathrm{H}^3(K,\mathbb{Z}/\ell\mathbb{Z})=0$. If $L/K$ is an algebraic extension and (i) holds, we have:
\[ \mathrm{H}^3(L,\mathbb{Z}/\ell\mathbb{Z}) = \begin{cases}
\mathbb{Z}/\ell\mathbb{Z}, & \ell^\infty\nmid [L:K]\\
0, & \text{otherwise}
\end{cases} \]
\end{lemma}

\begin{proposition}\label{2.6}
Let $K_0/K$ be an algebraic extension.
\begin{enumerate}[(i)]
\item Suppose there exists a prime number $\ell$ with $\ell^\infty\nmid[K_0:K]$ and such that for every algebraic extension $L/K_0$ we have: 
\[ \mathrm{H}^3(L,\mathbb{Z}/\ell\mathbb{Z}) = \begin{cases}
\mathbb{Z}/\ell\mathbb{Z}, & \ell^\infty\nmid [L:K_0]\\
0, & \text{otherwise}
\end{cases} \]
Then there exists a unique henselian $w\in|\operatorname{Spv}(K)|$.
\item Assume that the hypothesis of (i) is satisfied for two different prime numbers. Then $w$ is defectless.
\end{enumerate}
\end{proposition}

\section{Local Correspondence}

Let $K,K'$ be two finitely generated fields of characteristic $0$ and dimension $2$ and $\sigma:G_K\to G_{K'}$ be an isomorphism of profinite groups. For intermediate fields $L$ of $\overline{K}/K$ the corresponding intermediate field of $\overline{K'}/K'$ will be denoted by $L'$. In other words $\sigma G_L=G_{L'}$.

\begin{lemma}\label{3.1}
Let $w\in\operatorname{Spv}_\text{d}^{(2)}(K)$ and let $E=K_w$ be a henselization of $w$. Then $E'$ is a henselization of $K'$ with respect to a unique $w'\in\operatorname{Spv}_\text{d}^{(2)}(K')$. The map $w\mapsto w'$ does not depend on the choice of the henselization.
\end{lemma}

\begin{lemma}\label{3.2}
$\sigma$ induces a bijection $\sigma:\operatorname{Spv}_\text{d}^{(1)}(K) \to\operatorname{Spv}_\text{d}^{(1)}(K')$. If $E=K_v\subset\overline{K}$ is a henselization of $K$ with respect to $v\in\operatorname{Spv}_\text{d}^{(1)}(K)$, then $E'$ is a henselization of $K'$ with respect to $v'=\sigma(v)$. Furthermore, if $E^{\text{t}}$ and $E'^{\text{t}}$ denote the inertial fields of $E$ and $E'$ respectively, then $\sigma$ induces an isomorphism:
\[ G_{k(v)}\cong\operatorname{Gal}(E^\text{t}/E) \to\operatorname{Gal}(E'^{\text{t}}/E')\cong G_{k(v')} \]
which is induced by a unique isomorphism $\varphi_v^\text{s}:k(v')^\text{s}\to k(v)^\text{s}$. In particular $\varphi_v=\varphi_v^\text{s}|_{k(v')}:k(v')\to k(v)$ is an isomorphism.
\end{lemma}

\section{Global Correspondence mod $n$}

\begin{lemma}\label{4.1}
Let $E$ be a number field and let $X$ be a smooth projective geometrically connected curve over $E$. Let $\pi:X'\to X$ be a connected Galois covering of degree $d$ which splits completely, i.e. every closed point $P\in X_0$ has exactly $d$ pre-images in $X'$. Then $d=1$.
\end{lemma}

\paragraph{} Let $k$ and $k'$ be the algebraic closures of $\mathbb{Q}$ in $K$ and $K'$ respectively, and let $\overline{k}$ and $\overline{k'}$ be the algebraic closure of $k$ and $k'$ in $\overline{K}$ and $\overline{K'}$ respectively.

Let $X$ and $X'$ be the unique smooth projective curve over $k$ and $k'$ with function field $K$ and $K'$ respectively. We identify the set of closed points $X_0$ of $X$ with the set of valuations $v\in\operatorname{Spv}_\text{d}^{(1)}(K)$ with $\operatorname{char} k(v)=0$ and similarly identify $X'_0$ with the corresponding subset of $\operatorname{Spv}_\text{d}^{(1)}(K')$.

\begin{lemma}\label{4.2}
We have $(K\overline{k})'=K'\overline{k'}$, i.e. $\sigma(G_{K\overline{k}})=G_{K'\overline{k'}}$. $\sigma$ induces an isomorphism $G_k\to G_{k'}$ such that the following diagram commutes:
\[ \begin{tikzcd}
G_K \ar[r, "\sigma"] \ar[d, "\text{res}"'] & G_{K'}\ar[d, "\text{res}"]\\
G_k \ar[r, "\sigma"'] & G_{k'}
\end{tikzcd} \]
Moreover, $\sigma:G_k\to G_{k'}$ is induced by a unique isomorphism $\varphi:\overline{k'}\to\overline{k}$.
\end{lemma}

\paragraph{} Let $n\in\mathbb{N}$. For all intermediate fields $L$ of $\overline{K}/K$, $\sigma$ induces an isomorphism:
\[ \varphi_n:L'^\times/n \cong\mathrm{H}^1(L', \mu_n) \to \mathrm{H}^1(L,\mu_n) \cong L^\times/n \]
Obviously, for intermediate fields $L\subset M$ of $\overline{K}/K$ the following diagram commutes:
\[ \begin{tikzcd}
L'^\times/n \ar[r, "\varphi_n"]\dar & L^\times/n \dar\\
M'^\times/n \ar[r, "\varphi_n"'] & M^\times/n
\end{tikzcd} \]
Let $P\in X_0$ and $v_P:K\to\mathbb{Z}\cup\{\infty\}$ be the corresponding discrete normalized valuation and let $U_P$ be the group of units of the valuation ring $\mathcal{O}_P$. We denote the natural projection $U_P\to k(P)^\times,\ x\mapsto x\mod P$ by $r_P$. For $n\in\mathbb{N}$ the following sequence is exact:
\[ 0\to U_P/n\to K^\times/n\to\mathbb{Z}/n\mathbb{Z}\to 0 \]

\begin{lemma}\phantomsection\label{4.3}
\begin{enumerate}[(i)]
\item $\varphi_n$ induces an isomorphism $\varphi_n: U_{P'}/n\to U_P/n$.
\item The following diagram commutes:
\[ \begin{tikzcd}
U_{P'}/n \rar["\varphi_n"] \dar["r_{P'}/n"'] & U_P/n \dar["r_P/n"]\\
k(P')^\times/n \rar["\varphi_n"'] & k(P)^\times/n
\end{tikzcd} \]
\end{enumerate}
\end{lemma}

\begin{definition}\label{4.4}
For a finite subset $S\subset X_0$ set:
\[ K^S=\{x\in K^\times \mid v_P(x)=0\text{ for all }P\in X_0\setminus S \} \]
$K'^{S'}$ is defined similarly. For finite subsets $S$ of $X_0$, the corresponding subsets of $X_0'$ will be denoted by $S'$, i.e. $S'=\{\sigma P\mid P\in S \} $.
\end{definition}

\begin{lemma}\phantomsection\label{4.5}
\begin{enumerate}[(i)]
\item There exists a finite subset $S\subset X_0$ such that:
\[ \operatorname{Pic}(X\setminus S)=0=\operatorname{Pic}(X'\setminus S') \]
Such a set $S$ is called \textit{admissible}. If $V\subset X$ is an open subscheme, then one can choose $S$ in $V_0$.
\item Let $S$ be admissible. For all $n\in\mathbb{N}$ the following canonical maps are injective:
\[ K^S/n \to K^\times /n,\quad K'^{S'}/n\to K'^\times /n \]
$\varphi_n$ induces isomorphisms $\varphi_n: K'^{S'}/n\to K^S/n$.
\end{enumerate}
\end{lemma}

\section{The Proof}

\begin{definition}\label{5.1}
An abelian group is called \textit{free-by-finite} if its torsion subgroup $A_\text{tor}$ is finite and the quotien $A/A_\text{tor}$ is free abelian. $A$ is free-by-finite if and only if $A$ is the direct sum of a finite group and a free abelian group.
\end{definition}

\begin{lemma}\phantomsection\label{5.2}
\begin{enumerate}[(i)]
\item Let $A\to B$ be a homomorphism of abelian groups with finite kernel and cokernel. Then $A$ is free-by-finite iff $B$ is free-by-finite.
\item Let $0\to A'\stackrel{i}{\to}A\stackrel{p}{\to}A''\to 0$ be an exact sequence of abelian groups. If $A'$ and $A''$ are free-by-finite, then $A$ is free-by-finite. Conversely, if $A$ is free-by-finite, then $A'$ is free-by-finite. If additionally, $A'$ is finitely generated, then $A''$ is free-by-finite.
\item Let $A$ be free-by-finite. Then $\bigcap_{n\in\mathbb{N}}nA=0$.
\end{enumerate}
\end{lemma}

\begin{lemma}\label{5.3}
Let $F$ be a number field. Then:
\begin{enumerate}[(i)]
\item $F^\times$ is free-by-finite.
\item Let $F_1,\ldots,F_n$ be finite extensions of $F$. Then $\prod_{i=1}^{n}F_i^\times/F^\times$ is free-by-finite, where $F^\times$ is considered as a subgroup of $\prod_{i=1}^{n}F_i^\times$ via the diagonal embedding.
\end{enumerate}
\end{lemma}

\begin{lemma}\label{5.4}
Let $S\subset X_0$ be admissible. Then there exists a finite subset $T\subset X_0$ with $T\cap S=\varnothing$ such that the following map is injective:
\[ r_{S,T}:K^S \to\prod_{P\in T}k(P)^\times,\ x\mapsto(r_P(x))_{P\in T} \]
In this case, $\operatorname{coker}(r_{S,T})$ is free-by-finite.
\end{lemma}

\begin{definition}
For each admissible $S\subset X_0$ we can find a finite subset $T\subset X_0\setminus S$ such that both $r_{S,T}$ and $r_{S',T'}$ are injective. Such a $T$ will be called $S$-\textit{admissible} and $(S,T)$ is called \textit{admissible pair}.
\end{definition}

\begin{theorem}[\textit{Pop}]\label{1.1}
Let $K,K'$ be two finitely generated fields of characteristic $0$ and dimension $2$. Suppose there is an isomorphism $\sigma:G_K\to G_{K'}$ between their absolute Galois groups, then there exists a unique field isomorphism $\varphi:\overline{K'}\to\overline{K}$ with $\sigma(g)=\varphi^{-1}g\varphi$ for all $g\in G_K$. In particular $\varphi$ induces a field isomorphism $\varphi:K'\to K$.
\end{theorem}

\begin{step}
There exists a unique group isomorphism $\varphi:K'^\times\to K^\times$ such that for any $P\in X_0$ we have $\varphi(U_{P'})=U_P$ where $P'=\sigma(P)$ and the following diagram commutes:
\[ \begin{tikzcd}
U_{P'} \ar[r, "\varphi"]\ar[d, "r_{P'}"'] & U_P \ar[d, "r_P"]\\
k(P')^\times \ar[r, "\varphi_P"'] & k(P)^\times
\end{tikzcd} \]
\end{step}

\begin{proof}
Since any rational function on $X$ is uniquely determined by its values at those points where it is defined, $\varphi$ is unique. Furthermore, the following maps are injective for all finite subsets $S\subset X_0$:
\[ r_S:K^S \to\prod_{P\in X_0\setminus S} k(P)^\times,\ x\mapsto (r_P(x))_P \]
For existence, let $(S,T)$ be an admissible pair. First, we show that there is a unique homomorphism $\varphi^{S,T}$, making the following diagram commutative:
\[ (\star) \qquad \begin{tikzcd}
K'^{S'} \rar["\varphi^{S,T}", dashed] \dar["r_{S',T'}"'] & K^S \dar["r_{S,T}"]\\
\displaystyle \prod_{P\in T}k(\sigma P)^\times \rar["\prod\varphi_P"'] & \displaystyle \prod_{P\in T}k(P)^\times
\end{tikzcd} \phantom{\qquad (\star)} \]
The vertical maps are injective by definition. Thus, it suffices to show that the composition $\varrho:K'^{S'}\to\prod k(\sigma P)^\times\to \prod k(P)^\times\to\operatorname{coker}(r_{S,T})$ is trivial. Consider the commutative diagram which results from tensoring $\mathbb{Z}/n\mathbb{Z}$ with $(\star)$:
\[ \begin{tikzcd}
K'^{S'}/n \rar[dashed]\dar & K^S/n\dar\\
\displaystyle \prod_{P\in T}k(\sigma P)^\times/n \rar & \displaystyle \prod_{P\in T}k(P)^\times/n
\end{tikzcd} \]
According to Lemma~\ref{4.3} and Lemma~\ref{4.5} this diagram can now be completed by $\varphi_n: K'^{S'}/n\to K^S/n$. This implies $\varrho/n=0$, i.e. $\operatorname{im}(\varrho)\subset n\operatorname{coker}(r_{S,T})$ for every $n\in\mathbb{N}$. By Lemma~\ref{5.2} (iii) and Lemma~\ref{5.4} it follows $\varrho = 0$.

Let $T_1, T_2\subset X_0\setminus S$ be two $S$-admissible subsets and suppose $T_1\subset T_2$. We have:
\[ r_{S,T_1}\circ\varphi^{S,T_2}=\Big(\prod_{P\in T_1}\varphi_P\Big)\circ r_{S',T_1'} = r_{S,T_1}\circ\varphi^{S,T_1} \]
Since $r_{S,T_1}$ is injective we get $\varphi^{S,T_2}=\varphi^{S,T_1}$. Hence, $\varphi^{S,T}$ is actually independent of $T$, thus we can omit the $T$ from its notation. By symmetry, $\varphi^S$ is an isomorphism. 

Now, if $T$ ranges over all $S$-admissible subsets of $X_0$, then $(\star)$ shows:
\[ r_S\circ\varphi^S = \Big(\prod_{P\in X_0\setminus S}\varphi_P \Big)\circ r_{S'} \]
In other words, we can replace $T$ by $X_0\setminus S$ in $(\star)$. If $S_1\subset S_2$ are two admissible subsets, then:
\[ r_{S_2}\circ\varphi^{S_2}|_{K'^{S'_1}} = \Big(\prod_{P\in X_0\setminus S_2}\varphi_P \Big)\circ r_{S'_2}|_{K'^{S'_1}} = r_{S_2}\circ\varphi^{S_1} \]
Hence, $\varphi^{S_1}=\varphi^{S_2}|_{K'^{S'_1}}$. Thus, we can combine the isomorphisms $\varphi^S$, where $S$ ranges over the admissible subsets of $X_0$, into an isomorphism $\varphi = \varinjlim \varphi^S: K'^\times \to K^\times$. $\varphi$ has the desired properties.
\end{proof}

\begin{step}
Extending $\varphi$ to a map $K'\to K$ by setting $\varphi(0)=0$ yields a field isomorphism $\varphi:K'\to K$.
\end{step}

\begin{step}
$\varphi$ induces $\sigma:G_K\to G_{K'}$, i.e. $\sigma(g) = \varphi^{-1}g\varphi$ for all $g\in G_K$, and is uniquely determined by this property.
\end{step}