\chapter{Kohomologie}

Wir verwenden folgende Bezeichnungen für abelsche Kategorien:
\begin{itemize}
\item $\mathbf{Ab}$, die Kategorie der abelschen Gruppen
\item $\mathbf{Mod}(A)$, die Kategorie der Moduln über einem kommutativen Ring $A$ mit Eins
\item $\mathbf{Ab}(X)$, die Kategorie der Garben abelscher Gruppen über einem topologischen Raum $X$
\item $\mathbf{Mod}(X)$, die Kategorie der $\mathcal{O}_X$"~Moduln auf einem geringten Raum $(X,\mathcal{O}_X)$
\item $\mathbf{Qcoh}(X)$, die Kategorie der quasikohärenten $\mathcal{O}_X$"~Moduln auf einem Schema $X$
\item $\mathbf{Coh}(X)$, die Kategorie der kohärenten $\mathcal{O}_X$"~Modulgarben auf einem noetherschen Schema $X$
\end{itemize}

\section{Kohomologie von Garben}

\paragraph{Satz 9.15.}\label{9.15} Sei $A$ ein Ring. Dann ist jeder $A$"~Modul isomorph zu einem Untermodul eines injektiven $A$"~Moduls.

\paragraph{Satz 9.16.}\label{9.16} Sei $(X,\mathcal{O}_X)$ ein geringter Raum. Dann hat die Kategorie $\mathbf{Mod}(\mathcal{O}_X)$ genügend viele Injektive.

\paragraph{Beweis.} Sei $\mathcal{F}\in\mathbf{Mod}(\mathcal{O}_X)$. Dann ist $\mathcal{F}_x\in\mathbf{Mod}(\mathcal{O}_{X,x})$. Nach \hyperref[9.15]{Satz 9.15} gibt es ein Monomorphismus $\mathcal{F}_x\to I_x$ mit einem injektiven $\mathcal{O}_{X,x}$"~Modul $I_x$. Sei $j:\{x\}\hookrightarrow X$ die kanonische Inklusion und betrachte die $I_x$ als Garben auf $\{x\}$. Setze $\mathcal{J}=\prod_{x\in X}j_\ast I_x$. Dann gilt:
\[\operatorname{Hom}_{\mathcal{O}_X}(\mathcal{F},\mathcal{J}) = \prod_{x\in X}\operatorname{Hom}_{\mathcal{O}_X}(\mathcal{F},j_\ast I_x) = \prod_{x\in X}\operatorname{Hom}_{\mathcal{O}_{X,x}}(\mathcal{F}_x,I_x) \]
Die $\mathcal{F}_x\hookrightarrow I_x$ geben ein Monomorphismus $\mathcal{F}\hookrightarrow \mathcal{J}$. Ferner gilt:
\[\operatorname{Hom}_{\mathcal{O}_X}(-,\mathcal{J})=\prod_{x\in X}\operatorname{Hom}_{\mathcal{O}_{X,x}}((-)_x, I_x) \]
Da Halmbildung exakt und $I_x$ injektive Objekte sind, ist dies ein exakter Funktor. Somit ist $\mathcal{J}$ ein injektiver $\mathcal{O}_X$"~Modul.\qed

\paragraph{Korollar 9.17.}\label{9.17} Sei $X$ ein topologischer Raum. Dann besitzt $\mathbf{Ab}(X)$ genügend viele Injektive.

\paragraph{Beweis.} Sei $\mathcal{O}_X$ die konstante Garbe $\mathbb{Z}$ auf $X$. Dann ist $(X,\mathcal{O}_X)$ ein geringter Raum und es gilt $\mathbf{Mod}(\mathcal{O}_X)=\mathbf{Ab}(X)$.\qed

\paragraph{Definition 9.18.}\label{9.18} Sei $X$ ein topologischer Raum und $\Gamma(X,-):\mathbf{Ab}(X)\to \mathbf{Ab}$ der globale Schnittfunktor. Wir definieren:
\[\mathrm{H}^i(X,-)= \mathrm{R}^i\Gamma(X,-) \]
$\mathrm{H}^i(X,\mathcal{F})$ heißt \textit{$i$-te Kohomologiegruppe}\index{Kohomologiegruppe} von der Garbe $\mathcal{F}$. Haben $X$ und $\mathcal{F}$ Zusatzstrukturen, z.B. $X$ Schema oder $\mathcal{F}\in \mathbf{Qcoh}(X)$, so wird die Kohomologie trotzdem immer im obigen Sinne verstanden, d.h. $\mathcal{F}\in\mathbf{Ab}(X)$.

Sei $(X,\mathcal{O}_X)$ ein geringter Raum und $\mathcal{F}$ ein $\mathcal{O}_X$"~Modul. $\mathcal{F}$ heißt\textit{welk}\index{$\mathcal{O}_X$-Modul!welk}, falls für alle offenen $V\subset U$ die Restriktionsabbildung $\mathcal{F}(U)\to\mathcal{F}(V)$ surjektiv ist.

\paragraph{Lemma 9.19.}\label{9.19} Ist $(X,\mathcal{O}_X)$ ein geringter Raum, so ist jeder injektiver $\mathcal{O}_X$"~Modul $\mathcal{J}$ welk.

\paragraph{Beweis.} Seien $i:U\hookrightarrow X$ und $j:V\hookrightarrow X$ die natürlichen Inklusionen. Betrachte die Inklusion $j_!\mathcal{O}_X|_V\hookrightarrow i_!\mathcal{O}_X|_U$ von $\mathcal{O}_X$"~Moduln. Es folgt die Surjektivität von:
\[\mathcal{J}(U) = \operatorname{Hom}(i_!\mathcal{O}_X|_U,\mathcal{J})\to \operatorname{Hom}(j_!\mathcal{O}_X|_V,\mathcal{J})=\mathcal{J}(V)\qedhere \]

\paragraph{Satz 9.20.}\label{9.20} Sei $X$ ein topologischer Raum und $\mathcal{F}$ eine welke Garbe auf $X$. Dann ist $\mathrm{H}^i(X,\mathcal{F})=0$ für alle $i>0$.

\paragraph{Lemma 9.21.}\label{9.21} Sei $X$ ein topologischer Raum. Sei $0\to\mathcal{G}'\to\mathcal{G}\to\mathcal{G}''\to 0$ eine exakte Sequenz von Garben.
\begin{enumerate}[(i)]
\item Ist $\mathcal{G}'$ eine welke Garbe, so ist für alle $U\subset_\text{o}X$ die folgende Sequenz exakt:
\[0\longrightarrow \mathcal{G}'(U)\longrightarrow \mathcal{G}(U)\longrightarrow \mathcal{G}''(U)\longrightarrow 0 \]
\item Sind $\mathcal{G'}$ und $\mathcal{G}$ welk, so auch $\mathcal{G}''$.
\end{enumerate}

\paragraph{Beweis.}
\begin{enumerate}[(i)]
\item Sei o.B.d.A. $U=X$ und sei $s''\in\mathcal{G}''(X)$. Setze:
\[E=\{(U,s)\mid U\subset_\text{o}X,\ s\in\mathcal{G}(U),\ s\mapsto s''|_U \} \]
$E$ ist partiell geordnet bzgl. Inklusion. Jede Kette in $E$ besitzt eine obere Schranke in $E$. Nach Zorns Lemma gibt es ein maximales Element $(U,s)\in E$.

Gibt es ein $x\in X\setminus U$, so existiert eine offene Umgebung $V$ von $x$ und $t\in\mathcal{G}(V)$ mit $t\mapsto s''|_V$, da $\mathcal{G}\to\mathcal{G}''$ surjektiv ist. Da $\mathcal{G}$ welk ist, so setzt sich $s-t\in\mathcal{G}(U\cap V)$ zu einem Schnitt auf $\mathcal{G}(V)$ fort. Ersetzen wir $t$ durch $s-t\in\mathcal{G}(V)$, können wir o.B.d.A. $s=t$ auf $V$ annehmen. Die Existenz von $(V,s)$ ist ein Widerspruch zur Maximalität von $(U,s)$ Somit folgt $U=X$.
\item Für $V\subset U$ offen in $X$ kommutiert das folgende Diagramm mit exakten Zeilen:
\[\begin{tikzcd}
0 \rar & \mathcal{G}'(U)\rar\dar[two heads] & \mathcal{G}(U)\rar\dar[two heads] & \mathcal{G}''(U)\rar\dar & 0\\
0 \rar & \mathcal{G}'(V)\rar & \mathcal{G}(V)\rar & \mathcal{G}''(V)\rar & 0
\end{tikzcd} \]
Es folgt die Surjektivität von $\mathcal{G}''(U)\to\mathcal{G}''(V)$.\qed
\end{enumerate}

\paragraph{Beweis von \hyperref[9.20]{Satz 9.20}.} Sei $0\to\mathcal{F}\to\mathcal{J}\to\mathcal{G}\to 0$ exakt mit $\mathcal{J}$ injektiv. Nun sind $\mathcal{F}$ und $\mathcal{J}$ welk. Nach \hyperref[9.21]{Lemma 9.21} ist auch $\mathcal{G}$ welk und wir haben eine exakte Folge:
\[0\longrightarrow\mathcal{F}(X)\longrightarrow\mathcal{J}(X)\longrightarrow\mathcal{G}(X)\longrightarrow 0 \]
Da $\mathrm{H}^i(X,\mathcal{J})=0$ für alle $i>0$, gilt $\mathrm{H}^1(X,\mathcal{F})=0$ und $\mathrm{H}^i(X,\mathcal{F})=\mathrm{H}^{i-1}(X,\mathcal{G})$ für alle $i\geq 2$. Somit folgt die Aussage per Induktion nach $i$.\qed

\paragraph{Satz 9.22.}\label{9.22} Sei $(X,\mathcal{O}_X)$ ein geringter Raum. Dann sind die derivierten Funktoren von
\[\Gamma(X,-): \mathbf{Mod}(X)\to\mathbf{Ab} \]
gerade die Kohomologie-Funktoren.

\paragraph{Beweis.} $\mathrm{R}^i\Gamma(X,-)$ wird über eine injektive Auflösung in $\mathbf{Mod}(X)$ berechnet. Nun sind injektive Objekte in $\mathbf{Mod}(X)$ nach \hyperref[9.19]{Lemma 9.19} welk und nach \hyperref[9.20]{Satz 9.20} insbesondere azyklisch bzgl. $\Gamma(X,-)$.Somit ist $\mathrm{R}^i\Gamma(X,-)=\mathrm{H}^i(X,-)$.\qed

\paragraph{Bemerkung 9.23.}\label{9.23} Sei $(X,\mathcal{O}_X)$ ein geringter Raum und $\mathcal{F}$ ein $\mathcal{O}_X$"~Modul. Da die Kohomologie durch Auflösungen in der Kategorie $\mathbf{Mod}(X)$ berechnet werden kann, haben alle Kohomologiegruppen von $\mathcal{F}$ eine $A$"~Modulstruktur mit $A=\Gamma(X,\mathcal{O}_X)$.

\paragraph{Lemma 9.24.}\label{9.24} Sei $X$ ein noetherscher topologischer Raum und $\mathcal{F}=\varinjlim_\alpha\mathcal{F}_\alpha$. Dann gilt:
\begin{enumerate}[(i)]
\item $\mathcal{F}(X)=\varinjlim_\alpha\mathcal{F}_\alpha(X)$
\item Sind alle $\mathcal{F}_\alpha$ welk, so auch $\mathcal{F}$.
\end{enumerate}

\paragraph{Beweis.}\begin{enumerate}
\item[(ii)] Für alle $\alpha$ und $V\subset U$ offen in $X$ ist $\mathcal{F}_\alpha(U)\to\mathcal{F}_\alpha(V)$ surjektiv. Da $\varinjlim$ exakt ist, folgt mit (i) die Surjektivität von:
\[\mathcal{F}(U)=\varinjlim_\alpha\mathcal{F}_\alpha(U)\to\varinjlim\mathcal{F}_\alpha(V) = \mathcal{F}(V) \]
\item[(i)] Der Funktor $\Gamma(X,-)$ vertauscht mit $\varinjlim$, siehe z.B. Godement, Topologie algébrique et théorie des faisceaux, p. 169.\qed
\end{enumerate}

\paragraph{Satz 9.25.}\label{9.25} Sei $X$ ein noetherscher topologischer Raum und $(\mathcal{F}_\alpha)_{\alpha\in A}$ ein induktives System in $\mathbf{Ab}(X)$. Dann gibt es einen natürlichen Isomorphismus:
\[\varinjlim\mathrm{H}^i(X,\mathcal{F}_\alpha) \stackrel{\sim}{\to} \mathrm{H}^i(X,\varinjlim\mathcal{F}_\alpha) \]

\paragraph{Beweis.} Für alle $\alpha$ haben wir natürliche Abbildungen $\mathcal{F}_\alpha\to\varinjlim\mathcal{F}_\alpha$. Wir erhalten $\mathrm{H}^i(X,\mathcal{F}_\alpha)\to\mathrm{H}^i(X,\varinjlim\mathcal{F}_\alpha)$ und daher eine Abbildung:
\[\varinjlim\mathrm{H}^i(X,\mathcal{F}_\alpha)\to\mathrm{H}^i(X,\varinjlim\mathcal{F}_\alpha) \]
Dies ist nach \hyperref[9.24]{Lemma 9.24 (i)} für $i=0$ ein Isomorphismus. Betrachte die abelsche Kategorie $\mathbf{Ind}_A(\mathbf{Ab}(X))$ der direkten Limiten von Objekten aus $\mathbf{Ab}(X)$ indiziert durch $A$. Da $\varinjlim$ exakt ist, haben wir eine natürliche Transformation von $\delta$-Funktoren $\mathbf{Ind}_A(\mathbf{Ab}(X))\to\mathbf{Ab}$
\[\varinjlim\mathrm{H}^i(X,-)\to\mathrm{H}^i(X,\varinjlim -) \]
die für $i=0$ übereinstimmt. Jetzt genügt es zu zeigen, dass beide Funktoren auslöschbar sind. Dann sind beide universell und insbesondere gleich.

Sei $(\mathcal{F}_\alpha)\in\mathbf{Ind}_A(\mathbf{Ab}(X))$. Zu $\alpha$ definieren wir die Garbe $\mathcal{G}_\alpha$ wie folgt:
\[U\mapsto \Big\{s:U\to\coprod_{P\in U}(\mathcal{F}_\alpha)_P\mid s(P)\in (\mathcal{F}_\alpha)_P \Big\} \]
Dann ist $\mathcal{G}_\alpha$ welk und $\mathcal{F}_\alpha\hookrightarrow\mathcal{G}_\alpha$. Nach Konstruktion bilden die $\mathcal{G}_\alpha$ ein induktives System und wir erhalten einen Monomorphismus $u:(\mathcal{F}_\alpha)\hookrightarrow (\mathcal{G}_\alpha)$ in $\mathbf{Ind}_A(\mathbf{Ab}(X))$. Ferner gilt $\mathrm{H}^i(X,\mathcal{G}_\alpha)=0$ für alle $i>0$ nach \hyperref[9.20]{Satz 9.20}, also $\varinjlim\mathrm{H}^i(X,\mathcal{G}_\alpha)=0$ für alle $i>0$. Somit ist $\varinjlim\mathrm{H}^i(X,-)$ auslöschbar.

Da $\varinjlim\mathcal{G}_\alpha$ nach \hyperref[9.24]{Satz 9.24 (ii)} welk ist, ist auch $\mathrm{H}^i(X,\varinjlim\mathcal{G}_\alpha)=0$ für alle $i>0$, d.h. auch $\mathrm{H}^i(X,\varinjlim -)$ ist auslöschbar.\qed

\paragraph{Lemma 9.26.}\label{9.26} Sei $Y\subset X$ abgeschlossen mit natürlicher Inklusion $i:Y\hookrightarrow X$ und $\mathcal{F}\in\mathbf{Ab}(Y)$. Dann gilt:
\[\mathrm{H}^i(Y,\mathcal{F})=\mathrm{H}^i(X, i_\ast\mathcal{F}) \]

\paragraph{Beweis.} Sei $\mathcal{J}^\bullet$ eine injektive Auflösung von $\mathcal{F}$ auf $Y$; insbesondere ist $\mathcal{J}^\bullet$ eine welke Auflösung. Dann ist $i_\ast\mathcal{J}^\bullet$ eine welke Auflösung von $i_\ast\mathcal{F}$ auf $X$. Für alle $i$ gilt:
\[\Gamma(Y,\mathcal{J}^i) = \Gamma(X,i_\ast\mathcal{J}^i) \]
Somit ergeben sich die gleichen Kohomologiegruppen.\qed

\paragraph{Satz 9.27.}\label{9.27} \textit{(Verschwindungssatz von Grothendieck)}\index{Verschwindungssatz} Sei $X$ ein noetherscher topologischer Raum der Dimension $n$. Dann gilt für alle abelschen Garben $\mathcal{F}$ auf $X$:
\[\mathrm{H}^i(X,\mathcal{F})=0\quad\text{für alle }i>n \]

\paragraph{Beweis.} Sei $\mathcal{F}$ eine Garbe abelscher Gruppen auf $X$.
\begin{enumerate}
\item Schritt: \textit{(Reduktion auf irreduzible $X$)} Sei $X$ reduzibel und $Y\subset X$ eine irreduzible Komponente mit Einbettungen $i:Y\hookrightarrow X,\ j:U=X\setminus Y\hookrightarrow X$. Nach \hyperref[1.16]{1.16} ist die folgende Garbensequenz exakt:
\[0\longrightarrow \mathcal{F}_U \longrightarrow\mathcal{F}\longrightarrow\mathcal{F}_Y\longrightarrow 0 \]
wobei $\mathcal{F}_U= j_!(\mathcal{F}|_U)$ und $\mathcal{F}_Y=i_\ast(\mathcal{F}|_Y)$ ist. Es genügt zu zeigen, dass $\mathrm{H}^i(X,\mathcal{F}_U)=0=\mathrm{H}^i(X,\mathcal{F}_Y)$ für alle $i>n$ ist. Betrachte $g:\overline{U}\hookrightarrow X$. Nun hat $\overline{U}$ weniger irreduzible Komponenten als $X$. Setze $\mathcal{F}_{\overline{U}}=g^\ast\mathcal{F}_U$. Es folgt nach \hyperref[9.26]{Lemma 9.26}:
\[\mathrm{H}^i(\overline{U},\mathcal{F}_{\overline{U}})=\mathrm{H}^i(X,g_\ast\mathcal{F}_{\overline{U}}) = \mathrm{H}^i(X,\mathcal{F}_U) \]
\[\mathrm{H}^i(X,\mathcal{F}_Y)=\mathrm{H}^i(X,i_\ast\mathcal{F}|_Y)=\mathrm{H}^i(Y,\mathcal{F}|_Y) \]
Per Induktion nach Anzahl der irreduziblen Komponenten von $X$ können wir ohne Einschränkung $X$ als irreduzibel annehmen.
\item Schritt: Sei $X$ irreduzibel und $\dim(X)=0$. Die einzigen offenen Mengen von $X$ sind $\varnothing$ und $X$, d.h. $\Gamma(X,-):\mathbf{Ab}(X)\to\mathbf{Ab}$ ist eine Äquivalenz von Kategorien. Insbesondere ist $\Gamma(X,-)$ exakt, d.h. $\mathrm{H}^i(X,\mathcal{F})=0$ für alle $i>0$.
\item Schritt: Sei $X$ irreduzibel und $\dim(X)=n$. Sei $j:U\hookrightarrow X$ offen und $s\in\mathcal{F}(U)$. Da
\[\mathcal{F}(U)=\operatorname{Hom}(\mathbb{Z},\mathcal{F}(U))=\operatorname{Hom}_U(\mathbb{Z}|_U,\mathcal{F}|_U)=\operatorname{Hom}_X(j_!(\mathbb{Z}|_U),\mathcal{F}) \]
wobei $\mathbb{Z}|_U$ die konstante Garbe bezeichnet, entspricht $s$ einen Morphismus $j_!(\mathbb{Z}|_U)\to\mathcal{F}$. Setze $\mathcal{F}_s=\operatorname{im}(j_!(\mathbb{Z}|_U)\to\mathcal{F})$. $\mathcal{F}_s$ heißt die von $s$ erzeugte Untergarbe von $\mathcal{F}$. Sei $\alpha=(s_i)_{i\in I},\ s_i\in\mathcal{F}(U_i)$ eine beliebige Familie von Schnitten. Setze:
\[\mathcal{F}_\alpha = \sum_{i\in I}\mathcal{F}_{s_i}\subset\mathcal{F}\]
Sei $J$ die Menge aller endlichen Familien $\alpha=(s_i)$. $J$ wird mit der folgenden par\-tiel\-len Ordnung zu einem gerichteten System:
\[\alpha'\leq\alpha\iff \alpha'\text{ ist Unterfamilie von }\alpha \]
Für $\alpha'\leq\alpha$ gilt $\mathcal{F}_{\alpha'}\subset\mathcal{F}_\alpha$. Somit ist $\mathcal{F}=\varinjlim_{\alpha\in J}\mathcal{F}_\alpha$. Nach \hyperref[9.25]{Satz 9.25} reicht es zu zeigen, dass $\mathrm{H}^i(X,\mathcal{F}_\alpha)=0$ für alle $i>n$ und alle $\alpha\in J$.

Sei nun $\alpha'\leq\alpha$. Betrachte die exakte Sequenz $0\to\mathcal{F}_{\alpha'}\to\mathcal{F}_\alpha\to\mathcal{G}\to 0$. $\mathcal{G}$ ist erzeugt von $\#\alpha - \#\alpha'$ Schnitten über geeignete offenen Mengen. Per Induktion nach $\#\alpha$ können wir unter Betrachtung der langen exakten Kohomologiefolge o.B.d.A. annehmen, dass $\mathcal{F}$ von einem Schnitt über einer geeigneten offenen Menge $U$ erzeugt wird, d.h. $\mathcal{F}=\mathcal{F}_s$. Die zu $s$ gehörige Morphismus $\mathbb{Z}_U=j_!(\mathbb{Z}|_U)\to\mathcal{F}$ ist surjektiv.

Definiere die Garbe $\mathcal{R}$ durch die exakte Folge $0\to\mathcal{R}\to\mathbb{Z}_U\to\mathcal{F}\to 0$. Unter Betrachtung der langen exakten Kohomologiefolge genügt es zu zeigen, dass $\mathrm{H}^i(X,\mathcal{R})=0=\mathrm{H}^i(X,\mathbb{Z}_U)$ für alle $i>n$ ist.
\item Schritt: Wir zeigen $\mathrm{H}^i(X,\mathcal{R})=0$ für alle $i>n$ per Induktion über $n=\dim(X)$. Für $x\in U$ ist $\mathcal{R}_x$ eine Untergruppe von $\mathbb{Z}$. Ist $\mathcal{R} = 0$, so folgt $\mathcal{F}\cong\mathbb{Z}_U$. Sei nun $\mathcal{R}\neq 0$. Setze:
\[d=\min\{m>0\mid m\in\mathcal{R}_x,\ x\in U\} \]
Es existiert ein $\varnothing\neq V\subset_\text{o}U$ mit $\mathcal{R}|_V\cong d\cdot\mathbb{Z}|_V\subset \mathbb{Z}|_V$. Wir erhalten $\mathcal{R}_V=j_!(\mathcal{R}|_V)=\mathbb{Z}_V$, wobei $j:V\hookrightarrow X$ die kanonische Inklusion ist. Wir erhalten die exakte Sequenz:
\[0\longrightarrow\mathbb{Z}_V\longrightarrow\mathcal{R}\longrightarrow\mathcal{R}/\mathbb{Z}_V\longrightarrow 0 \]
Nun gilt $\operatorname{supp}(\mathcal{R}/\mathbb{Z}_V)\subset\overline{U\setminus V}\subset X$. Da $X$ irreduzibel ist, gilt $\dim(\overline{U\setminus V})<n$. Aus der Induktionsannahme folgt mit \hyperref[9.26]{Lemma 9.26} für alle $i\geq n$:
\[\mathrm{H}^i(X,\mathcal{R}/\mathbb{Z}_V) = \mathrm{H}^i(\overline{U\setminus V},\mathcal{R}/\mathbb{Z}_V|_{\overline{U\setminus V}})=0 \]
Nach der langen exakten Kohomologiesequenz reicht es zu zeigen, dass $\mathrm{H}^i(X,\mathbb{Z}_V)=0$ für $i>n$ gilt.
\item Schritt: Wir zeigen $\mathrm{H}^i(X,\mathbb{Z}_V)=0$ für alle $i>n$ per Induktion über $n=\dim(X)$. Sei $U\subset_\text{o}X$ und $Y=X\setminus U$. Dann ist $0\to\mathbb{Z}_U\to\mathbb{Z}\to\mathbb{Z}_Y\to 0$ exakt nach \hyperref[1.16]{1.16}. Wegen $\dim(Y)<\dim(X)$ folgt nach der Induktionsannahme und \hyperref[9.26]{Lemma 9.26} für alle $i\geq n$:
\[\mathrm{H}^i(X,\mathbb{Z}_Y)=\mathrm{H}^i(Y,\mathbb{Z}|_Y)=0 \]
Ferner ist die konstante Garbe $\mathbb{Z}$ welk. Nach \hyperref[9.20]{Satz 9.20} folgt $\mathrm{H}^i(X,\mathbb{Z})=0$ für alle $i>0$. Die lange exakte Kohomologiesequenz liefert $\mathrm{H}^i(X,\mathbb{Z}_U)=0$ für $i>n$.\qed
\end{enumerate}

\paragraph{Satz 9.28.}\label{9.28} Sei $X=\operatorname{Spec}(A)$ noethersch. Dann ist $\mathrm{H}^i(X,\mathcal{F})=0$ für alle $i>0$ und alle quasikohärenten Garben $\mathcal{F}$.

\paragraph{Bemerkung.} Dieser Satz gilt auch für nicht-noethersche Ringe $A$.

\paragraph{Satz von Artin-Rees.}\index{Satz von Artin-Rees}\label{prop:artin-rees} Sei $A$ ein noetherscher Ring, $\mathfrak{a}\subset A$ ein Ideal und $M\subset N$ endlich erzeugte $A$"~Moduln. Dann ist die $\mathfrak{a}$-adische Topologie auf $M$ induziert von der $\mathfrak{a}$-adischen Topologie auf $N$, d.h:
\[\forall n\geq 0,\ \exists m\geq n:\ \mathfrak{a}^nM\supset\mathfrak{a}^mN\cap M \]

\paragraph{Beweis.} Siehe beispielsweise Bosch: Algebraic Geometry and Commutative Algebra, 2.3. Lemma 1.\qed

\paragraph{Definition.} Sei $A$ ein Ring, $\mathfrak{a}\subset A$ ein Ideal und $M$ ein $A$"~Modul. Wir definieren:
\[\Gamma_\mathfrak{a}(M)=\{m\in M\mid\exists n>0:\ \mathfrak{a}^nm=0 \} \]

\paragraph{Lemma 9.29.}\label{9.29} Sei $A$ noethersch, $\mathfrak{a}\subset A$ ein Ideal und $I$ ein injektiver $A$"~Modul. Dann ist $\Gamma_\mathfrak{a}(I)$ ebenfalls injektiv.

\paragraph{Beweis.} Setze $J=\Gamma_\mathfrak{a}(I)$. Wir zeigen zunächst: Für alle Ideale $\mathfrak{b}\subset A$ und alle $A$-Modulhomomorphismen $\varphi:\mathfrak{b}\to J$ gibt es einen $A$-Modulhomomorphismus $\psi:A\to J$ mit $\psi|_\mathfrak{b}=\varphi$.

Sei $\mathfrak{b}\subset A$ ein notwendigerweise endlich erzeugtes Ideal und $\varphi:\mathfrak{b}\to J$ ein $A$-Modulhomomorphismus. Zu $x\in J$ gibt es nach Definition ein $n>0$ mit $\mathfrak{a}^nx=0$. Da $\mathfrak{b}$ endlich erzeugt ist, existiert ein $n$ mit:
\[0=\mathfrak{a}^n\varphi(\mathfrak{b})=\varphi(\mathfrak{a}^n\mathfrak{b}) \]
Nach dem \hyperref[prop:artin-rees]{Satz von Artin-Rees} gibt es ein $m\geq n$ mit $\mathfrak{a}^n\mathfrak{b}\supset\mathfrak{b}\cap\mathfrak{a}^m$, also $\varphi(\mathfrak{b}\cap\mathfrak{a}^m)=0$. Somit faktorisiert $\varphi$ über $\mathfrak{b}/(\mathfrak{b}\cap\mathfrak{a}^m)$:
\[\begin{tikzcd}
A \rar & A/\mathfrak{a}^m\ar[rrd, dashed] &&\\
\mathfrak{b} \uar[hook]\rar & \mathfrak{b}/(\mathfrak{b}\cap\mathfrak{a}^m) \uar[hook]\rar & J\rar[hook] & I
\end{tikzcd}\]
Da $I$ injektiv ist, gibt es eine Fortsetzung $\psi':A/\mathfrak{a}^m\to I$ von $\varphi$, das obiges Diagramm kommutativ macht. Nun gilt $\mathfrak{a}^m\psi'(A/\mathfrak{a}^m)=0$, d.h. $\psi'(A/\mathfrak{a}^m)\subset J$. Somit ist $\psi:A\to A/\mathfrak{a}^m\to J$ eine Fortsetzung von $\varphi:\mathfrak{b}\to J$.

Sei nun $X$ ein $A$-Modul, $X'\subset X$ ein Untermodul und $f:X'\to J$ ein Homomorphismus. Betrachte:
\[\Sigma=\{(Y,g)\mid Y\subset X\text{ Untermodul},\ X'\subset Y,\ g:Y\to J,\ g|_{X'}=f \} \]
Dann ist $(X',f)\in\Sigma$ und $\Sigma$ ist induktiv geordnet. Nach Zorn gibt es ein maximales Element $(Y,g)\in\Sigma$. Angenommen, $Y\neq X$. Dann gibt es ein $x\in X\setminus Y$. Betrachte das Ideal $\mathfrak{b}=\{\lambda\in A\mid\lambda x\in Y\}\subset A$ und den $A$-Modulhomomorphismus $\varphi:\mathfrak{b}\to J,\ \lambda\mapsto g(\lambda x)$. Diese können wir zu einem $\psi:A\to J$ fortsetzen. Definiere $g':\langle Y,x\rangle\to J$ durch $g'|_Y=g$ und $g'(x)=\psi(1)$. Diese Abbildung ist ein wohldefinierter $A$-Modulhomomorphismus, da $g'(y+\lambda x)=g'(y)+\psi(\lambda)$ für $y\in Y,\ \lambda\in A$ und ist $\lambda x\in Y$, so folgt $\lambda\in\mathfrak{b}$, d.h. $\psi(\lambda)=\varphi(\lambda)=g(\lambda x)=g'(\lambda x)$. Dies ist ein Widerspruch zur Maximalität von $(Y,g)$, daher folgt $Y=X$.\qed

\paragraph{Satz 9.30.}\label{9.30} Sei $A$ ein noetherscher Ring und $I$ ein injektiver $A$-Modul. Dann ist die Garbe $\widetilde{I}$ welk über $\operatorname{Spec}(A)$.

\paragraph{Lemma 9.31.}\label{9.31} Sei $A$ ein noetherscher Ring und $I$ ein injektiver $A$-Modul. Dann ist für jedes $f\in A$ die natürliche Abbildung $\theta:I\to I_f$ surjektiv.

\paragraph{Beweis.} Betrachte die folgenden Ideale in $A$:
\[\mathfrak{b}_i=\operatorname{ann}(f^i)= \{x\in A\mid f^ix=0\} \]
Dann gilt $\mathfrak{b}_1\subset\mathfrak{b}_2\subset\ldots\subset\mathfrak{b}_r=\mathfrak{b}_{r+1}=\ldots$ für ein $r$. Sei $x\in I_f$. Dann gibt es ein $y\in I$ und ein $n\geq 0$ mit $x=\frac{\theta(y)}{f^n}$. Definiere den $A$-Modulhomomorphismus:
\[\varphi: (f^{n+r})\to I,\ f^{n+r}\mapsto f^ry \]
Dieser ist wohldefiniert, da $\operatorname{ann}(f^{n+r})= \mathfrak{b}_{n+r}=\mathfrak{b}_r = \operatorname{ann}(f^r)\subset\operatorname{ann}(f^ry)$. Da $I$ injektiv ist, gibt es eine Fortsetzung $\psi:A\to I$ von $\varphi$. Sei $z=\psi(1)$. Dann gilt $f^{n+r}z=f^ry$, also $\theta(z)=\frac{\theta(y)}{f^n}=x$.\qed

\paragraph{Beweis von \hyperref[9.30]{9.30}.} Sei $Y=\overline{\operatorname{supp}(\widetilde{I})}$. Besteht $Y$ nur aus einem abgeschlossenen Punkt, so ist $\widetilde{I}$ eine Wolkenkratzergarbe und somit welk.

Sei nun $Y$ größer und die Aussage für alle kleineren $Y$ bewiesen. Es genügt zu zeigen, dass alle Restriktionsabbildungen der Form $\Gamma(X,\widetilde{I})\to\Gamma(U,\widetilde{I})$ surjektiv sind. Sei $U\subset_\text{o}X$. Ist $Y\cap U=\varnothing$, so ist nichts zu zeigen. Sei $Y\cap U\neq\varnothing$. Es existiert ein $f\in A$ mit $X_f=D(f)\subset U$ und $X_f\cap Y\neq\varnothing$. Setze $Z=X\setminus X_f$ und $\Gamma_Z(U,\widetilde{I})=\{s\in\Gamma(U,\widetilde{I})\mid\operatorname{supp}(s)\subset Z \}$. Betrachte:
\[\begin{tikzcd}
\Gamma(X,\widetilde{I})\rar & \Gamma(U,\widetilde{I})\rar & \Gamma(X_f,\widetilde{I})\\
\Gamma_Z(X,\widetilde{I})\rar\uar[hook] & \Gamma_Z(U,\widetilde{I}) \uar[hook]&
\end{tikzcd} \]
Sei $s\in\Gamma(U,\widetilde{I})$ und betrachte die Restriktion $s'\in\Gamma(X_f,\widetilde{I})=I_f$. Nach \hyperref[9.31]{Lemma 9.31} gibt es ein $t\in I=\Gamma(X,\widetilde{I})$ mit $t|_{X_f}=s'$. Sein $t'$ die Restriktion von $t$ auf $U$. Dann ist $(s-t')|_{X_f}=0$, also $\operatorname{supp}(s-t')\subset Z$. Bleibt noch zu zeigen, dass $\Gamma_Z(X,\widetilde{I})\to\Gamma_Z(U,\widetilde{I})$ surjektiv ist. Dann existiert nämlich ein Urbild $x\in\Gamma_Z(X,\widetilde{I})$ von $s-t'$ und daher:
\[(x+t)|_U=s-t'+t'=s\]
Sei $J=\Gamma_Z(X,\widetilde{I})$ und $\mathfrak{a}=(f)\subset A$. Dann ist $J=\Gamma_\mathfrak{a}(I)$. Nach \hyperref[9.29]{Lemma 9.29} ist $J$ ein injektiver $A$-Modul mit $\operatorname{supp}(\widetilde{J})\subset Y\cap Z\subsetneq Y$. Nach Induktionsannahme ist $\widetilde{J}$ welk, also ist die Restriktionsabbildung surjektiv:
\[\Gamma_Z(X,\widetilde{I}) = \Gamma(X,\widetilde{J})\to \Gamma(U,\widetilde{J})=\Gamma_Z(U,\widetilde{I})\qedhere \]

\paragraph{Beweis von \hyperref[9.28]{9.28}.} Sei $\mathcal{F}=\widetilde{M}$ eine quasikohärente Garbe über einem noetherschen Ring $A$. Sei $0\to M\to I^\bullet$ eine injektive Auflösung in $\mathbf{Mod}(A)$. Wir erhalten eine exakte Folge $0\to\mathcal{F}\to\widetilde{I}^\bullet$. Nach \hyperref[9.30]{Satz 9.30} sind alle $\widetilde{I}^i$ welk und somit azyklisch für $\Gamma(X,-)$. Daher kann diese Auflösung zur Kohomologieberechnung benutzt werden. Anwendung von $\Gamma(X,-)$ gibt gerade die exakte Sequenz $0\to M\to I^\bullet$. Also ist $\mathrm{H}^i(X,\mathcal{F})=0$ für alle $i>0$.\qed