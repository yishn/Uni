\documentclass[11pt,b5paper,openany]{memoir}
\usepackage{fontspec}
\usepackage[ngerman]{babel}
\usepackage{amsmath, amsfonts, amssymb}
\usepackage{paralist}
\usepackage{makeidx}
\usepackage{graphicx, tikz}
\usetikzlibrary{cd, babel}
\usepackage{hyperref}
\hypersetup{
    bookmarksopen=true,
    pdfstartview=FitH
}
\usepackage[all]{xy}
\usepackage[left=1.3cm,right=1.3cm,top=2cm,bottom=2cm]{geometry}
\author{Yichuan Shen}
\title{Algebraische Geometrie}
\linespread{1.1}
\makeindex
\chapterstyle{tandh}

\def \: {\mathbin:}
\def \qed {$\hfill\square$}
\def \qedhere {\tag*{$\square$}}

\begin{document}

\frontmatter
\pagenumbering{gobble}

\begin{center}
\vspace*{0cm}
\vfill {
    \begin{Huge}Algebraische Geometrie\end{Huge}\\
    \vspace{1mm}
    Vorlesung von \textsc{prof. dr. kay wingberg}
}
\vfill {
    \includegraphics[scale=.2]{seal.pdf}\\
    gesetzt von\\\textsc{yichuan shen} \\\ \\
    2014/2015\\
    \today
}
\end{center}

\clearpage

\vspace*{0cm}
\vfill
\paragraph{Lizenz \& Quellcode.} Dieses Dokument steht unter einer Creative Commons Attribution Lizenz. Weitere Informationen findet man auf \url{http://creativecommons.org/licenses/by/3.0/de/}. Der Quellcode ist auf \url{http://github.com/yishn/Uni} zu finden.

\mainmatter
\pagenumbering{arabic}
\setcounter{page}{3}
\tableofcontents

\chapter{Varietäten}

Sei $k$ ein algebraisch abgeschlossener Körper.

\section{Affine Varietäten}

\paragraph{Definition.} \begin{enumerate}[(i)]
\item Die Menge aller $n$-Tupeln über $k$
\[\mathbf{A}^n = \mathbf{A}_k^n = \{(a_1,\ldots,a_n)\mid a_i\in k\} \]
heißt \textit{affiner $n$-dimensionaler Raum}\index{affiner Raum} über $k$. Ein Element $P=(a_1,\ldots,a_n)\in\mathbf{A}^n$ heißt \textit{Punkt}\index{Punkt} und die $a_i$ heißen \textit{Koordinaten}\index{Koordinate} von $P$.
\item Der Polynomring über $k$ in $n$ Variablen bezeichnen wir mit $A=k[X_1,\ldots,X_n]$. Für $T\subset A$ definieren wir die \textit{Nullstellenmenge}\index{Nullstellenmenge} von $T$ wie folgt:
\[Z(T)=\{P\in\mathbf{A}^n\mid \forall f\in T\colon f(P)=0\} \]
Es gilt $Z(T)=Z(\mathfrak{a})$, wobei $\mathfrak{a}$ das von $T$ erzeugte Ideal in $A$ ist.
\item Eine Teilmenge $Y\subset\mathbf{A}^n$ der Form $Y=Z(T)$ für ein $T\subset A$ heißt \textit{affine algebraische Menge}\index{algebraische Menge!affin}. Für algebraische Mengen $Y_i=Z(\mathfrak{a}_i)$ mit Ideale $\mathfrak{a}_i\subset A,\ i\in I$ gilt:
\[Y_1\cup Y_2=Z(\mathfrak{a}_1\cap\mathfrak{a}_2),\quad \bigcap_{i\in I}Y_i = Z\Big(\sum_{i\in I}\mathfrak{a}_i\Big) \]
Ferner gilt $\mathbf{A}^n=Z(0)$ und $\varnothing = Z(1)$. Wir statten $\mathbf{A}^n$ mit der sogenannten \textit{Zariski-Topologie}\index{Zariski-Topologie} aus, in dem wir eine Menge $U\subset\mathbf{A}^n$ genau dann offen nennen, wenn $\mathbf{A}^n\setminus U$ eine algebraische Menge ist.
\item Eine \textit{affine Varietät}\index{Varietät!affin} $V$ ist eine \textit{irreduzible}\index{irreduzibel} abgeschlossene Menge in $\mathbf{A}^n$, d.h. aus $V=V_1\cup V_2$ mit abgeschlossenen Mengen $V_1,V_2\subset\mathbf{A}^n$ folgt $V_1=\varnothing$ oder $V_2=\varnothing$. 

Eine offene Teilmenge einer affinen Varietät bzgl. der induzierten Topologie nennt man \textit{quasi-affine Varietät}\index{Varietät!quasi-affin}. 
\item Sei $Y\subset\mathbf{A}^n$ eine algebraische Menge. Dann definieren wir das Ideal:
\[I(Y)=\{f\in A\mid \forall P\in Y\colon f(P)=0\} \]
Der \textit{Koordinatenring}\index{Koordinatenring} von $Y$ ist definiert als $A(Y)=A/I(Y)$.
\end{enumerate}

\paragraph{Definition.} Sei $\mathfrak{a}\subset A$ ein Ideal. Das \textit{Radikal}\index{Radikal} von $\mathfrak{a}$ ist definiert als:
\[\operatorname{Rad}(\mathfrak{a})=\{f\in A\mid\exists r>0\colon f^r\in\mathfrak{a} \} \]
Ein Ideal $\mathfrak{a}\subset A$ heißt \textit{Radikalideal}\index{Radikalideal}, wenn $\mathfrak{a}=\operatorname{Rad}(\mathfrak{a})$ gilt.

\paragraph{Satz.} Es gibt eine inklusionsumkehrende Bijektion:
\[\{\text{Algebraische Mengen in }\mathbf{A}^n\} \to \{\text{Radikalideale in }k[X_1,\ldots,X_n]\},\ Y\mapsto I(Y) \]
mit der Umkehrabbildung $\mathfrak{a}\mapsto Z(\mathfrak{a})$. Eine algebraische Menge $Y$ ist genau dann irreduzibel, wenn $I(Y)\subset A$ ein Primideal ist.

\section{Projektive Varietäten}

\paragraph{Definition.}\begin{enumerate}[(i)]
\item Zwei Punkte $(a_0,\ldots,a_n),(b_0,\ldots,b_n)\in \mathbf{A}^{n+1}$ heißen äquivalent, wenn ein $\lambda\in k^\times$ existiert, so dass $a_i=\lambda b_i$ für alle $i$ gilt. Die Äquivalenzklasse von $(a_0,\ldots,a_n)$ wird mit $(a_0\: \ldots\: a_n)$ bezeichnet. Der \textit{projektiver $n$-dimensionaler Raum}\index{projektiver Raum} über $k$ wird definiert als:
\[\mathbf{P}^n=\mathbf{P}_k^n = \{(a_0\: \ldots\: a_n) \mid a_i\in k\text{ nicht alle $0$} \} \]
Ein Element $P=(a_0\: \ldots\: a_n)\in\mathbf{P}^n$ heißt \textit{Punkt}\index{Punkt} und die $a_i$ heißen \textit{homogene Koordinaten} von $P$.\index{Koordinate!homogen}
\item Der Polynomring über $k$ in $n+1$ Variablen $S=k[X_0,\ldots,X_n]$ wird mit der folgenden Zerlegung zu einem graduierten Ring:
\[S=\bigoplus_{d\geq 0}S_d,\quad S_d=\Big\{ \sum a_{i_0,\ldots,i_n} X_0^{i_0}\cdots X_n^{i_n}\Bigm| a_{i_0,\ldots,i_n}\in k,\ \sum_{j=0}^n i_j=d \Big\} \]
Die Elemente in $S_d$ heißen \textit{homogene Elemente vom Grad $d$}\index{homogene Elemente}. Ein Ideal $\mathfrak{a}\subset S$ heißt \textit{homogenes Ideal}\index{homogenes Ideal}, wenn $\mathfrak{a}=\bigoplus_{d\geq 0}(S_d\cap\mathfrak{a})$ gilt.
\item Sei $\mathfrak{a}\subset S$ ein homogenes Ideal. Dann setzen wir:
\[Z(\mathfrak{a}) = \{P\in\mathbf{P}^n\mid \forall f\in\mathfrak{a}\text{ homogen}\colon f(P)=0 \} \]
Diese ist wohldefiniert, da $f(\lambda a_0,\ldots,\lambda a_n)=\lambda^d f(a_0,\ldots,a_n)$ für $f\in S_d$. Eine Menge $Y\subset\mathbf{P}^n$ heißt \textit{projektive algebraische Menge}\index{algebraische Menge!projektiv}, wenn $Y=Z(\mathfrak{a})$ für ein homogenes Ideal $\mathfrak{a}\subset S$ gilt.

Analog wie im affinen Fall, können wir auch $\mathbf{P}^n$ mit der \textit{Zariski-Topologie} ausstatten, d.h. eine Menge $U\subset\mathbf{P}^n$ ist genau dann offen, wenn $\mathbf{P}^n\setminus U$ eine projektive algebraische Menge ist.\index{Zariski-Topologie}
\item Eine \textit{projektive Varietät}\index{Varietät!projektiv} $V$ ist eine irreduzible abgeschlossene Menge in $\mathbf{P}^n$. Eine offene Teilmenge einer projektiven Varietät bzgl. der induzierten Topologie nennt man \textit{quasi-projektive Varietät}\index{Varietät!quasi-projektiv}. 
\item Sei $Y\subset\mathbf{P}^n$ eine algebraische Menge. Dann setzen wir $I(Y)$ als das Ideal in $S$, das von der folgenden Menge erzeugt wird:
\[\{f\in S \text{ homogen}\mid \forall P\in Y\colon f(P)=0\} \]
$I(Y)$ ist ein homogenes Ideal in $S$. Der \textit{homogene Koordinatenring}\index{Koordinatenring!homogen} von $Y$ ist definiert als $S(Y)=S/I(Y)$.
\end{enumerate}

\paragraph{Satz.} Wir haben eine inklusionsumkehrende Bijektion:
\[\{\text{Algebraische Mengen in }\mathbf{P}^n\}\to\{\text{Radikalideale in }k[X_0,\ldots,X_n]\},\ Y\mapsto I(Y) \]
mit der Umkehrabbildung $\mathfrak{a}\mapsto Z(\mathfrak{a})$. 

\paragraph{Satz.} Sei $Y$ eine (quasi-)projektive Varietät. Dann wird $Y$ von offenen Mengen der Form $Y\cap U_i,\ i=0,\ldots,n$ überdeckt mit:
\[U_i=\{(a_0\: \ldots\: a_n)\in \mathbf{P}^n\mid a_i\neq 0\} \]
Die Abbildungen $\varphi_i: U_i\to\mathbf{A}^n,\ (a_0\: \ldots\: a_n)\mapsto \big(\frac{a_0}{a_i},\ldots,\widehat{\frac{a_i}{a_i}},\ldots,\frac{a_n}{a_i}\big)$ sind wohldefiniert und Ho\-möo\-mor\-phis\-men, d.h. die $Y\cap U_i$ sind (quasi-)affine Varietäten.

\section{Morphismen}

\paragraph{Definition.} \begin{enumerate}[(i)]
\item Sei $Y\subset\mathbf{A}^n$ eine quasi-affine Varietät. Eine Abbildung $f:Y\to k$ heißt \textit{reguläre Funktion} in $P\in Y$, wenn eine offene Umgebung $U\subset Y$ mit $P\in U$ existiert, so dass $f=\frac{g}{h}$ auf $U$ für gewisse $g,h\in A$ gilt.

Sei $Y\subset\mathbf{P}^n$ eine quasi-projektive Varietät. Eine Abbildung $f:Y\to k$ heißt \textit{reguläre Funktion}\index{reguläre Funktion} in $P\in Y$, wenn eine offene Umgebung $U\subset Y$ mit $P\in U$ existiert, so dass $f=\frac{g}{h}$ auf $U$ für gewisse homogene Polynome $g,h\in S$ vom gleichen Grad.

Identifizieren wir $k\cong \mathbf{A}^1$ so ist eine reguläre Funktion notwendigerweise stetig.
\item Eine stetige Abbildung $\varphi:X\to Y$ zwischen zwei (quasi-projektive) Varietäten heißt \textit{Morphismus}\index{Morphismus!Varietäten}, wenn für jede offene Menge $V\subset Y$ und reguläre Funktion $f:V\to k$ auch $f\circ\varphi:\varphi^{-1}(V)\to k$ regulär ist.

Damit erhält man die Kategorie $\textbf{Var$(k)$}$ aller Varietäten auf $k$.
\end{enumerate}

%\paragraph{Bemerkung.} Sei $Y$ eine Varietät und $f,g:Y\to k$ reguläre Funktionen. Existiert eine nichtleere offene Menge $U\subset Y$ mit $f=g$ auf $U$, so folgt $f=g$ auf ganz $Y$: Sei dazu o.B.d.A. $U$ so klein, dass $f$ und $g$ als Brüche homogener Polynome vom gleichen Grad dargestellt werden kann. Dann ist die Nullstellenmenge $\{P\in Y\mid (f-g)(P)=0\}$ abgeschlossen und dicht in $Y$.

\paragraph{Definition.} Sei $Y$ eine Varietät und $P\in Y$ ein Punkt.
\begin{enumerate}[(i)]
\item Wir bezeichnen den Ring aller regulären Funktionen auf $Y$ mit $\mathcal{O}(Y)$.
\item $\mathcal{O}_{P,Y}=\{\langle U,f\rangle\mid P\in U\subset_\text{o} Y,\ f\text{ ist auf $U$ regulär}\}$ heißt der Ring der \textit{Keime} regulärer Funktionen auf $Y$ in $P$.\index{Keim} Wir identifizieren zwei Keime $\langle U,f\rangle=\langle V,g\rangle$, wenn $f=g$ auf $U\cap V$ gilt.

$\mathcal{O}_{P,Y}$ ist ein lokaler Ring, dessen Maximalideal wir mit $\mathfrak{m}_P$ bezeichnen.
\item $K(Y)=\{\langle U,f\rangle\mid\varnothing\neq U\subset_\text{o}Y,\ f\text{ ist auf $U$ regulär}\}$ heißt der \textit{Funktionenkörper}\index{Funktionenkörper}  von $Y$. Die Elemente von $K(Y)$ heißen \textit{rationale Funktionen} auf $Y$\index{rationale Funktion}.
\end{enumerate}
Es gilt $\mathcal{O}(Y)\subset\mathcal{O}_{P,Y}\subset K(Y)$.

\paragraph{Theorem.} Sei $Y\subset\mathbf{A}^n$ eine affine Varietät. Dann gilt:
\begin{enumerate}[(i)]
\item $\mathcal{O}(Y)\cong A(Y)$
\item Die Abbildung $Y\to \{\text{Maximale Ideale in }A(Y)\},\ P\mapsto \mathfrak{m}_P\subset\mathcal{O}_{P,Y}$ ist eine Bijektion.
\item $\mathcal{O}_{P,Y}\cong A(Y)_{\mathfrak{m}_P}$ und $\dim\mathcal{O}_{P,Y}=\dim Y$.
\item $K(Y)\cong\operatorname{Quot}(A(Y))$
\end{enumerate}

\paragraph{Theorem.} Sei $Y\subset\mathbf{P}^n$ eine projektive Varietät. Dann gilt:
\begin{enumerate}[(i)]
\item $\mathcal{O}(Y)=k$
\item $\mathcal{O}_{P,Y}\cong S(Y)_{(\mathfrak{m}_P)}$
\item $K(Y)\cong S(Y)_{((0))}$
\end{enumerate}

\paragraph{Theorem.} Sei $X$ eine beliebige Varietät und $Y$ eine affine Varietät. Dann haben wir eine Bijektion:
\[\operatorname{Mor}_{\textbf{Var}}(X,Y) \to \operatorname{Hom}_{\textbf{$k$-Alg}}(A(Y),\mathcal{O}(X)),\ f\mapsto(\varphi\mapsto \varphi\circ f) \]
Ist $X$ ebenfalls affin, so gilt $X\cong Y$, genau dann wenn $A(X)\cong A(Y)$. Der Funktor $\textbf{affine Var$(k)$}\to\textbf{nullteilerfreie $k$-Alg},\ X\mapsto A(X)$ ist eine pfeilumkehrende Äquivalenz von Kategorien.

\chapter{Schemata}
\section{Garben}

\paragraph{Definition 1.1.}\label{1.1} Sei $X$ ein topologischer Raum.
\begin{enumerate}[(i)]
\item Die Menge aller offenen Teilmengen in $X$ bilden zusammen mit den natürlichen Inklusionen eine Kategorie $\textbf{Top$(X)$}$.
\item Eine \textit{Prägarbe}\index{Prägarbe} $F$ abelscher Gruppen ist nichts anderes als ein kontravarianter Funk\-tor $F:\textbf{Top$(X)$}\to\textbf{Ab}$ mit $F(\varnothing)=0$.
\end{enumerate}

\paragraph{Bemerkung.}\begin{enumerate}
\item Eine Prägarbe besteht also aus abelschen Gruppen $F(U),\ U\subset_\text{o}X$ und Homomorphismen abelscher Gruppen $\operatorname{res}_V^U:F(U)\to F(V)$ für alle offenen $V\subset U$, so dass $\operatorname{res}_U^U=\operatorname{id}_{F(U)}$. Für offene Mengen $W\subset V\subset U$ gelte ferner $\operatorname{res}_W^U=\operatorname{res}_W^V\circ\operatorname{res}_V^U$.
\item Ebenso können Prägarben in eine beliebige Kategorie gebildet werden, z.B. $\textbf{Ringe}$ und $\textbf{Mengen}$.
\item Die Elemente von $F(U)$ heißen \textit{Schnitte} von $F$ über $U$.\index{Schnitt} Manchmal schreiben wir auch $\Gamma(U,F)=F(U)$. Die $\operatorname{res}_V^U$ heißen \textit{Restriktionsabbildungen}\index{Restriktionsabbildung}. Wir schreiben auch $\operatorname{res}_V^U(s)=s|_V$.
\end{enumerate}

\paragraph{Definition 1.2.}\label{1.2} Eine Prägarbe $F$ auf einem topologischen Raum $X$ heißt \textit{Garbe}\index{Garbe}, falls die folgenden Diagramme exakt sind:
\[\begin{tikzcd}
0\ar[r] & F(U)\ar[r, "\operatorname{res}"] & \displaystyle\prod_i F(U_i)\ar[r, "\operatorname{res}", shift left] \ar[r, shift right] & \displaystyle\prod_{i,j}F(U_i\cap U_j)
\end{tikzcd} \]
für alle $U\subset_\text{o}X$ und jede offene Überdeckung $U=\bigcup_i U_i$, d.h:
\begin{enumerate}[(i)]
\item $s\mapsto (\operatorname{res}^U_{U_i}(s))_i$ ist injektiv, d.h. aus $s|_{U_i}=0$ für alle $i$ folgt $s=0$.
\item Sei $s_i\in F(U_i)$ für alle $i$ gegeben, so dass $s_i|_{U_i\cap U_j}=s_j|_{U_i\cap U_j}$ für alle $i,j$. Dann gibt es ein $s\in F(U)$ mit $s|_{U_i}=s_i$ für alle $i$.
\end{enumerate}

\paragraph{Definition 1.3.}\label{1.3} \begin{enumerate}[(i)]
\item Ein \textit{Morphismus}\index{Morphismus!Prägarben} $\varphi:F\to G$ von Prägarben auf $X$ ist ein Morphismus kontravarianter Funktoren, d.h. eine Kollektion von Morphismen $(\varphi(U))_{U\subset_\text{o}X}$, so dass für alle offenen Mengen $V\subset U$ folgendes Diagramm kommutativ ist:
\[\begin{tikzcd}
F(U)\ar[rr, "\varphi(U)"]\ar[d, "\operatorname{res}^U_V"'] && G(U)\ar[d, "\operatorname{res}^U_V"]\\
F(V)\ar[rr, "\varphi(V)"'] && G(V)
\end{tikzcd} \]
\item Ein \textit{Morphismus}\index{Morphismus!Garben} von Garben ist ein Morphismus von Prägarben. Die (Prä-)Garben bilden eine Kategorie.
\end{enumerate}

\paragraph{Beispiel 1.4.}\label{1.4} \begin{enumerate}
\item Sei $X$ eine Varietät über $k$. Betrachte den Funktor $\mathcal{O}:\textbf{Top$(X)$}\to\textbf{komm Ringe}$ mit den gewöhnlichen Restriktionsabbildungen $\operatorname{res}^U_V:\mathcal{O}(U)\to\mathcal{O}(V)$ ist offensichtlich eine Prägarbe von Ringen. Da ferner reguläre Funktionen $0$ ist, wenn sie lokal $0$ ist, und eine lokal reguläre Funktion auch global regulär ist, ist $\mathcal{O}$ auch eine Garbe.
\item Sei $X$ ein topologischer Raum und $A$ eine abelsche Gruppe. Die \textit{konstante Garbe}\index{Garbe!konstant} $\mathcal{A}$ auf $X$ ist folgendermaßen definiert: Wir statten $A$ mit der diskreten Topologie aus. Für jedes $U\subset_\text{o}X$ setze:
\[\mathcal{A}(U)=\{f:U\to A\mid f\text{ stetig}\} \]
Ist $U$ zusammenhängend, so gilt $\mathcal{A}(U)\stackrel{\sim}{\to} A,\ f\mapsto f(x)$, wobei $x\in U$ beliebig.
\end{enumerate}

\paragraph{Definition 1.5.}\label{1.5} Sei $F$ eine Prägarbe auf $X$ und $P\in X$. Der \textit{Halm}\index{Halm} $F_P$ von $F$ in $P$ ist definiert als:
\[F_P =\varinjlim_{\substack{U\subset_\text{o}X\\ P\in U}} F(U)=\coprod_{\substack{U\subset_\text{o}X\\ P\in U}} F(U)\Big/{\sim} \]
wobei zwei Elemente $s\in F(U),\ t\in F(V)$ genau dann äquivalent $s\sim t$ sind, wenn es ein $\varnothing \neq W\subset_\text{o}X$ mit $W\subset U\cap V$ existiert, so dass $s|_W=t|_W$ gilt. Die Elemente eines Halms heißen \textit{Keime}\index{Keim} der Schnitte von $F$ in $P$.

\paragraph{Beispiel.} Sei $X$ eine Varietät, $P\in X$ ein Punkt und $\mathcal{O}$ die Garbe der regulären Funktionen. Dann ist der Halm in $P$ gerade der lokale Ring $\mathcal{O}_{P,X}$.

\paragraph{Bemerkung.} Ein Morphismus $\phi:F\to G$ von Prägarben induziert für alle $P\in X$ ein Gruppenhomomorphismus $\phi_P:F_P\to G_P$.

\paragraph{Satz 1.6.}\label{1.6} Sei $\phi:F\to G$ ein Morphismus von Garben auf einem topologischen Raum $X$. Dann gilt:
\[\phi:F\to G\text{ ist Isomorphismus}\iff \phi_P:F_P\to G_P\text{ ist Isomorphismus für alle }P\in X \]
Für Prägarben gilt dieser Satz im Allgemeinen nicht.

\paragraph{Beweis.} Ist $\phi$ ein Isomorphismus, so auch alle $\phi_P,\ P\in X$. Sei umgekehrt $\phi_P$ Isomorphismen für alle $P\in X$. Es genügt zu zeigen, dass $\phi(U):F(U)\to G(U)$ für alle $U\subset_\text{o}X$ ein Isomorphismus ist. Sei $U\subset_\text{o}X$ und setze $\varphi=\phi(U)$.
\begin{itemize}
\item \textit{Injektivität von $\varphi$:} Sei $s\in F(U)$ mit $0=\varphi(s)\in G(U)$. Dann gilt für das Bild $\varphi(s)_P$ von $\varphi(s)$ im Halm $0=\varphi(s)_P\in F_P$. Wegen $\varphi(s)_P=\phi_P(s_P)$ für das Bild $s_P\in F_P$ von $s$, folgt wegen der Injektivität von $\phi_P$ nun $s_P=0$ für alle $P\in U$. 

Per Definition gibt es für jedes $P\in U$ eine offene Umgebung $W_P\subset_\text{o}X$ von $P$ mit $W_P\subset U$, so dass $s|_{W_P}=0$ gilt. Dann bilden die $W_P$ eine offene Überdeckung von $U=\bigcup_{P\in U}W_P$. Da $F$ eine Garbe ist, folgt $s=0$.

Wir haben gezeigt, dass $\phi(U)$ für alle $U\subset_\text{o}X$ injektiv ist, genau dann wenn $\phi_P$ für alle $P\in X$ injektiv ist.
\item \textit{Surjektivität von $\varphi$:} Sei $t\in G(U)$ ein Schnitt und $t_P\in G_P$ sein Keim in $P$. Da $\phi_P$ surjektiv ist, existiert ein $s_P\in F_P$ mit $\phi_P(s_P)=t_P$. Sei $s_P$ durch den Schnitt $s(P)\in F(V_P)$ mit $V_P\subset_\text{o}U,\ P\in V_P$ repräsentiert. Dann sind $\phi(V_P)(s(P))$ und $t|_{V_P}$ zwei Elemente aus $G(V_P)$ mit demselben Keim. Durch das Verkleinern von $V_P$ folgt $\phi(V_P)(s(P))=t|_{V_P}$ in $G(V_P)$.

Dann bilden die $V_P$ eine offene Überdeckung von $U=\bigcup_{P\in U}V_P$. Es gilt außerdem $s(P)|_{V_P\cap V_Q}=s(Q)|_{V_P\cap V_Q}$ für alle $P,Q\in U$, denn beide Elemente sind Schnitte aus $F(V_P\cap V_Q)$, die durch $\phi(V_P\cap V_Q)$ auf $t|_{V_P\cap V_Q}$ abgebildet werden, und $\phi(V_P\cap V_Q)$ aus dem ersten Teil injektiv ist.

Da $F$ eine Garbe ist, existiert ein $s\in F(U)$ mit $s|_{V_P}=s(P)$ für alle $P\in U$. Schließlich gilt $\varphi(s)|_{V_P}=t|_{V_P}$ für alle $P\in U$, d.h. $(\varphi(s)-t)|_{V_P}=0$. Da $G$ eine Garbe ist, folgt $\varphi(s)=t$.\qed
\end{itemize}

\paragraph{Definition 1.7.}\label{1.7} Sei $\varphi:F\to G$ ein Morphismus von Prägarben. Die Prägarben
\[U\mapsto \ker\varphi(U),\quad U\mapsto\operatorname{coker}\varphi(U),\quad U\mapsto\operatorname{im}\varphi(U) \]
heißen \textit{Prägarbenkern}, \textit{-kokern} und \textit{-bild} von $\varphi$.\index{Kern}\index{Kokern}\index{Bild} Sind $F$ und $G$ Garben, so sind Kokern und Bild nicht notwendig Garben.

\paragraph{Satz \& Definition 1.8.}\label{1.8} Sei $F$ eine Prägarbe. Dann existiert eine Garbe $F^+$ und ein Morphismus von Prägarben $\theta:F\to F^+$ mit folgender Universaleigenschaft:

Sei $G$ eine Garbe und $\phi:F\to G$ ein Morphismus von Prägarben. Dann existiert ein eindeutig bestimmter Morphismus $\psi:F^+\to G$, so dass das folgende Diagramm kommutiert:
\[\begin{tikzcd} 
F\ar[rr, "\theta"]\ar[d, "\phi"'] && F^+\ar[dll, "\psi", dashed]\\
G
\end{tikzcd} \]
$F^+$ ist somit eindeutig bestimmt und heißt die zu $F$ \textit{assoziierte Garbe}\index{assoziierte Garbe!Prägarbe}.

\paragraph{Beweis.} Für jede offene Menge $U\subset X$ setze $F^+(U)$ als die Menge aller Abbildungen $s:U\to\coprod_{P\in U}F_P$, so dass:
\begin{enumerate}[(i)]
\item Für alle $P\in U$ gilt $s(P)\in F_P$.
\item Für alle $P\in U$ gibt es eine offene Umgebung $V$ von $P$ mit $V\subset U$ und ein Element $t\in F(V)$, so dass für alle $Q\in V$ der Keim $t_Q$ von $t$ in $Q$ gleich $s(Q)$ ist.
\end{enumerate}
Somit wird $F^+$ zu einer Garbe bzgl. der natürlichen Restriktionsabbildungen und besitzt die verlangte Universaleigenschaft. Für jeden Punkt $P\in X$ gilt $F^+_P=F_P$. Ist $F$ eine Garbe, so ist $F^+\cong F$ via $\theta$.\qed 

\paragraph{Definition 1.9.}\label{1.9} 
\begin{enumerate}[(i)]
\item Eine \textit{Untergarbe}\index{Garbe!Untergarbe} von $F$ ist eine Garbe $F'$ derart, dass:
\begin{enumerate}[(a)]
\item $F'(U)\subset F(U)$ ist eine Untergruppe für alle $U\subset_\text{o}X$.
\item Für offene Mengen $V\subset U$ gilt $\operatorname{res}'{}^U_V=\operatorname{res}^U_V|_{F'(U)}$.
\end{enumerate}
Insbesondere ist $F'_P\subset F_P$ eine Untergruppe.
\item Der \textit{Kern}\index{Kern} von $\varphi$ ist die Prägarbe $\ker(\varphi)$, die bereits eine Garbe ist. \textit{Grund:}

Sei $U\subset_\text{o}X$ und $U=\bigcup U_i$ eine offene Überdeckung. Sei $s\in\ker\varphi(U)$ mit $s|_{U_i}=0$ für alle $i$. Da $F$ eine Garbe ist und $s\in F(U)$, folgt $s=0$. Sei nun $s_i\in \ker\varphi(U_i)$ für alle $i$ gegeben mit $s_i|_{U_i\cap U_j}=s_j|_{U_i\cap U_j}$ für alle $i,j$. Da $F$ eine Garbe ist, existiert ein $s\in F(U)$ mit $s|_{U_i}=s_i$ für alle $i$. Zu zeigen ist noch $s\in \ker\varphi(U)$. Es gilt für alle $i$:
\[0=\varphi(U_i)(s_i)=\varphi(U_i)(s|_{U_i}) = \varphi(U)(s)|_{U_i}\in G(U_i) \]
Da nun auch $G$ eine Garbe ist, folgt $\varphi(U)(s)=0$.
\item $\varphi$ heißt \textit{injektiv}\index{Injektivität}, falls $\ker(\varphi)=0$. 

Mit anderen Worten: $\varphi$ ist genau dann injektiv, wenn $\varphi(U):F(U)\to G(U)$ für alle $U\subset_\text{o}X$ injektiv ist.
\item Das \textit{Bild}\index{Bild} $\operatorname{im}(\varphi)$ von $\varphi$ ist die assoziierte Garbe des Prägarbenbilds von $\varphi$.

Nach der Universaleigenschaft gibt es einen natürlichen Morphismus $\psi:\operatorname{im}(\varphi)\to G$. Dieser ist injektiv, da $(\operatorname{im}\varphi)_P:\operatorname{im}(\varphi_P) \to G_P$ für alle $P\in X$ injektiv ist.
\item $\varphi$ heißt \textit{surjektiv}\index{Surjektivität}, wenn $\operatorname{im}(\varphi)=G$.
\item Eine Garbensequenz
\[\begin{tikzcd}
\cdots \ar[r] & F^i\ar[r, "\varphi^i"] & F^{i+1}\ar[r, "\varphi^{i+1}"] & F^{i+2}\ar[r] & \cdots 
\end{tikzcd} \]
heißt \textit{exakt}\index{Exaktheit}, falls $\ker(\varphi^{i+1})=\operatorname{im}(\varphi^i)$ für alle $i$ gilt.
\item Sei $F'$ eine Untergarbe von $F$. Die \textit{Quotientengarbe}\index{Quotientengarbe} $F/F'$ ist die assoziierte Garbe zur Prägarbe $U\mapsto F(U)/F'(U)$. Offensichtlich gilt $(F/F')_P=F_P/F'_P$ für alle $P\in X$.
\item Der \textit{Kokern}\index{Kokern} von $\varphi$ ist die assoziierte Garbe zum Prägarbenkokern von $\varphi$.
\end{enumerate}

\paragraph{Regeln 1.10.}\label{1.10} Seien $F,G$ Garben auf $X$ und $\varphi:F\to G$ ein Morphismus von Garben.
\begin{enumerate}[(i)]
\item $\varphi$ ist genau dann injektiv, wenn $0\to F\to G$ exakt ist und genau dann surjektiv, wenn $F\to G\to 0$ exakt ist.
\item Eine Garbensequenz $\cdots \to F^i\to F^{i+1}\to F^{i+2}\to\cdots$ ist genau dann exakt, wenn ihre entsprechenden Halmsequenzen in allen Punkten $P\in X$ exakt ist. \textit{Grund:}
\[(\operatorname{im} \varphi^i)_P=\operatorname{im}(\varphi^i_P),\quad (\ker\varphi^{i+1})_P=\ker(\varphi^{i+1}_P) \]
Insbesondere ist ein Garbenmorphismus genau dann injektiv bzw. surjektiv falls alle Halmabbildungen injektiv bzw. surjektiv sind.
\item $\varphi$ ist genau dann surjektiv, wenn für alle $U\subset_\text{o}X$ und $s\in G(U)$ eine Überdeckung $U=\bigcup_{i\in I} U_i$ mit Schnitten $t_i\in F(U_i),\ i\in I$ existieren, so dass $\varphi(t_i)=s|_{U_i}$.

Ist $\varphi$ surjektiv, so muss im Allgemeinen $\varphi(U):F(U)\to G(U)$ nicht surjektiv sein.
\item Sei $0\to F'\to F\to F''$ eine exakte Garbensequenz und $U\subset_\text{o}X$. Dann ist auch die folgende Sequenz exakt:
\[\begin{tikzcd}[cramped, sep=small]
0 \ar[r] & \Gamma(U,F') \ar[r]& \Gamma(U,F) \ar[r]& \Gamma(U,F'')
\end{tikzcd} \]
Der Funktor $\Gamma(U,-)$ ist linksexakt, aber nicht exakt.
\item Sei $\varphi:F\to G$ ein injektiver Morphismus von Prägarben. Dann ist der induzierte Morphismus $\varphi^+:F^+\to G^+$ der assoziierten Garben auch injektiv. Der Funktor $-^+$ ist sogar exakt.
\end{enumerate}

\paragraph{Regeln 1.11.}\label{1.11} \begin{enumerate}[(i)]
\item Sei $F'$ eine Untergarbe von $F$. Dann ist die Sequenz $0\to F'\to F\to F/F'\to 0$ exakt, da sie halmweise exakt ist.
\item Sei $\varphi:F\to G$ ein Garbenmorphismus. Dann gilt:
\[\operatorname{im}(\varphi)\cong F/\ker(\varphi),\quad \operatorname{coker}(\varphi)\cong G/ \operatorname{im}(\varphi) \]
d.h. die Folgen $0\to\ker(\varphi)\to F\to\operatorname{im}(\varphi)\to 0$ und $0\to\operatorname{im}(\varphi)\to G\to\operatorname{coker}(\varphi)\to 0$ sind exakt.
\end{enumerate}

\paragraph{Definition 1.12.}\label{1.12} \begin{enumerate}[(i)]
\item Seien $F$ und $G$ Garben auf $X$. Die \textit{Summe}\index{Summe} $F\oplus G$ von $F$ und $G$ ist die Garbe $U\mapsto F(U)\oplus G(U)$.
\item Sei $(F_i,\varphi_{i,j})$ ein direktes System von Garben auf $X$. Der \textit{direkte Limes}\index{Limes!direkt} $(\varinjlim F_i,\varphi_i)$ ist die assoziierte Garbe zur Prägarbe $U\mapsto \varinjlim F_i(U)$.

Der direkte Limes besitzt die übliche Universaleigenschaft: Sei $G$ eine Garbe und $\psi_i:F_i\to G$ Morphismen mit $\varphi_k\varphi_{ik}=\psi_i$ für alle $i\leq k$. Dann existiert ein eindeutig bestimmter Morphismus $\psi:\varinjlim F_i\to G$, so dass das folgende Diagramm kommutiert:
\[\begin{tikzcd}
F_i\ar[rr, "\varphi_i"]\ar[d, "\psi_i"'] & & \varinjlim F_i\ar[lld, "\psi", dashed]\\
G
\end{tikzcd} \]
\item Ebenso wird der \textit{projektive Limes}\index{Limes!projektiv} definiert, wobei alle Pfeile umgedreht werden. Es ist außerdem $U\mapsto\varprojlim F_i(U)$ bereits eine Garbe.
\item Sei $F$ eine Garbe auf $X$ und $s\in F(U)$ ein Schnitt über $U\subset_\text{o}X$. Dann heißt
\[\operatorname{Supp}(s)=\{P\in U\mid s_P\neq 0\} \]
wobei $s_P\in F_P$ der Keim von $s$ in $P$ bezeichnet, der \textit{Support}\index{Support} von $s$. $\operatorname{Supp}(s)$ ist abgeschlossen in $U$.
\[\operatorname{Supp}(F)=\{P\in X\mid F_P\neq 0\} \]
heißt \textit{Support} von $F$. Dieser ist nicht notwendigerweise abgeschlossen.
\item Seien $F,G$ Garben abelscher Gruppen auf $X$. Für ein $U\subset_\text{o}X$ sei $F|_U$ die Einschränkung von $F$ auf $U$, d.h. $F|_U(V)=F(V)$ für alle $V\subset_\text{o}U$. Dann ist die Menge $\operatorname{Hom}(F|_U,G|_U)$ der Morphismen von $F|_U$ nach $G|_U$ eine abelsche Gruppe.
\[U\mapsto\operatorname{Hom}(F|_U,G|_U)\]
definiert eine Garbe und wird die \textit{Hom-Garbe}\index{Hom-Garbe} genannt. Sie wird mit $\operatorname{\mathcal{H}om}(F,G)$ bezeichnet.
\end{enumerate} 

\paragraph{Definition 1.13.}\label{1.13} Sei $f:X\to Y$ eine stetige Abbildung topologischer Räume, $F$ eine Garbe auf $X$ und $G$ eine Garbe auf $Y$.
\begin{enumerate}[(i)]
\item Die \textit{direkte Bildgarbe}\index{direkte Bildgarbe} $f_\ast F$ von $F$ auf $Y$ ist die Garbe
\[V\mapsto (f_\ast F)(V)=F(f^{-1}(V)) \]
\item Die \textit{Urbildgarbe}\index{Urbildgarbe} $f^{-1}G$ von $G$ auf $X$ ist die assoziierte Garbe zur Prägarbe
\[U\mapsto (f^{-1}G)(U)= \varinjlim_{\substack{V\subset_\text{o}Y\\ f(U)\subset V}}G(V) \]
\end{enumerate}

\paragraph{Regeln 1.14.}\label{1.14} Seien $X,Y$ topologische Räume.
\begin{enumerate}[(i)]
\item Sei $Z\subset X$ ein Teilraum mit der Inklusionsabbildung $i:Z\hookrightarrow X$ und $F$ eine Garbe auf $X$. Dann gilt:
\[i^{-1}F = F|_Z \]
Offensichtlich gilt $(F|_Z)_P=F_P$ für $P\in Z$.
\item Seien $\mathbf{Ab}(X)$ und $\mathbf{Ab}(Y)$ die Kategorien der Garben auf $X$ bzw. $Y$. Dann sind $f_\ast:\mathbf{Ab}(X)\to\mathbf{Ab}(Y)$ und $f^{-1}:\mathbf{Ab}(Y)\to\mathbf{Ab}(X)$ Funktoren.
\item Sei $f:X\to Y$ stetig. Dann sind die \textit{Adjunktionsabbildungen}\index{Adjunktionsabbildung} 
\[\operatorname{ad}:f^{-1}f_\ast F\to F,\quad\operatorname{ad} :G\to f_\ast f^{-1}G \]
für Garben $F$ auf $X$ bzw. $G$ auf $Y$ Garbenmorphismen und wie folgt definiert:
\begin{itemize}
\item Sei $U\subset_\text{o}X$ und $s\in (f^{-1}f_\ast F)(U)$ ein Schnitt, das durch $s'\in (f_\ast F)(V)$ mit $f(U)\subset V\subset_\text{o}Y$ repräsentiert wird, d.h. $s'\in F(f^{-1}(V))$ mit $U\subset f^{-1}(V)$. Dann setzen wir: 
\[(f^{-1}f_\ast F)(U)\to F(U),\ s\mapsto \operatorname{res}_U^{f^{-1}(V)}s'\in F(U)\]
\item Sei $V\subset_\text{o}Y$. Es besteht $(f_\ast f^{-1}G)(V)=(f^{-1}G)(f^{-1}(V))$ aus Abbildungen der Form $f^{-1}(V)\to\coprod_{P\in f^{-1}(V)}(f^{-1}G)_P$. Setze nun:
\[G(V)\to (f_\ast f^{-1}G)(V),\ s\mapsto (s\circ f: f^{-1}(V)\to\coprod  f^{-1}(G)_P,\ P\mapsto s_{f(P)}) \]
\end{itemize}
Es existiert eine natürliche Bijektion:
\[\operatorname{Hom}_X(f^{-1}G,F)\cong\operatorname{Hom}_Y(G,f_\ast F) \]
in dem wir ein $\varphi:f^{-1}(G)\to F$ auf $\psi: G\stackrel{\operatorname{ad}}{\longrightarrow} f_\ast f^{-1}G\stackrel{f_\ast(\varphi)}{\longrightarrow} f_\ast F$ schicken und ein $\psi:G\to f_\ast F$ auf $\varphi:f^{-1}G \stackrel{f^{-1}(\psi)}{\longrightarrow} f^{-1}f_\ast F\stackrel{\operatorname{ad}}{\longrightarrow}F$ schicken. Somit ist $f^{-1}$ linksadjungiert zu $f_\ast$.
\end{enumerate}

\paragraph{Definition 1.15.}\label{1.15} Sei $X$ ein topologischer Raum mit $P\in X$ und $A$ eine abelsche Gruppe. Sei $\mathcal{A}$ die konstante Garbe auf $\overline{\{P\}}$ und $i:\overline{\{P\}}\hookrightarrow X$ die natürliche Inklusion. Dann heißt die Garbe $i_\ast \mathcal{A}$ \textit{Wolkenkratzergarbe}. Es gilt:
\[(i_\ast\mathcal{A})(U) = \begin{cases}
A, &\text{wenn } P\in U\\
0, &\text{sonst}
\end{cases},\quad (i_\ast \mathcal{A})_Q = \begin{cases}
A,&\text{wenn } Q\in\overline{\{P\}}\\
0,&\text{sonst}
\end{cases} \]

\paragraph{Definition 1.16.}\label{1.16} Sei $X$ ein topologischer Raum, $Z\subset X$ abgeschlossen und $U=X\setminus Z$. Seien $j:U\hookrightarrow X$, $i:Z\hookrightarrow X$ die Inklusionsabbildungen. Ist $F$ eine Garbe auf $Z$, so gilt:
\[(i_\ast F)_P=\begin{cases}
F_P, &\text{wenn }P\in Z\\
0, &\text{sonst}
\end{cases} \]
Sei $F$ eine Garbe auf $U$ und $j_!(F)$ die Garbe auf $X$, die zur Prägarbe
\[V\mapsto\begin{cases}
F(V),&\text{wenn }V\subset U\\
0,&\text{sonst}
\end{cases} \]
assoziiert ist. Sie heißt die \textit{außerhalb $U$ durch Null fortgesetzte Garbe}\index{Garbe!Fortsetzung} von $F$. Es gilt:
\[(j_!F)_P=\begin{cases}
F_P,&\text{wenn }P\in U\\
0,&\text{sonst}
\end{cases} \]
Sei $F$ eine Garbe auf $X$. Dann ist die folgende Sequenz exakt:
\[\begin{tikzcd}[cramped, sep=small]
0\ar[r] & j_!(F|_U)\ar[r] & F \ar[r]& i_\ast(F|_Z)\ar[r] & 0
\end{tikzcd} \]

\section{Schemata}

Sei $A$ ein kommutativer Ring mit Eins und $\operatorname{Spec}(A)$ die Menge aller Primideale von $A$. Für ein Ideal $\mathfrak{a}\subset A$, setzen wir:
\[V(\mathfrak{a}) = \{\mathfrak{p}\in\operatorname{Spec}(A)\mid \mathfrak{a}\subset\mathfrak{p}\} \]

\paragraph{Lemma 2.1.}\label{2.1} \begin{enumerate}[(i)]
\item Sind $\mathfrak{a},\mathfrak{b}$ zwei Ideale von $A$, so gilt $V(\mathfrak{ab})=V(\mathfrak{a})\cup V(\mathfrak{b})$.
\item Sind $\mathfrak{a}_i\subset A,\ i\in I$ Ideale, so gilt $V\big(\sum\mathfrak{a}_i\big)=\bigcap V(\mathfrak{a}_i)$.
\item Sind $\mathfrak{a},\mathfrak{b}\subset A$ Ideale, gilt: $V(\mathfrak{a})\subset V(\mathfrak{b}) \iff \operatorname{Rad}(\mathfrak{a})\supset\operatorname{Rad}(\mathfrak{b})$
\end{enumerate}
Wegen $V(A)=\varnothing$ und $V(0)=\operatorname{Spec}(A)$ sehen wir, dass wir Teilmengen der Form $V(\mathfrak{a})$ zu abgeschlossene Mengen in $\operatorname{Spec}(A)$ erklären können. Somit erhalten wir die \textit{Zariski-Topologie}\index{Zariski-Topologie} auf $\operatorname{Spec}(A)$.

Setzen wir $D(f)=\operatorname{Spec}(A)\setminus V(f)$ für $f\in A$, so bilden diese offene Mengen eine Basis der Topologie auf $\operatorname{Spec}$.

\paragraph{Definition 2.2.}\label{2.2} Wir definieren eine Ringgarbe $\mathcal{O}$ auf $\operatorname{Spec}(A)$ wie folgt: Sei $U\subset_\text{o}\operatorname{Spec}(A)$. Setze $\mathcal{O}(U)$ als die Menge aller Abbildungen $s:U\to\coprod_{\mathfrak{p}\in U}A_\mathfrak{p}$, so dass:
\begin{enumerate}[(i)]
\item Für alle $\mathfrak{p}\in U$ gilt $s(\mathfrak{p})\in A_\mathfrak{p}$.
\item Für alle $\mathfrak{p}\in U$ gibt es eine offene Umgebung $V$ von $\mathfrak{p}$ mit $V\subset U$ und Elemente $a,f\in A$, so dass für alle $\mathfrak{q}\in V$ stets $f\not\in\mathfrak{q}$ und $s(\mathfrak{q})=\frac{a}{f}$ in $A_\mathfrak{q}$ gilt.
\end{enumerate}
Offensichtlich ist mit $s,t\in\mathcal{O}(U)$ auch $s+t,st\in\mathcal{O}(U)$. Ferner ist für $V\subset U$ offen $\operatorname{res}^U_V:\mathcal{O}(U)\to\mathcal{O}(V)$ ein Ringhomomorphismus. $\mathcal{O}$ heißt \textit{Strukturgarbe}\index{Strukturgarbe}. Das \textit{Spektrum}\index{Spektrum} von $A$ ist das Paar $(\operatorname{Spec}A,\mathcal{O})$.

\paragraph{Satz 2.3.}\label{2.3} Sei $A$ ein Ring und $(\operatorname{Spec} A,\mathcal{O})$ sein Spektrum. Dann gilt:
\begin{enumerate}[(i)]
\item Der Halm $\mathcal{O}_\mathfrak{p}$ ist isomorph zu $A_\mathfrak{p}$ für alle $\mathfrak{p}\in\operatorname{Spec}(A)$.
\item $\mathcal{O}(D(f))\cong A_f$ für alle $f\in A$
\item $\Gamma(\operatorname{Spec}(A),\mathcal{O})=A$
\end{enumerate}

\paragraph{Beweis.} (iii) folgt aus (ii) mit $f=1$.
\begin{enumerate}[(i)]
\item Die Abbildungen $\mathcal{O}(U)\to A_\mathfrak{p},\ s\mapsto s(\mathfrak{p})$ mit $\mathfrak{p}\in U\subset_\text{o}\operatorname{Spec}(A)$ sind kompatibel und induzieren einen Homomorphismus $\varphi:\mathcal{O}_\mathfrak{p}\to A_\mathfrak{p}$.
\begin{itemize}
\item \textit{Surjektivität:} Sei $\frac{a}{f}\in A_\mathfrak{p}$ mit $a,f\in A,\ f\not\in\mathfrak{p}$. Dann ist $D(f)$ eine offene Umgebung von $\mathfrak{p}$ und es gibt ein $s\in\mathcal{O}(D(f))$ mit $s(\mathfrak{p})=\frac{a}{f}$.
\item \textit{Injektivität:} Sei $U$ eine Umgebung von $\mathfrak{p}$ und $s,t\in\mathcal{O}(U)$ mit $s(\mathfrak{p})=t(\mathfrak{p})$. Verkleinern wir $U$ wenn nötig, so können wir o.B.d.A. annehmen, dass:
\[s=\frac{a}{f},\quad t=\frac{b}{g}\quad \text{ für gewisse } a,b,g,f\in A,\ g,f\not\in\mathfrak{p} \]
Wegen $\frac{a}{f}=\frac{b}{g}$ in $A_\mathfrak{p}$, gibt es ein $h\not\in\mathfrak{p}$, so dass $h(ga-bf)=0$ in $A$. Insbesondere ist $\frac{a}{f}=\frac{b}{g}$ in $A_\mathfrak{q}$ für alle $\mathfrak{q}\in\operatorname{Spec}(A)$ mit $g,f,h\not\in\mathfrak{q}$. Somit ist $s=t$ auf der offenen Umgebung $D(f)\cap D(g)\cap D(h)$ von $\mathfrak{p}$ und haben daher denselben Keim.
\end{itemize}
\item Sei $f\in A$ und $\mathfrak{p}\in D(f)$, d.h. $(f)\subset A\setminus\mathfrak{p}$. Betrachte die kanonische Abbildung $\lambda_\mathfrak{p}:A_f\to A_\mathfrak{p}$. Setze:
\[\psi: A_f \to\mathcal{O}(D(f)),\ \frac{a}{f^n}\mapsto \Big(\mathfrak{p}\mapsto \lambda_\mathfrak{p}\Big(\frac{a}{f^n}\Big)\Big) \]
\begin{itemize}
\item \textit{Injektivität:} Sei $\psi\big(\frac{a}{f^n}\big)=\psi \big(\frac{b}{f^m}\big)$ und $\mathfrak{p}\in D(f)$. Dann ist $\frac{a}{f^n}=\frac{b}{f^m}$ in $A_\mathfrak{p}$, d.h. es gibt ein $h\not\in\mathfrak{p}$ mit $h(f^ma-f^nb)=0$. Setze $\mathfrak{a}=\operatorname{Ann}(f^ma-f^nb)$. Dann ist $\mathfrak{a}\not\subset\mathfrak{p}$, da $h\in\mathfrak{a}$, also folgt $\mathfrak{p}\not\in V(\mathfrak{a})$. Wir haben also $V(\mathfrak{a})\subset V(f)$ gezeigt. Nach \hyperref[2.1]{Lemma 2.1 (iii)} folgt $f\in\operatorname{Rad}(\mathfrak{a})$, d.h. $f^e\in\mathfrak{a}$ für ein $e>0$. Per Definition gilt $f^e(f^na-f^mb)=0$, also $\frac{a}{f^n}=\frac{b}{f^m}$ in $A_f$.
\item \textit{Surjektivität:} Sei $s\in\mathcal{O}(D(f))$. Nach Definition von $\mathcal{O}$ ist $D(f)=\bigcup V_i$ mit $s=\frac{a_i}{g_i}$ auf $V_i$ für gewisse $a_i,g_i\in A,\ g_i\not\in\mathfrak{p}$ für alle $\mathfrak{p}\in V_i$. Insbesondere gilt $V_i\subset D(g_i)$. Da die $D(h)$ eine Basis der Topologie bilden, können wir o.B.d.A. $V_i=D(h_i)$ annehmen, also $D(h_i)\subset D(g_i)$. Es folgt $V(h_i)\supset V(g_i)$ und nach \hyperref[2.1]{Lemma 2.1 (iii)} auch $\operatorname{Rad}(h_i)\subset\operatorname{Rad}(g_i)$. Wähle ein $n$, so dass $h_i^n\in (g_i)$ für alle $i$, d.h. $h_i^n=c_ig_i$ für ein $c_i\in A$. Es folgt:
\[\frac{a_i}{g_i}=\frac{c_ia_i}{h_i^n}\]
Ersetzt man $a_i$ durch $c_ia_i$ und $h_i$ durch $h_i^n$, so können wir o.B.d.A. $D(f)\subset\bigcup D(h_i)$ und $s=\frac{a_i}{h_i}$ auf $D(h_i)$ annehmen.

Wir zeigen nun, dass $D(f)$ durch endlich viele $D(h_i)$ überdeckt werden kann. Wir haben mit \hyperref[2.1]{Lemma 2.1} Äquivalenzen:
\begin{align*}
D(f)\subset \bigcup_{i}D(h_i) &\iff V(f)\supset\bigcap_i V(h_i) = V\Big(\sum_i (h_i)\Big)\\
&\iff f\in\operatorname{Rad}\Big(\sum_i (h_i)\Big)\\
&\iff \exists n\in\mathbb{N}\colon f^n\in\sum_i (h_i)
\end{align*}
Daher ist $f^n$ eine endliche Summe der Form $f^n=\sum_{i=1}^r b_ih_i$ für gewisse $b_i\in A$, d.h. $D(f)\subset D(h_1)\cup\cdots\cup D(h_r)$.

Nun gilt: \[D(h_i)\cap D(h_j)=\operatorname{Spec}(A)\setminus (V(h_i)\cup V(h_j))=\operatorname{Spec}(A)\setminus V(h_ih_j)=D(h_ih_j)\]
Auf $D(h_ih_j)$ wird $s$ repräsentiert durch $\frac{a_i}{h_i}$ und $\frac{a_j}{h_j}$ in $A_{h_ih_j}$. Wenden wir die Injektivität von $\psi$ auf $D(h_ih_j)$ an, so erhalten wir $\frac{a_i}{h_i}=\frac{a_j}{h_j}$ in $A_{h_ih_j}$. Es folgt $(h_ih_j)^n(h_ja_i-h_ia_j)=0$ für ein $m$. Sei $m$ so groß, dass dies für alle endlich vielen $i,j$ gilt, also gilt für alle $i,j$:
\[h_j^{m+1}(h_i^ma_i)-h_i^{m+1}(h_j^ma_j)=0 \]
Ersetzen wir nun $h_i$ durch $h_i^{m+1}$ und $a_i$ durch $a_ih_i^m$, so wird $s$ auf $D(h_i)$ immer noch durch $\frac{a_i}{h_i}$ repräsentiert und es gilt $h_ja_i=h_ia_j$ für alle $i,j$. 

Schreibe nun $f^n=\sum b_ih_i$ für ein $n$ und setze $a=\sum b_ia_i$. Es folgt für alle $j$:
\[h_ja = \sum_i b_ia_ih_j=\sum _i b_ih_ia_j=f^na_j \]
Also gilt $\frac{a}{f^n}=\frac{a_j}{h_j}$ auf $D(h_j)$ für alle $j$, d.h. $\psi\big(\frac{a}{f^n}\big)=s$.\qed
\end{itemize}
\end{enumerate}

\paragraph{Definition 2.4.}\label{2.4} \begin{enumerate}[(i)]
\item Ein \textit{geringter Raum}\index{geringter Raum} $(X,\mathcal{O}_X)$ besteht aus einem topologischen Raum $X$ und einer Ringgarbe $\mathcal{O}_X$ auf $X$. Ein \textit{Morphismus von geringten Räumen}\index{Morphismus!geringte Räume} ist ein Paar
\[(f,f^\sharp):(X,\mathcal{O}_X)\to (Y,\mathcal{O}_Y) \]
wobei $f:X\to Y$ eine stetige Abbildung und $f^\sharp:\mathcal{O}_Y\to f_\ast\mathcal{O}_X$ ein Morphismus von Ringgarben auf $Y$ ist.
\item Ein geringter Raum $(X,\mathcal{O}_X)$ heißt \textit{lokal}\index{geringter Raum!lokal}, falls für alle $P\in X$ der Halm $\mathcal{O}_{X,P}$ ein lokaler Ring ist. Ein \textit{Morphismus von lokal geringten Räumen}\index{Morphismus!lokal geringte Räume} ist ein Morphismus $(f,f^\sharp)$ von geringten Räumen derart, dass für alle $P\in X$ die induzierte Abbildung $f_P^\sharp:\mathcal{O}_{Y,f(P)}\to\mathcal{O}_{X,P}$ lokale Homomorphismen sind.
\end{enumerate}

\paragraph{Bemerkung 2.5.}\label{2.5}\begin{itemize}
\item Die (lokal) geringte Räume bilden eine Kategorie.
\item Ein Morphismus $(f,f^\sharp)$ von (lokal) geringten Räumen ist genau dann ein Isomorphismus, wenn $f$ ein Homöomorphismus ist und $f^\sharp$ ein Garbenisomorphismus ist.
\end{itemize}

\paragraph{Satz 2.6.}\label{2.6} Seien $A,B$ Ringe.
\begin{enumerate}[(i)]
\item $(\operatorname{Spec}A,\mathcal{O})$ ist ein lokal geringter Raum.
\item Sei $\varphi:A\to B$ ein Ringhomomorphismus. Dann induziert $\varphi$ einen natürlichen Morphismus von lokal geringten Räumen:
\[(f,f^\sharp):(\operatorname{Spec}B,\mathcal{O}_B)\to (\operatorname{Spec}A,\mathcal{O}_A),\ f(\mathfrak{p})=\varphi^{-1}(\mathfrak{p}) \]
\item Jeder Morphismus $(f,f^\sharp):(\operatorname{Spec}B,\mathcal{O}_B)\to(\operatorname{Spec}A,\mathcal{O}_A)$ von lokal geringten Räu\-men ist induziert von einem Ringhomomorphismus $\varphi:A\to B$.
\end{enumerate}

\paragraph{Beweis.} \begin{enumerate}[(i)]
\item folgt aus \hyperref[2.3]{Satz 2.3 (i)}.
\item Definiere $f$ durch $f(\mathfrak{p})=\varphi^{-1}(\mathfrak{p})$ für alle $\mathfrak{p}\in\operatorname{Spec}(B)$. Sei $\mathfrak{a}\subset A$ ein Ideal. Dann ist $f^{-1}(V(\mathfrak{a}))=V(\varphi(\mathfrak{a}))$. Daher ist $f$ stetig. Sei $\mathfrak{p}\in\operatorname{Spec}(B)$. Dann liefert $\varphi$ einen lokalen Homomorphismus $\varphi_\mathfrak{p}:A_{\varphi^{-1}(\mathfrak{p})}\to B_\mathfrak{p}$. Das liefert für $V\subset_\text{o}\operatorname{Spec}(A)$ einen Ringhomomorphismus:
\[f^\sharp(V):\mathcal{O}_A(V)\to\mathcal{O}_B(f^{-1}(V)) \]
indem man eine Abbildung $s:V\to\coprod_{\mathfrak{q}\in V} A_\mathfrak{q}$ auf die folgende Abbildung $f^\sharp(s): f^{-1}(V)\to\coprod_{\mathfrak{p}\in f^{-1}(V)}B_\mathfrak{p}$ schickt:
\[ \begin{tikzcd}[row sep=.1em]
f^{-1}(V) \ar[r, "f"] & V\ar[r, "s"] & A_{\varphi^{-1}(\mathfrak{p})}\ar[r, "\varphi_\mathfrak{p}"] & B_\mathfrak{p}\\
\mathfrak{p}\ar[r, mapsto] & f(\mathfrak{p}) = \varphi^{-1}(\mathfrak{p})\ar[r, mapsto] & s(\varphi^{-1}(\mathfrak{p}))\ar[r, mapsto] & \varphi_\mathfrak{p}(s(\varphi^{-1}(\mathfrak{p})))
\end{tikzcd} \]
Die $f^\sharp(V)$ gibt uns einen Garbenmorphismus $f^\sharp:\mathcal{O}_A\to f_\ast \mathcal{O}_B$. Die durch $f^\sharp$ induzierte Abbildungen auf den Halmen sind gerade die $\varphi_\mathfrak{p}$. Somit ist $(f,f^\sharp)$ ein Morphismus lokal geringten Räumen.
\item Sei $(f,f^\sharp):(\operatorname{Spec}B,\mathcal{O}_B)\to(\operatorname{Spec}A,\mathcal{O}_A)$ ein Morphismus von lokal geringten Räu\-men. $f^\sharp$ induziert einen Ringhomomorphismus:
\[\varphi: A=\Gamma(\operatorname{Spec}A,\mathcal{O}_A)\to\Gamma(\operatorname{Spec}B,\mathcal{O}_B) =B \]
Sei $\mathfrak{p}\in\operatorname{Spec}(B)$. Dann haben wir induzierte lokale Homomorphismen mit kommutativem Diagramm:
\[\begin{tikzcd}
A_{f(\mathfrak{p})} = \mathcal{O}_{A,f(\mathfrak{p})}  \ar[r, "f^\sharp_\mathfrak{p}"] & \mathcal{O}_{B,\mathfrak{p}} = B_\mathfrak{p}\\
A \ar[r, "\varphi"']\ar[u] & B\ar[u]
\end{tikzcd} \]
Da die $f^\sharp_\mathfrak{p}$ lokale Homomorphismen sind, folgt $f(\mathfrak{p})=\varphi^{-1}(\mathfrak{p})$. Somit ist $f^\sharp$ von dem Ringhomomorphismus $\varphi$ induziert.\qed
\end{enumerate}

\paragraph{Definition 2.7.}\label{2.7} \begin{enumerate}[(i)]
\item Ein \textit{affines Schema}\index{Schema!affin} ist ein lokal geringter Raum $(X,\mathcal{O}_X)$, der als lokal geringter Raum isomorph zum Spektrum eines Rings ist.
\item Ein \textit{Schema}\index{Schema} ist ein lokal geringter Raum $(X,\mathcal{O}_X)$ derart, dass jeder Punkt $P\in X$ eine offene Umgebung $U\subset X$ besitzt, so dass $(U,\mathcal{O}_X|_U)$ ein affines Schema ist.
\end{enumerate}
$\mathcal{O}_X$ heißt \textit{Strukturgarbe}\index{Strukturgarbe}. Ein \textit{Morphismus}\index{Morphismus!Schemata} von Schemata ist ein Morphismus von lokal geringten Räumen.

\paragraph{Beispiel 2.8.}\label{2.8}\begin{enumerate}
\item Sei $k$ ein Körper. $\operatorname{Spec}(k)$ ist ein affines Schema, dessen topologischer Raum aus einem Punkt besteht.
\end{enumerate}

\paragraph{Definition 2.9.}\label{2.9} Sei $K$ ein Körper. Eine \textit{diskrete Bewertung}\index{Bewertung!diskret} von $K$ ist eine Abbildung $v:K\to\mathbb{Z}\cup\{\infty\}$, so dass für alle $x,y\in K$ gilt:
\begin{enumerate}[(i)]
\item $v(xy)=v(x)+v(y)$
\item $v(x+y)\geq \min\{v(x),v(y)\}$
\item $v(x)=\infty\iff x=0$
\end{enumerate}
$R=\{x\in K\mid v(x)\geq 0\}$ definiert einen Teilring von $K$ und heißt \textit{diskreter Be\-wer\-tungs\-ring}\index{Bewertungsring!diskret} von $v$. $R$ ist ein lokaler Hauptidealring mit Maximalideal $\mathfrak{m}=\{x\in K\mid v(x) >0\}$. $R/\mathfrak{m}$ heißt \textit{Restklassenkörper}\index{Restklassenkörper} von $v$. Ein \textit{diskreter Bewertungsring}\index{Bewertungsring!diskret} $A$ ist ein nullteilerfreier Ring, der diskreter Bewertungsring für eine Bewertung seines Quotientenkörpers ist.

\paragraph{Beispiel 2.8}\begin{enumerate}
\item[2.] Sei $R$ ein diskreter Bewertungsring. Es ist $T=\operatorname{Spec}(R)$ ein affines Schema, bestehend aus zwei Punkten:
\begin{itemize}
\item Der Punkt $t_0=\mathfrak{m}\in\operatorname{Spec}(R)$ ist abgeschlossen, da $V(\mathfrak{m})=\{\mathfrak{m}\}$ und besitzt $R=R_{t_0}$ als lokalen Ring.
\item Der Punkt $t_1=(0)\in\operatorname{Spec}(R)$ ist offen und dicht in $\operatorname{Spec}(R)$, da $V(0)=\operatorname{Spec}(R)$. $t_1$ besitzt $K=\operatorname{Quot}(R)=R_{t_1}$ als lokalen Ring.
\end{itemize}
\[\begin{tikzcd}[row sep=.1em]
\operatorname{Spec}(K)\ar[r] & \operatorname{Spec}(R) &\ar[l] \operatorname{Spec}(R/\mathfrak{m})\\
(0)\ar[r, mapsto] & t_1\quad t_0 & \ar[l, mapsto] (0)
\end{tikzcd} \]
\item[3.] Sei $k$ ein Körper. Die \textit{affine Gerade}\index{affine Gerade} $\mathbf{A}_k^1$ über $k$ ist $\operatorname{Spec}k[X]$. Sei $\xi$ das Nullideal in $\operatorname{Spec}k[X]$. Dann ist $\overline{\{\xi\}}=\mathbf{A}_k^1$. Ein solcher Punkt heißt \textit{generischer Punkt}\index{generischer Punkt}. Alle anderen Punkte sind abgeschlossen, da diese den maximalen Idealen in $k[X]$ entsprechen. Es besteht eine Bijektion zwischen den irreduziblen, nicht-konstanten, normierten Polynomen aus $k[X]$ und den abgeschlossenen Punkten von $\mathbf{A}_k^1$.

Ist $k$ algebraisch abgeschlossen, so besteht eine Bijektion zwischen den Elementen aus $k$ und den abgeschlossenen Punkten von $\mathbf{A}_k^1$.
\item[4.] Allgemeiner definieren wir den \textit{affinen $n$-dimensionalen Raum}\index{affiner Raum} über $k$ als: \[\mathbf{A}_k^n=\operatorname{Spec}k[X_1,\ldots,X_n]\]
\item[5.] Sei $k$ algebraisch abgeschlossen. Dann entsprechen die abgeschlossenen Punkte von $\mathbf{A}_k^n$ nach dem hilbertschen Nullstellensatz bijektiv den $n$-Tupeln von Elementen aus $k$. Ferner gibt es einen generischen Punkt $\xi$, der dem Nullideal in $k[X_1,\ldots,X_n]$ entspricht, d.h. $\overline{\{\xi\}}=\mathbf{A}_k^n$.
\end{enumerate}

\paragraph{Definition 2.10.}\label{2.10} \textit{(Offene Unterschemata)}\index{Schema!Unterschema!offen} Sei $(X,\mathcal{O}_X)$ ein Schema und $U\subset X$ offen. Dann ist $(U,\mathcal{O}_X|_U)$ ein Schema. Diese Aussage ist nichttrivial und wir werden sie \hyperref[2.10-beweis]{später} zeigen. Die offene Menge $U$ besitzt die \textit{induzierte Unterschemastruktur}.

\paragraph{Definition 2.11.}\label{2.11} \textit{(Verkleben von Schemata)}\index{Verklebung!Schema} Sei $\{X_i\}$ eine Familie von Schemata und $U_{ij}\subset X_i,\ i\neq j$ offene Teilmengen mit induzierter Struktur. Ferner haben wir für  $i\neq j$ Isomorphismen von Schemata:
\[\varphi_{ij}: (U_{ij},\mathcal{O}_{X_i}|_{U_{ij}}) \stackrel{\sim}{\to} (U_{ji},\mathcal{O}_{X_j}|_{U_{ji}}) \]
mit $\varphi_{ij}=\varphi_{ji}^{-1}$ und $\varphi_{ij}(U_{ij}\cap U_{ik})=U_{ji}\cap U_{jk}$ und $\varphi_{jk}=\varphi_{ik}\circ\varphi_{ij}$ auf $U_{ij}\cap U_{ik}$ für alle paar\-wei\-se verschiedene $i,j,k$. Wir erhalten ein Schema $X$ durch \textit{Verkleben} der $X_i$ längst $U_{ij}$ bzgl. $\varphi_{ij}$:
\[X=\dot{\bigcup_i}\ X_i\Big/{\sim},\quad x_i\sim\varphi_{ij}(x_i)\text{ für alle }x_i\in U_{ij},\ i\neq j \]
$X$ besitze die Quotiententopologie. Es existiert für jedes $j$ ein Morphismus $\psi_j:X_j\to X$ von Schemata, das ein Isomorphismus auf einem offenen Unterschema in $X$ induziert mit $X=\bigcup\psi_j(X_j)$ und $\psi_i(U_{ij})=\psi_i(X_i)\cap \psi_j(X_j)$ und $\psi_i=\psi_j\circ\varphi_{ij}$ auf $U_{ij}$ für alle $i\neq j$. Die Strukturgarbe auf $X$ ist folgendermaßen gegeben: Sei $V\subset_\text{o} X$. Setze:
\[\mathcal{O}_X(V)=\Big\{ (s_i)\in\prod_i \mathcal{O}_{X_i}(\psi_i^{-1}(V))\Bigm| \varphi_{ij}(s_i|_{\psi_i^{-1}(V)\cap U_{ij}}) = s_j|_{\psi_j^{-1}(V)\cap U_{ji}} \Big\} \]
Somit ist $(X,\mathcal{O}_X)$ ein lokal geringter Raum. Da alle $X_i$ Schemata sind, besitzt jeder Punkt von $X$ eine affine Umgebung. Also ist $X$ ein Schema.

\paragraph{Beispiel 2.12.}\label{2.12} Sei $k$ ein Körper, $X_1=X_2=\mathbf{A}_k^1$ und $U_1=U_2=\mathbf{A}_k^1\setminus\{P\}$ mit einem abgeschlossenen Punkt $P$. Ist $\varphi:U_1\to U_2$ die Identität, so ist die Verklebung $X$ von $X_1$ und $X_2$ längst $\varphi$ die affine Gerade, wobei der Punkt $P$ verdoppelt wurde. $X$ ist selbst nicht mehr affin.

\paragraph{Satz 2.13.}\label{2.13} Sei $A$ ein Ring und $(X,\mathcal{O}_X)$ ein Schema. Dann ist die Abbildung bijektiv:
\[\alpha:\operatorname{Hom}_\mathbf{Sch}(X,\operatorname{Spec}A)\to \operatorname{Hom}_\textbf{Ringe}(A,\Gamma(X,\mathcal{O}_X)) \]
wobei wir $(f:X\to\operatorname{Spec}A ,\ f^\sharp:\mathcal{O}_A\to f_\ast\mathcal{O}_X)$ durch das Nehmen der globalen Schnitte auf den Ringhomomorphismus $A=\Gamma(\operatorname{Spec}A,\mathcal{O}_A)\to\Gamma(X,\mathcal{O}_X)$ schicken.

\paragraph{Beweis.} Sei $X=\bigcup_\nu X_\nu$ eine affine Überdeckung. Ein Morphismus $(f,f^\sharp)$ ist eindeutig durch seine Einschränkungen $(f_\nu,f_\nu^\sharp)$ auf $X_\nu$ bestimmt. Diese sind nach \hyperref[2.6]{Satz 2.6} wiederum eindeutig bestimmt durch $\alpha(f_\nu)$. Ferner kommutiert das folgende Diagramm:
\[\begin{tikzcd}
A\ar[d, "\alpha(f)"']\ar[rr, "\alpha(f_\nu)"] && \Gamma(X_\nu,\mathcal{O}_{X_\nu})\\
\Gamma(X,\mathcal{O}_X)\ar[rru, "\operatorname{res}"']
\end{tikzcd} \]
weshalb $\alpha(f)$ schon alle $\alpha(f_\nu)$ eindeutig bestimmt. Somit ist $\alpha$ injektiv. Sei nun ein Ringhomomorphismus $h:A\to\Gamma(X,\mathcal{O}_X)$ gegeben und $h_\nu:A\to\Gamma(X,\mathcal{O}_X)\to\Gamma(X_\nu,\mathcal{O}_X)$. Nach \hyperref[2.6]{Satz 2.6 (iii)} existiert ein $f_\nu:X_\nu\to\operatorname{Spec}(A)$ mit $\alpha(f_\nu)=h_\nu$. Für alle $\nu,\mu$ ist das folgende Diagramm kommutativ:
\[\begin{tikzcd}
& \Gamma(X_\nu,\mathcal{O}_X)\ar[dr, "\operatorname{res}"] &\\
A\ar[ur, "h_\nu"] \ar[dr, "h_\mu"'] & & \Gamma(X_\nu\cap X_\mu,\mathcal{O}_X)\\
& \Gamma(X_\mu,\mathcal{O}_X)\ar[ur, "\operatorname{res}"'] &
\end{tikzcd} \]
Aus der Injektivität von $\alpha$ folgt $f_\nu=f_\mu$ auf $X_\nu\cap X_\mu$. Kleben wir die $f_\nu$ nun zusammen, so erhalten wir ein $f:X\to\operatorname{Spec}(A)$ mit $\alpha(f)=h$.\qed 

\paragraph{Korollar 2.14.}\label{2.14} Es gibt eine pfeilumkehrende Kategorienäquivalenz zwischen der Kategorie der affinen Schemata und der Kategorie der kommutativen Ringe mit $1$.

\paragraph{Korollar 2.15.}\label{2.15} $\operatorname{Spec}(\mathbb{Z})$ ist Endobjekt in der Kategorie der Schemata, d.h. für jedes Schema $X$ existiert ein eindeutiger Morphismus $f:X\to\operatorname{Spec}(\mathbb{Z})$.

\paragraph{Satz 2.16.}\label{2.16} Sei $A$ ein Ring, $X=\operatorname{Spec}(A)$ und $f\in A$. Dann gilt:
\[(D(f),\mathcal{O}_X|_{D(f)})\cong \operatorname{Spec}(A_f) \]

\paragraph{Beweis.} Die natürliche Abbildung $\varphi:A\to A_f$ induziert einen Homöomorphismus:
\[\psi:\operatorname{Spec}(A_f) \to \{\mathfrak{p}\in\operatorname{Spec}(A)\mid f\not\in\mathfrak{p} \}=D(f),\ \mathfrak{p}\mapsto\varphi^{-1}(\mathfrak{p})\qedhere \]

\paragraph{Beweis zu \hyperref[2.10]{Definition 2.10}.}\label{2.10-beweis} Sei $(X,\mathcal{O}_X)$ ein Schema und $U\subset X$ offen. Dann ist $(U,\mathcal{O}_X|_U)$ ein lokal geringter Raum. Sei $P\in U$. Wir zeigen, dass eine Umgebung $V\subset_\text{o} U$ von $P$ existiert, so dass $(V,\mathcal{O}_X|_V)$ affin ist. Da $(X,\mathcal{O}_X)$ ein Schema ist, existiert ein $V'\subset_\text{o}X,\ P\in V'$ mit $(V',\mathcal{O}_X|_{V'})$ affin. Sei also $V'=\operatorname{Spec}(A)$ für einen Ring $A$. Da die $D(f),\ f\in A$ eine Basis der Topologie auf $X$ bilden, existiert ein $f\in A$, so dass $P\in D(f)\subset V'\cap U$. Wegen \hyperref[2.16]{Satz 2.16} ist $(D(f),\mathcal{O}_X|_{D(f)})\cong\operatorname{Spec}(A_f)$ affin.\qed

\paragraph{Satz 2.17.}\label{2.17} Sei $X$ ein Schema und $Z\subset X$ irreduzibel und abgeschlossen. Dann existiert genau ein Punkt $\xi\in Z$ derart, dass $\overline{\{\xi\}}=Z$. $\xi$ heißt \textit{generischer Punkt}\index{generischer Punkt} von $Z$.

\paragraph{Beweis.}\begin{itemize}
\item \textit{Existenz:} Sei $U\subset_\text{o}X$ affin mit $Z\cap U\neq\varnothing$. Da $Z\cap U$ abgeschlossen in $U$ ist, gibt es ein Radikalideal $\mathfrak{p}$ mit $Z\cap U=V(\mathfrak{p})$. 

Nun ist $Z\cap U$ irreduzibel, da für offene Mengen $V_1,V_2\subset_\text{o} Z\cap U$ stets $V_1,V_2\subset_\text{o}Z$ gilt und aus der Irreduzibilität von $Z$ stets $V_1\cap V_2\neq\varnothing$ folgt. 

Ferner ist $\mathfrak{p}$ ein Primideal, denn ist $\mathfrak{ab}\subset\mathfrak{p}$ für gewisse Ideale $\mathfrak{a}$ und $\mathfrak{b}$, so folgt $\operatorname{Rad}(\mathfrak{ab})\subset\operatorname{Rad}(\mathfrak{p})=\mathfrak{p}$. Dies ist äquivalent zu $V(\mathfrak{p})\subset V(\mathfrak{ab})=V(\mathfrak{a})\cup V(\mathfrak{b})$, daher:
\[Z\cap U =V(\mathfrak{p})=(V(\mathfrak{p})\cap V(\mathfrak{a}))\cup (V(\mathfrak{p})\cap V(\mathfrak{b})) \]
Da $Z\cap U$ irreduzibel ist, folgt o.B.d.A. $V(\mathfrak{p})=V(\mathfrak{p})\cap V(\mathfrak{a})\subset V(\mathfrak{a})$, d.h. $\mathfrak{a}\subset\operatorname{Rad}(\mathfrak{a})\subset\operatorname{Rad}(\mathfrak{p})=\mathfrak{p}$.

$\mathfrak{p}$ ist nun generischer Punkt von $Z\cap U$, da:
\[\overline{\{\mathfrak{p}\}}=\bigcap_{\mathfrak{p}\in V(\mathfrak{a})}V(\mathfrak{a})=\bigcap_{\mathfrak{a}\subset\mathfrak{p}} V(\mathfrak{a})\subset V(\mathfrak{p}) \]
Die andere Inklusion folgt aus $\mathfrak{a}\subset\mathfrak{p}$. Da $\mathfrak{p}$ prim ist, folgt $\operatorname{Rad}(\mathfrak{a})\subset\mathfrak{p}$ und somit $V(\mathfrak{p})\subset V(\mathfrak{a})$. Somit ist $Z\cap U=\overline{\{\mathfrak{p}\}}$.

Bezeichne mit $\overline{\{\mathfrak{p}\}}^X$ den topologischen Abschluss in $X$. Dann ist $Z\cap U\subset\overline{\{\mathfrak{p}\}}^X$. Da $Z$ irreduzibel ist, gilt $Z=\overline{Z\cap U}\subset\overline{\{\mathfrak{p}\}}^X$, also $Z=\overline{\{\mathfrak{p}\}}^X$.
\item \textit{Eindeutigkeit:} Sei $\overline{\{\xi\}}^X=Z=\overline{\{\xi'\}}^X$. Sei $U\subset_\text{o}X$ eine affine Umgebung von $\xi\in U$. Dann ist $\xi'\in U$, da aus $\xi'\in Z\setminus U$ der Widerspruch $\xi\in Z=\overline{\{\xi'\}}^X\subset X\setminus U$ folgt. Wähle nun Radikalideale $\mathfrak{p}'$ und $\mathfrak{p}$ mit:
\[V(\mathfrak{p}')= \overline{\{\xi'\}}^U=U\cap \overline{\{\xi'\}}^X=U\cap\overline{\{\xi\}}^X=\overline{\{\xi\}}^U=V(\mathfrak{p}) \]
Es folgt $\xi'=\mathfrak{p}'=\operatorname{Rad}(\mathfrak{p}')=\operatorname{Rad}(\mathfrak{p})=\mathfrak{p}=\xi$.\qed
\end{itemize}

\paragraph{Satz 2.18.}\label{2.18} Sei $X$ ein Schema und $K$ ein Körper. Dann gibt es eine Bijektion:
\[\operatorname{Hom}_\mathbf{Sch}(\operatorname{Spec}K,X)\stackrel{\sim}{\to} \{(x,i)\mid x\in X,\ i:\kappa(x)\hookrightarrow K\text{ Ringhomomorphismus} \} \]
wobei $\kappa(x)=\mathcal{O}_{X,x}/\mathfrak{m}_x$ der Restklassenkörper von $\mathcal{O}_{X,x}$ bezeichnet. Die Elemente heißen \textit{$K$-wertige Punkte} von $X$.\index{Punkt!$K$-wertig}

\paragraph{Beweis.} Sei ein Morphismus $f:\operatorname{Spec}(K)\to X$ gegeben. Setze $x=f((0))\in X$. $f^\sharp$ induziert einen lokalen Homomorphismus $\mathcal{O}_{X,x}\to\mathcal{O}_{K,(0)}=K$. $f^\sharp$ faktorisiert daher über $\mathcal{O}_{X,x}/\mathfrak{m}_x=\kappa(x)$ und induziert einen Homomorphismus $i:\kappa(x)\hookrightarrow K$.

Sei nun umgekehrt ein $x\in X$ und $i:\kappa(x)\hookrightarrow K$ gegeben. $i$ definiert nach \hyperref[2.6]{Satz 2.6} ein Schemamorphismus:
\[f:\operatorname{Spec}(K)\to\operatorname{Spec}\kappa(x)\to\operatorname{Spec}(\mathcal{O}_{X,x})\stackrel{\psi}{\to} X \]
wobei $\psi$ die folgende kanonische Abbildung ist: 

Sei $U\subset_\text{o}X$ eine affine Umgebung von $x$ mit $U=\operatorname{Spec}(A)$. Nach \hyperref[2.3]{Satz 2.3 (i)} gilt $\mathcal{O}_{X,x}=\mathcal{O}_{U,x}=A_x$. Die kanonische Abbildung $A\to A_x$ induziert $\psi:\operatorname{Spec}(\mathcal{O}_{X,x})\to U\hookrightarrow X$. Diese Abbildung ist unabhängig von $U$: 

Sei $U'\subset_\text{o}X$ eine weitere affine Umgebung von $x$. Dann existiert eine affine Umgebung $U''\subset_\text{o}U\cap U'$ mit $x\in U''$, also können wir o.B.d.A. $\operatorname{Spec}(A)= U\subset U'=\operatorname{Spec}(A')$ annehmen. Es existiert ein kanonischer Homomorphismus $A'\to A$ derart, dass das folgende Diagramm kommutiert:
\[\begin{tikzcd}
A'\ar[rd]\ar[r] & A\ar[d]\\
& A'_x= \mathcal{O}_{X,x}= A_x &
\end{tikzcd} \]
Somit kommutiert:
\[\begin{tikzcd}
X & U'\ar[l] & U\ar[l]\\
 && \operatorname{Spec}(\mathcal{O}_{X,x})\ar[lu]\ar[u]
\end{tikzcd}\qedhere\]

\paragraph{Satz 2.19.}\label{2.19} \begin{enumerate}[(i)]
\item Sei $A$ ein Ring und $f\in A$. Dann gilt:
\[D(f)=\varnothing\iff f\text{ nilpotent} \]
\item Sei $\varphi:A\to B$ ein Ringhomomorphismus und $f: Y=\operatorname{Spec}(B)\to\operatorname{Spec}(A)=X$ der durch $\varphi$ induzierte Morphismus affiner Schemata. Dann gilt:
\begin{enumerate}[(a)]
\item $\varphi$ ist genau dann injektiv, wenn $f^\sharp:\mathcal{O}_X\to f_\ast\mathcal{O}_Y$ injektiv ist. In diesem Fall ist $f$ \textit{dominant}\index{dominant}, d.h. $f(Y)\subset X$ ist dicht.
\item $\varphi$ ist genau dann surjektiv, wenn $f^\sharp$ surjektiv ist und $f$ ein Homöomorphismus auf eine abgeschlossene Teilmenge von $X$ ist.
\end{enumerate}
\end{enumerate}

\paragraph{Beweis.} Übung.\qed

% \paragraph{Beispiel.} Das Spektrum $T=\operatorname{Spec}( k[X,Y]/(X^2))$ entspricht der $Y$-Achse, also besitzt der Ring $\mathcal{O}_T(T)= k[X,Y]/(X^2)$ nilpotente Elemente.

\paragraph{Definition 2.20.}\label{2.20} Ein Schema $(X,\mathcal{O}_X)$ heißt \textit{reduziert}\index{Schema!reduziert}, falls $\mathcal{O}_X(U)$ für alle $U\subset_\text{o}X$ reduziert sind, d.h. keine nilpotente Elemente besitzt.

\paragraph{Regeln 2.21.}\label{2.21} Sei $(X,\mathcal{O}_X)$ ein Schema.
\begin{enumerate}[(i)]
\item $(X, \mathcal{O}_X)$ ist genau dann reduziert, wenn $\mathcal{O}_{X,P}$ für alle $P\in X$ keine nilpotente Elemente besitzt.
\item Sei $\mathcal{O}_X^\text{red}$ die assoziierte Garbe zur folgenden Prägarbe:
\[ U \mapsto \mathcal{O}_X(U)/\mathfrak{N}_U \]
wobei $\mathfrak{N}_U$ das Nilradikal von $\mathcal{O}_X(U)$ bezeichnet. Dann ist $(X,\mathcal{O}_X^\text{red})$ ein Schema, das zu $X$ assoziierte \textit{reduzierte Schema}\index{Schema!reduziert} $X_\text{red}$. Es gibt einen Morphismus $f:X_\text{red}\to X$ mit dem Homöomorphismus $\operatorname{id}$ auf den unterliegenden topologischen Räumen und $f^\sharp:\mathcal{O}_X\to f_\ast\mathcal{O}_X^\text{red}$ gegeben durch:
\[\mathcal{O}_X(U)\to\mathcal{O}_X^\text{red}(U),\ s\mapsto \Big(U\stackrel{s}{\to}\coprod_{P\in U}\mathcal{O}_{X,P} \to\coprod_{P\in U}\mathcal{O}_{X,P}^\text{red}\Big) \]
\item Sei $f:X\to Y$ ein Morphismus von Schemata mit $X$ reduziert. Setze $g=f$ auf den unterliegenden topologischen Räumen. Da $X$ reduziert ist, faktorisiert $f^\sharp:\mathcal{O}_Y\to f_\ast\mathcal{O}_X$ über $\mathcal{O}_Y^\text{red}$. $f^\sharp$ induziert $g^\sharp:\mathcal{O}_Y^\text{red}\to g_\ast\mathcal{O}_X=f_\ast\mathcal{O}_X$. Es gibt also einen eindeutig bestimmten Morphismus $g:X\to Y_\text{red}$ mit kommutativem Diagramm:
\[\begin{tikzcd}
X\ar[rr, "f"]\ar[drr, "g"'] && Y\\
&& Y_\text{red}\ar[u]
\end{tikzcd} \]
\end{enumerate}

\paragraph{} Sei $S=\bigoplus_{d\geq 0} S_d$ ein graduierter Ring und $S_+=\bigoplus_{d>0}S_d$. Ein Primideal $\mathfrak{p}\subset S$ ist genau dann homogen, wenn aus $fg\in\mathfrak{p}$ für gewisse homogene Elemente $f,g\in S$ stets $f\in\mathfrak{p}$ oder $g\in\mathfrak{p}$ folgt. Für ein homogenes Ideal $\mathfrak{a}\subset S$ setze:
\[\operatorname{Proj}(S)=\{\mathfrak{p}\subset S\text{ homogenes Primideal}\mid S_+\not\subset\mathfrak{p}\},\quad V(\mathfrak{a})=\{\mathfrak{p}\in\operatorname{Proj}(S)\mid\mathfrak{a}\subset\mathfrak{p}\} \]

\paragraph{Lemma 2.22.}\label{2.22}\begin{enumerate}[(i)]
\item Sind $\mathfrak{a},\mathfrak{b}\subset S$ homogene Ideale, so gilt $V(\mathfrak{ab})=V(\mathfrak{a})\cup V(\mathfrak{b})$.
\item Ist $(\mathfrak{a}_i)_i$ eine Familie homogener Ideale in $S$, so folgt $V\big(\sum\mathfrak{a}_i\big)=\bigcap V(\mathfrak{a}_i)$.
\end{enumerate}
Damit wird auf $\operatorname{Proj}(S)$ eine Topologie definiert. Die abgeschlossenen Mengen sind genau die Mengen der Form $V(\mathfrak{a})$ für ein homogenes Ideal $\mathfrak{a}\subset S$.

\paragraph{Beweis.} Wie in \hyperref[2.1]{Lemma 2.1} unter Beachtung, dass homogene Ideale von homogene Elemente erzeugt werden.\qed

\paragraph{Definition.} Sei $T\subset S$ multiplikativ abgeschlossen, die aus homogenen Elementen besteht. Dann wird $T^{-1}S = \bigoplus_{i\geq 0} (T^{-1}S)_i$ zu einem graduierten Ring:
\[(T^{-1}S)_i=\Big\{\frac{s}{t}\in T^{-1}S\Bigm| s\in S\text{ homogen},\ t\in T,\ \deg(s)-\deg(t)=i \Big\} \]
Ist $\mathfrak{p}\subset S$ ein homogenes Primideal und $f\in S$ ein homogenes Element, so ist die \textit{homogene Lokalisierung}\index{homogene Lokalisierung} bzgl. $\mathfrak{p}$ bzw. $f$ definiert als:
\[S_{(\mathfrak{p})} = (S_\mathfrak{p})_0,\quad S_{(f)} = (S_f)_0 \]

\paragraph{Definition.} Wir definieren eine Ringgarbe $\mathcal{O}$ auf $\operatorname{Proj}(S)$ wie folgt: Sei $U\subset_\text{o}\operatorname{Proj}(S)$ und setze $\mathcal{O}(U)$ als die Menge aller Abbildungen $s:U\to\coprod_{\mathfrak{p}\in U}S_{(\mathfrak{p})}$, so dass:
\begin{enumerate}[(i)]
\item Für alle $\mathfrak{p}\in U$ gilt $s(\mathfrak{p})\in S_{(\mathfrak{p})}$.
\item Für alle $\mathfrak{p}\in U$ existiert eine offene Umgebung $V$ von $\mathfrak{p}$ mit $V\subset U$ und homogene Elemente $a,f\in S$ mit $\deg(a)=\deg(f)$ derart, dass für alle $\mathfrak{q}\in V$ stets $f\not\in\mathfrak{q}$ und $s(\mathfrak{q})=\frac{a}{f}$ in $S_{(\mathfrak{q})}$ gilt.
\end{enumerate}

\paragraph{Satz 2.23.}\label{2.23} Sei $S$ ein graduierter Ring. Dann gilt:
\begin{enumerate}[(i)]
\item $\mathcal{O}_\mathfrak{p}\cong S_{(\mathfrak{p})}$ für alle $\mathfrak{p}\in\operatorname{Proj}(S)$.
\item Für ein homogenes $f\in S_+$ setze $D_+(f)=\{\mathfrak{p}\in\operatorname{Proj}(S)\mid f\not\in\mathfrak{p}\}$. Dann ist $D_+(f)$ offen in $\operatorname{Proj}(S)$ und es gilt:
\[\operatorname{Proj}(S) =\bigcup_{f\in S_+\text{ homogen}}D_+(f) \]
Es gibt einen Isomorphismus lokal geringter Räume $(D_+(f),\mathcal{O}|_{D_+(f)})\cong\operatorname{Spec}S_{(f)}$.
\item $(\operatorname{Proj}S,\mathcal{O})$ ist ein Schema.
\end{enumerate}

\paragraph{Beweis.}\begin{enumerate}[(i)]
\item Die Abbildung $\mathcal{O}_\mathfrak{p}\to S_{(\mathfrak{p})},\ s_\mathfrak{p}\mapsto s(\mathfrak{p})$, wobei $s$ ein Repräsentant von $s_\mathfrak{p}$ ist, ist ein Isomorphismus. Beweis analog wie \hyperref[2.3]{Satz 2.3 (i)}.
\item Da $D_+(f)=\operatorname{Proj}(S)\setminus V(f)$, ist $D_+(f)$ offen. Sei $\mathfrak{p}\in\operatorname{Proj}(S)$, d.h. $\mathfrak{p}\subset S$ ist ein homogenes Primideal mit $S_+\not\subset\mathfrak{p}$. Sei $f\in S_+\setminus\mathfrak{p}$. Dann ist $\mathfrak{p}\not\in V(f)$, also $\mathfrak{p}\in D_+(f)$. Daher ist $\operatorname{Proj}(S)=\bigcup D_+(f)$.

Sei $f\in S_+$. Wir definieren ein Morphismus lokal geringter Räume $(\phi,\phi^\sharp): D_+(f)\to\operatorname{Spec}S_{(f)}$ wie folgt: Sei $S\rightarrow S_f$ der natürliche Homomorphismus. Sei $\mathfrak{a}\subset S$ ein homogenes Ideal und setze $\phi(\mathfrak{a})=\mathfrak{a}S_f\cap S_{(f)}$. Beachte, dass $S_{(f)}=(S_f)_0\subset S_f$ ein Teilring ist. Für $\mathfrak{p}\in D_+(f)$ ist $\phi(\mathfrak{p})\in\operatorname{Spec}S_{(f)}$, siehe \hyperref[2.16]{Satz 2.16}, und $\phi$ ist bijektiv. Sei $\mathfrak{a}\subset S$ ein homogenes Ideal. Dann ist $\mathfrak{p}\supset\mathfrak{a}$ genau dann, wenn $\phi(\mathfrak{p})\supset\phi(\mathfrak{a})$. Daher ist $\phi$ ein Homöomorphismus.

$\phi^\sharp:\mathcal{O}_{\operatorname{Spec} S_{(f)}}\to\phi_\ast \big(\mathcal{O}_{\operatorname{Proj}(S)}|_{D_+(f)}\big)$ wird wie folgt definiert:
\[\mathcal{O}_{\operatorname{Spec}S_{(f)}}(U)\to\mathcal{O}_{\operatorname{Proj}(S)}(\phi^{-1}(U)),\ s\mapsto \Big(\phi^{-1}(U)\stackrel{s\circ\phi}{\longrightarrow}\coprod (S_{(f)})_{\phi(\mathfrak{p})} \cong\coprod S_{(\mathfrak{p})}\Big) \]
Dieses ist ein Isomorphismus.
\item folgt aus (i) und (ii).\qed
\end{enumerate}

\paragraph{Beispiel 2.24.}\label{2.24} Sei $A$ ein Ring. Dann heißt
\[\mathbf{P}_A^n = \operatorname{Proj} A[X_0,\ldots,X_n] \]
der \textit{$n$-dimensionaler projektiver Raum}\index{projektiver Raum} über $A$. Ist speziell $A=k$ ein algebraisch abgeschlossener Körper, so ist $\mathbf{P}_k^n$ ein Schema. Dessen Teilraum aller abgeschlossenen Punkte ist homöomorph zur projektiven $n$-dimensionalen Varietät.

\paragraph{Definition 2.25.} \label{2.25} Sei $S$ ein beliebiges Schema. Ein \textit{Schema über $S$}\index{Schema!über $S$} ist ein Schema $X$ zusammen mit einem Morphismus $X\to S$, der sogenannte \textit{Strukturmorphismus}\index{Strukturmorphismus}. Ein Morphismus zweier Schemata $X$ und $Y$ über $S$ ist ein Morphismus $f:X\to X'$, so dass das folgende Diagramm kommutiert:
\[\begin{tikzcd}
X\ar[rr, "f"]\ar[d] && X'\ar[lld]\\
S
\end{tikzcd} \]
So ein Morphismus nennt man auch \textit{$S$-Morphismus}\index{Morphismus!$S$-Schemata}. Bezeichne die Kategorie aller Schemata über $S$ mit $S$-Morphismen mit $\mathbf{Sch}(S)$. Für einen Ring $A$ setzen wir auch $\mathbf{Sch}(A)=\mathbf{Sch}(\operatorname{Spec}A)$.

\paragraph{Satz 2.26.}\label{2.26} Sei $k$ ein algebraisch abgeschlossener Körper. Dann gibt es einen natürlichen Funktor
\[t:\mathbf{Var}(k)\to\mathbf{Sch}(k) \]
der \textit{volltreu}\index{volltreuer Funktor} ist, d.h. für zwei Varietäten $V,W$ ist die durch $t$ auf den Morphismen induzierte Abbildung $\operatorname{Hom}_{\mathbf{Var}(k)}(V,W)\to \operatorname{Hom}_{\mathbf{Sch}(k)}(tV,tW)$ bijektiv. Für eine Varietät $V$ setze $\mathfrak{M}(V)$ als die Menge aller abgeschlossenen Punkte des Schemas $tV$ mit der Teilraumtopologie. Es gibt einen Homöomorphismus topologischer Räume $V\cong\mathfrak{M}(V)$. Die Garbe der regulären Funktionen ist via diesen Homöomorphismus isomorph zu $\mathcal{O}_{tV}|_{\mathfrak{M}(V)}$.

\paragraph{Beweis.} Siehe z.B. Hartshorne Kapitel II, Proposition 2.6 oder Mumford, Theorem 2 auf Seite 168.

\section{Erste Eigenschaften von Schemata}

\paragraph{Definition 3.1.}\label{3.1} \begin{enumerate}[(i)]
\item Ein Schema heißt \textit{zusammenhängend}\index{Schema!zusammenhängend}, falls es als topologischer Raum zu\-sam\-men\-hän\-gend ist.
\item Ein Schema heißt \textit{irreduzibel}\index{Schema!irreduzibel}, falls es als topologischer Raum irreduzibel ist.
\item Ein Schema heißt \textit{reduziert}\index{Schema!reduziert} falls für alle $U\subset_\text{o}X$ der Ring $\mathcal{O}_X(U)$ reduziert ist.
\item Ein Schema heißt \textit{integer}\index{Schema!integer}, falls für alle $U\subset_\text{o}X$ der Ring $\mathcal{O}_X(U)$ nullteilerfrei ist.
\end{enumerate}

\paragraph{Beispiel 3.2.}\label{3.2} Sei $X=\operatorname{Spec}(A)$ ein affines Schema. Dann gilt:
\begin{enumerate}[(i)]
\item $X$ ist irreduzibel $\iff \mathfrak{N}(A)$ ist ein Primideal
\item $X$ ist reduziert $\iff A$ ist reduziert $\iff\mathfrak{N}(A)=0$
\item $X$ ist integer $\iff A$ ist nullteilerfrei
\end{enumerate}

\paragraph{Beweis.} (i) und (ii) sind klar. Ist $X$ integer, so ist $A=\mathcal{O}_X(X)$ nullteilerfrei. Sei nun umgekehrt $A$ nullteilerfrei, d.h. $\mathfrak{N}(A)=0$ ist ein Primideal. Nach (i) und (ii) ist $X$ irreduzibel und reduziert. Daher folgt die Aussage aus dem nächsten Satz.\qed

\paragraph{Satz 3.3.}\label{3.3} Ein Schema $X$ ist genau dann integer, wenn $X$ reduziert und irreduzibel ist.

\paragraph{Beweis.} Sei $X$ integer. Dann ist $X$ offensichtlich reduziert. Wäre $X$ nicht irreduzibel, so gäbe es $\varnothing\neq U_1,U_2\subset_\text{o}X$ mit $U_1\cap U_2=\varnothing$. Dann ist:
\[\mathcal{O}_X(U_1\cup U_2)=\mathcal{O}_X(U_1)\times\mathcal{O}_X(U_2) \]
Somit ist $\mathcal{O}_X(U_1\cup U_2)$ nicht nullteilerfrei. Sei nun umgekehrt $X$ irreduzibel und reduziert.

Sei $V\subset_\text{o}X$ affin mit $V=\operatorname{Spec}(A)$. Sei $a,b\in A$ mit $ab=0$. Es folgt:
\[\operatorname{Spec}(A)=V(ab)=V(a)\cup V(b) \]
Da $V$ irreduzibel ist, folgt o.B.d.A. $\operatorname{Spec}(A)=V(a)$. Da $A=\mathcal{O}_X(V)$ reduziert ist, folgt $\operatorname{Rad}(a)=(0)$, also $a=0$. Daher ist $\mathcal{O}_X(V)$ für jedes affine $V\subset_\text{o}X$ nullteilerfrei. 

Sei nun $U\subset_\text{o}X$ beliebig und $f,g\in\mathcal{O}_X(U)$ mit $fg=0$. Für $V\subset_\text{o}U$ affin, folgt aus $f|_V\cdot g|_V=0$ o.B.d.A. $f|_V=0$. Nun ist $U$ der Abschluss von $V$ in $U$. Sei $x\in U$ und $U(x)\subset_\text{o}U$ eine affine Umgebung von $x$ mit $U(x)=\operatorname{Spec}(B)$. Sei $f(x)=\frac{a}{h}\in B_x$ für alle $x\in U(x)$. Es ist $U(x)\cap V\neq\varnothing$, da $U$ irreduzibel ist. Wähle ein $y\in U(x)\cap V$; es folgt $0=f(y)=\frac{a}{h}\in B_y$, also gibt es ein $k\in B\setminus y$ mit $ka=0$. Da $B$ nullteilerfrei ist, folgt $a=0$ und $f=0$ auf $U(x)$. Somit folgt $f=0$.\qed

\paragraph{Definition.} \begin{enumerate}[(i)]
\item Ein Schema $X$ heißt \textit{quasikompakt}\index{quasikompakt}, wenn sein unterliegender topologischer Raum quasikompakt ist.
\item Ein topologischer Raum $X$ heißt \textit{noethersch}\index{noethersch}, wenn jede absteigende Kette von abgeschlossenen Teilmengen in $X$ stationär wird.
\end{enumerate}

\paragraph{Bemerkung.} Sei $X$ ein noetherscher Raum. Nach Zorns Lemma besitzt jede nichtleere Menge $\Sigma$ von abgeschlossenen Mengen in $X$ ein minimales Element, da jede Kette in $\Sigma$ ein minimales Element besitzt.

\paragraph{Satz 3.4.}\label{3.4} \begin{enumerate}[(i)]
\item Sei $X$ ein topologischer Raum. Dann ist $X$ genau dann noethersch, wenn alle offenen Teilmengen $U\subset X$ quasikompakt sind.
\item Sei $X$ ein affines Schema. Dann ist $X$ quasikompakt, aber nicht notwendig noethersch.
\item Sei $A$ ein noetherscher Ring. Dann ist der $\operatorname{Spec}(A)$ unterliegender Raum noethersch.
\end{enumerate}

\paragraph{Beweis.}\begin{enumerate}[(i)]
\item Sei $U\subset_\text{o}X$ und $U=\bigcup_{i\in I}U_i$ eine offene Überdeckung. Für eine endliche Teilmenge $J\subset I$ setze $V_J=\bigcup_{i\in J}U_i$. Dann ist $V_J\subset X$ offen und es gilt:
\[U=\bigcup_{J\subset I\text{ endlich}}V_J \]
Wählt man aus $\Sigma=\{X\setminus V_J\mid J\subset I\text{ endlich}\}$ ein minimales Element $X\setminus V_{J'}$. Dann gilt $V_{J'}\supset V_J$ für alle $J$. Also ist $U=V_{J'}=\bigcup_{i\in J'}U_i$ eine endliche Teilüberdeckung.

Sei umgekehrt $Y_1\supset Y_2\supset\ldots$ eine Kette abgeschlossener Mengen in $X$, so ist die Menge $U=\bigcup_{j\geq 1}X\setminus Y_j$ offen in $X$. Wir erhalten eine endliche Teilüberdeckung $U=\bigcup_{r\geq j\geq 1}X\setminus Y_j=X\setminus Y_r$, also folgt $Y_s=Y_r$ für alle $s\geq r$.
\item Sei $X=\operatorname{Spec}(A)=\bigcup_{i\in I}U_i$ eine offene Überdeckung. Wir können o.B.d.A. $U_i=D(f_i)$ für gewisse $f_i\in A$ annehmen. Sei $\mathfrak{a}=(f_i\mid i\in I)\subset A$. Dann gilt:
\[X=\bigcup_{i\in I}X\setminus V(f_i) = X\setminus\bigcap_{i\in I}V(f_i)=X\setminus V(\mathfrak{a}) \]
Es folgt $V(\mathfrak{a})=\varnothing$, also $1\in\mathfrak{a}$. Somit gibt es endlich viele $g_j\in A$ und $i_j\in I$ mit $1=\sum_j g_jf_{i_j}$. Wir erhalten die endliche Teilüberdeckung $X=\bigcup_j D(f_{i_j})$.
\item Sei $V(\mathfrak{a}_1)\supset V(\mathfrak{a}_2)\supset\ldots$ eine Kette abgeschlossener Mengen in $\operatorname{Spec}(A)$ mit Ra\-di\-kal\-idealen $\mathfrak{a}_i\subset A$. Sie wird stationär, da die Kette $\mathfrak{a}_1\subset \mathfrak{a}_2\subset\ldots$ stationär wird.\qed
\end{enumerate}

\paragraph{Definition 3.5.}\label{3.5} Sei $X$ ein Schema.
\begin{enumerate}[(i)]
\item $X$ heißt \textit{lokal noethersch}\index{noethersch!lokal}, falls $X$ von offenen, affinen Teilmengen $\operatorname{Spec}(A_i)$ mit noetherschen Ringen $A_i$ überdeckt werden kann.
\item $X$ heißt \textit{noethersch}\index{noethersch}, falls $X$ lokal noethersch und quasikompakt ist. Dies ist äquivalent dazu, dass $X$ von endlich vielen offenen, affinen Teilmengen $\operatorname{Spec}(A_i)$ mit noetherschen Ringen $A_i$ überdeckt werden kann.
\end{enumerate}

\paragraph{Bemerkung.} Ist ein Schema $X$ noethersch, so ist nach \hyperref[3.4]{Satz 3.4 (iii)} der unterliegender Raum von $X$ noethersch. Die Umkehrung gilt im Allgemeinen nicht.

\paragraph{Satz 3.6.}\label{3.6} Sei $X$ ein Schema. Dann ist $X$ genau dann lokal noethersch, wenn für alle offenen, affinen Teilmengen $U=\operatorname{Spec}(A)$ stets $A$ ein noetherscher Ring ist.

\paragraph{Beweis.} Die Rückrichtung ist trivial. Sei also $X$ lokal noethersch und $U=\operatorname{Spec}(A)$ offen in $X$. Wir haben eine offene affine Überdeckung $X=\bigcup_i\operatorname{Spec}(B_i)$ mit noetherschen Ringen $B_i$. Da die offenen Mengen $D(f)$ eine offene Basis der Topologie bilden, haben wir eine Darstellung:
\[U=\bigcup_{i,j} D(f_{ij}),\quad f_{ij}\in B_i,\ D(f_{ij})\subset U\cap \operatorname{Spec}(B_i) \]
mit $D(f_{ij})=\operatorname{Spec}(B_i)_{f_{ij}}$. Da $B_i$ noethersch sind, sind die Lokalisierungen $(B_i)_{f_{ij}}$ ebenfalls noethersche Ringe. Da $U$ affin und somit quasikompakt ist, kann $U$ von endlich vielen Spektren noetherscher Ringe überdeckt werden.

Sei $V=\operatorname{Spec}(B)$ offen in $U$ mit noetherschen Ring $B$. Sei $f\in A$ und betrachte $D(f)\subset V$. Die natürliche Inklusion $V\hookrightarrow U$ induziert einen Ringhomomorphismus $A\to B$. Sei $\overline{f}$ das Bild von $f$ in $B$. Es gilt:
\[\operatorname{Spec}(A_f)=D(f)=D(\overline{f})=\operatorname{Spec}(B_{\overline{f}})\]
Es folgt $A_f\cong B_{\overline{f}}$ und $A_f$ ist noethersch. Wir haben nun gezeigt, dass $U$ von endlich vielen offenen Mengen der Form $D(f)=\operatorname{Spec}(A_f)$ überdeckt werden kann mit noetherschen Ringen $A_f$. Somit folgt die Aussage aus dem nächsten Lemma.\qed

\paragraph{Lemma 1.}\label{3.6-lemma1} Sei $A$ ein Ring und $f_1,\ldots,f_r\in A$ mit $1=(f_1,\ldots,f_r)$. Sind alle $A_{f_i}$ noethersch, so ist auch $A$ noethersch.

\paragraph{Lemma 2.}\label{3.6-lemma2} Sei $A$ ein Ring und $f_1,\ldots,f_r\in A$ mit $1=(f_1,\ldots,f_r)$. Sei $\mathfrak{a}\subset A$ ein Ideal und $\varphi_i:A\to A_{f_i}$ die Lokalisierungsabbildung. Dann gilt:
\[\mathfrak{a} = \bigcap_{i=1}^r \varphi_i^{-1}(\varphi_i(\mathfrak{a}) A_{f_i}) \]

\paragraph{Beweis.} Für die nichttriviale Inklusion sei $b\in A$ mit $\varphi_i(b)\in \varphi(\mathfrak{a})A_{f_i}$ für alle $i$. Schreibe:
\[\varphi_i(b)=\frac{a_i}{f_i^{n_i}}\in A_{f_i},\quad a_i\in\mathfrak{a},\ n_i>0 \]
Sei o.B.d.A. $n=n_1=\ldots=n_r$. Somit gibt es für alle $i$ ein $m_i\geq 0$ mit:
\[f_i^{m_i}(f_i^nb-a_i)=0 \]
Sei o.B.d.A. $m=m_1=\ldots=m_r$. Es folgt $f_i^{m+n}b\in\mathfrak{a}$ für alle $i$. Aus $1=(f_1,\ldots,f_r)$ folgt $1=(f_1^N,\ldots,f_r^N)$ für alle $N\geq 0$, insbesondere für $N=m+n$. Sei also $1=\sum_{i=1}^rc_if_i^N$ für gewisse $c_i\in A$. Dann gilt:
\[b=\sum_{i=1}^rc_if_i^Nb\in\mathfrak{a}\qedhere \]

\paragraph{Beweis von \hyperref[3.6-lemma1]{Lemma 1}.} Sei $\mathfrak{a}_1\subset\mathfrak{a}_2\subset\ldots$ eine Kette von Idealen in $A$. Diese induziert für alle $i$ eine Kette von Idealen in $A_{f_i}$:
\[\varphi_i(\mathfrak{a}_1)A_{f_i}\subset\varphi_i(\mathfrak{a}_2) A_{f_i}\subset\ldots \]
Da $A_{f_i}$ noethersch ist, wird diese Kette stationär für alle $i$. Es existiert also ein $s$ mit $\varphi_i(\mathfrak{a}_s)A_{f_i}=\varphi(\mathfrak{a}_{s+1})A_{f_i}=\ldots$ für alle $i$. Mit \hyperref[3.6-lemma2]{Lemma 2} wird auch die ursprüngliche Kette von Idealen in $A$ stationär.\qed

\paragraph{Definition.} Sei $f:X\to Y$ ein Morphismus von Schemata.
\begin{enumerate}[(i)]
\item $f$ heißt \textit{lokal von endlichem Typ}\index{von endlichem Typ!lokal}, falls $Y$ eine offene affine Überdeckung $\bigcup_i\operatorname{Spec}(B_i)$ besitzt, so dass $f^{-1}(\operatorname{Spec}B_i)$ für alle $i$ eine offene affine Überdeckung $\bigcup_j\operatorname{Spec}(A_{ij})$ besitzt, wobei alle $A_{ij}$ endlich erzeugte $B_i$-Algebren sind.
\item $f$ heißt \textit{von endlichem Typ}\index{von endlichem Typ}, falls $Y$ eine offene affine Überdeckung $\bigcup_i\operatorname{Spec}(B_i)$ besitzt, so dass $f^{-1}(\operatorname{Spec}B_i)$ für alle $i$ eine endliche, offene affine Überdeckung $\bigcup_j\operatorname{Spec}(A_{ij})$ besitzt, wobei alle $A_{ij}$ endlich erzeugte $B_i$-Algebren sind.
\item $f$ heißt \textit{endlich}\index{endlich}, falls eine offene affine Überdeckung $Y=\bigcup_i\operatorname{Spec}(B_i)$ existiert, so dass $f^{-1}(\operatorname{Spec}B_i)=\operatorname{Spec}(A_i)$ für alle $i$ ist, wobei $A_i$ eine $B_i$-Algebra ist, die als $B_i$-Modul endlich erzeugt ist.
\end{enumerate}

\paragraph{Bemerkung 3.7.}\label{3.7} Sei $f:X\to Y$ ein Morphismus von Schemata.
\begin{enumerate}[(i)]
\item $f$ ist genau dann lokal von endlichem Typ, wenn für alle offene, affine $V=\operatorname{Spec}(B)$ in $Y$ es eine offene affine Überdeckung $f^{-1}(V)=\bigcup_j\operatorname{Spec}(A_j)$ gibt, wobei $A_j$ endlich erzeugte $B$-Algebren sind.
\item $f$ ist genau dann von endlichem Typ, wenn für alle offene, affine $V=\operatorname{Spec}(B)$ in $Y$ es eine endliche, offene affine Überdeckung $f^{-1}(V)=\bigcup_j\operatorname{Spec}(A_j)$ gibt, wobei $A_j$ endlich erzeugte $B$-Algebren sind.
\item $f$ ist genau dann endlich, wenn für alle offene, affine $V=\operatorname{Spec}(B)$ in $Y$ die Menge $f^{-1}(V)=\operatorname{Spec}(A)$ affin ist, wobei $A$ ein endlich erzeugter $B$-Modul ist.
\end{enumerate}

\paragraph{Beweis.} Dies werden wir später zeigen.

\paragraph{Beispiel 3.8.}\label{3.8} Sei $V$ eine Varietät über einem algebraisch abgeschlossenen Körper $k$. Dann ist das Schema $t(V)$ in \hyperref[2.26]{Satz 2.26} ein integres, noethersches Schema von endlichem Typ über $k$.

\paragraph{Beispiel 3.9.}\label{3.9} Sei $P$ ein Punkt einer Varietät und $\mathcal{O}_P$ der zugehörige Halm. Dann ist $\operatorname{Spec}(\mathcal{O}_P)$ ein integres, noethersches Schema, aber nicht von endlichem Typ über $k$.

\paragraph{Definition 3.10.}\label{3.10} Ein \textit{offenes Unterschema}\index{Schema!Unterschema!offen} eines Schemas $X$ ist ein Schema $U$, dessen unterliegender topologischer Raum eine offene Teilmenge von $X$ ist und dessen Strukturgarbe $\mathcal{O}_U$ isomorph zu $\mathcal{O}_X|_U$ ist.

Ein Schemamorphismus $f:X\to Y$ heißt \textit{offene Immersion}\index{Immersion!offen}, falls $f$ ein Isomorphismus auf ein offenes Unterschema in $Y$ induziert.

\paragraph{Bemerkung.} Jede offene Teilmenge eines Schemas $X$ trägt eine eindeutig bestimmte Struktur als offenes Unterschema, siehe auch \hyperref[2.10]{Definition 2.10}.

\paragraph{Definition 3.11.}\label{3.11} Ein \textit{abgeschlossenes Unterschema}\index{Schema!Unterschema!abgeschlossen} eines Schemas $X$ ist ein Schema $Y$, zusammen mit einem Morphismus $(i,i^\sharp):Y\to X$, so dass:
\begin{enumerate}[(i)]
\item Der $Y$ unterliegender Raum ist eine abgeschlossene Teilmenge von $X$.
\item $i:Y\hookrightarrow X$ ist die natürliche Inklusion.
\item $i^\sharp:\mathcal{O}_X\to i_\ast\mathcal{O}_Y$ ist surjektiv.
\end{enumerate}

Ein Schemamorphismus $f:X\to Y$ heißt \textit{abgeschlossene Immersion}\index{Immersion!abgeschlossen}, falls $f$ ein Isomorphismus auf ein abgeschlossenes Unterschema in $Y$ induziert.

\paragraph{Beispiel 3.12.}\label{3.12} Sei $A$ ein Ring, $\mathfrak{a}\subset A$ ein Ideal und $X=\operatorname{Spec}(A),\ Y=\operatorname{Spec}(A/\mathfrak{a})$. Der Ringhomomorphismus $\varphi:A\to A/\mathfrak{a}$ induziert $f:Y\to X,\ \mathfrak{p}\mapsto\varphi^{-1}(\mathfrak{p})$. Dies induziert ein Morphismus von Schemata. $f$ ist ein Homöomorphismus von $Y$ auf $V(\mathfrak{a})$ und die Abbildung $f^\sharp:\mathcal{O}_X\to f_\ast\mathcal{O}_Y$ induziert: 
\[f_\mathfrak{p}^\sharp:\mathcal{O}_{X, f(\mathfrak{p})}=A_{\varphi^{-1}(\mathfrak{p})}\to (A/\mathfrak{a})_\mathfrak{p}=\mathcal{O}_{Y,\mathfrak{p}}\]
$f_\mathfrak{p}^\sharp$ ist surjektiv für alle $\mathfrak{p}\in Y$, also ist nach \hyperref[1.10]{1.10 (ii)} auch $f^\sharp$ surjektiv. Somit erhält man für jedes $\mathfrak{a}\subset A$ auf $V(\mathfrak{a})\subset X$ eine Struktur als abgeschlossenes Unterschema in $X$.

\paragraph{Bemerkung.} Ist $Y\subset X=\operatorname{Spec}(A)$ eine abgeschlossene Teilmenge, so existieren auf $Y$ viele abgeschlossene Unterschamestrukturen. Wir werden später sehen, das sie genau den Idealen $\mathfrak{a}\subset A$ mit $Y=V(\mathfrak{a})$ entsprechen.

\paragraph{Satz 3.14.}\label{3.14} Sei $X$ ein Schema und $Y$ eine abgeschlossene Teilmenge von $X$. Dann besitzt $Y$ eine eindeutig bestimmte induzierte Struktur als reduziertes, abgeschlossenes Unterschema.

\iffalse
\paragraph{Beweis.}
\begin{enumerate}
\item Fall: Sei zunächst $X=\operatorname{Spec}(A)$ affin. Betrachte das folgende Ideal in $A$:
\[\mathfrak{a}=\bigcap_{\mathfrak{p}\in Y}\mathfrak{p} \]
Dann ist $V(\mathfrak{a})=Y$ und $\mathfrak{a}$ ist maximal unter allen Idealen $\mathfrak{a}'$ mit $V(\mathfrak{a}')=Y$. Die Unterschemastruktur $Y=V(\mathfrak{a})=\operatorname{Spec}(A/\mathfrak{a})$ auf $Y$ ist reduziert, da $A/\mathfrak{a}$ reduziert ist, wegen $\mathfrak{a}=\operatorname{Rad}(\mathfrak{a})$.

Für die Eindeutigkeit sei $Y'=V(\mathfrak{a}')=\operatorname{Spec}(A/ \mathfrak{a}')$ eine weitere abgeschlossene Unterschemastruktur. Dann ist $\mathfrak{a}'\subset\mathfrak{a}$. Die kanonische Abbildung $A/\mathfrak{a}'\to A/\mathfrak{a}$ induziert ein Morphismus $Y\to Y'$. Da $Y$ reduziert ist, gibt es einen eindeutigen Morphismus $Y\to Y'_\text{red}$, so dass folgendes Diagramm kommutiert:
\[\xymatrix{
Y\ar[rr]\ar@{-->}[rrd] && Y'\\
&& Y'_\text{red}\ar[u]
} \]
\end{enumerate}
\fi

\paragraph{Lemma 3.15.}\label{3.15} Sei $X$ ein topologischer Raum und $X=\bigcup_i U_i$ eine offene Überdeckung. Ferner sei für jedes $i$ eine Garbe $F_i$ auf $U_i$ gegeben und für alle $i,j$ seien Isomorphismen gegeben:
\[\varphi_{ij}:F_i|_{U_i\cap U_j}\stackrel{\sim}{\to} F_j|_{U_i\cap U_j} \]
so dass $\varphi_{ii}=\operatorname{id}$ und $\varphi_{ik}=\varphi_{jk}\varphi_{ij}$ auf $U_i\cap U_j\cap U_k$ gilt. Dann existiert eine eindeutig bestimmte Garbe $F$ auf $X$ und Isomorphismen $\psi_i:F|_{U_i}\stackrel{\sim}{\to}F_i$ mit $\psi_j=\varphi_{ij}\psi_i$ auf $U_i\cap U_j$. Wir sagen auch, dass $F$ durch \textit{Verkleben} der $F_i$ längst $\varphi_{ij}$ entsteht.\index{Verklebung!Garbe}

\paragraph{Beweis.} Folgt direkt aus der Definition einer Garbe.\qed

\paragraph{Definition 3.16.}\label{3.16} \begin{enumerate}[(i)]
\item Die \textit{Dimension} $\dim(X)$ eines Schemas $X$ ist die Dimension von $X$ als topologischer Raum, d.h:\index{Dimension}
\begin{align*}
\dim(X)=\sup\{n\mid&\text{ es gibt eine Kette }Z_0\subsetneq Z_1\subsetneq\ldots\subsetneq Z_n\\
&\text{ von irreduziblen, abgeschlossenen Teilmengen in }X \} 
\end{align*}
\item Sei $Z$ eine irreduzible, abgeschlossene Teilmenge eines Schemas $X$. Dann ist die \textit{Kodimension}\index{Kodimension} $\operatorname{codim}(Z,X)$ von $Z$ in $X$ definiert als:
\begin{align*}
\operatorname{codim}(Z,X)=\sup\{n\mid & \text{ es gibt eine Kette }Z=Z_0\subsetneq Z_1\subsetneq\ldots\subsetneq Z_n\\
&\text{ von irreduziblen, abgeschlossenen Teilmengen in }X\}
\end{align*}
Ist $Y$ eine abgeschlossene Teilmenge von $X$, so setzen wir:
\[\operatorname{codim}(Y,X)=\inf \{\operatorname{codim}(Z,X)\mid Z\subset Y\text{ irreduzibel, abgeschlossen} \} \]
\end{enumerate}

\paragraph{Definition 3.17.}\label{3.17} Sei $S$ ein Schema und $X,Y$ $S$-Schemata. Das \textit{Faserprodukt}\index{Faserprodukt} $X\times_SY$ von $X$ und $Y$ über $S$ ist ein Schema, zusammen mit Projektionsmorphismen $p_1:X\times_SY\to X$ und $p_2:X\times_SY\to Y$ derart, dass:
\begin{enumerate}[(i)]
\item Das folgende Diagramm kommutiert:
\[ \begin{tikzcd}
X\times_S Y\ar[r, "p_2"]\ar[d, "p_1"'] & Y\ar[d]\\
X\ar[r] & S
\end{tikzcd}\]
\item Ist $Z$ ein $S$-Schema und $f:Z\to X,\ g:Z\to Y$ Morphismen derart, dass das folgende Diagramm kommutiert:
\[\begin{tikzcd}
Z\ar[rrd, bend left, "g"]\ar[ddr, "f"', bend right]\ar[rd, "\theta", dashed] &&\\
&X\times_S Y\ar[r, "p_2"']\ar[d, "p_1"] & Y\ar[d]\\
&X\ar[r] & S
\end{tikzcd} \]
so existiert einen eindeutig bestimmten Morphismus $\theta:Z\to X\times_SY$ mit $f=p_1\theta$ und $g=p_2\theta$.
\end{enumerate}
Wird für Schemata $X$ und $Y$ kein Bezug zu einer Basis angegeben, so ist immer das Endobjekt $S=\operatorname{Spec}(\mathbb{Z})$ gemeint, d.h. $X\times Y=X\times_{\operatorname{Spec}(\mathbb{Z})}Y$.

\paragraph{Theorem 3.18.}\label{3.18} Seien $X$ und $Y$ $S$-Schemata. Dann existiert das Faserprodukt $X\times_S Y$ und ist auf Isomorphie eindeutig.

\paragraph{Lemma 3.19.}\label{3.19} \textit{(Verkleben von Morphismen, vgl. \hyperref[2.13]{Satz 2.13})}\index{Verklebung!Morphismus} Seien $X,Y$ Schemata und $\{U_i\}$ eine offene Überdeckung von $X$. Ferner seien $f_i:U_i\to Y$ Morphismen gegeben, wobei $U_i$ mit der offenen Unterschemastruktur versehen ist. Es gelte $f_i|_{U_i\cap U_j}=f_j|_{U_i\cap U_j}$ für alle $i,j$. Dann gibt es einen Morphismus $f:X\to Y$ mit $f|_{U_i}=f_i$ für alle $i$.

\paragraph{Beweis von \hyperref[3.18]{Theorem 3.18}.} Die Eindeutigkeit ist klar.
\begin{enumerate}
\item Schritt: Seien $X=\operatorname{Spec}(A),\ Y=\operatorname{Spec}(B),\ S=\operatorname{Spec}(R)$ affin. Somit sind $A$ und $B$ $R$-Algebren. Wir zeigen $X\times_S Y=\operatorname{Spec}(A\otimes_R B)$. Die Projektionsabbildung $p_1:\operatorname{Spec}(A\otimes_RB)\to \operatorname{Spec}(A)$ ist durch die natürliche Abbildung $\tilde{p}_1:A\to A\otimes_RB$ gegeben, analog für $p_2$. Offensichtlich kommutiert:
\[\begin{tikzcd}
 A\ar[r, "\tilde{p}_1"] & A\otimes_RB\\
 R\ar[r]\ar[u] & B\ar[u, "\tilde{p}_2"']
\end{tikzcd} \]
Sei also $Z$ ein $S$-Schema und Morphismen $f:Z\to X,\ g:Z\to Y$ gegeben, die über $S$ gleich sind. Diese entsprechen Ringhomomorphismen $\tilde{f}:A\to\Gamma(Z,\mathcal{O}_Z)$ und $\tilde{g}:B\to\Gamma(Z,\mathcal{O}_Z)$ nach \hyperref[2.13]{Satz 2.13}. Es kommutiert:
\[\begin{tikzcd}
&&\Gamma(Z,\mathcal{O}_Z)\\
A\ar[r, "\tilde{p}_2"']\ar[rru, "\tilde{f}", bend left]& A\otimes_RB\ar[ru, "\tilde{\theta}"', dashed] &\\
R\ar[r]\ar[u] & B\ar[u, "\tilde{p}_2"]\ar[ruu, "\tilde{g}"', bend right]
\end{tikzcd} \]
Wegen der Universaleigenschaft des Tensorprodukts gibt es genau einen Ringhomomorphismus $\tilde{\theta}:A\otimes_RB\to\Gamma(Z,\mathcal{O}_Z)$ mit $\tilde{f}\tilde{\theta}=\tilde{p}_1$ und $\tilde{g}\tilde{\theta}=\tilde{p}_2$. \hyperref[2.13]{Satz 2.13} liefert ein eindeutiges $\theta:Z\to\operatorname{Spec}(A\otimes_RB)$ mit $f=p_1\theta$ und $g=p_2\theta$.
\item Schritt: Seien $X,Y$ beliebige $S$-Schemata und $U\subset_\text{o}X$. Wir nehmen an, dass das Faserprodukt $X\times_SY$ mit Projektionen $p_1,p_2$ existiert. Wir zeigen, dass für die offene Teilmenge $p_1^{-1}(U)\subset X\times_SY$ stets $p_1^{-1}(U)=U\times_SY$ gilt.

Da $p_1^{-1}(U)\subset X\times_SY$, kommutiert das Diagramm:
\[\begin{tikzcd}
p_1^{-1}(U)\ar[r, "p_2"]\ar[d, "p_1"'] & Y\ar[d]\\
U\ar[r] & S
\end{tikzcd} \]
Sei $Z$ ein $S$-Schema und Morphismen $f:Z\to U,\ g:Z\to Y$ gegeben, so dass $(Z\stackrel{f}{\to} U\stackrel{i}{\hookrightarrow} X\to S)=(Z\stackrel{g}{\to} Y\to S)$. Nach der Universaleigenschaft von $X\times_SY$ existiert ein eindeutiges $\theta:Z\to X\times_SY$ mit $if=p_1\theta$ und $g=p_2\theta$. Insbesondere gilt $\theta(Z)\subset p_1^{-1}(U)$, also $\theta:Z\to p_1^{-1}(U)$. Somit erfüllt $p_1^{-1}(U)$ die Universaleigenschaft von $U\times_SY$.
\item Schritt: Seien $X,Y$ $S$-Schemata und $\{X_i\}$ eine offene Überdeckung von $X$. Wir nehmen an, dass alle Faserprodukte $X_i\times_SY$ mit Projektionen $p_{1i},p_{2i}$ existieren. Wir zeigen, dass in diesem Fall auch $X\times_SY$ existiert.

Setze $X_{ij}=X_i\cap X_j$ und $U_{ij}=p_{1i}^{-1}(X_{ij})\subset X_i\times_SY$. Nach Schritt 2 folgt nun $U_{ij}=X_{ij}\times_SY$. Wegen Eindeutigkeit existieren nun Isomorphismen $\varphi_{ij}:U_{ij}\stackrel{\sim}{\to}U_{ji}$ mit $\varphi_{ij}=\varphi_{ji}^{-1}$, $\varphi_{ij}(U_{ij}\cap U_{ik})=U_{ji}\cap U_{jk}$ und $\varphi_{ik}=\varphi_{jk}\circ\varphi_{ij}$ auf $U_{ij}\cap U_{jk}$. Mithilfe \hyperref[2.11]{2.11} verkleben wir die $X_i\times_SY$ via $\varphi_{ij}$ und erhalten so ein Schema $Z$ mit Morphismen $\psi_j:X_j\times_SY\to Z$, die Isomorphismen auf einem offenen Unterschema induzieren. Seien $p_1,p_2$ die Morphismen, die durch Verkleben der $p_{1i}$ bzw. $p_{2i}$ entstehen, siehe \hyperref[3.19]{Lemma 3.19}. Wir zeigen nun, dass $Z$ gerade das Faserprodukt $X\times_SY$ mit Projektionsmorphismen $p_1,p_2$ ist.

Es gilt $Z=\bigcup_j\psi_j(X_j\times_SY)$. Also folgt die Kommutativität des zweiten Diagramms aus dem ersten:
\[\begin{tikzcd}
X_j\times_S Y\ar[r, "p_{1j}"]\ar[d, "p_{2j}"'] & X_j\ar[d]\\
Y\ar[r] & S
\end{tikzcd}\qquad\qquad \begin{tikzcd}
Z\ar[d, "p_2"']\ar[r, "p_1"] & X\ar[d]\\
Y\ar[r] & S
\end{tikzcd}\]
Sei nun $Z'$ ein weiteres $S$-Schema und $f:Z'\to X,\ g:Z'\to Y$ gegeben, die über $S$ gleich sind. Setze $Z_i'=f^{-1}(X_i)$ für alle $i$. Zu jedem $i$ existiert genau ein Morphismus ${\theta}_i:Z_i'\to X_i\times_SY\hookrightarrow Z$ mit $f|_{Z_i'}=p_{1i}\circ{\theta}_i$ und $g|_{Z_i'}=p_{2i}\circ{\theta}_i$. Es kommutiert:
\[\begin{tikzcd}
X_i\times_SY\ar[r, hook]\ar[d, "p_{1i}"'] & Z\ar[d, "p_1"]\\
X_i\ar[r, hook] & X
\end{tikzcd} \]
Es gilt $Z_i'\cap Z_j'=f^{-1}(X_i\cap X_j)=f^{-1}(X_{ij})$ und daher $f|_{Z'_i\cap Z'_j}=p_{1i}\circ\theta_i|_{Z'_i\cap Z'_j}=p_{1j}\circ\theta_j|_{Z_i'\cap Z_j'}$, entsprechend für $g$. Wegen Eindeutigkeit folgt $\theta_i|_{Z_i'\cap Z_j'}=\theta_j|_{Z'_i\cap Z'_j}$. Daher können wir die $\theta_i$ zu einem Morphismus $\theta:Z'\to Z$ verkleben mit $f=p_1\theta$ und $g=p_2\theta$. $\theta$ ist eindeutig, da $\theta|_{Z_i'}=\theta_i$ und alle $\theta_i$ eindeutig sind.
\item Schritt: Seien $X,Y$ $S$-Schemata und $S$ affin. Wir zeigen, dass $X\times_SY$ existiert.

Seien $X=\bigcup_iX_i$ und $Y=\bigcup_jY_j$ offene affine Überdeckungen. Nach Schritt 1 existieren $X_i\times_S Y_j$ für alle $i,j$. Nach Schritt 3 existieren $X\times_SY_j$ für alle $j$. Wegen Symmetrie, existiert somit $ X\times_SY$ nach Schritt 3.
\item Schritt: Seien $X,Y$ $S$-Schemata mit $S$ beliebig. Wir zeigen, dass $X\times_SY$ existiert.

Seien $q:X\to S$ und $r:Y\to S$ die Strukturmorphismen und $S=\bigcup_iS_i$ eine offene affine Überdeckung. Setze $X_i=q^{-1}(S_i)$ und $Y_i=r^{-1}(S_i)$. Nach Schritt 4 existieren $X_i\times_{S_i}Y_i$. Wir zeigen, dass $X_i\times_{S_i}Y_i$ die Universaleigenschaft von $X_i\times_SY$ erfüllt.

Seien $f:Z\to X_i$ und $g:Z\to Y$ gegeben, die über $S$ gleich sind. Dann gilt $rg(Z)=qf(Z)\subset q(X_i)\subset S_i$, also $g(Z)\subset Y_i$. Wir erhalten kommutatives Diagramm:
\[\begin{tikzcd}
Z\ar[ddr, "f"', bend right]\ar[rrd, "g"', bend left]\ar[rrrd, "g", bend left]\ar[rd, "\theta"', dashed]\\
& X_i\times_{S_i}Y_i\ar[r, "p_2"']\ar[d, "p_1"] & Y_i\ar[r, hook]\ar[d] & Y\ar[dd, "r"]\\
& X_i\ar[d, hook]\ar[r] & S_i\ar[dr, hook] &\\
& X\ar[rr]_q & & S
\end{tikzcd} \]
Es existiert genau ein $\theta:Z\to X_i\times_{S_i} Y_i$ mit $f=p_1\theta$ und $g=p_2\theta$. Somit existieren auch $X_i\times_SY$. Nach Schritt 3 existiert auch $X\times_SY$.\qed
\end{enumerate}

\paragraph{Lemma 3.20.}\label{3.20} Seien $X,Y$ $S$-Schemata mit Strukturmorphismen $\xi:X\to S,\ \eta:Y\to S$ und $U,V,W$ offen in $X,Y$ bzw. $S$, so dass $\xi(U)\subset W$ und $\eta(V)\subset W$. Ferner seien $p_1,p_2$ die Projektionsmorphismen von $X\times_SY$. Dann gibt es einen $S$-Schemaisomorphismus:
\[p_1^{-1}(U)\cap p_2^{-1}(V)\cong U\times_WV=U\times_SV \]

\paragraph{Beweis.} $E=p_1^{-1}(U)\cap p_2^{-1}(V)$ ist offenes Unterschema von $X\times_SY$. Die Projektionsmorphismen von $X\times_SY$ induzieren $p_1:E\to U,\ p_2:E\to V$. Sei $Z$ ein $W$-Schema und Morphismen $\varphi:Z\to U,\ \psi:Z\to V$ gegeben, die über $W$ gleich sind. Wir erhalten Morphismen $\varphi':Z\to U\hookrightarrow X,\ \psi':Z\to V\hookrightarrow Y$, die über $S$ gleich sind. Es existiert einen eindeutigen Morphismus $\theta:Z\to X\times_SY$ mit $\varphi'=p_1\theta$ und $\psi'=p_2\theta$. Nun gilt $\theta(Z)\subset E$, da $p_1\theta(Z)=\varphi'(Z)\subset U$ und $p_2\theta(Z)=\psi'(Z)\subset V$. Somit ist $\theta:Z\to E$ mit $\varphi=p_1\theta,\ \psi=p_2\theta$. Es folgt $E\cong U\times_WV$. Da $W$ beliebig war, folgt auch $E\cong U\times_SV$.\qed 

\paragraph{Definition.} Ein \textit{Unterschema}\index{Schema!Unterschema} $Y$ eines Schemas $X$ ist ein abgeschlossenes Unterschema eines offenen Unterschemas von $X$. Mit anderen Worten haben wir eine abgeschlossene Immersion $\phi$ und eine offene Immersion $\psi$:
\[Y\stackrel{\phi}{\hookrightarrow} U\stackrel{\psi}{\hookrightarrow}X \]
Ein Morphismus $i:Y\to X$ heißt \textit{Immersion}\index{Immersion}, wenn $i$ einen Isomorphismus von $Y$ auf ein Unterschema in $X$ induziert.

\paragraph{Satz 3.21.}\label{3.21} Seien $X,Y,Z$ $S$-Schemata und $W$ ein $Z$-Schema. Dann gilt:
\begin{enumerate}[(i)]
\item $X\times_SS\cong X$
\item $X\times_SY\cong Y\times_SX$
\item $(X\times_SY)\times_SZ\cong X\times_S(Y\times_SZ)$
\item $(X\times_SZ)\times_ZW\cong X\times_SW$
\item $(X\times_SY)\times_SZ\cong (X\times_SZ)\times_Z(Y\times_SZ)$
\item Ist $\sigma:S\to T$ eine Immersion, so ist $X\times_SY\cong X\times_TY$.
\end{enumerate}

\paragraph{Beweis.} (i) bis (v) folgen direkt aus der Universaleigenschaft des Faserprodukts. Für (vi) seien $\varphi:Z\to X,\ \psi:Z\to Y$ Morphismen über $T$ und $\xi:X\to S,\ \eta:Y\to S$ die Strukturmorphismen. Wir erhalten folgendes kommutative Diagramm:
\[ \begin{tikzcd}
Z\ar[rrd, "\varphi", bend left]\ar[rd, "\theta"', dashed]\ar[ddr, "\psi"', bend right]\\
& X\ar[r]\ar[d]\times_SY & X\ar[d, "\xi"']\ar[ddr, bend left] &\\
& Y\ar[r, "\eta"]\ar[rrd, bend right] & S\ar[dr, "\sigma"]&\\
&&&T
\end{tikzcd}\]
Wegen der Injektivität von $\sigma$, folgt aus $\sigma\xi\varphi=\sigma\eta\psi$ stets $\xi\varphi=\eta\psi$. Es existiert ein eindeutiges, kompatibles $\theta:Z\to X\times_SY$. Somit erfüllt $X\times_SY$ die Universaleigenschaft von $X\times_TY$.\qed

\paragraph{Definition 3.22.}\label{3.22} Seien $f:X_1\to X_2$ und $g:Y_1\to Y_2$ $S$-Morphismen mit kommutativen Diagramm:
\[\begin{tikzcd}
&X_1 \ar[r, "f"]& X_2\\
X_1\times_SY_1\ar[dr, "p_2"']\ar[ur, "p_1"]\ar[rrr, "f\times_Sg", dashed] &&& X_2\times_SY_2\ar[lu, "q_1"']\ar[ld, "q_2"]\\
& Y_1\ar[r, "g"']&Y_2
\end{tikzcd} \]
Es existiert genau ein Morphismus $f\times_Sg: X_1\times_SY_1\to X_2\times_SY_2$ mit $fp_1=q_1(f\times_Sg)$ und $gp_2=q_2(f\times_Sg)$.

\paragraph{Definition 3.23.}\label{3.23} Sei $f:X\to S$ ein Morphismus von Schemata. Für ein Morphismus $g:T\to S$ erhält man ein $T$-Schema $X_T=X\times_ST$ mit:
\[\begin{tikzcd}
X_T\ar[r]\ar[d] & T\ar[d, "g"]\\
X\ar[r, "f"'] & S
\end{tikzcd} \]
Man sagt, dass $X_T$ durch \textit{Basiswechsel} von $X$ über $S$ durch $T$ erhalten bleibt.\index{Basiswechsel}

\paragraph{Lemma 3.24.}\label{3.24} Sei $U\subset S$ ein offenes Unterschema. Dann gilt für den Basiswechsel $X_U$ von $f:X\to S$:
\[X_U=X\times_SU\cong f^{-1}(U) \]

\paragraph{Beweis.} Nach \hyperref[3.21]{Satz 3.21 (i)} ist die Projektion $p_1:X\times_SS\stackrel{\sim}{\to} X$ ein Isomorphismus. Sei in \hyperref[3.20]{Lemma 3.20} $Y=S$ und setze $f=p_2:X\to S$. Wir erhalten: \[f^{-1}(U)=p_1^{-1}(U)\cap p_2^{-1}(U)\cong X\times_SU\qedhere\]

\paragraph{Definition.} Sei $\mathcal{P}$ eine Eigenschaft eines Morphismus' $f:X\to S$. Man sagt, dass $\mathcal{P}$ bei Basiswechsel \textit{erhalten} bleibt, wenn $f_T:X_T\to T$ die Eigenschaft $\mathcal{P}$ besitzt für alle Morphismen $g:T\to S$.

\paragraph{Bemerkung.} Hat $f:X \to S$ eine Eigenschaft $\mathcal{P}$, die stabil unter Basiswechsel ist, so hat insbesondere $f|_{f^{-1}(U)}:f^{-1}(U)\to U$ für alle offenen $U\subset S$ die Eigenschaft $\mathcal{P}$.

\paragraph{Satz 3.25.}\label{3.25}\begin{enumerate}[(i)]
\item Die Eigenschaft (abgeschlossene, offene) Immersion zu sein bleibt bei Basiswechsel erhalten.
\item Die Eigenschaft von endlichem Typ zu sein bleibt stabil unter Basiswechsel.
\end{enumerate}

\paragraph{Beweis.} Wird ausgelassen.

\paragraph{Bemerkung.} Irreduzibilität, Reduziertheit und Integrität bleiben unter Basiswechsel nicht notwendig erhalten.

\paragraph{Lemma 3.26.}\label{3.26} Sei $\psi:A\to B$ ein Ringhomomorphismus, so dass $\alpha(\psi):\operatorname{Spec}(B)\to\operatorname{Spec}(A)$ eine abgeschlossene Immersion ist. Sei ferner $C$ ein beliebiger Ring. Dann ist $\alpha(\psi\otimes\operatorname{id}_C):\operatorname{Spec}(B\otimes_AC)\to\operatorname{Spec}(A\otimes_AC)=\operatorname{Spec}(C)$ eine abgeschlossene Immersion, wobei $\alpha$ die Abbildung aus \hyperref[2.13]{Satz 2.13} bezeichnet.

\paragraph{Beweis.} Da $\tilde{\psi}=\alpha(\psi)$ eine abgeschlossene Immersion ist, ist $\tilde{\psi}^\sharp:\mathcal{O}_A\to\tilde{\psi}_\ast\mathcal{O}_B$ surjektiv. $\tilde{\psi}$ ist Homöomorphismus auf eine abgeschlossene Teilmenge, daher ist $(\tilde{\psi}_\ast\mathcal{O}_B)_{\tilde{\psi}(\mathfrak{p})}=\mathcal{O}_{B,\mathfrak{p}}$ für alle $\mathfrak{p}\in\operatorname{Spec}(B)$. Nach \hyperref[3.12]{Beispiel 3.12} ist die auf den Halmen induzierte Abbildung $\tilde{\psi}^\sharp_\mathfrak{p}$ surjektiv:
\[\tilde{\psi}^\sharp_\mathfrak{p} : A_{\psi^{-1}(\mathfrak{p})}=\mathcal{O}_{A,\tilde{\psi}(\mathfrak{p})}\to (\tilde{\psi}_\ast\mathcal{O}_B)_{\tilde{\psi}(\mathfrak{p})}=B_\mathfrak{p} \]
Betrachte nun die exakte $A$-Modulsequenz $A\stackrel{\psi}{\to}B\to B/\psi(A)\to 0$. Diese induziert unter $-\otimes_AA_\mathfrak{q}$ für alle $\mathfrak{q}\in\operatorname{Spec}(A)$ die exakte Folge:
\[A_\mathfrak{q}\to B\otimes_A A_\mathfrak{q}\to B/\psi(A)\otimes_A A_\mathfrak{q}\to 0 \]
Sei das Bild von $\tilde{\psi}$ von der Form $V(\mathfrak{a})\cong\operatorname{Spec}(B)$ für ein Ideal $\mathfrak{a}\subset A$. Für ein Primideal $\mathfrak{q}\in\operatorname{Spec}(A)$ gilt:
\begin{align*}
B\otimes_A A_\mathfrak{q}\neq 0 &\iff \mathfrak{q}\in\operatorname{Supp}(B)\text{, wobei $B$ als $A$-Modul aufgefasst wird}\\
&\iff \mathfrak{q}\supset\mathfrak{a}\\
&\iff \mathfrak{q}\in\operatorname{im}(\tilde{\psi})\\
&\iff\mathfrak{q}=\psi^{-1}(\mathfrak{p})\text{ für ein }\mathfrak{p}\in\operatorname{Spec}(B)\\
&\implies B/\psi(A)\otimes_AA_\mathfrak{q}=0
\end{align*}
Somit ist $A_\mathfrak{q}\to B\otimes A_\mathfrak{q}$ surjektiv für alle $\mathfrak{q}\in\operatorname{Spec}(A)$. Es folgt die Surjektivität von $\psi$. Wir erhalten unter $-\otimes_AC$ die surjektive Abbildung $C\to B\otimes_AC$. Nach \hyperref[3.12]{Beispiel 3.12} folgt die Behauptung.\qed

\paragraph{Lemma 3.27.}\label{3.27} Sei $f:X\to Y$ ein Morphismus und $Y=\bigcup_\lambda Y_\lambda$ eine offene Überdeckung. Dann ist $f$ genau dann eine (abgeschlossene, offene) Immersion, wenn $f_\lambda=f|_{X_\lambda}:X_\lambda \to Y_\lambda$ mit $X_\lambda=f^{-1}(Y_\lambda)$ die Eigenschaft hat.

\paragraph{Beweis.} Kein Beweis.

\paragraph{Bemerkung 3.28.}\label{3.28} Sind $Y_1,Y_2$ Unterschemata von $X$, so ist $Y_1\times_XY_2\to X$ ein Unterschema von $X$, d.h. eine Immersion.

\paragraph{Definition 3.29.}\label{3.29} Sei $f:X\to Y$ ein $S$-Morphismus von $S$-Schemata. Der Morphismus $\Gamma_f=(\operatorname{id}_X,f)_S:X\to X\times_SY$ heißt \textit{$S$-Graph}\index{Graph} von $f$ und $\Gamma_f(X)$ heißt \textit{Graph} von $f$. Ist $f=\operatorname{id}_X$ die Identität, so heißt $\Delta_{X/S}=\Gamma_{\operatorname{id}_X}$ der \textit{Diagonalmorphismus}\index{Diagonalmorphismus}.

\paragraph{Satz 3.30.}\label{3.30} Sei $f:X\to Y$ ein $S$-Morphismus von $S$-Schemata. \begin{enumerate}[(i)]
\item Sind $Y$ und $S$ affine Schemata, so ist $\Gamma_f$ eine abgeschlossene Immersion. Insbesondere ist für $X,S$ affin $\Delta_{X/S}$ eine abgeschlossene Immersion.
\item $\Gamma_f$ ist allgemein eine Immersion.
\end{enumerate}

\paragraph{Beweis.} \begin{enumerate}[(i)]
\item Sei $X=\bigcup_\alpha X_\alpha$ eine offene affine Überdeckung und $X\times_SY$ das Faserprodukt. Nach \hyperref[3.24]{Lemma 3.24} ist $p_1^{-1}(X_\alpha)\cong X_\alpha\times_SY$ affin. Da $p_1\circ\Gamma_f=\operatorname{id}_X$, gilt $\Gamma_f^{-1}(p_1^{-1}(X_\alpha))=X_\alpha$. Da $\Gamma_f: \bigcup_\alpha\Gamma_f^{-1}(X_\alpha\times_SY) \to \bigcup_\alpha(X_\alpha\times_SY)=X\times_SY$, können wir nach \hyperref[3.27]{Lemma 3.27} o.B.d.A. annehmen, dass $X$ affin ist.

Sei also $S=\operatorname{Spec}(R),\ X=\operatorname{Spec}(B)$ und $Y=\operatorname{Spec}(A)$. Sei $f$ induziert von $\varphi:A\to B$ und $\Gamma_f$ induziert von $\psi:B\otimes_RA\to A,\ b\otimes a\mapsto b\varphi(a)$. $\psi$ ist offensichtlich surjektiv, daher ist $\Gamma_f$ nach \hyperref[2.19]{Satz 2.19 (ii)} eine abgeschlossene Immersion.
\item Sei zunächst $S$ affin und $Y=\bigcup_\alpha Y_\alpha$ eine offene affine Überdeckung. Setze $X_\alpha=f^{-1}(Y_\alpha)$, $\Gamma_{f\alpha}=\Gamma_f|_{X_\alpha}$ und $f_\alpha=f|_{X_\alpha}:X_\alpha\to Y_\alpha$. Es ist $\Gamma_f(X_\alpha)\subset X_\alpha\times_SY_\alpha$ nach (i) ein abgeschlossenes Unterschema und $X_\alpha\times_SY_\alpha\subset X\times_SY_\alpha$ nach \hyperref[3.25]{Satz 3.25 (i)} ein offenes Unterschema ist. Nach (i) ist $\Gamma_{f\alpha}:X_\alpha\to X_\alpha\times_SY_\alpha$ eine Immersion. Nach \hyperref[3.27]{Lemma 3.27} ist auch $\Gamma_f$ eine Immersion.

Sei nun $S$ beliebig und $S=\bigcup_\lambda S_\lambda$ eine offene affine Überdeckung. Nach \hyperref[3.21]{Satz 3.21 (v)} ist $(X\times_SY)\times_SS_\lambda=X_\lambda\times_{S_\lambda}Y_\lambda$, wobei $X_\lambda=X\times_SS_\lambda$ und $Y_\lambda=Y\times_SS_\lambda$. Ferner ist $\Gamma_{f_\lambda}$ der Graph von $f_\lambda:X_\lambda\to Y_\lambda$ eine Immersion. Nach \hyperref[3.27]{Lemma 3.27} ist auch $\Gamma_f$ eine Immersion.\qed
\end{enumerate}

\paragraph{Definition 3.31.}\label{3.31} Sei $f:X\to Y$ ein Morphismus von Schemata und $y\in Y$ ein Punkt mit Restklassenkörper $\kappa(y)=\mathcal{O}_{Y,y}/\mathfrak{m}_{Y,y}$. Sei weiter $i:\operatorname{Spec}\kappa(y)\to Y$ der natürliche Morphismus gegeben durch $(y,\operatorname{id}_{\kappa(y)})$ (siehe \hyperref[2.18]{Satz 2.18}). Dann heißt das Faserprodukt $X_y=X\times_Y\operatorname{Spec}\kappa(y)$ die \textit{Faser}\index{Faser} von $f$ über dem Punkt $y$.
\[\begin{tikzcd}
X_y\ar[r, "p_2"]\ar[d, "p_1"'] & \operatorname{Spec}\kappa(y)\ar[d, "i"]\\
X\ar[r, "f"'] & Y
\end{tikzcd}\]

\paragraph{Satz.} $X_y$ ist ein $\kappa(y)$-Schema mit $f^{-1}(y)$ als unterliegender topologischer Raum.

\paragraph{Beweis.} Es ist $fp_1(X_y)=ip_2(X_y)=y$, also folgt $p_1(X_y)\subset f^{-1}(y)$ und somit $X_y=p_1^{-1}(f^{-1}(y))$. Sei $V\subset_\text{o}X$ affin mit $V\cap f^{-1}(y)\neq\varnothing$. Dann gilt:
\[p_1^{-1}(V)=V_y\cong X\times_Y\operatorname{Spec}\kappa(y)\times_XV=X_y\times_XV \]
Wir können somit o.B.d.A. $X$ und $Y$ als affin annehmen, sei also $X=\operatorname{Spec}(B),\ Y=\operatorname{Spec}(A),\ y=\mathfrak{p}$. Sei $\varphi:A\to B$ assoziiert zu $f$. Setze $\varphi':A_\mathfrak{p}\to B'=B\otimes_AA_\mathfrak{p}=\varphi(A\setminus\mathfrak{p})^{-1}B$. Es gibt eine Folge von Homöomorphismen:
\begin{align*}
f^{-1}(y) &= \{\mathfrak{q}\in\operatorname{Spec}(B)\mid\varphi^{-1}(\mathfrak{q})=\mathfrak{p} \}\\
&\cong \{\mathfrak{q}\in\operatorname{Spec}(B')\mid\varphi'^{-1}(\mathfrak{q})=\mathfrak{p}A_\mathfrak{p} \}\\
&\cong \{ \mathfrak{q}\in\operatorname{Spec}(B')\mid\mathfrak{q}\supset (\mathfrak{p}A_\mathfrak{p})B' \} \\
&=\operatorname{Spec}(B'/(\mathfrak{p}A_\mathfrak{p})B')\\
&=\operatorname{Spec}(B'\otimes_{A_\mathfrak{p}}A_\mathfrak{p}/\mathfrak{p}A_\mathfrak{p})\\
&=\operatorname{Spec}((B\otimes_AA_\mathfrak{p})\otimes_{A_\mathfrak{p}}A_\mathfrak{p}/\mathfrak{p}A_\mathfrak{p}) =\operatorname{Spec}(B\otimes_A\kappa(\mathfrak{p}))\qedhere
\end{align*}

\paragraph{Definition 3.32.}\label{3.32} Sei $f:X\to Y$ ein Morphismus von Schemata, so dass $f(X)\subset Y$ dicht liegt, und $Y$ irreduzibel mit generischer Punkt $\xi$ mit $\xi\in f(X)$. Dann heißt $f^{-1}(\xi)$ \textit{generische Faser}\index{generische Faser} von $f$.

\paragraph{Beispiel 3.33.}\label{3.33} Sei $k$ ein algebraisch abgeschlossener Körper und:
\[X=\operatorname{Spec}k[X,Y,t]/(tY-X^2),\quad Y=\operatorname{Spec}k[T] \]
Sei $f:X\to Y$ gegeben durch die kanonische Abbildung $\varphi:k[t]\to k[X,Y,t]/(tY-X^2)$. $X$ und $Y$ sind integre Schemata von endlichem Typ über $k$. Ferner ist $f$ surjektiv, da $\varphi$ injektiv ist (vgl. \hyperref[2.19]{Satz 2.19 (ii)}). Abgeschlossene Punkte von $Y$ entsprechen $k$. Sei $a\in Y$ ein abgeschlossener Punkt und betrachte die Faser: \[X_a=X\times_Y\operatorname{Spec}\kappa(a)=\operatorname{Spec}k[X,Y]/(aY-X^2)\]
Im Fall $a\neq 0$ ist $X_a$ eine ebene Kurve in $\mathbf{A}_k^2$, die irreduzibel und reduziert ist. Für $a=0$ ist $X_0=\operatorname{Spec}k[X,Y]/(X^2)$ die $Y$-Achse. Diese ist nicht reduziert.

\section{Separierte \& eigentliche Morphismen}

\paragraph{Definition 4.1.}\label{4.1} Sei $f:X\to Y$ ein Morphismus von Schemata. $f$ heißt \textit{separiert}, falls $\Delta_{X/Y}:X\to X\times_YX$ eine abgeschlossene Immersion ist. In diesem Fall sagen wir auch, dass $X$ über $Y$ separiert ist. 

Ein Schema $X$ heißt \textit{separiert}\index{separiert}, falls $X\to\operatorname{Spec}(\mathbb{Z})$ separiert ist.

\paragraph{Beispiel 4.2.}\label{4.2} Sei $k$ ein Körper und $X$ die affine Gerade über $k$ mit doppeltem Nullpunkt wie in \hyperref[2.12]{Beispiel 2.12}. Dann ist $X\times_kX$ die affine Ebene mit vierfachen Nullpunkt und $\Delta(X)$ die gewöhnliche Diagonale, die zwei Nullpunkte besitzt. $\Delta(X)$ ist nicht abgeschlossen, da $\overline{\Delta(X)}$ vier Nullpunkte besitzt.

\paragraph{Beispiel 4.3.}\label{4.3} Ist $V$ eine Varietät über eine algebraisch abgeschlossenen Körper $k$, so ist $t(V)$ separiert über $k$. Dies werden wir später zeigen.

\paragraph{Beispiel 4.4.}\label{4.4} Ist $f:X\to Y$ ein Morphismus affiner Schemata, so ist $f$ separiert nach \hyperref[3.30]{Satz 3.30 (i)}.

\paragraph{Beispiel 4.5.}\label{4.5} Ist $f:X\to Y$ eine Immersion, so ist $f$ separiert, da nach \hyperref[3.21]{Satz 3.21 (vi)} $X\cong X\times_XX\cong X\times_YX$ gilt, d.h. $\Delta_{X/Y}:X\to X$ ist ein Isomorphismus.

\paragraph{Satz 4.6.}\label{4.6} Sei $f:X\to Y$ ein Morphismus von Schemata. Dann ist $f$ genau dann separiert, wenn $\Delta(X)\subset X\times_YX$ abgeschlossen ist.

\paragraph{Beweis.} Für die nicht-triviale Richtung sei $\Delta(X)\subset X\times_YX$ abgeschlossen. Wir müssen zeigen, dass $X\to \Delta(X)$ ein Homöomorphismus ist, und dass der Morphismus $\mathcal{O}_{X\times_YX}\to\Delta_\ast\mathcal{O}_X$ surjektiv ist.
\begin{itemize}
\item Sei $p_1:X\times_YX\to X$ die erste Projektion. Da $p_1\circ\Delta=\operatorname{id}_X$, induziert $\Delta$ ein Homöomorphismus auf sein Bild. 
\item Sei $P\in X$ und $V\subset_\text{o} Y$ affin. Wähle $U\subset_\text{o}X$ affin mit $P\in U$ und $f(U)\subset V$. Dann ist $U\times_VU$ eine offene, affine Umgebung von $\Delta(P)$. Nach \hyperref[4.4]{Beispiel 4.4} ist $\Delta:U\to U\times_VU$ eine abgeschlossene Immersion. Somit ist $\mathcal{O}_{X\times_YX}|_{U\times_VU}\to \Delta_\ast\mathcal{O}_X|_U$ surjektiv.\qed
\end{itemize}

\paragraph{Definition 4.7.}\label{4.7} Ein Ring $R$ heißt \textit{Bewertungsring}\index{Bewertungsring}, wenn $R$ nullteilerfrei ist, wenn von der folgenden Form ist:
\[R=\{x\in K^\times\mid v(x)\geq 0\} \]
wobei $v:K^\times\to G$ eine Bewertung ist, d.h. $v(xy)=v(x)+v(y)$ und $v(x+y)\geq\min\{v(x),v(y)\}$, und $G$ eine total geordnete abelsche Gruppe ist. $R$ ist dann lokaler Ring mit Maximalideal $\mathfrak{m}=\{x\in K^\times\mid v(x)>0\}\cup\{0\}$ und $K=\operatorname{Quot}(R)$.

\paragraph{Theorem 4.8.}\label{4.8} \textit{(Bewertungstheoretisches Kriterium für Separiertheit)}\index{Bewertungstheoretisches Kriterium} Sei $f:X\to Y$ ein Morphismus von Schemata mit $X$ noethersch. Es sind folgende Aussagen äquivalent:
\begin{enumerate}[(i)]
\item $f$ ist separiert.
\item Sei $R$ ein Bewertungsring des Körpers $K=\operatorname{Quot}(R)$ und $i:U=\operatorname{Spec}(K)\hookrightarrow T=\operatorname{Spec}(R)$ die kanonische Inklusion. Seien ferner Morphismen $T\to Y,\ U\to X$ derart gegeben, dass das folgende Diagramm kommutativ ist:
\[\begin{tikzcd}
U\ar[r]\ar[d, "i"'] & X\ar[d, "f"]\\
T\ar[r]\ar[ur, "h", dashed] & Y
\end{tikzcd}\]
Dann gibt es höchstens eine Abbildung $h:T\to X$, die das obige Diagramm kommutativ macht. Mit anderen Worten: Die folgende kanonische Abbildung ist für jeden Bewertungsring $R$ über $Y$ injektiv:
\[\operatorname{Hom}_Y(\operatorname{Spec}(R),X)\to\operatorname{Hom}_Y(\operatorname{Spec}(K),X),\ h\mapsto h\circ i \]
\end{enumerate}

\paragraph{Definition 4.9.}\label{4.9}  Sei $X$ ein topologischer Raum und $x\in X$. Ein Punkt $x'\in X$ heißt \textit{Spezialisierung}\index{Spezialisierung} von $x$, wenn $x'\in\overline{\{x\}}$ gilt. Ist $x''$ Spezialisierung von $x'$ und $x'$ Spezialisierung von $x$, so ist $x''$ auch Spezialisierung von $x$.

Eine Teilmenge $Z\subset X$ heißt \textit{stabil unter Spezialisierungen}, falls mit $x\in Z$ auch jede Spezialisierung in $Z$ ist. Abgeschlossene Mengen sind stabil unter Spezialisierungen. Die Umkehrung gilt im Allgemeinen nicht.

Ein Morphismus von Schemata $f$ heißt \textit{quasikompakt}\index{quasikompakt}, falls es eine offene affine Ü\-ber\-deckung $Y=\bigcup_\alpha Y_\alpha$ existiert, so dass alle $f^{-1}(Y_\alpha)$ quasikompakt sind. Es gilt:
\[f\text{ quasikompakt}\iff\forall V\subset_\text{o}Y\text{ affin}\colon f^{-1}(V)\text{ quasikompakt} \]

\paragraph{Satz 4.10.}\label{4.10} Sei $f:X\to Y$ ein quasikompakter Morphismus von Schemata. Dann sind folgende Aussagen äquivalent:
\begin{enumerate}[(i)]
\item $f$ ist abgeschlossen.
\item Ist $x\in X$ und $y'$ eine Spezialisierung von $y=f(x)$, so gibt es eine Spezialisierung $x'$ von $x$ mit $f(x')=y'$. Mit anderen Worten: $f(\overline{\{x\}})$ ist stabil unter Spezialisierungen für alle $x\in X$.
\end{enumerate}

\paragraph{Beweis.} (i)$\implies$(ii) ist trivial. Sei also $X'\subset X$ abgeschlossen und setze $Y'=\overline{f(X')}$. Wir zeigen $Y'=f(X')$. Wir versehen $X'$ und $Y'$ mit der eindeutig bestimmten, reduzierte abgeschlossene Unterschemastruktur in $X$ bzw. $Y$. Seien $i:X'\hookrightarrow X$ und $j:Y'\hookrightarrow Y$ die natürlichen Inklusionen. Das Urbildschema $(f\circ i)^{-1}(Y')$ ist reduziert, besitzt $X'$ als unterliegender topologischer Raum und ist als Schema gleich $X'$, da die reduzierte Unterschemastrukturen eindeutig sind. Somit existiert ein eindeutig bestimmter Morphismus $f':X'\to Y'$:
\[\begin{tikzcd}
X'\ar[r, "f'"]\ar[d, "i"'] & Y'\ar[d, "j"]\\
X\ar[r, "f"'] & Y
\end{tikzcd} \]
$f'$ erfüllt ebenfalls die Voraussetzung (ii). Ferner ist $f'$ quasikompakt. $i$ ist als abgeschlossene Immersion quasikompakt, also auch $f\circ i=j\circ f':X'\to Y$. Unter Basiswechsel sehen wir, dass $X'\times_YY'\to Y'$ quasikompakt ist. Da $j$ eine Immersion ist, ist $X'\cong X'\times_{Y'}Y' = X'\times_YY'$, also ist $X'\to Y'$ quasikompakt.

Wir müssen noch zeigen: Ist $f:X\to Y$ ein quasikompakter, dominanter Morphismus reduzierter Schema, welcher (ii) erfüllt, so ist $f$ surjektiv. Sei dazu $y'\in Y$ gegeben und $y$ ein generischer Punkt der irreduziblen Komponente von $Y$, in der $y'$ liegt. Somit ist $y'$ eine Spezialisierung von $y$. Gilt nun $f^{-1}(y)\neq\varnothing$, so gibt es ein $x\in X$ mit $f(x)=y$. Wegen (ii) gibt es eine Spezialisierung $x'$ von $x$ mit $f(x')=y'$ und wir sind fertig. Somit ist noch das nächste Lemma zu zeigen.\qed

\paragraph{Lemma 4.12.}\label{4.12} Für einen quasikompakten Morphismus $f:X\to Y$ ist äquivalent:
\begin{enumerate}[(i)]
\item $f$ ist dominant.
\item Für die generischen Punkte $y$ von $Y$ der irreduziblen Komponenten gilt $f^{-1}(y)\neq\varnothing$.
\end{enumerate}

\iffalse
\paragraph{Beweis.} (ii)$\implies$(i) ist trivial. Für die andere Richtung ersetzen wir $f$ durch $f_\text{red}$ und nehmen $X,Y$ als reduziert an. Sei $y\in Y$ ein generischer Punkt einer irreduziblen Komponente von $Y$ und $U\subset_\text{o}Y$ eine affine Umgebung von $y$. $U$ ist insbesondere quasikompakt, also auch $f^{-1}(U)$, d.h. $f^{-1}(U)=\bigcup_iV_i$ hat eine endliche Überdeckung durch offene, affine $V_i$. Da $f$ nach Voraussetzung dominant ist, folgt $U=\overline{ff^{-1}(U)}=\bigcup_i\overline{f(V_i)}$. Daher gibt es ein $i$ mit $y\in\overline{f(V_i)}$. $\overline{f(V_i)}\subset U$ wird mit der reduzierten, abgeschlossenen Unterschemastruktur versehen. Da alle $V_i$ reduziert sind, faktorisiert $f|_{V_i}:V_i\to U$ eindeutig in der Form $V_i\stackrel{g}{\to} \overline{f(V_i)}\hookrightarrow U$.

Somit genügt es zu zeigen, dass $g^{-1}(y)\neq \varnothing$, da $g^{-1}(y)=f^{-1}(y)\cap V_i$. $y$ ist generischer Punkt einer irreduziblen Komponente von $Y$, daher gilt $\mathcal{O}_{U,y}=\mathcal{O}_{Y,y}$.
\fi

\paragraph{Korollar 4.13.}\label{4.13} Eine quasikompakte Immersion $f:X\to Y$ ist genau dann eine abgeschlossene Immersion, wenn $f(X)$ in $Y$ stabil unter Spezialisierungen ist.

\paragraph{Beweis.} $f$ faktorisiert in eindeutiger Weise in der Form $X\stackrel{\sim}{\to}Z\stackrel{i}{\hookrightarrow} Y$, wobei $i$ die kanonische Inklusion des Unterschema $Z\subset Y$ ist. Es ist $i$ quasikompakt. Ist $f(X)=Z$ stabil unter Spezialisierungen, ist (ii) in \hyperref[4.10]{Satz 4.10} erfüllt, daher ist $i:Z\to Y$ abgeschlossen und $Z$ ein abgeschlossenes Unterschema in $Y$.\qed

\paragraph{Lemma 4.14.}\label{4.14} Sei $R$ ein Bewertungsring eines Körpers $K$ und $T=\operatorname{Spec}(R),\ U=\operatorname{Spec}(K)$. Sei $X$ ein Schema. Dann gilt:
\begin{enumerate}[(i)]
\item $\operatorname{Hom}(U,X)$ ist die Menge aller Paare $(x,i)$ mit $x\in X$ und Körperhomomorphismus $i:\kappa(x)\hookrightarrow K$, wobei $\kappa(x)$ der Restklassenkörper in $x$ bezeichnet.
\item $\operatorname{Hom}(T,X)$ ist die Menge aller Tripel $(x_0,x_1,i)$ mit $x_0,x_1\in X,\ x_0\in\overline{\{x_1\}}$ und Körperhomomorphismus $\kappa(x_1)\hookrightarrow K$, so dass $R$ über $\mathcal{O}_{\overline{\{x_1\}},x_0}$ \textit{dominiert}\index{dominiert}, wobei $\overline{\{x_1\}}$ mit der reduzierten Unterschemastruktur ausgestattet ist.
\end{enumerate}

\paragraph{Definition 4.15.}\label{4.15} Seien $(A,\mathfrak{m}_A),(B,\mathfrak{m}_B)$ lokale Ringe, die in einem Körper $K$ eingebettet sind. Wir sagen $B$ \textit{dominiert}\index{dominiert} $A$, wenn $A\subset B$ und $\mathfrak{m}_B\cap A=\mathfrak{m}_A$ gilt. Mit anderen Worten, wenn die natürliche Inklusion $A\hookrightarrow B$ ein lokaler Homomorphismus ist.

\paragraph{Beweis von \hyperref[4.14]{Lemma 4.14}.} (i) ist gerade \hyperref[2.18]{Satz 2.18}. Setze $t_0=\mathfrak{m}_R\in T$ und $t_1=(0)\in T$. Für (ii) sei $f:T\to X$ ein Morphismus. Setze $x_i=f(t_i)$ und $Z=\overline{\{x_1\}}$ mit der reduzierten Unterschemastruktur. Dann gilt $f^{-1}(Z)=T$ als Mengen. Nach \hyperref[2.21]{Regeln~2.21~(iii)} faktorisiert $f$ in der Form:
\[\begin{tikzcd}
T\ar[rr, "f"]\ar[rrd] &&X\\
&& Z\ar[u]
\end{tikzcd} \]
Beachte, dass für $U\subset_\text{o} Z$ mit $x_0\in U$ auch $x_1\in U$ gilt. Das induziert $\mathcal{O}_{Z,x_0}\to\mathcal{O}_{Z,x_1}$, analog haben wir $\mathcal{O}_{T,t_0}\to\mathcal{O}_{T,t_1}$. Wir erhalten folgendes kommutative Diagramm:
\[\begin{tikzcd}
\mathcal{O}_{Z,x_1} \ar[r]& \mathcal{O}_{T,t_1}=K\\
\mathcal{O}_{Z,x_0} \ar[r]\ar[u]& \mathcal{O}_{T,t_0}=R\ar[u, hook]\\
\mathfrak{m}_{Z,x_0}\ar[r]\ar[u, hook] & \mathfrak{m}_{T,t_0}\ar[u, hook]
\end{tikzcd}\]
Da $Z$ reduziert und irreduzibel ist, ist $Z$ integer. Daher ist $\mathcal{O}_{Z,x_1}=\kappa(x_1)$, alle Abbildungen sind injektiv und wir sehen, dass $R$ über $\mathcal{O}_{Z,x_0}$ dominiert.

Sei umgekehrt $x_0,x_1\in X$ mit $x_0\in\overline{\{x_1\}}= Z$ und $\mathcal{O}_{Z,x_0}\hookrightarrow R$ ein lokaler Homomorphismus. Das induziert $T=\operatorname{Spec}(R)\to\operatorname{Spec}(\mathcal{O}_{Z,x_0})\to Z\to X$. Diese beiden Konstruktionen sind invers zueinander.\qed

\paragraph{Beweis zu \hyperref[4.8]{Theorem 4.8}.} Sei $f:X\to Y$ ein Morphismus von Schemata mit $X$ noethersch.
\begin{itemize}
\item Sei zunächst $f$ separiert und $R$ ein Bewertungsring des Körpers $K=\operatorname{Quot}(R)$ mit Inklusion $i:U=\operatorname{Spec}(K)\hookrightarrow\operatorname{Spec}(R)=T$. Seien $T\to Y,\ U\to X$ gegeben und $h_1,h_2:T\to X$ mit kommutativem Diagramm:
\[\begin{tikzcd}
U\ar[d, "i"']\ar[r] & X\ar[d, "f"]\\
T\ar[r]\ar[ur, "h_1", shift left]\ar[ur, "h_2"', shift right] & Y
\end{tikzcd} \]
Setze $t_0=\mathfrak{m}_R\in T$ und $t_1=(0)\in T$. Aus der Kommutativität $h_1i=h_2i$ folgt insbesondere $h_1(t_1)=h_2(t_1)$. Setze $h''=(h_1,h_2)_Y:T\to X\times_Y X$. Betrachte nun folgendes kommutative Diagramm:
\[\begin{tikzcd}
U\ar[r]\ar[d, "i"]\ar[rdd, "h_2i"', bend right=90]\ar[rrd, "h_1i", bend left=50] & X\ar[d, "\Delta"] &\\
T\ar[r, "h''"'] & X\times_Y X\ar[r, "p_1"]\ar[d, "p_2"'] & X\ar[d, "f"]\\
& X\ar[r, "f"'] & Y
\end{tikzcd}\]
Es gilt $\{h''(t_1)\}= h''i(U)\subset\Delta(X)$. Da $f$ separiert ist, ist $\Delta(X)$ abgeschlossen und es folgt $h''(t_0)\in h''(\overline{\{t_1\}})\subset\overline{\{h''(t_1)\}}\subset\Delta(X)$. Es folgt:
\[h_1(t_0)=p_1h''(t_0)=p_2h''(t_0)=h_2(t_0) \]
Aus der Kommutativität folgt, dass $h_1^\sharp$ und $h_2^\sharp$ dieselbe Abbildung $\kappa(x_1)\hookrightarrow K$ mit $x_1=h_1(t_1)=h_2(t_1)$ induzieren. Mit \hyperref[4.14]{Lemma 4.14 (ii)} folgt $h_1=h_2$.
\item Für die andere Richtung genügt es nach \hyperref[4.6]{Satz 4.6} zu zeigen, dass $\Delta(X)\subset X\times_YX$ abgeschlossen ist. Nach \hyperref[3.30]{Satz 3.30 (ii)} ist $\Delta$ eine Immersion. Da $X$ noethersch ist, ist $\Delta$ quasikompakt. Nach \hyperref[4.13]{Korollar 4.13} genügt es zu zeigen, dass $\Delta(X)$ stabil unter Spezialisierungen ist.

Sei also $\xi_1\in\Delta(X)$ und $\xi_0\in\overline{\{\xi_1\}}$. Sei $K=\kappa(\xi_1)$ und $\mathcal{O}=\mathcal{O}_{\overline{\{\xi_1\}},\xi_0}$, wobei $\overline{\{\xi_1\}}$ mit der reduzierten Unterschemastruktur versehen ist, d.h. $\mathcal{O}$ ist ein lokaler Ring in $K$. Mit \hyperref[4.16]{Satz 4.16} und dem Lemma von Zorn sehen wir, dass für jeden lokalen Ring $\mathcal{O}$ in $K$ einen Bewertungsring $R$ gibt, der $\mathcal{O}$ dominiert. Sei $R$ ein solcher Bewertungsring. Nach \hyperref[4.14]{4.14 (ii)} haben wir einen Morphismus $h:T=\operatorname{Spec}(R)\to X\times_YX$ mit $t_i\mapsto\xi_i$, wobei $t_0=\mathfrak{m}_R\in T,\ t_1=(0)\in T$. Betrachte das kommutative Diagramm:
\[\begin{tikzcd}
\operatorname{Spec}(K)\ar[r, "i"] & T\ar[dr, "h"]\ar[rrd, "p_1h", bend left]\ar[ddr, "p_2h"', bend right]\\
& & X\times_YX\ar[r, "p_1"']\ar[d, "p_2"] & X\ar[d, "f"]\\
&& X\ar[r, "f"'] & Y
\end{tikzcd}\]
Ist $x\in X$, so ist $\kappa(x)\cong\kappa(\Delta(x))$, wegen:
\[\begin{tikzcd}
X\ar[rr, "\Delta"]\ar[rrd, "\operatorname{id}"'] & & X\times_Y X\ar[d, "p_i"]\\
&& X
\end{tikzcd} \]
Betrachte nun das folgende Diagramm:
\[\begin{tikzcd}
\operatorname{Spec}(K)\ar[rd, "hi"']\ar[r, "g", dashed] & X\ar[d, "\Delta"] \\
& X\times_YX 
\end{tikzcd}\]
Wähle ein $x_1\in X$ mit $\Delta(x_1)=\xi_1$ und $j:\kappa(x_1)\hookrightarrow K$ als den von $hi$ induzierten $\kappa(\Delta(x_1))\hookrightarrow K$. Nach \hyperref[4.14]{Lemma 4.14 (i)} erhalten wir den zu $(x_1, j)$ passenden Morphismus $g:\operatorname{Spec}(K)\to X$, das das obige Diagramm kommutativ macht. Also faktorisiert $\operatorname{Spec}(K)\to X\times_YX$ über $\operatorname{Spec}(K)\to X\stackrel{\Delta}{\to}X\times_YX$, d.h. $p_1hi=p_2hi$. Nach Voraussetzung folgt $p_1h=p_2h$, d.h. auch $T\to X\times_YX$ faktorisiert über $T\to X\stackrel{\Delta}{\to}X\times_YX$. Somit folgt $\xi_0\in \Delta(X)$.\qed
\end{itemize}

\paragraph{Satz 4.16.}\label{4.16} Sei $K$ ein Körper und $R\subset K$ ein lokaler Ring. Dann ist $R$ genau dann ein Bewertungsring, wenn für jeden lokalen Ring $R\subset S\subset K$ mit lokalen Homomorphismus $R\hookrightarrow S$ stets $R=S$ folgt.

\paragraph{Beweis.} Siehe z.B. Bourbaki, Algebra VI §13.

\paragraph{Satz 4.17.}\label{4.17} Alle involvierten Schemata seine noethersch. Dann gilt:
\begin{enumerate}[(i)]
\item Offene und abgeschlossene Immersionen sind separiert.
\item Die Komposition separierter Morphismen ist separiert.
\item Separiertheit ist stabil unter Basiswechsel.
\item Sind $f:X\to Y,\ f':X'\to Y'$ separierte Morphismen von $S$-Schemata, so ist auch $f\times_Sf':X\times_SX'\to Y\times_SY'$ separiert.
\item Ist $f\circ f$ separiert, so ist auch $f$ separiert.
\item $f:X\to Y$ ist genau dann separiert, wenn es eine offene Überdeckung $Y=\bigcup_\alpha Y_\alpha$ gibt, so dass alle $f^{-1}(Y_\alpha)\to Y_\alpha$ separiert sind.
\end{enumerate}

\paragraph{Beweis.} Folgt alles aus \hyperref[4.8]{Theorem 4.8}.\qed

\paragraph{Definition 4.18.}\label{4.18} Ein Morphismus von Schemata $f:X\to Y$ heißt \textit{eigentlich}\index{eigentlich}, wenn gilt:
\begin{enumerate}[(i)]
\item $f$ ist von endlichem Typ.
\item $f$ ist separiert.
\item $f$ ist \textit{universell abgeschlossen}\index{universell abgeschlossen}, d.h. für jeden Morphismus $Y'\to Y$ ist der Morphismus $f':X'=X\times_YY'\to Y'$ abgeschlossen.
\end{enumerate}
In diesem Fall heißt $X$ auch \textit{eigentlich über} $Y$.

\paragraph{Beispiel 4.19.}\label{4.19} Sei $k$ ein Körper und $X=\mathbf{A}_k^1$ die affine Gerade. Dann ist $X$ separiert und von endlichem Typ. Basiserweiterung mit $X\to k$ ergibt $X\times_kX\to X$, die Projektionsabbildung $\mathbf{A}_k^2\to\mathbf{A}_k^1$. Sei $Y=\operatorname{Spec}(k[X,Y]/(XY-1))$ die hyperbolische Kurve in $\mathbf{A}_k^2$. $Y\subset\mathbf{A}_k^2$. Diese ist abgeschlossen in $\mathbf{A}_k^2$ und projeziert sich in $\mathbf{A}_k^1$ auf $\mathbf{A}_k^1\setminus\{0\}$, die in $\mathbf{A}_k^1$ nicht abgeschlossen ist. Daher ist die Projektionsabbildung nicht abgeschlossen. In diesem Beispiel fehlt der unendliche Punkt von $Y$. Wir werden später zeigen, dass $X$ eigentlich über $k$ ist, wenn $X$ eine sogenannte projektive Varietät.

\paragraph{Theorem 4.20.}\label{4.20} \textit{(Bewertungstheoretisches Kriterium für Eigentlichkeit)}\index{Bewertungstheoretisches Kriterium} Sei $f:X\to Y$ ein Morphismus von endlichem Typ mit $X$ noethersch. Dann sind äquivalent:
\begin{enumerate}[(i)]
\item $f$ ist eigentlich.
\item Sei $R$ ein Bewertungsring des Körpers $K=\operatorname{Quot}(R)$ und $i:U=\operatorname{Spec}(K)\hookrightarrow T=\operatorname{Spec}(R)$ die kanonische Inklusion. Seien ferner Morphismen $T\to Y$ und $U\to X$ derart gegeben, dass das folgende Diagramm kommutativ ist:
\[\begin{tikzcd}
U\ar[r, "u"]\ar[d, "i"'] & X\ar[d, "f"]\\
T\ar[r, "t"']\ar[ru, "h", dashed] & Y
\end{tikzcd}\]
Dann existiert ein eindeutig bestimmter Morphismus $h:T\to X$, der das Diagramm kommutativ macht. Mit anderen worten ist die folgende Abbildung bijektiv:
\[\operatorname{Hom}_Y(T,X)\to\operatorname{Hom}_Y(U,X),\ h\mapsto h\circ i \]
\end{enumerate}

\paragraph{Lemma 4.21.}\label{4.21} \begin{enumerate}[(i)]
\item Abgeschlossene Immersinoen sind von endlichem Typ. Quasikompakte offene Immersionen sind von endlichem Typ.
\item Kompositum zweier Morphismen von endlichem Typ ist von endlichem Typ.
\item Ein Morphismus $f:X\to Y$ ist genau dann von endlichem Typ, wenn $f$ lokal von endlichem Typ und quasikompakt ist.
\item Seien $X\stackrel{f}{\to}Y\stackrel{g}{\to}Z$ Morphismen mit $f$ quasikompakt und $g\circ f$ von endlichem Typ. Dann ist $f$ von endlichem Typ.
\end{enumerate}

\paragraph{Beweis von \hyperref[4.20]{Theorem 4.20}.}\begin{itemize}
\item Sei $f$ eigentlich. Nach Definition ist $f$ separiert, somit ist ein Morphismus $h:T\to X$ wie oben eindeutig bestimmt nach \hyperref[4.8]{Theorem 4.8}. Es bleibt die Existenz zu zeigen. Betrachte die Basiserweiterung $X_T=X\times_YT$ von $X$ mit $T\to Y$:
\[\begin{tikzcd}
U \ar[dr, "\theta"', dashed]\ar[rrd, "u", bend left]\ar[ddr, "i"', bend right] &&\\
& X_T\ar[r, "p_1"']\ar[d, "f'"] & X\ar[d, "f"]\\
& T\ar[r, "t"'] & Y
\end{tikzcd} \]
Wir erhalten ein $\theta:U\to X_T$, so dass das obige Diagramm kommutiert. Sei $\xi_1\in X_T$ das Bild des einzigen Punktes $t_1\in U$ und $Z=\overline{\{\xi_1\}}\subset X_T$ mit der reduzierten Unterschemastruktur. Da $f$ universell abgeschlossen ist, ist $f'$ abgeschlossen, somit ist $f'(Z)\subset T$ abgeschlossen. Da $f'(\xi_1)=t_1$ und $t_1$ der generische Punkt von $T$ ist, gilt $f'(Z)=T$. Es existiert also ein $\xi_0\in Z$ mit $f'(\xi_0)=t_0$, wobei $t_0$ der abgeschlossene Punkt in $T$ ist. 

Betrachte den zu $f'$ gehörigen lokalen Homomorphismus $R\hookrightarrow\mathcal{O}_{Z,\xi_0}$. Da $\xi_1$ der generische Punkt von $Z$ ist, gilt $\kappa(\xi_1)=\mathcal{O}_{Z,\xi_1}$ und nach \hyperref[4.14]{Lemma 4.14 (i)} erhalten wir $\kappa(\xi_1)\subset K$, der durch $U\to Z$ induziert wird. Nach \hyperref[4.16]{Satz 4.16} ist $R$ als Bewertungsring maximal unter allen lokalen Ringen in $K$ bzgl. Dominanz. Ferner gilt $\mathcal{O}_{Z,\xi_0}\subset\mathcal{O}_{Z,\xi_1}=\kappa(\xi_1)\subset K$ und $\mathcal{O}_{Z,\xi_0}\subsetneq K$, da $\xi_0\neq\xi_1$. Wegen $f'(\xi_0)=t_0$ dominiert $\mathcal{O}_{Z,\xi_0}$ den Ring $R$, d.h. $R\cong\mathcal{O}_{Z,\xi_0}$, insbesondere dominiert $R$ über $\mathcal{O}_{Z,\xi_0}$. Nach \hyperref[4.14]{Lemma 4.14 (ii)} erhalten wir Morphismus $h':T\cong\operatorname{Spec}(\mathcal{O}_{Z,\xi_0})\to X_T,\ t_i\mapsto \xi_i$. Wir erhalten $h:T\stackrel{h'}{\to}X_T\stackrel{p_1}{\to}X$ mit $fh=fp_1h'=tf'h'=t$ und $hi=p_1h'i=p_1h'f'\theta = p_1\theta = u$.
\item Es gelte (ii). $f$ ist nach Voraussetzung von endlichem Typ und separiert nach \hyperref[4.8]{Theorem 4.8}. Sei also $Y'\to Y$ ein Morphismus und $f':X'=X\times_YY'\to Y'$ die Basiserweiterung von $f$. Wir zeigen, dass $f'$ abgeschlossen ist. Sei $Z\subset X'$ abgeschlossen mit der reduzierten Unterschemastruktur. Betrachte das kommutative Diagramm:
\[ \begin{tikzcd}
Z\ar[r, hook]\ar[dr] & X'\ar[r]\ar[d, "f'"] & X\ar[d, "f"]\\
& Y'\ar[r] & Y
\end{tikzcd} \]
Da $f$ von endlichem Typ ist, ist nach \hyperref[3.25]{Satz 3.25 (ii)} auch $f'$ von endlichem Typ. Es folgt, dass $f'|_Z:Z\to Y'$ von endlichem Typ ist und somit quasikompakt. $f'|_Z$ faktorisiert über $Z\to f'(Z)\hookrightarrow Y'$. Wir sehen, dass $f'(Z)\hookrightarrow Y'$ quasikompakt ist. Nach \hyperref[4.13]{Korollar 4.13} genügt es zu zeigen, dass $f'(Z)$ stabil unter Spezialisierungen ist.

Sei also $z_1\in Z$ und $y_1=f'(z_1),\ y_0\in\overline{\{y_1\}}$. Wir versehen $\overline{\{y_1\}}$ mit der reduzierten Unterschemastruktur. Sei $\mathcal{O}=\mathcal{O}_{\overline{\{y_1\}},y_0 }$. Dann ist $\operatorname{Quot}(\mathcal{O})=\kappa(y_1)\hookrightarrow \kappa(z_1)=:K$.  Sei $R$ ein Bewertungsring von $K$, der $\mathcal{O}$ dominiert. Nach \hyperref[4.14]{Lemma 4.14} haben wir Morphismen:
\[U=\operatorname{Spec}(K)\to Z,\ t_1\mapsto z_1 \]
\[T=\operatorname{Spec}(R)\to Y',\ t_i\mapsto y_i \]
Betrachte folgendes kommutative Diagramm:
\[\begin{tikzcd}
U\ar[r] \ar[d, "i"'] & Z\ar[r, hook] & X'\ar[r]& X\ar[d, "f"]\\
T\ar[rr]\ar[rrru, dashed]\ar[rru, dashed] & & Y'\ar[r]\ar[from=u, "f'", crossing over]\ar[from=ul, crossing over] & Y
\end{tikzcd} \]
Nach Voraussetzung gibt es ein Morphismus $h:T\to X$, der das Diagramm kommutativ macht. Aus der Universaleigenschaft des Faserprodukts $X'$ erhalten wir ein Morphismus $h':T\to X'$, der $h:T\to X$ liftet. Nach Voraussetzung ist $Z$ abgeschlossen und da der generische Punkt $t_1\in T$ auf $z_1\in X'$ abgebildet wird, faktorisiert $h':T\to X'$ über $h':T\to Z\to X'$. Setze $z_0=h'(t_0)\in Z$. Dann ist $f'(z_0)=y_0$, also $y_0\in f'(Z)$.\qed
\end{itemize}

\paragraph{Korollar 4.22.}\label{4.22} Alle vorkommenden Schemata seien noethersch. \begin{enumerate}[(i)]
\item Abgeschlossene Immersionen sind eigentlich.
\item Kompositum zweier eigentlicher Morphismen ist eigentlich.
\item Eigentliche Morphismen sind stabil unter Basiserweiterung.
\item Seien $f:X\to Y$ und $f':X'\to Y'$ eigentliche Morphismen von $S$-Schemata. Dann ist $f\times f':X\times_SX'\to Y\times_SY'$ eigentlich.
\item Seien $f:X\to Y,\ g:Y\to Z$ Morphismen. Ist $g\circ f$ eigentlich und $g$ separabel, so ist $f$ eigentlich.
\item Ein Morphismus $f:X\to Y$ ist genau dann eigentlich, wenn es eine offene Ü\-ber\-deckung $Y=\bigcup_\alpha Y_\alpha$ gibt, so dass $f^{-1}(Y_\alpha)\to Y_\alpha$ für alle $\alpha$ eigentlich ist.
\end{enumerate}

\paragraph{Beweis.} Wir zeigen nur (v). Da $X$ noethersch ist, ist $f$ quasikompakt. Nach \hyperref[4.21]{Lemma 4.21} ist $f$ von endlichem Typ. Nach \hyperref[4.17]{Satz 4.17} ist $f$ separiert. Sei nun $R$ ein Bewertungsring und seien Morphismen $U\to X,\ T\to Y$ gegeben mit kommutativem Diagramm:
\[\begin{tikzcd}
U \ar[r, "u"]\ar[d, "i"']& X\ar[d, "f"]\\
T\ar[r, "t"]\ar[dr, "t'"'] & Y\ar[d, "g"]\\
& Z
\end{tikzcd} \]
Wir wollen ein $h:T\to X$ konstruieren mit $fh=t$ und $hi=u$. Setze $t'=gt$. Da $gf$ eigentlich ist, gibt es genau ein $h:T\to X$ mit $gfh=t'$ und $hi=u$. Sei $t''=fh$. Nun ist $g$ separabel und $ti=fu=fhi=t''i$ und $gt=t'=gfh=gt''$. Nach \hyperref[4.8]{Theorem 4.8} folgt $t=t''$ und somit $fh=t$.\qed

\paragraph{} Ist $\varphi:A\to B$ ein Ringhomomorphismus und $\operatorname{Spec}(B)\to\operatorname{Spec}(A)$ der zugehörige Morphismus, so gilt:
\[\mathbf{P}_B^n\cong\mathbf{P}_A^n\times_{\operatorname{Spec}(A)}\operatorname{Spec}(B) \]
Insbesondere gilt $\mathbf{P}_A^n\cong\mathbf{P}_\mathbb{Z}^n\times_\mathbb{Z}A$.

\paragraph{Definition 4.23.}\label{4.23} \begin{enumerate}[(i)]
\item Sei $Y$ ein Schema. Dann heißt $\mathbf{P}_Y^n=\mathbf{P}_\mathbb{Z}^n\times_\mathbb{Z}Y$ der \textit{$n$-dimensionale projektive Raum über $Y$}\index{projektiver Raum}.
\item Ein Morphismus $f:X\to Y$ von Schemata heißt \textit{projektiv}, wenn er eine Faktorisierung $f:X\stackrel{i}{\hookrightarrow} \mathbf{P}_Y^n\stackrel{p_2}{\to} Y$ besitzt, wobei $i$ eine abgeschlossene Immersion ist.
\item Ein Morphismus $f:X\to Y$ von Schemata heißt \textit{quasiprojektiv}\index{quasiprojektiv}, falls er eine Faktorisierung der Form $f:X\stackrel{i}{\hookrightarrow} X'\stackrel{p}{\to}Y$ besitzt, wobei $i$ eine offene Immersion und $p$ projektiv ist.
\end{enumerate}

\paragraph{Theorem 4.24.}\label{4.24} Ein projektiver Morphismus von noetherschen Schemata ist eigentlich. Ein quasiprojektiver Morphismus von noetherschen Schemata ist von endlichem Typ und separiert.

\paragraph{Beweis.} Es reicht zu zeigen, dass $\mathbf{P}_\mathbb{Z}^n$ eigentlich über $\mathbb{Z}$ ist, da der Basiswechsel nach \hyperref[4.22]{Korollar 4.22 (iii)} eigentlich über $Y$ ist. Ist $f:X\hookrightarrow \mathbf{P}^n_Y \to Y$ projektiv, so ist er als Verkettung von eigentlichen Morphismen wieder eigentlich, siehe \hyperref[4.22]{Korollar 4.22}. Ist $f:X\hookrightarrow X'\to Y$ quasiprojektiv, so ist $f$ als Verkettung separierter Morphismen separiert, siehe \hyperref[4.17]{Korollar 4.17}. Da $X$ noethersch ist, ist $X\hookrightarrow X'$ eine quasikompakte offene Immersion, also nach \hyperref[4.21]{Lemma 4.21 (i)} vom endlichem Typ.

Sei also $X=\mathbf{P}_\mathbb{Z}^n$ und $X=\bigcup V_i$ eine offene affine Überdeckung mit $V_i=D_+(x_i)=\operatorname{Spec}\mathbb{Z}\big[\frac{x_0}{x_i},\ldots,\frac{x_n}{x_i} \big]$. Es ist also $X$ von endlichem Typ. Sei nun $R$ ein Bewertungsring mit Quotientenkörper $K=\operatorname{Quot}(R)$ und $U\to X,\ T\to\operatorname{Spec}(\mathbb{Z})$ Morphismen mit kommutativem Diagramm:
\[\begin{tikzcd}
U\ar[r, "u"]\ar[d, "i"'] & X\ar[d]\\
T\ar[r, "t"'] & \operatorname{Spec}(\mathbb{Z})
\end{tikzcd}\]
Wir zeigen nun per Induktion, dass genau ein Morphismus $h:T\to X$ existiert, der das obige Diagramm kommutativ ergänzt. Für $n=0$ ist $X=\operatorname{Spec}(\mathbb{Z})$ und die Aussage ist klar. Sei $n\geq 1$ und $\xi_1$ das Bild des einzigen Punktes aus $U$ in $X$. Ist $\xi_1\in X\setminus V_i$ für ein $i$, so folgt die Behauptung nach Induktionsvoraussetzung, da die Hyperebene $X\setminus V_i$ isomorph zu $\mathbf{P}_\mathbb{Z}^{n-1}$ ist.

Sei also $\xi_1\in\bigcap V_i$, d.h. alle Funktionen der Form $\frac{x_i}{x_j}$ sind invertierbar in $\mathcal{O}_{\xi_1}$. Der Morphismus $U\to X$ liefert Inklusion $\kappa(\xi_1)\hookrightarrow K$. Sei weiter $f_{ij}\in K^\times$ das Bild von $\frac{x_i}{x_j}$ unter $\mathcal{O}_{\xi_1}\to \kappa(\xi_1)\hookrightarrow K$. Dann folgt $f_{ik}=f_{ij}f_{jk}$ für alle $i,j,k$. Sei $v:K\to G$ die zu $R$ gehörige Bewertung und setze $g_i=v(f_{i0})$ für alle $i$. Sei $k$ derart, dass $g_k$ minimal unter $g_0,\ldots,g_n$ ist. Es gilt für alle $i$:
\[v(f_{ik})=v(f_{i0})-v(f_{k0})=g_i-g_k\geq 0 \]
Es folgt $f_{ik}\in R$ für alle $i$. Es gibt Homomorphismen:
\[\begin{tikzcd}
\mathbb{Z}\big[\frac{x_0}{x_k},\ldots,\frac{x_n}{x_k}\big]\ar[rr, "\varphi", "\frac{x_i}{x_k}\mapsto f_{ik}"'] \ar[d]&& R\ar[d]\\
\mathcal{O}_{\xi_1}\ar[r] & \kappa(\xi_1)\ar[r] & K
\end{tikzcd}\]
Wir erhalten den zu $\varphi$ gehörige Morphismus $T\to V_k$ und somit $h:T\to V_k\hookrightarrow X$. Offensichtlich ist $T\to X\to\operatorname{Spec}(\mathbb{Z})$ der Morphismus $t$ unt $U\to T\to X$ der Morphismus $u$. Ferner ist $h$ eindeutig nach Konstruktion.\qed

\paragraph{Satz 4.25.}\label{4.25} Sei $k$ ein algebraisch abgeschlossener Körper. Das Bild des Funktors $t:\mathbf{Var}(k)\to \mathbf{Sch}(k)$ ist die Menge aller quasiprojektiven, integren Schemata über $k$. Insbesondere ist $t(V)$ integer, separiert und von endlichem Typ für jede Varietät $V$.

\paragraph{Beweis.} Ohne Beweis.

\paragraph{Satz 4.26.}\label{4.26} Sei $X$ ein Schema von endlichem Typ über einem Körper $k$. Dann ist die Menge der abgeschlossenen Punkte in $X$ dicht in $X$.

\iffalse
\paragraph{Beweis.} Es genügt die Aussage für affine Schemata zu zeigen. Sei $X=\operatorname{Spec}(A)$ mit einer endlich erzeugten $k$-Algebra $A$ und $\mathfrak{M}$ die Menge der abgeschlossenen Punkte von $\operatorname{Spec}(A)$. Sei $f\in A$ mit $D(f)\neq\varnothing$. Wir zeigen $\mathfrak{M}\cap D(f)\neq\varnothing$. Es ist $D(f)\cong\operatorname{Spec}(A_f)$. Es existiert ein abgeschlossener Punkt $y\in \operatorname{Spec}(A_f)$, d.h. ein Maximalideal in $A_f$. Nun ist $A_f=A\big[\frac{1}{f}\big]$ endlich erzeugt über $k$, also $\kappa(y)=A_f/y$ eine algebraische Erweiterung von $k$ nach dem Hilbertschen Nullstellensatz. Sei $x$ der zu $y$ gehörige Punkt in $D(f)$.
\fi

\paragraph{Definition 4.27.}\label{4.27} Eine \textit{(abstrakte) Varietät}\index{Varietät!abstrakt} ist ein integres, separiertes Schema $X$ von endlichem Typ über einem algebraisch abgeschlossenen Körper $k$. Ist $X$ über $k$ eigentlich, so heißt $X$ \textit{vollständig}\index{Varietät!eigentlich}. Quasiprojektive abstrakte Varietäten entsprechen den klassischen Varietäten.

\paragraph{Bemerkung 4.28.}\label{4.28}\begin{enumerate}[(i)]
\item Eine projektive, abstrakte Varietät ist vollständig nach \hyperref[4.24]{Theorem 4.24}.
\item Es gibt vollständige Varietäten, die nicht projektiv sind, d.h. die Klasse der abstrakten Varietäten ist größer als die Klasse der klassischen Varietäten.
\item Jede vollständige abstrakte Varietät der Dimension $1$, d.h. eine vollständige Kurve, ist projektiv.
\item Jede Varietät kann als offene Menge in eine vollständige Varietät eingebettet werden.
\end{enumerate}

\section{Modulgarben}

\paragraph{Definition 5.1.}\label{5.1} Sei $(X,\mathcal{O}_X)$ ein geringter Raum.
\begin{enumerate}[(i)]
\item Eine \textit{Garbe von $\mathcal{O}_X$"~Moduln} bzw. \textit{$\mathcal{O}_X$"~Modul}\index{$\mathcal{O}_X$-Modul} ist eine Garbe $\mathcal{F}$ auf $X$ derart, dass $\mathcal{F}(U)$ ein $\mathcal{O}_X(U)$"~Modul für alle $U\subset_\text{o}X$ ist und alle $\operatorname{res}:\mathcal{F}(U)\to\mathcal{F}(V),\ V\subset U$ verträglich mit den Modulstrukturen via $\operatorname{res}:\mathcal{O}_X(U)\to\mathcal{O}_X(V)$ ist.
\item Ein \textit{Morphismus}\index{Morphismus!$\mathcal{O}_X$-Moduln} $\mathcal{F}\to\mathcal{G}$ von $\mathcal{O}_X$-Moduln ist ein Garbenmorphismus, so dass $\mathcal{F}(U)\to\mathcal{G}(U)$ ein $\mathcal{O}_X(U)$"~Modulhomomorphismus für alle $U\subset_\text{o}X$ ist.
\item Eine Sequenz von $\mathcal{O}_X$"~Moduln heißt \textit{exakt}\index{Exaktheit}, wenn sie exakt als Garbensequenz abelscher Garben ist.
\item Sind $\mathcal{F},\mathcal{G}$ $\mathcal{O}_X$"~Moduln, so bezeichnet $\operatorname{Hom}_{\mathcal{O}_X}(\mathcal{F},\mathcal{G})$ die Gruppe der Morphismen von $\mathcal{F}$ nach $\mathcal{G}$.
\item Sind $\mathcal{F},\mathcal{G}$ $\mathcal{O}_X$"~Moduln, so heißt die Garbe $U\mapsto \operatorname{Hom}_{\mathcal{O}_X|_U}(\mathcal{F}|_U,\mathcal{G}|_U)$ die \textit{Hom-Garbe}\index{Hom-Garbe} und wird mit $\operatorname{\mathcal{H}om}(\mathcal{F},\mathcal{G})$ bezeichnet. Diese ist ein $\mathcal{O}_X$"~Modul.
\item Seien $\mathcal{F},\mathcal{G}$ $\mathcal{O}_X$"~Moduln. Dann heißt die zur Prägarbe $U\mapsto\mathcal{F}(U)\otimes_{\mathcal{O}_X(U)}\mathcal{G}(U)$ assoziierte Garbe das \textit{Tensorprodukt}\index{Tensorprodukt} von $\mathcal{F}$ und $\mathcal{G}$. Diese wird  mit $\mathcal{F}\otimes_{\mathcal{O}_X}\mathcal{G}$ bezeichnet.
\item Ein $\mathcal{O}_X$"~Modul $\mathcal{F}$ heißt \textit{frei}\index{$\mathcal{O}_X$-Modul!frei}, wenn $\mathcal{F}$ isomorph zu einer direkten Summe von Exemplaren von $\mathcal{O}_X$ ist.
\item Ein $\mathcal{O}_X$"~Modul $\mathcal{F}$ heißt \textit{lokal frei}\index{$\mathcal{O}_X$-Modul!lokal frei}, wenn $X$ durch offene Mengen $U$ überdeckt werden kann, so dass $\mathcal{F}|_U$ freier $\mathcal{O}_X|_U$"~Modul ist.

Der \textit{Rang} $r$\index{Rang} von $\mathcal{F}$ auf $U$ ist gerade die Anzahl der Kopien von $\mathcal{O}_X|_U$. Wir schreiben $\operatorname{rang}(\mathcal{F}|_U)=r$. Ist $X$ zusammenhängend, so ist dieser Rang überall gleich. 
\item Ein lokal freier $\mathcal{O}_X$"~Modul vom Rang $1$ heißt \textit{invertierbare Garbe}\index{Garbe!invertierbar}.
\item Eine \textit{Idealgarbe} auf $X$\index{Idealgarbe} ist ein $\mathcal{O}_X$"~Modul $\mathcal{J}$, der Untergarbe von $\mathcal{O}_X$ ist, d.h. $\mathcal{J}(U)$ ist ein Ideal von $\mathcal{O}_X(U)$ für $U\subset_\text{o}X$.
\item Sei $f:(X,\mathcal{O}_X)\to (Y,\mathcal{O}_Y)$ ein Morphismus von geringten Räumen und $\mathcal{F}$ ein $\mathcal{O}_X$"~Modul. Dann ist $f_\ast\mathcal{F}$ ein $f_\ast\mathcal{O}_X$"~Modul und somit auch ein $\mathcal{O}_Y$"~Modul unter dem Garbenmorphismus $f^\sharp:\mathcal{O}_Y\to f_\ast\mathcal{O}_X$. $f_\ast\mathcal{F}$ heißt \textit{direktes Bild} von $\mathcal{F}$ unter $f$.\index{direktes Bild}

Sei $\mathcal{G}$ ein $\mathcal{O}_Y$"~Modul. Dann ist $f^{-1}\mathcal{G}$ ein $f^{-1}\mathcal{O}_Y$"~Modul. Betrachte das Bild $\theta$ von $f^\sharp$ unter dem Adjunktionsisomorphismus $\operatorname{Hom}_Y(\mathcal{O}_Y,f_\ast\mathcal{O}_X)\to\operatorname{Hom}_X(f^{-1}\mathcal{O}_Y,\mathcal{O}_X)$. Durch $\theta$ wird $\mathcal{O}_X$ zu einem $f^{-1}\mathcal{O}_Y$"~Modul. Wir definieren: 
\[f^\ast\mathcal{G}= f^{-1}\mathcal{G}\otimes_{f^{-1}\mathcal{O}_Y}\mathcal{O}_X\]
Somit ist $f^\ast\mathcal{G}$ ein $\mathcal{O}_X$"~Modul, das \textit{Urbild}\index{Urbild} von $\mathcal{G}$ unter $f$. 

Für jeden $\mathcal{O}_X$"~Modul $\mathcal{F}$ und $\mathcal{O}_Y$"~Modul $\mathcal{G}$ gilt:
\[\operatorname{Hom}_{\mathcal{O}_X}(f^\ast\mathcal{G},\mathcal{F})\cong\operatorname{Hom}_{\mathcal{O}_Y}(\mathcal{G},f_\ast\mathcal{F}) \]
Somit ist $f^\ast$ linksadjungiert zu $f_\ast$.
\end{enumerate}

\paragraph{Bemerkung.}\begin{enumerate}[(i)]
\item Kern, Bild und Kokern eines Morphismus von $\mathcal{O}_X$"~Moduln ist wieder ein $\mathcal{O}_X$"~Modul.
\item Sind $\mathcal{F},\mathcal{F}'$ $\mathcal{O}_X$"~Moduln und $\mathcal{F}'$ eine Untergarbe von $\mathcal{F}$, so ist $\mathcal{F}/\mathcal{F}'$ ein $\mathcal{O}_X$-Modul.
\item Direkte Summen, direkte Produkte und projektive Limiten von $\mathcal{O}_X$"~Moduln sind wieder $\mathcal{O}_X$"~Moduln.
\end{enumerate}

\paragraph{Definition.} Sei $A$ ein Ring und $M$ ein $A$-Modul. Wir definieren die \textit{zu $M$ assoziierte Garbe}\index{assoziierte Garbe!Modul} $\widetilde{M}$ auf $\operatorname{Spec}(A)$ wie folgt: Für $U\subset_\text{o}\operatorname{Spec}(A)$ setzen wir $\widetilde{M}(U)$ als die Menge aller Abbildungen $s:U\to\coprod_{\mathfrak{p}\in U}M_\mathfrak{p}$, so dass:
\begin{enumerate}[(i)]
\item Für alle $\mathfrak{p}\in U$ gilt $s(\mathfrak{p})\in M_\mathfrak{p}$.
\item Für alle $\mathfrak{p}\in U$ gibt es eine offene Umgebung $V$ von $\mathfrak{p}$ mit $V\subset U$ und Elemente $m\in M,\ f\in A$, so dass für alle $\mathfrak{q}\in V$ stets $f\not\in\mathfrak{q}$ und $s(\mathfrak{q})=\frac{m}{f}\in M_\mathfrak{q}$ gilt.
\end{enumerate}
Dann ist $\widetilde{M}$ mit den gewöhnlichen Restriktionsabbildungen eine Garbe.

\paragraph{Satz 5.2.}\label{5.2} Sei $A$ ein Ring und $M$ ein $A$-Modul mit der assoziierten Garbe $\widetilde{M}$ auf $X=\operatorname{Spec}(A)$. Dann gilt:\begin{enumerate}[(i)]
\item $\widetilde{M}$ ist ein $\mathcal{O}_X$"~Modul.
\item Für jedes $\mathfrak{p}\in X$ gilt für den Halm von $\widetilde{M}$ in $\mathfrak{p}$ stets $\widetilde{M}_\mathfrak{p}\cong M_\mathfrak{p}$.
\item Für alle $f\in A$ gibt es einen $A_f$-Modulisomorphismus $\widetilde{M}(D(f))\cong M_f$.
\item Insbesondere gilt $\Gamma(X,\widetilde{M})\cong M$.
\end{enumerate}

\paragraph{Beweis.} (i) ist klar. (iv) folgt aus (iii) mit $f=1$. (ii) und (iii) gehen analog zu \hyperref[2.3]{2.3}.\qed

\paragraph{Satz 5.3.}\label{5.3} Sei $A$ ein Ring und $X=\operatorname{Spec}(A)$. Ferner sei $\varphi:A\to B$ ein Ringhomomorphismus und $f:\operatorname{Spec}(B)\to \operatorname{Spec}(A)$ der entsprechende Morphismus. Dann gilt:
\begin{enumerate}[(i)]
\item Die Abbildung $M\mapsto\widetilde{M}$ liefert einen exakten, volltreuen Funktor von der Kategorie der $A$"~Moduln in die Kategorie der $\mathcal{O}_X$"~Moduln.
\item Seien $M,N$ $A$"~Moduln. Dann gilt $\widetilde{M\otimes_AN}\cong \widetilde{M}\otimes_{\mathcal{O}_X}\widetilde{N}$.
\item Sei $(M_i)_i$ eine Familie von $A$-Moduln. Dann gilt $\widetilde{\bigoplus_i M_i}\cong\bigoplus_i\widetilde{M_i}$.
\item Sei $N$ ein $B$-Modul. Dann gilt $f_\ast\widetilde{N}\cong \widetilde{_AN}$, wobei $_AN$ den Modul $N$ als $A$"~Modul via $\varphi$ bezeichnet.
\item Sei $M$ ein $A$"~Modul. Dann gilt $f^\ast\widetilde{M}\cong \widetilde{M\otimes_A B}$.
\end{enumerate}

\paragraph{Beweis.} Für (i) zeigen wir: \begin{itemize}
\item \textit{Funktorialität:} Sei $\psi:M\to N$ ein $A$"~Modulhomomorphismus. Dieser induziert für alle $f\in A$ einen $A_f$"~Modulhomomorphismus $\psi_f:M_f\to N_f$. Ist $D(f)\supset D(g)$, so erhalten wir das folgende kommutative Diagramm:
\[\begin{tikzcd}
M_f\ar[rr, "\psi_f"]\ar[d] && N_f\ar[d]\\
M_g\ar[rr, "\psi_g"'] && N_g
\end{tikzcd} \]
Da $D(f),\ f\in A$ eine Basis der Topologie bilden, induzieren $\psi_f,\ f\in A$ einen $\widetilde{A}$"~Homomorphismus $\widetilde{\psi}:\widetilde{M}\to\widetilde{N}$ mit $\widetilde{\psi}|_{D(f)}=\psi_f$.
\item \textit{Volltreu:} Die Umkehrabbildung zu $\operatorname{Hom}_A(M,N)\to\operatorname{Hom}_{\widetilde{A}}(\widetilde{M},\widetilde{N})$ ist gegeben durch das Bilden der globalen Schnitte: 
\[\operatorname{Hom}_{\widetilde{A}}(\widetilde{M},\widetilde{N})\to\operatorname{Hom}_A(\Gamma(X,\widetilde{M}),\Gamma(X,\widetilde{N}))=\operatorname{Hom}_A(M,N)\]
\item \textit{Exaktheit:} Sei $0\to M'\to M\to M''\to 0$ eine kurze exakte Folge von $A$"~Moduln. Da Lokalisierungen exakt sind, ist $0\to M'_\mathfrak{p}\to M_\mathfrak{p}\to M''_\mathfrak{p}\to 0$ exakt. Nach \hyperref[5.2]{Satz 5.2 (ii)} ist $\widetilde{M}_\mathfrak{p}=M_\mathfrak{p}$. Da jede Halmsequenz exakt ist, folgt nach \hyperref[1.10]{1.10} die Exaktheit von $0\to\widetilde{M'}\to\widetilde{M}\to\widetilde{M''}\to 0$.
\end{itemize}
(ii) und (iii) folgen, da direkte Summen und Tensorprodukte mit Lokalisierungen kommutieren. Für (iv) sei $g\in A$. Es gilt:
\[\Gamma(D(g),f_\ast\widetilde{N}) = \Gamma(f^{-1}(D(g)),\widetilde{N}) = \Gamma(D(\varphi(g)),\widetilde{N}) \cong N_{\varphi(g)} = N_g\cong \Gamma(D(g),\widetilde{N})\]
Für (v) sei $N$ ein $B$-Modul. Dann gilt:
\begin{align*}
\operatorname{Hom}_{\widetilde{B}}(f^\ast\widetilde{M},\widetilde{N})&=\operatorname{Hom}_{\widetilde{A}}(\widetilde{M},f_\ast \widetilde{N}) \stackrel{\text{(iv)}}{=}\operatorname{Hom}_{\widetilde{A}}(\widetilde{M}, \widetilde{_AN})\\
&= \operatorname{Hom}_A(M, {_AN}) \stackrel{\star}{\cong}\operatorname{Hom}_B(M\otimes_A B,N)\cong\operatorname{Hom}_{\widetilde{B}}(\widetilde{M\otimes_A B},\widetilde{N})
\end{align*}
wobei $\star$ durch die Abbildung $\eta\mapsto (m\otimes b\mapsto \eta(m)b)$ gegeben ist.\qed

\paragraph{Definition 5.4.}\label{5.4} \begin{enumerate}[(i)]
\item Sei $(X,\mathcal{O}_X)$ ein Schema. Ein $\mathcal{O}_X$"~Modul $\mathcal{F}$ heißt \textit{quasikohärent}\index{$\mathcal{O}_X$-Modul!quasikohärent}, falls es eine offene affine Überdeckung $U_i=\operatorname{Spec}(A_i),\ i\in I$ von $X$ gibt, so dass für jedes $i$ ein $A_i$"~Modul $M_i$ existiert mit $\mathcal{F}|_{U_i}\cong\widetilde{M_i}$.
\item $\mathcal{F}$ heißt \textit{kohärent}\index{kohärent}, falls $\mathcal{F}$ quasikohärent ist und alle vorkommenden $M_i$ in (i) endlich erzeugte $A_i$"~Moduln sind.
\end{enumerate}

\paragraph{Beispiel 5.5.}\label{5.5} Für jedes Schema $X$ ist $\mathcal{O}_X$ kohärent, da $\mathcal{O}_X|_{\operatorname{Spec}(A)}=\widetilde{A}$.

\paragraph{Beispiel 5.6.}\label{5.6} Sei $X=\operatorname{Spec}(A)$ affin und $Y\subset X$ ein abgeschlossenes Unterschema, das durch das Ideal $\mathfrak{a}\subset A$ definiert ist. Sei $i:Y\hookrightarrow X$ die natürliche Inklusion. Es ist $\mathcal{O}_Y\cong\widetilde{A/\mathfrak{a}}$ und somit $i_\ast\mathcal{O}_Y=\widetilde{A/\mathfrak{a}}$, wobei hier $A/\mathfrak{a}$ als $A$"~Modul aufgefasst wird. Somit ist $i_\ast\mathcal{O}_Y$ ein kohärenter $\mathcal{O}_X$"~Modul.

\paragraph{Beispiel 5.7.}\label{5.7} Sei $X=\operatorname{Spec}(A)$ affin und $U\subsetneq_\text{o}X$ mit der natürlichen Inklusion $j:U\hookrightarrow X$. Betrachte die Garbe $j_!\mathcal{O}_U$, die außerhalb $U$ durch Null fortgesetzte Garbe von $\mathcal{O}_U$. $j_!\mathcal{O}_U$ ist nicht quasikohärent:

Sei $X$ irreduzibel und $V=\operatorname{Spec}(A)\subset_\text{o}X$ mit $V\subsetneq U$. Wäre $j_!\mathcal{O}_U|_V\cong\widetilde{M}$ für einen $A$"~Modul $M$., so ist $(j_!\mathcal{O}_U|_V)(V)=M$, aber $(j_!\mathcal{O}_U|_V)(V)=0$ und $j_!\mathcal{O}_U|_V\neq 0$.

\paragraph{Lemma 5.8.}\label{5.8} Sei $X=\operatorname{Spec}(A)$ ein affines Schema, $f\in A$ und $\mathcal{F}$ ein quasikohärenter $\mathcal{O}_X$"~Modul.
\begin{enumerate}[(i)]
\item Sei $s\in\Gamma(X,\mathcal{F})$ mit $s|_{D(f)}=0$. Dann existiert ein $n>0$ mit $f^ns=0$.
\item Sei $t\in\Gamma(D(f),\mathcal{F})$. Dann existiert ein $n>0$ und $t'\in\mathcal{F}(X)$ mit $f^nt=t'|_{D(f)}$.
\end{enumerate}

\paragraph{Beweis.} Wir zeigen zunächst, dass es eine Überdeckung der Form $X=\bigcup_{i=1}^m D(g_i)$ gibt, so dass $\mathcal{F}|_{D(g_i)}\cong\widetilde{M_i}$ für einen $A_{g_i}$"~Modul $M_i$.

Da $\mathcal{F}$ quasikohärent ist, existiert eine offene affine Überdeckung aus Mengen der Form $V=\operatorname{Spec}(B)$ mit $\mathcal{F}|_V=\widetilde{M}$ für einen $B$"~Modul $M$. Wir schreiben $V=\bigcup_{\text{gewisse }g\in A}D(g)$. Die natürlichen Morphismen $D(g)\hookrightarrow V$ liefern Ringhomomorphismen $B\to A_g$. Nach \hyperref[5.3]{Satz 5.3 (v)} ist $\mathcal{F}|_{D(g)}\cong \widetilde{M\otimes_B A_g}$. Da $X$ affin und somit quasikompakt ist, kann $X$ durch solche Mengen endlich überdeckt werden.
\begin{enumerate}[(i)]
\item Setze $s_i$ als das Bild von $s|_{D(g_i)}$ unter $\Gamma(D(g_i),\mathcal{F})=\Gamma(D(g_i),\widetilde{M_i})\cong M_i$. Wegen $D(fg_i)=D(f)\cap D(g_i)$ folgt nach \hyperref[5.2]{Satz 5.2 (iii)} $\Gamma(D(fg_i),\mathcal{F})=(M_i)_f$. Also ist $s_i=0$ in $(M_i)_f$. Nach Definition gibt es ein $n_i\in\mathbb{N}$ mit $f^{n_i}s_i=0$ in $M_i$. Sei $n$ das Maximum aller $n_i,\ i=1,\ldots,m$. Dann folgt $f^ns=0$ aus der ersten Garbeneigenschaft.
\item Betrachte die Einschränkungen $t\in\Gamma(D(fg_i),\mathcal{F})=(M_i)_f$. Für alle $i$ gibt es ein $n_i\in\mathbb{N}$, so dass $f^{n_i}t=t_i|_{D(fg_i)}$ für ein $t_i\in \Gamma(D(g_i),\mathcal{F})=M_i$. Sei $n$ das Maximum aller $n_i,\ i=1,\ldots,m$. Dann gibt es für alle $i$ ein $t_i\in\Gamma(D(g_i),\mathcal{F})$ mit $f^n t = t_i|_{D(fg_i)}$.

Auf $D(g_i)\cap D(g_j)=D(g_ig_j)$ haben wir Schnitte $t_i,t_j$ konstruiert, die auf $D(fg_ig_j)$ übereinstimmen. Nach (i) gibt es ein $m_{ij}$, so dass $f^{m_{ij}}(t_i-t_j)=0$ auf $D(g_ig_j)$ gilt. Sei $m$ das Maximum aller $m_{ij}$, so dass $f^m(t_i-t_j)=0$ auf $D(g_ig_j)$ für alle $i,j$ gilt. Die lokalen Schnitte $f^mt_i$ in $\Gamma(D(g_i),\mathcal{F})$ verkleben sich somit zu einem globalen Schnitt $t'$ von $\mathcal{F}$ zusammen mit $t'|_{D(f)}=f^{n+m}t$.\qed
\end{enumerate}

\paragraph{Satz 5.9.}\label{5.9} Sei $X$ ein Schema und $\mathcal{F}$ ein $\mathcal{O}_X$"~Modul. Dann ist $\mathcal{F}$ genau dann quasikohärent, wenn für alle affinen $U=\operatorname{Spec}(A)\subset_\text{o}X$ ein $A$"~Modul $M$ existiert mit $\mathcal{F}|_U\cong\widetilde{M}$.

Ist $X$ noethersch, so ist $\mathcal{F}$ genau dann kohärent, wenn für alle affinen $U=\operatorname{Spec}(A)\subset_\text{o}X$ ein endlich erzeugter $A$"~Modul $M$ existiert mit $\mathcal{F}|_U\cong\widetilde{M}$.

\paragraph{Lemma.} Sei $X=\operatorname{Spec}(A)$ affin, $M$ ein $A$"~Modul und $\mathcal{F}$ ein $\mathcal{O}_X$"~Modul. Dann ist
\[\operatorname{Hom}_A(M,\Gamma(X,\mathcal{F}))\to\operatorname{Hom}_{\widetilde{A}}(\widetilde{M},\mathcal{F}),\ \varphi\mapsto\widetilde{\varphi} \]
ein Isomorphismus mit Umkehrabbildung $\theta\mapsto\Gamma(X,\theta)$.

\paragraph{Beweis von \hyperref[5.9]{Satz 5.9}.} Die Rückrichtungen der beiden Aussagen sind trivial. Sei $U\subset_\text{o}X$ affin. Nach dem Beweis von \hyperref[5.8]{Lemma 5.8} gibt es eine Basis der Topologie von $U$, bestehend aus affinen Teilmengen $V_i$ derart, dass $\mathcal{F}|_{V_i}\cong\widetilde{N_i}$ mit einem Modul $N_i$ ist. Somit ist $\mathcal{F}|_U$ quasikohärent. Wir können somit o.B.d.A. $X=U=\operatorname{Spec}(A)$ als affin annehmen.

Setze $M=\Gamma(X,\mathcal{F})$ und $\alpha:\widetilde{M}\to\mathcal{F}$ als das Bild von $\operatorname{id}_M$ unter der Abbildung im vorherigen Lemma. Wie im Beweis von \hyperref[5.8]{Lemma 5.8} gezeigt, gibt es eine Überdeckung der Form $X=\bigcup_{i=1}^mD(g_i)$ mit $\mathcal{F}|_{D(g_i)}\cong\widetilde{M_i}$ für einen $A_{g_i}$"~Modul $M_i$. Es gilt $M_i=\mathcal{F}(D(g_i))\cong M_{g_i}$ und wir haben ein kommutatives Diagramm:
\[\begin{tikzcd}
\mathcal{F}(X)_{g_i}\ar[rr, "\cong"] && \mathcal{F}(D(g_i))\\
\mathcal{F}(X)\ar[u]\ar[urr, "\operatorname{res}"']
\end{tikzcd} \]
Somit ist $\alpha|_{D(g_i)}$ ein Isomorphismus für alle $i$. Da die $D(g_i)$ ganz $X$ überdecken, ist $\alpha$ ein Isomorphismus.

Sei nun $X$ zusätzlich noethersch. Dann sind die $A_{g_i}$"~Moduln $M_{g_i}$ endlich erzeugt. Wir zeigen, dass $M$ endlich erzeugt ist. Da $A$ noethersch ist, sind alle $A_{g_i}$ noethersch. Also sind die endlich erzeugten $A_{g_i}$"~Moduln $M_{g_i}$ noethersch. Analog zu \hyperref[3.6]{Satz 3.6} folgt $M$ noethersch. Insbesondere ist $M$ endlich erzeugt. \qed

\paragraph{Korollar 5.10.}\label{5.10} Sei $A$ ein Ring und $X=\operatorname{Spec}(A)$. Dann ist
\[\{\text{Kategorie der $A$-Moduln}\}\to\{\text{Kategorie der quasikohärenten $\mathcal{O}_X$-Moduln}\},\ M\mapsto\widetilde{M} \]
ist eine Kategorienäquivalenz mit der Umkehrabbildung $\mathcal{F}\mapsto\Gamma(X,\mathcal{F})$. Ist $A$ noethersch, so geben dieselben Funktoren eine Kategorienäquivalenz zwischen den endlich erzeugten $A$-Moduln und den kohärenten $\mathcal{O}_X$"~Moduln.

\paragraph{Beweis.} Sei $\mathcal{F}$ ein quasikohärenter $\mathcal{O}_X$"~Modul. Nach \hyperref[5.9]{Satz 5.9} existiert ein $A$"~Modul $M$ mit $\mathcal{F}\cong\widetilde{M}$. Nach \hyperref[5.2]{Satz 5.2 (iv)} ist $\Gamma(X,\mathcal{F})=M$, also:
\[M\mapsto\widetilde{M}\mapsto \Gamma(X,\widetilde{M})=M,\quad\mathcal{F}=\widetilde{M}\mapsto \Gamma(X,\widetilde{M})=M\mapsto\widetilde{M}\qedhere \]

\paragraph{Satz 5.11.}\label{5.11} Sei $X=\operatorname{Spec}(A)$ ein affines Schema und sei $0\to\mathcal{F}'\to\mathcal{F}\to\mathcal{F}''\to 0$ eine exakte Sequenz von $\mathcal{O}_X$"~Moduln, wobei $\mathcal{F}'$ quasikohärent ist. Dann ist die Sequenz der globalen Schnitte ebenfalls exakt:
\[0\to\Gamma(X,\mathcal{F}')\to\Gamma(X,\mathcal{F})\to\Gamma(X,\mathcal{F}'')\to 0 \]

\paragraph{Beweis.} Da $\Gamma(X,-)$ linksexakt ist, bleibt nur die Surjektivität von $\Gamma(X,\mathcal{F})\to\Gamma(X,\mathcal{F}'')$ zu zeigen. Sei $s\in\Gamma(X,\mathcal{F}'')$ gegeben. Für $x\in X$ gibt es ein $f\in A$ mit $x\in D(f) \subset X$, so dass $s|_{D(f)}$ sich zu einem $t\in\mathcal{F}(D(f))$ liftet. Wir zeigen nun, dass es ein $r>0$ existiert, so dass sich $f^rs$ zu einem $t''\in\mathcal{F}(X)$ liftet.

Sei $X=\bigcup_i D(g_i)$ eine endliche offene Überdeckung, so dass sich $s|_{D(g_i)}$ zu einem $t_i\in\mathcal{F}(D(g_i))$ liftet. Auf $D(f)\cap D(g_i)=D(fg_i)$ liften $t_i,t\in\mathcal{F}(D(fg_i))$ beide $s$. Aus der Linksexaktheit von $\Gamma(D(fg_i), -)$ folgt $t-t_i\in\mathcal{F}'(D(fg_i))$. Da $\mathcal{F}'$ quasikohärent ist, folgt aus \hyperref[5.8]{Lemma 5.8 (ii)} die Existenz eines $n>0$ und $u_i\in\mathcal{F}'(D(g_i))$ mit $u_i|_{D(fg_i)}=f^n(t-t_i)$. Wir können $n$ unabhängig von $i$ wählen. Setze $t_i'=f^nt_i+u_i\in\mathcal{F}(D(g_i))$. Dann ist $t'_i$ ein Lift von $f^ns|_{D(g_i)}$ und $\star\ t_i'=f^nt$ auf $\mathcal{F}(D(fg_i))$. Auf $D(g_ig_j)$ liften $t_i'$ und $t_j'$ beide $f^ns$, also $t_i'-t_j'\in\mathcal{F}'(D(g_ig_j))$ und daher $t'_i=t'_j$ auf $\mathcal{F}(D(fg_ig_j))$ wegen $\star$. Nach \hyperref[5.8]{Lemma 5.8 (i)} existiert ein $m>0$, so dass $f^m(t_i'-t_j')=0$ auf $\mathcal{F}(D(g_ig_j))$. Wir können $m$ unabhängig von $i,j$ wählen. Somit verkleben sich die $f^mt_i'$ zu einem $t''\in\mathcal{F}(X)$ zusammen und $t''$ ist ein Lift von $f^{n+m}s$.

Sei nun $X=\bigcup_i D(f_i)$ eine endliche offene Überdeckung, so dass sich $s|_{D(f_i)}$  zu einem Schnitt aus $\mathcal{F}(D(f_i))$ liften lässt. Für alle $i$ existiert ein $n$ und ein Lift $t_i\in\Gamma(X,\mathcal{F})$ von $f_i^ns$. Wir können o.B.d.A. $n$ unabhängig von $i$ annehmen. Da $X=\bigcup_i D(f_i)$, gilt $(f_1^n,\ldots,f_k^n)=A$, also $1=\sum_{i=1}^k a_if_i^n$ für gewisse $a_i\in A$. Setze $t=\sum_{i=1}^k a_it_i\in\mathcal{F}(X)$. Dann ist $t$ ein Lift von $\sum_{i=1}^k a_if_i^ns=s\in\mathcal{F}''(X)$.\qed

\paragraph{Satz 5.12.}\label{5.12} Sei $X$ ein Schema.
\begin{enumerate}[(i)]
\item Kern, Kokern und Bild eines Morphismus von quasikohärenten Garben ist wieder quasikohärent.
\item Ist $0\to\mathcal{F}'\to\mathcal{F}\to\mathcal{F}''\to 0$ eine kurze exakte Sequenz von $\mathcal{O}_X$"~Moduln und $\mathcal{F}',\mathcal{F}''$ quasikohärent, so ist $\mathcal{F}$ quasikohärent.
\item Ist $X$ noethersch, so gilt (ii) auch für kohärente Garben.
\end{enumerate}

\paragraph{Beweis.} Da Quasikohärenz bzw. Kohärenz eine lokale Eigenschaft ist, können wir ohne Einschränkung $X=\operatorname{Spec}(A)$ als affin annehmen. Nach \hyperref[5.10]{Korollar 5.10} gelten (i) und (ii) für Modulgarben der Form $\widetilde{M}$. Da $M\mapsto \widetilde{M}$ nach \hyperref[5.3]{5.3 (i)} ein exakter, volltreuer Funktor ist, folgt die Aussage über Kern, Kokern und Bild.

Sei nun $0\to\mathcal{F}'\to\mathcal{F}\to\mathcal{F}''\to 0$ eine kurze exakte Sequenz von $\mathcal{O}_X$"~Moduln mit $\mathcal{F}',\mathcal{F}''$ quasikohärent. Nach \hyperref[5.11]{Satz 5.11} ist die folgende Folge exakt:
\[0\to\Gamma(X,\mathcal{F}')\to\Gamma(X,\mathcal{F})\to\Gamma(X,\mathcal{F}'')\to 0 \]
Da $M\mapsto\widetilde{M}$ ein exakter Funktor ist, folgt die Exaktheit von:
\[\begin{tikzcd}
0  \ar[r]& \widetilde{\Gamma(X,\mathcal{F}')} \ar[d] \ar[r]& \widetilde{\Gamma(X,\mathcal{F})} \ar[d] \ar[r]& \widetilde{\Gamma(X,\mathcal{F}'')} \ar[d] \ar[r]& 0\\
0 \ar[r] & \mathcal{F}'  \ar[r]& \mathcal{F}  \ar[r]&  \ar[r]\mathcal{F}''  \ar[r]& 0
\end{tikzcd} \]
Die vertikalen Pfeile links und rechts sind nach \hyperref[5.10]{Korollar 5.10} Isomorphismen. Nach dem 5er Lemma ist somit auch der mittlere Pfeil ein Isomorphismus und $\mathcal{F}$ ist quasikohärent.

Sei nun $X$ noethersch und $\mathcal{F}',\mathcal{F}''$ kohärent. Dann sind $\Gamma(X,\mathcal{F}'),\Gamma(X,\mathcal{F}'')$ endlich erzeugt und somit auch $\Gamma(X,\mathcal{F})$. Es folgt die Kohärenz von $\mathcal{F}$.\qed

\paragraph{Satz 5.13.}\label{5.13} Sei $f:X\to Y$ ein Morphismus von Schemata.
\begin{enumerate}[(i)]
\item Sei $\mathcal{G}$ ein quasikohärenter $\mathcal{O}_Y$"~Modul. Dann ist $f^\ast\mathcal{G}$ quasikohärenter $\mathcal{O}_X$"~Modul.
\item Seien $X,Y$ noethersch und $\mathcal{G}$ kohärenter $\mathcal{O}_Y$"~Modul. Dann ist $f^\ast\mathcal{G}$ kohärent.
\item Sei $X$ noethersch oder $f$ quasikompakt und separiert. Ist $\mathcal{F}$ ein quasikohärenter $\mathcal{O}_X$"~Modul, so ist $f_\ast\mathcal{F}$ quasikohärenter $\mathcal{O}_Y$"~Modul.
\end{enumerate}

\paragraph{Lemma 5.14.}\label{5.14} Sei $Y$ ein affines Schema, $f:X\to Y$ ein separierter Morphismus und $U,V\subset_\text{o}X$ affin. Dann ist $U\cap V$ ein abgeschlossenes Unterschema eines affinen Schemas. Wir werden in \hyperref[5.18]{Korollar 5.18} sehen, dass $U\cap V$ sogar affin ist. Insbesondere ist $U\cap V$ quasikompakt.

\paragraph{Beweis.} Betrachte das kartesische Diagramm:
\[\begin{tikzcd}
X\times_Y X\ar[r, "p_2"]\ar[d, "p_1"'] & X\ar[d, "f"]\\
X\ar[r, "f"'] & Y
\end{tikzcd} \]
Nach \hyperref[3.20]{Lemma~3.20} ist $p_1^{-1}(U)\cap p_2^{-1}(V)=U\times_YV$ affin. Es gilt: \[U\cap V\cong \Delta_{X/Y}(X)\cap U\times_YV\subset U\times_YV\] Nun ist $\Delta_{X/Y}(X)$ abgeschlossen in $X\times_YX$, d.h. $U\cap V$ ist ein abgeschlossenes Un\-ter\-sche\-ma in $U\times_YV$.\qed

\paragraph{Beweis von \hyperref[5.13]{Satz 5.13}.} Für (i) und (ii) ist die Aussage lokal in $X$ und in $Y$. Daher können wir o.B.d.A. $X=\operatorname{Spec}(B)$ und $Y=\operatorname{Spec}(A)$ als affin annehmen. Dann folgt die Behauptung aus der Kategorienäquivalenz \hyperref[5.10]{Korollar 5.10} und \hyperref[5.3]{Satz 5.3 (v)} $f^\ast\widetilde{M} = \widetilde{M\otimes_AB}$.

Für (iii) können wir nur $Y$ als affin annehmen. Sei $X=\bigcup_i U_i$ eine endliche, offene affine Überdeckung, da in beiden Fällen $X$ quasikompakt ist. Setze $U_{ij}=U_i\cap U_j$. In beiden Fällen sind $U_{ij}$ quasikompakt, siehe \hyperref[5.14]{Lemma 5.14} und \hyperref[3.5]{3.5}, \hyperref[3.4]{3.4}. Sei also $U_{ij}=\bigcup_k U_{ijk}$ eine endliche, offene affine Überdeckung. Nach der Garbeneigenschaft haben wir die folgende exakte Sequenz:
\[0\to f_\ast\mathcal{F} \to\bigoplus_if_\ast(\mathcal{F}|_{U_i}) \to\bigoplus_{i,j,k}f_\ast(\mathcal{F}|_{U_{ijk}}) \]
Wegen \hyperref[5.3]{Satz 5.3 (iv)} sind $f_\ast(\mathcal{F}|_{U_i})$ und $f_\ast(\mathcal{F}|_{U_{ijk}})$ quasikohärent. Nach \hyperref[5.12]{Satz 5.12 (ii)} sind $\bigoplus_i f_\ast(\mathcal{F}|_{U_i})$ und $\bigoplus_{i,j,k}f_\ast(\mathcal{F}|_{U_{ijk}})$ quasikohärent. Nach \hyperref[5.12]{5.12 (i)} ist daher auch $f_\ast\mathcal{F}$ quasikohärent.\qed

\paragraph{Bemerkung.} Sind $X,Y$ noethersch und $\mathcal{F}$  kohärent, so muss $f_\ast\mathcal{F}$ nicht notwendigerweise kohärent sein.

\paragraph{Definition 5.16.}\label{5.16} Sei $Y$ ein abgeschlossenes Unterschema von $X$ und $i:Y\hookrightarrow X$ der Inklusionsmorphismus. Dann ist die \textit{zu $Y$ gehörige Idealgarbe} auf $X$.\index{Idealgarbe} wie folgt definiert:
\[\mathcal{J}_Y=\ker(i^\sharp:\mathcal{O}_X\to i_\ast\mathcal{O}_Y) \]

\paragraph{Satz 5.17.}\label{5.17} Sei $X$ ein Schema. Dann gilt:
\begin{enumerate}[(i)]
\item Ist $Y$ ein abgeschlossenes Unterschema von $X$, so ist $\mathcal{J}_Y$ eine quasikohärente Idealgarbe auf $X$.
\item Ist $X$ zusätzlich noethersch, so ist $\mathcal{J}_Y$ kohärent.
\item Jede quasikohärente Idealgarbe auf $X$ bestimmt in eindeutiger Weise ein abgeschlossenes Unterschema.
\end{enumerate}

\paragraph{Beweis.}\begin{enumerate}[(i)]
\item Sei $Y\subset X$ abgeschlossen. Dann ist $i:Y\hookrightarrow X$ ein quasikompakter Morphismus. Wegen \hyperref[4.17]{Satz 4.17 (i)} ist $i$ separiert. Nach \hyperref[5.13]{Satz 5.13 (iii)} ist $i_\ast\mathcal{O}_Y$ quasikohärent, also $\mathcal{J}_Y$ quasikohärent nach \hyperref[5.12]{Satz 5.12 (i)}.
\item Ist $X$ noethersch und $U\subset_\text{o}X$ affin mit $U=\operatorname{Spec}(A)$, so ist auch $A$ noethersch nach \hyperref[3.6]{Satz 3.6}. Daher ist $I=\Gamma(U,\mathcal{J}_Y|_U)$ ein endlich erzeugtes Ideal in $A$. Nach \hyperref[5.9]{Satz 5.9} ist $\mathcal{J}_Y$ kohärent.
\item Sei $\mathcal{J}$ eine quasikohärente Idealgarbe auf $X$. Setze:
\[Y=\operatorname{Supp}(\mathcal{O}_X/\mathcal{J})=\{x\in X\mid (\mathcal{O}_X/\mathcal{J})_x\neq 0 \}\subset X \]
Wir zeigen, dass $(Y,\mathcal{O}_X/\mathcal{J})$ ein abgeschlossenes Unterschema von $X$ ist. Dies ist eine lokale Frage, sei o.B.d.A. $X=\operatorname{Spec}(A)$ affin. Da $\mathcal{J}$ quasikohärent ist, folgt $\mathcal{J}=\widetilde{\mathfrak{a}}$ für ein Ideal $\mathfrak{a}\subset A$. Es gilt:
\begin{align*}
Y&=\operatorname{Supp}(\mathcal{O}_X/\mathcal{J})=\operatorname{Supp}(\widetilde{A/\mathfrak{a}})\\
&= \{\mathfrak{p}\in X\mid (A/\mathfrak{a})_\mathfrak{p}\neq 0\}\\
&=\{\mathfrak{p}\in X\mid\mathfrak{a}\subset\mathfrak{p}\}=\operatorname{Spec}(A/\mathfrak{a})
\end{align*}
Die Eindeutigkeit ist klar.\qed
\end{enumerate}

\paragraph{Korollar 5.18.}\label{5.18} Sei $X=\operatorname{Spec}(A)$ affin. Dann haben wir eine Bijektion:
\[\{\mathfrak{a}\mid \mathfrak{a}\subset A \text{ Ideal}\}\to \{Y\mid Y\subset X \text{ abgeschlossenes Unterschema}\},\ \mathfrak{a}\mapsto \operatorname{Spec}(A/\mathfrak{a}) \]
Insbesondere ist jedes abgeschlossenes Unterschema eines affinen Schemas wieder affin.

\paragraph{Definition 5.19.}\label{5.19} Sei $S=\bigoplus_d S_d$ ein graduierter Ring. Ein $S$"~Modul heißt \textit{graduierter $S$"~Modul}\index{Modul!graduiert}, falls $M=\bigoplus_d M_d$ mit $S_d\cdot M_e\subset S_{d+e}$ gilt. Sei $\ell\in\mathbb{Z}$. Dann definieren wir den \textit{getwisteten $S$"~Modul}\index{Modul!getwistet} $M(\ell)$ von $M$ durch:
\[M(\ell)_d = M_{d+\ell} \]

\paragraph{Definition 5.20.}\label{5.20} Sei $S$ ein graduierter Ring und $M$ ein graduierter $S$"~Modul. Die zu $M$ assoziierte Garbe $\widetilde{M}$ auf $\operatorname{Proj}(S)$ ist wie folgt definiert: Sei $U\subset_\text{o}\operatorname{Proj}(S)$ und setze $\widetilde{M}(U)$ als die Menge aller Abbildungen $s:U\to\coprod_{\mathfrak{p}\in U}M_{(\mathfrak{p})}$, so dass:
\begin{enumerate}[(i)]
\item Für alle $\mathfrak{p}\in U$ gilt $s(\mathfrak{p})\in M_{(\mathfrak{p})}$.
\item Für alle $\mathfrak{p}\in U$ existiert eine offene Umgebung $V$ von $\mathfrak{p}$ mit $V\subset U$ und homogene Elemente $m\in M,\ f\in S$ mit $\deg(m)=\deg(f)$ derart, dass für alle $\mathfrak{q}\in V$ stets $f\not\in\mathfrak{q}$ und $s(\mathfrak{q})=\frac{m}{f}$ in $M_{(\mathfrak{q})}$ gilt.
\end{enumerate}
$\widetilde{M}$ wird zu einer Garbe mit den gewöhnlichen Restriktionsabbildungen.

\paragraph{Satz 5.21.}\label{5.21} Sei $S$ ein graduierter Ring, $M$ ein graduierter $S$"~Modul und $X=\operatorname{Proj}(S)$. Dann gilt:
\begin{enumerate}[(i)]
\item $(\widetilde{M})_\mathfrak{p}=M_{(\mathfrak{p})}$ für alle $\mathfrak{p}\in X$.
\item Für alle homogene Elemente $f\in S_+$ ist $\widetilde{M}|_{D_+(f)}\cong\widetilde{M_{(f)}}$ bzgl. des Isomorphismus' $D_+(f)\cong\operatorname{Spec}S_{(f)}$.
\item $\widetilde{M}$ ist ein quasikohärenter $\mathcal{O}_X$"~Modul. Ist $X$ noethersch und $M$ endlich erzeugt, so ist $\widetilde{M}$ kohärent.
\end{enumerate}

\paragraph{Beweis.} (i) und (ii) sind analog zu \hyperref[2.23]{Satz 2.23}. (iii) folgt aus (ii).\qed

\paragraph{Definition 5.22.}\label{5.22} Sei $S$ ein graduierter Ring, $X=\operatorname{Proj}(S)$ und $\mathcal{F}$ ein $\mathcal{O}_X$"~Modul. Für $n\in\mathbb{Z}$ definieren wir die \textit{getwistete Garbe}\index{Garbe!getwistet} $\mathcal{F}(n)$ von $\mathcal{F}$ wie folgt:
\[\mathcal{F}(n)=\mathcal{F}\otimes_{\mathcal{O}_X}\mathcal{O}_X(n),\quad \mathcal{O}_X(n)=\widetilde{S(n)} \]

\paragraph{Satz 5.23.}\label{5.23} Sei $S$ ein graduierter Ring und $X=\operatorname{Proj}(S)$, wobei $S$ als $S_0$"~Algebra von $S_1$ erzeugt wird. Dann gilt:
\begin{enumerate}[(i)]
\item $\mathcal{O}_X(n)$ ist eine invertierbare Garbe auf $X$.
\item Sind $M,N$ graduierte $S$"~Moduln, so ist $\widetilde{M\otimes_SN}\cong\widetilde{M}\otimes_{\mathcal{O}_X}\widetilde{N}$. Insbesondere gilt $\widetilde{M(n)}\cong\widetilde{M}(n)$ und $\mathcal{O}_X(n+m)\cong\mathcal{O}_X(n)\otimes\mathcal{O}_X(m)$.
\item Sei $T$ ein weiterer graduierter Ring, der von $T_1$ als $T_0$"~Algebra erzeugt wird und $\varphi:S\to T$ ein Homomorphismus graduierter Ringe. Sei $U\subset_\text{o} Y=\operatorname{Proj}(T)$ und $f:U\to X$ der durch $\varphi$ induzierte Morphismus. Dann gilt für jeder graduierte $S$"~Modul $M$ und jeder graduierte $T$"~Modul $N$:
\[f^\ast(\widetilde{M})\cong\widetilde{M\otimes_ST}|_U,\quad f_\ast(\widetilde{N}|_U)\cong (\widetilde{_SN}) \]
Insbesondere gilt $f^\ast(\mathcal{O}_X(n))\cong\mathcal{O}_Y(n)|_U$ und $f_\ast(\mathcal{O}_X(n)|_U)=(f_\ast\mathcal{O}_U)(n)$.
\end{enumerate}

\paragraph{Beweis.}\begin{enumerate}[(i)]
\item Sei $f\in S_1$ und betrachte $\mathcal{O}_X(n)|_{D_+(f)}\cong\widetilde{S(n)_{(f)}}$ auf $\operatorname{Spec}S_{(f)}$. Es ist $S(n)_{(f)}$ freier $S_{(f)}$"~Modul vom Rang $1$ via dem Isomorphismus $(S_f)_0\to (S_f)_n,\ s\mapsto f^ns$ für alle $n$. Da $S$ von $S_1$ als $S_0$"~Algebra erzeugt wird, gilt $X=\bigcup_{f\in S_1}D_+(f)$. Daher ist $\mathcal{O}_X(n)$ invertierbar.
\item Sei $f\in S_1$. Es gilt $(M\otimes_SN)_{(f)}=M_{(f)}\otimes_{S_{(f)}}N_{(f)}$. Da $S$ von $S_1$ erzeugt wird, folgt $\widetilde{M\otimes_SN}\cong\widetilde{M}\otimes_{\mathcal{O}_X}\widetilde{N}$.
\item Analog wie im affinen Fall.\qed
\end{enumerate}

\paragraph{Definition 5.24.}\label{5.24} Sei $S$ ein graduierter Ring, $X=\operatorname{Proj}(S)$ und $\mathcal{F}$ ein $\mathcal{O}_X$"~Modul. Der zu $\mathcal{F}$ \textit{assoziierte graduierte $S$"~Modul}\index{Modul!assoziiert} $\Gamma_\ast(\mathcal{F})$ ist definiert als die Gruppe
\[\Gamma_\ast(\mathcal{F})=\bigoplus_{n\in\mathbb{Z}}\Gamma(X,\mathcal{F}(n)) \]
mit der folgenden $S$"~Wirkung: Für $s\in S_d\cong\Gamma(X,\mathcal{O}_X(d))$ und $t\in\Gamma(X,\mathcal{F}(n))$ setze $st= s\otimes t\in\Gamma(X,\mathcal{F}(n)\otimes_{\mathcal{O}_X}\mathcal{O}_X(d))=\Gamma(X,\mathcal{F}(n+d))$.

\paragraph{Satz 5.25.}\label{5.25} Sei $A$ ein Ring und $X=\mathbf{P}_A^r$ mit $r\geq 1$ und $S=A[X_0,\ldots,X_r]$. Dann gilt:
\[\Gamma_\ast(\mathcal{O}_X)= S \]

\paragraph{Beweis.} Sei $X=\bigcup_{i=0}^r D_+(X_i)$. Ein $t\in\Gamma(X,\mathcal{O}_X(n))$ entspricht eine Familie $t_i\in\Gamma(D_+(X_i),\mathcal{O}_X(n)),\ i=1,\ldots,r$ mit $t_i=t_j$ auf $D_+(X_iX_j)$. $t_i$ ist ein homogenes Element $s_i\in S_{X_i}$ vom Grad $n$ und $t_i|_{D_+(X_iX_j)}$ entspricht dem Bild von $s_i$ in $S_{X_iX_j}$. Es folgt:
\[\Gamma_\ast(\mathcal{O}_X)=\Big\{ (t_0,\ldots,t_r)\in\prod_{i=0}^r S_{X_i}\mid t_i=t_j\text{ auf } S_{X_iX_j}\text{ für alle }i,j \Big\} \]
Da keine $X_i$ Nullteiler sind, haben wir Inklusionen $S\hookrightarrow S_{X_i}\hookrightarrow S_{X_iX_j}\hookrightarrow S_{X_0\cdots X_r}$ und:
\[\Gamma_\ast(\mathcal{O}_X)=\bigcap_{i=0}^r S_{X_i}\subset S_{X_0\cdots X_r} \]
Jedes homogene $t\in S_{X_0\cdots X_r}$ lässt sich eindeutig in der folgenden Form schreiben:
\[t=X_0^{i_0}\cdots X_r^{i_r}f,\quad i_j\in\mathbb{Z}\]
wobei $f\in S$ ein homogenes Element ist, das durch kein $X_i$ teilbar ist. Es ist $t$ genau dann in $S_{X_i}$, wenn $i_j\geq 0$ für alle $j\neq i$ gilt. Also ist $\bigcap S_{X_i}=S$.\qed

\paragraph{Lemma 5.26.}\label{5.26} Sei $X$ ein Schema, $\mathcal{L}$ eine invertierbare Garbe auf $X$ und $f\in\Gamma(X,\mathcal{L})$. Setze $X_f=\{x\in X\mid f_x\not\in\mathfrak{m}_x\mathcal{L}_x\}\subset_\text{o}X$ und sei $\mathcal{F}$ eine quasikohärente Garbe auf $X$.
\begin{enumerate}[(i)]
\item Sei $X$ quasikompakt und $s\in\Gamma(X,\mathcal{F})$ mit $s|_{X_f}=0$. Dann gibt es ein $n>0$, so dass $f^ns=0\in\Gamma(X,\mathcal{F}\otimes\mathcal{L}^{\otimes n})$.
\item Sei $X=\bigcup U_i$ eine endliche, offene affine Überdeckung, so dass $\mathcal{L}|_{U_i}$ für alle $i$ frei und $U_i\cap U_j$ für alle $i,j$ quasikompakt sind. Zu $t\in\Gamma(X_f,\mathcal{F})$ gibt es ein $n>0$, so dass sich $f^nt\in\Gamma(X_f,\mathcal{F}\otimes\mathcal{L}^{\otimes n})$ zu einem globalen Schnitt auf ganz $X$ fortsetzen lässt.
\end{enumerate}

\paragraph{Bemerkung 5.27.}\label{5.27} Voraussetzungen in \hyperref[5.26]{Lemma 5.26 (i) und (ii)} sind erfüllt, wenn $X$ noethersch ist, oder wenn $X$ quasikompakt und separiert ist.

\paragraph{Beweis von \hyperref[5.26]{Lemma 5.26}.} \begin{enumerate}[(i)]
\item Sei $X=\bigcup U_i$ eine endliche, offene affine Überdeckung mit $\mathcal{L}|_{U_i}$ frei. Betrachte $U=U_i$ und sei $\psi:\mathcal{L}|_U\stackrel{\sim}{\to}\mathcal{O}_U$ ein Isomorphismus. Da $\mathcal{F}$ quasikohärent ist, folgt $\mathcal{F}|_U\cong\widetilde{M}$ für ein $A$"~Modul $M$, wobei $U=\operatorname{Spec}(A)$. Für ein $s\in\Gamma(X,\mathcal{F})$ ist $s|_U\in M$. Setze $g:=\psi(f|_U)\in A$. Es ist $X_f\cap U=D(g)$ und $s|_{X_f}=0$. Nach \hyperref[5.8]{Lemma 5.8 (i)} gibt es ein $n>0$ mit $g^ns=0\in M$. Der Isomorphismus
\[\operatorname{id}\otimes \psi^{\otimes n}:\mathcal{F}\otimes \mathcal{L}^{\otimes n}|_U\to\mathcal{F}|_U \]
liefert $0=f^ns=\Gamma(U, \mathcal{F}\otimes\mathcal{L}^{\otimes n})$ für alle $U=U_i$. Wählt man $n$ so groß, dass die obige Aussage für alle $U_i$ gilt, so folgt $f^ns=0$ auf $X$.
\item Analog zu (i) mit \hyperref[5.8]{Lemma 5.8 (ii)}.\qed
\end{enumerate}

\paragraph{Satz 5.28.}\label{5.28} Sei $S$ ein graduierter Ring, der durch $S_1$ als $S_0$"~Algebra endlich erzeugt wird. Sei $X=\operatorname{Proj}(S)$ und $\mathcal{F}$ ein quasikohärenter $\mathcal{O}_X$"~Modul. Dann gibt es einen natürlichen Isomorphismus:
\[\beta:\widetilde{\Gamma_\ast(\mathcal{F})}\stackrel{\sim}{\to}\mathcal{F} \]

\paragraph{Beweis.} Es ist $X=\bigcup_{f\in S_1}D_+(f)$ eine endliche Vereinigung. Für $f\in S_1$ definiere:
\[\beta_f:\widetilde{\Gamma_\ast(\mathcal{F})_{(f)}}=\widetilde{\Gamma_\ast(\mathcal{F})}|_{D_+(f)}\to\mathcal{F}|_{D_+(f)} \]
durch $\bigoplus_d\Gamma(D_+(f),\mathcal{F}(d))\to\Gamma(D_+(f),\mathcal{F})$, das induziert wird durch:
\[\Gamma(D_+(f),\mathcal{F}(d))\to\Gamma(D_+(f),\mathcal{F}(d)\otimes \mathcal{O}_X(-d))= \Gamma(D_+(f),\mathcal{F}),\ \frac{m}{f^d}\mapsto m\otimes f^{-d} \]
Wir erhalten eine Abbildung $\beta:\widetilde{\Gamma_\ast(\mathcal{F})}\to\mathcal{F}$. Wir zeigen nun, dass alle $\beta_f$ Isomorphismen sind. Es genügt zu zeigen, dass $\Gamma_\ast(\mathcal{F})_{(f)}\to\Gamma(D_+(f),\mathcal{F})$ Isomorphismen sind. \hyperref[5.26]{Lemma 5.26 (i)} liefert die Injektivität und \hyperref[5.26]{Lemma 5.26 (ii)} die Surjektivität.\qed

\paragraph{Korollar 5.29.}\label{5.29} Sei $A$ ein Ring. Dann gilt:
\begin{enumerate}[(i)]
\item Ist $Y\hookrightarrow\mathbf{P}_A^r$ ein abgeschlossenes Unterschema, so existiert ein homogenes Ideal $I\subset S=A[X_0,\ldots,X_r]$, so dass $Y=\operatorname{Proj}(S/I)\hookrightarrow\operatorname{Proj}(S)=X$.
\item Sei $Y$ ein Schema über $\operatorname{Spec}(A)$. Dann ist $Y$ genau dann projektiv, wenn $Y\cong\operatorname{Proj}(S)$ für einen graduierten Ring $S$, der von $S_1$ als $S_0=A$"~Algebra endlich erzeugt wird.
\end{enumerate}

\paragraph{Beweis.} \begin{enumerate}[(i)]
\item Sei $\mathcal{J}_Y\subset\mathcal{O}_X$ die Idealgarbe von $Y$ auf $X=\mathbf{P}_A^r$. Da $\mathcal{J}_Y(d)\subset\mathcal{O}_X(d)$, folgt $\Gamma_\ast(\mathcal{J}_Y)\subset\Gamma_\ast(\mathcal{O}_X)$. Nach \hyperref[5.25]{Satz 5.25} ist $\Gamma_\ast(\mathcal{O}_X)=S$, d.h. $I=\Gamma_\ast(\mathcal{J}_Y)$ ist ein homogenes Ideal in $S$. Setze $Y'=\operatorname{Proj}(S/I)$. $Y'$ ist ein abgeschlossenes Unterschema von $X$ mit Idealgarbe $\mathcal{J}_{Y'}=\widetilde{I}$. Da $\mathcal{J}_Y$ nach \hyperref[5.17]{Satz 5.17 (i)} quasikohärent ist, folgt $\mathcal{J}_Y\cong\widetilde{\Gamma_\ast(\mathcal{J}_Y)}$ nach \hyperref[5.28]{Satz 5.28}. Nun gilt:
\[\mathcal{J}_Y\cong\widetilde{\Gamma_\ast(\mathcal{J}_Y)}=\widetilde{I}=\mathcal{J}_{Y'} \]
Nach \hyperref[5.17]{Satz 5.17} folgt $Y=Y'$, also ist $Y$ das abgeschlossene Unterschema, das durch $I$ definiert ist.
\item Es gilt:
\begin{align*}
Y\text{ projektiv} &\iff Y\text{ ist abgeschlossenes Unterschema von }\mathbf{P}_A^r\text{ für ein }r\\
&\iff Y\cong\operatorname{Proj}(S'/I)\text{ für ein homogenes Ideal }I\subset S' = A[X_0,\ldots,X_r]
\end{align*}
Wir zeigen zunächst, dass $I$ und $I'=\bigoplus_{d\geq d_0}I_d$ dasselbe abgeschlossene Unterschema bestimmen. Dafür zeigen wir $\mathfrak{p}\supset I$, wenn $\mathfrak{p}\supset I'$ für alle $\mathfrak{p}\in\operatorname{Proj}(S')$. Wegen $\mathfrak{p}\not\supset S_+$ gibt es ein $x_i\not\in\mathfrak{p}$. Sei nun $x\in I_r,\ r<d_0$, dann ist $x_i^{d_0-r}x\in I_{d_0}\in\mathfrak{p}$. Da $\mathfrak{p}$ ein Primideal ist, folgt $x\in\mathfrak{p}$.

Somit können wir o.B.d.A. $I\subset S'_+$ annehmen. Also ist $A=(S'/I)_0$ und $S=S'/I$ wird als $A$"~Algebra von $S_1$ endlich erzeugt. 

Umgekehrt ist jeder graduierte Ring $S$, der von $S_1$ als $S_0=A$"~Algebra endlich erzeugt ist, Quotient des Polynomrings und $\operatorname{Proj}(S)$ ist projektiv.\qed
\end{enumerate}

\paragraph{Definition 5.30.}\label{5.30} Sei $Y$ ein Schema. Der kanonische Morphismus $g:\mathbf{P}_Y^r=\mathbf{P}_\mathbb{Z}^r\times Y\to\mathbf{P}_\mathbb{Z}^r$ definiert die \textit{getwistete Garbe} $\mathcal{O}(1)$ auf $\mathbf{P}_Y^r$ durch:
\[\mathcal{O}(1)=g^\ast\mathcal{O}(1) \]

\paragraph{Bemerkung.} Ist $Y=\operatorname{Spec}(A)$ affin, so ist $\mathcal{O}(1)$ die bereits in \hyperref[5.22]{Definition 5.22} definierte Garbe auf $\mathbf{P}_A^r$.

\paragraph{Definition 5.31.}\label{5.31} Sei $X$ ein Schema über $Y$. Eine invertierbare Garbe $\mathcal{L}$ auf $X$ heißt \textit{sehr ampel}\index{$\mathcal{O}_X$-Modul!sehr ampel} bzgl. $Y$, wenn es eine Immersion $i:X\hookrightarrow\mathbf{P}_Y^r$ für ein $r$ gibt, so dass $i^\ast(\mathcal{O}(1))\cong\mathcal{L}$.

\paragraph{Satz 5.32.}\label{5.32} Sei $Y$ ein noethersches Schema und $X$ ein Schema über $Y$. Dann ist $X$ genau dann projektiv über $Y$, wenn:
\begin{enumerate}[(i)]
\item $X$ ist eigentlich über $Y$.
\item Es gibt eine sehr ample Garbe auf $X$ bzgl. $Y$.
\end{enumerate}

\paragraph{Beweis.} Sei $X$ projektiv. Dann folgt (i) aus \hyperref[4.24]{Theorem~4.24} und es gibt eine abgeschlossene Immersion $i:X\to\mathbf{P}_Y^r$ für ein $r$, so dass $i^\ast\mathcal{O}(1)$ sehr ampel ist.

Sei umgekehrt $i:X\hookrightarrow\mathbf{P}_Y^r$ eine Immersion und $\mathcal{L}\cong i^\ast(\mathcal{O}(1))$ eine sehr ample Garbe auf $X$ bzgl. $Y$. Betrachte das kartesische Quadrat:
\[\begin{tikzcd}
\mathbf{P}^r_Y = Y\times \mathbf{P}_\mathbb{Z}^r \ar[r]\ar[d]& Y\ar[d]\\
\mathbf{P}_\mathbb{Z}^r\ar[r] & \mathbb{Z}
\end{tikzcd} \]
$\mathbf{P}_\mathbb{Z}^r\to\mathbb{Z}$ ist separiert, da projektiv, also ist auch der Basiswechsel $\mathbf{P}_Y^r\to Y$ separiert. Ferner ist $X\to Y$ eigentlich, nach \hyperref[4.22]{Korollar 4.22 (v)} ist auch $X\hookrightarrow \mathbf{P}_Y^r$ eigentlich, also insbesondere abgeschlossen.\qed

\paragraph{Definition 5.33.}\label{5.33} Sei $X$ ein Schema und $\mathcal{F}$ ein $\mathcal{O}_X$"~Modul. $\mathcal{F}$ heißt \textit{von globalen Schnitten erzeugt}\index{von globalen Schnitten erzeugt}, wenn es eine Familie von globalen Schnitten $s_i\in\Gamma(X,\mathcal{F}),\ i\in I$ gibt, so dass für alle $x\in X$ die Bilder der $s_i$ den Halm $\mathcal{F}_x$ als $\mathcal{O}_X$"~Modul erzeugen. Dies ist äquivalent zu: Es gibt einen surjektiven Garbenmorphismus $\bigoplus_{i\in I}\mathcal{O}_X\to \mathcal{F}$.

\paragraph{Beispiel 5.34.}\label{3.34} Sei $X=\operatorname{Spec}(A)$ und $\mathcal{F}=\widetilde{M}$ für ein $A$"~Modul $M$. Dann wird $\mathcal{F}$ von globalen Schnitten erzeugt; jedes Erzeugendensystem von $M$ als $A$"~Modul liefern solche Schnitte. Die Surjektion $A^{(I)}\to M$ induziert surjektives $\mathcal{O}_X^{(I)}\to\mathcal{F}$.

\paragraph{Beispiel 5.35.}\label{3.35} Sei $X=\operatorname{Proj}(S)$ mit einem graduierten Ring $S$, der von $S_1$ als $S_0$"~Algebra erzeugt wird. Dann liefern die Elemente aus $S_1$ globale Schnitte von $\mathcal{O}_X(1)$ und erzeugen diesen quasikohärenten Modul.

\paragraph{Lemma 5.36.}\label{5.36} \begin{enumerate}[(i)]
\item Abgeschlossene Immersionen sind endliche Morphismen.
\item Sei $f:X\to Y$ ein endlicher Morphismues noetherscher Schemata und $\mathcal{F}$ ein kohärenter $\mathcal{O}_X$"~Modul. Dann ist $f_\ast\mathcal{F}$ kohärent.
\end{enumerate}

\paragraph{Beweis.}\begin{enumerate}[(i)]
\item Sei $f:X\to Y$ eine abgeschlossene Immersion. Sei $V=\operatorname{Spec}(B)\subset_\text{o}Y$. Es ist $f^{-1}(V)=X\times_YV\to V$ als Basiswechsel von $f$ eine abgeschlossene Immersion. Somit ist $f^{-1}(V)\cong\operatorname{Spec}(B/I)$ für ein Ideal $I\subset B$ affin und ferner $B/I$ ein endlich erzeugter $B$"~Modul.
\item Nach \hyperref[5.13]{Satz 5.13 (iii)} ist $f_\ast\mathcal{F}$ quasikohärent. Sei $V=\operatorname{Spec}(B)\subset_\text{o}Y$. Da $f$ endlich ist, ist $f^{-1}(V)=\operatorname{Spec}(A)$ affin und es gilt $\mathcal{F}|_{f^{-1}(V)}=\widetilde{N}$ für ein endlich erzeugter $A$"~Modul $N$. Nach \hyperref[5.3]{Satz 5.3 (iv)} gilt:
\[f_\ast\mathcal{F}|_V = f_\ast\widetilde{N} = \widetilde{_BN} \]
Da $f$ endlich ist, ist $A$ ein endlich erzeugter $B$"~Modul und somit auch $_BN$.\qed
\end{enumerate}

\paragraph{Satz 5.37.}\label{5.37} \textit{(Serre)} Sei $X$ ein projektives Schema über $\operatorname{Spec}(A)$ mit $A$ noethersch. Sei $\mathcal{O}_X(1)$ eine sehr ample Garbe auf $X$ und $\mathcal{F}$ ein kohärenter $\mathcal{O}_X$"~Modul. Dann existiert ein $n_0\in\mathbb{Z}$, so dass für alle $n\geq n_0$ der getwistete $\mathcal{O}_X$"~Modul $\mathcal{F}(n)$ von endlich vielen globalen Schnitten erzeugt wird.

\paragraph{Beweis.} Sei $i:X\hookrightarrow \mathbf{P}_A^r$ eine abgeschlossene Immersion mit $\mathcal{O}_X(1)\cong i^\ast(\mathcal{O}(1))$. Nach \hyperref[5.36]{Lemma 5.36} ist $i_\ast\mathcal{F}$ kohärent auf $\mathbf{P}_A^r$ und nach \hyperref[5.23]{Satz 5.23 (iii)} gilt $(i_\ast\mathcal{F})(n)=i_\ast(\mathcal{F}(n))$. Wird nun $i_\ast\mathcal{F}(n)$ von endlich vielen globalen Schnitten erzeugt, so ist dies auch für $\mathcal{F}(n)$ der Fall, betrachte dafür:
\[\Gamma(\mathbf{P}_A^r, i_\ast\mathcal{F}(n)) = \mathcal{F}(n)(i^{-1}(\mathbf{P}_A^r))=\Gamma(X,\mathcal{F}(n)) \]
\[(i_\ast\mathcal{F}(n))_x = \begin{cases}
\mathcal{F}(n)_x, & x\in i(X)\\
0, &\text{sonst}
\end{cases} \]
Sei also o.B.d.A. $X=\mathbf{P}_A^r=\operatorname{Proj}A[X_0,\ldots,X_r]$. Es ist $X=\bigcup_{i=0}^r D_+(X_i)$. Für jedes $i$ ist $\mathcal{F}|_{D_+(X_i)}\cong\widetilde{M_i}$ für ein endlich erzeugter Modul $M_i$ über $B_i=A\big[\frac{X_0}{X_i},\ldots,\widehat{\frac{X_i}{X_i}},\ldots,\frac{X_r}{X_i}\big]$. Sei $(s_{ij})_j$ ein Erzeugendensystem von $M_i$. Wegen \hyperref[5.26]{Lemma 5.26} existiert ein $n_0>0$, so dass für alle $n\geq n_0$ die Schnitte $X_i^ns_{ij}$ sich zu globalen Schnitten $t_{ij}$ von $\mathcal{F}(n)$ liften lassen. Wir können $n_0$ unabhängig von $i$ und $j$ wählen. Sei $\mathcal{F}(n)|_{D_+(X_i)}\cong\widetilde{M_i'}$ für ein $B_i$"~Modul $\widetilde{M_i'}$. Die Abbildungen $X_i^n:\mathcal{F}\to\mathcal{F}(n)$ induzieren Isomorphismen $M_i\to M_i'$. Da $\{X_i^ns_{ij}\mid j\}$ ganz $M_i'$ erzeugen, erzeugen $t_{ij}\in\Gamma(X,\mathcal{F}(n))$ als globale Schnitte ganz $\mathcal{F}(n)$.\qed

\paragraph{Korollar 5.38.}\label{5.38} Sei $X$ ein projektives Schema über $\operatorname{Spec}(A)$ mit $A$ noethersch. Dann gibt es für jeden kohärenten $\mathcal{O}_X$"~Modul $\mathcal{F}$ eine Surjektion $\mathcal{O}_X(n)^N\to\mathcal{F}$ mit $n,N\in\mathbb{Z}$.

\paragraph{Beweis.} Nach \hyperref[5.37]{Satz 5.37} gibt es eine Surjektion $\mathcal{O}_X^N\to\mathcal{F}(n)$. Tensorieren mit $\mathcal{O}_X(-n)$ gibt die Behauptung.\qed

\paragraph{Satz 5.39.}\label{5.39} Sei $k$ ein Körper, $A$ eine endlich erzeugte $k$"~Algebra und $X$ projektives Schema über $\operatorname{Spec}(A)$. Ferner sei $\mathcal{F}$ ein kohärenter $\mathcal{O}_X$"~Modul. Dann ist $\Gamma(X,\mathcal{F})$ ein endlich erzeugter $A$"~Modul.

\paragraph{Beweis.} Wir werden diesen Satz später kohomologisch beweisen.

\paragraph{Korollar 5.40.}\label{5.40} Sei $f:X\to Y$ ein projektiver Morphismus von Schemata von endlichem Typ über einem Körper $k$. Ist $\mathcal{F}$ kohärent auf $X$, so ist auch $f_\ast\mathcal{F}$ kohärent auf $Y$. Insbesondere ist für $A=k$ der Modul $\Gamma(X,\mathcal{F})$ ein endlich dimensionierter $k$"~Vektorraum.

\paragraph{Beweis.} Sei o.B.d.A. $Y=\operatorname{Spec}(A)$ affin, wobei $A$ eine endlich erzeugte $k$"~Algebra ist. Da $f$ projektiv ist, ist $f$ eigentlich und somit separiert und von endlichem Typ, also quasikompakt. Wegen \hyperref[5.13]{Satz 5.13 (iii)} ist $f_\ast\mathcal{F}$ quasikohärent. Es gilt:
\[f_\ast\mathcal{F} = \widetilde{\Gamma(Y,f_\ast\mathcal{F})}=\widetilde{\Gamma(X,\mathcal{F})} \]
Aber $\Gamma(X,\mathcal{F})$ ist endlich erzeugter $A$"~Modul nach \hyperref[5.39]{Satz 5.39}.\qed

\section{Divisoren}

\paragraph{Definition 6.1.}\label{6.1} \begin{enumerate}[(i)]
\item Ein noetherscher lokaler Ring $(R,\mathfrak{m})$ heißt \textit{regulär}\index{regulär}, falls für $k=R/\mathfrak{m}$ gilt:
\[\dim(R)=\dim_k (\mathfrak{m}/\mathfrak{m}^2)\]
Ist $R$ ein noetherscher lokaler Ring, so gilt stets $\dim(R)\leq\dim_k(\mathfrak{m}/\mathfrak{m}^2)$.
\item Ein Schema $X$ heißt \textit{regulär in Kodimension $1$}\index{regulär}, falls jeder Halm $\mathcal{O}_{X,x}$ von $X$ mit $\dim(\mathcal{O}_{X,x})=1$ regulär ist.
\end{enumerate}

\paragraph{Definition 6.3.}\label{6.3}
Ein Ring heißt \textit{normal}\index{Ring!normal}, wenn er ganzabgeschlossen und nullteilerfrei ist.
Ein Schema heißt \textit{normal}\index{Schema!normal}, wenn seine Halme normal sind.

\paragraph{Theorem 6.4.}\label{6.4} Sei $R$ ein noetherscher, normaler Ring und $\mathfrak{p}$ ein Primideal der Hö\-he $1$. Dann ist $R_\mathfrak{p}$ regulär. Genauer: Sei $(R,\mathfrak{m})$ ein noetherscher lokaler Ring der Dimension $1$. Dann sind folgende Aussagen äquivalent:
\begin{enumerate}[(i)]
\item $R$ ist ein diskreter Bewertungsring.
\item $R$ ist ganzabgeschlossen.
\item $R$ ist regulär.
\item $\mathfrak{m}$ ist ein Hauptideal.
\end{enumerate}

\paragraph{Beweis.} Siehe z.B. Matsumura: \glqq Commutative Algebra\grqq, Theorem 3.9 und Ati\-yah-Mac\-Do\-nald: \glqq Introduction to Commutative Algebra\grqq, Proposition 9.2.\qed

\paragraph{Definition.} Ein Schema habe die Eigenschaft $(\star)$, wenn es noethersch, separiert, integer und regulär in Kodimension $1$ ist.

\paragraph{Definition 6.5.}\label{6.5} Sei $X$ ein Schema mit $(\star)$. Dann gilt:
\begin{enumerate}[(i)]
\item Ein \textit{Primdivisor}\index{Primdivisor} auf $X$ ist ein abgeschlossenes, integres Unterschema der Kodimension $1$.
\item Ein \textit{Weil-Divisor}\index{Divisor!Weil} ist ein Element der freien abelschen Gruppe $\operatorname{Div}(X)$, die von den Primdivisoren erzeugt wird. Wir schreiben ein Divisor als $D=\sum_i n_iY_i$ mit Primdivisoren $Y_i$ und $n_i\in\mathbb{Z}$ mit $n_i=0$ für fast alle $i$. Ein solcher Divisor heißt \textit{effektiv}\index{Divisor!effektiv}, falls alle $n_i\geq 0$ sind.
\item Sei $Y$ ein Primdivisor auf $X$ und $\eta$ ein generischer Punkt in $Y$. Dann ist $\mathcal{O}_{X,\eta}$ nach \hyperref[6.4]{Theorem 6.4} ein diskreter Bewertungsring mit Quotientenkörper $K$, der \textit{Funktionenkörper}\index{Funktionenkörper} von $X$. Sei $v_Y:K^\times\to\mathbb{Z}$ die zugehörige diskrete Bewertung. Sei $f\in K^\times$. Ist $v_Y(f)>0$, so sagen wir, dass $f$ eine \textit{Nullstelle}\index{Nullstelle} entlang $Y$ von der Ordnung $v_Y(f)$ hat. Ist $v_Y(f)<0$, so sagen wir, dass $f$ ein \textit{Pol}\index{Pol} entlang $Y$ von der Ordnung $-v_Y(f)$ besitzt.
\end{enumerate}

\paragraph{Lemma 6.6.}\label{6.6} Sei $X$ ein Schema mit $(\star)$ und $f\in K^\times$. Dann ist $v_Y(f)=0$ für fast alle Primdivisoren $Y$.

\paragraph{Beweis.} Sei $\varnothing \neq U = \operatorname{Spec}(A)\subset_\text{o}X$ affin mit $f|_U$ \textit{regulär}, d.h. $f|_U\in\mathcal{O}_X(U)$. Sei $Z=X\setminus U\subsetneq X$ abgeschlossen. Da $X$ noethersch ist, enthält $Z$ höchstens endlich viele Primdivisoren von $X$, alle anderen treffen $U$. Es genügt also zu zeigen, dass es nur endlich viele Primdivisoren $Y$ in $U$ gibt mit $v_Y(f)\neq 0$, d.h. $v_Y(f)>0$. Es gilt mit $Y=\overline{\{ \eta\}}$:
\[v_Y(f)>0\iff f\in\mathcal{O}_{U,\eta}=A_\eta\iff f\in\eta\iff Y=V(\eta)\subset V(f)\subsetneq U \]
$V(f)$ enthält aber nur endlich viele abgeschlossene irreduzible Teilmengen der Kodimension $1$.\qed

\paragraph{Definition 6.7.}\label{6.7} Sei $X$ ein Schema mit $(\star)$ und $f\in K^\times$. Der Divisor $\operatorname{div}(f)$ von $f$ ist definiert als:
\[\operatorname{div}(f)=\sum_Y v_Y(f)Y \]
wobei $Y$ über die Primdivisoren in $X$ läuft. Diese Summe ist nach \hyperref[6.6]{Lemma 6.6} endlich und somit wohldefiniert. Jeder Divisor der Form $\operatorname{div}(f)$ heißt \textit{Hauptdivisor} oder \textit{prinzipal}.\index{Hauptdivisor}\index{Divisor!prinzipal}.

\paragraph{Bemerkung 6.8.}\label{6.8} Sei $f,g\in K^\times$. Dann gilt:
\[\operatorname{div}\big(\tfrac{f}{g}\big)=\operatorname{div}(f)-\operatorname{div}(g) \]
Somit ist $K^\times\to\operatorname{Div}(X),\ f\mapsto\operatorname{div}(f)$ ein Gruppenhomomorphismus, dessen Bild gerade die Gruppe der Hauptdivisoren in $X$ sind.

\paragraph{Definition 6.9.}\label{6.9} Sei $X$ ein Schema mit $(\star)$. Zwei Divisoren $D,D'$ heißen \textit{linear äquivalent}\index{linear äquivalent} $D\sim D'$, wenn $D-D'$ ein Hauptdivisor ist. Die Gruppe der zugehörigen Äquivalenzklassen $\operatorname{Cl}(X)$ heißt \textit{Divisorenklassengruppe}\index{Divisorenklassengruppe}. Wir haben eine exakte Folge:
\[K^\times\to\operatorname{Div}(X)\to\operatorname{Cl}(X)\to 0 \]

\paragraph{Satz 6.10.}\label{6.10} Sei $A$ ein noetherscher, nullteilerfreier Ring. Dann gilt:
\[A\text{ ist faktoriell}\iff\operatorname{Cl}(\operatorname{Spec}(A))=0 \]

\paragraph{Beweis.} Siehe z.B. Bourbaki: Algèbre Commutative, Chapitre 7 §3 Proposition 2.\qed

\paragraph{Beispiel 6.11.}\label{6.11} \begin{enumerate}
\item Sei $X=\mathbf{A}_k^n$ für ein Körper $k$. Dann gilt $\operatorname{Cl}(X)=0$, da $k[X_1,\ldots,X_n]$ faktoriell ist.
\item Sei $A$ ein Dedekindring. Dann ist $\operatorname{Cl}(\operatorname{Spec}(A))$ gerade die Idealklassengruppe.
\end{enumerate}

\paragraph{Satz 6.12.}\label{6.12} Sei $k$ ein Körper.
\begin{enumerate}[(i)]
\item Sei $X=\mathbf{A}_k^n$ und $Y\subset X$ ein abgeschlossenes Unterschema. Dann ist $Y$ genau dann ein Primdivisor, wenn:
\[Y=V(f)\quad\text{für ein irreduzibles, nicht-konstantes }f\in k[X_1,\ldots,X_n]  \]
\item Sei $X=\mathbf{P}_k^n$ und $Y\subset X$ ein abgeschlossenes Unterschema. Dann ist $Y$ genau dann ein Primdivisor, wenn:
\[Y=V_+(f)\quad\text{für ein homogenes, irreduzibles }f\in k[X_0,\ldots,X_n],\ \deg(f)=r>0 \]
\end{enumerate}

\paragraph{Definition 6.13.}\label{6.13} Sei $X=\mathbf{P}^n_k$. Jeder Primdivisor $Y$ in $X$ hat die Form $Y=V_+(f_Y)$. Betrachte die Abbildung $Y\mapsto\deg(f_Y)$. Diese induziert ein Gruppenhomomorphismus $\operatorname{Div}(X)\to\mathbb{Z}$. Wir zeigen, dass dieser über $\operatorname{Cl}(X)$ faktorisiert. Sei $f\in K^\times$. Dann gilt:
\[\deg\operatorname{div}(f)=\sum_Y v_Y(f) \deg(Y)=\sum_Y v_Y(f) \deg(f_Y) \]
$f=\frac{g}{h}$ mit homogenen $g,h$ vom selben Grad $d$. Sei $g=g_1^{n_1}\cdots g_r^{n_r}$ eine Zerlegung in irreduzible Elemente $g_i$ vom Grad $d_i$. Dann sind $Y_i=\operatorname{div}(g_i)$ nach \hyperref[6.12]{Satz 6.12 (ii)} Primdivisoren.

\printindex
\end{document}