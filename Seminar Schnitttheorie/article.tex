\documentclass[10pt,a4paper]{article}
\usepackage{fontspec}
\usepackage{amsmath}
\usepackage{amsfonts}
\usepackage{amssymb}
\usepackage{amsthm}
\usepackage{paralist}
\usepackage{tikz-cd}
\linespread{1.1}

\author{von \textsc{yichuan shen}}
\title{Vektorbündel \& Gysin-Homomorphismus}
\begin{document}

\theoremstyle{plain}
\theoremstyle{definition}
\newtheorem{theorem}{Theorem}
\newtheorem{lemma}[theorem]{Lemma}
\newtheorem{proposition}[theorem]{Satz}
\newtheorem{corollary}[theorem]{Korollar}
\theoremstyle{definition}
\newtheorem*{definition}{Definition}
\newtheorem*{example}{Beispiel}
\theoremstyle{remark}
\newtheorem*{remark}{Bemerkung}

\maketitle

\begin{definition}
Sei $X$ ein Schema.
\begin{enumerate}[(i)]
\item Sei $p: E\to X$ ein Vektorbündel vom Rang $n$ und $X=\bigcup_\alpha U_\alpha$ eine offene Überdeckung mit $U_\alpha$-Isomorphismen $\varphi_j: f^{-1}(U_\alpha)\to \mathbf{A}_{U_\alpha}^n$. Wir bezeichnen den Nullschnitt mit $i_0: X\to E$.
\item Für eine lokal freie Garbe $\mathcal{E}$ vom Rang $r$ auf $X$ definiert
\[ \mathbf{V}(\mathcal{E}) = \mathbf{Spec}_X(\operatorname{Sym}^\bullet\mathcal{E})\to X \]
ein Vektorbündel auf $X$.
\item Für einen Vektorbündel $E\to X$ vom Rang $r$ definiert
\[ \Gamma(E) = (U\mapsto \operatorname{Hom}_X(U, E)) \]
eine lokal freie Garbe auf $X$ vom Rang $r$.
\end{enumerate}
\end{definition}

\begin{proposition}
Sei $X$ ein Schema. Für eine lokal freie Garbe $\mathcal{E}$ auf $X$ und einen Vektorbündel $E\to X$ gilt:
\[ \Gamma(\mathbf{V}(\mathcal{E})) = \mathcal{E}^\vee \]
\[ \mathbf{V}(\Gamma(E)) = E^\vee \]
\end{proposition}

\begin{definition}
Sei $X$ ein Schema.
\begin{enumerate}[(i)]
\item Das \textit{projektive Vektorbündel} $p: \mathbf{P}(\mathcal{E})\to X$ einer lokal freien Garbe $\mathcal{E}$ über $X$ ist gegeben durch:
\[ \mathbf{P}(\mathcal{E}) = \mathbf{Proj}_X(\operatorname{Sym}^\bullet\mathcal{E})\to X \]
Es gibt einen natürlichen surjektiven Morphismus $p^\ast\mathcal{E} \to \mathcal{O}_\mathcal{E}(1)$.
\item Das \textit{projektive Vektorbündel} $p: P(E)\to X$ eines Vektorbündels $E\to X$ über $X$ ist gegeben durch:
\[ P(E) = (E\setminus i_0(X))/\mathbb{G}_\text{m}\to X \]
\end{enumerate}
\end{definition}

\begin{proposition}
Sei $X$ ein Schema.
\begin{enumerate}[(i)]
\item Sei $\mathcal{E}$ eine lokal freie Garbe und $E=\mathbf{V}(\mathcal{E})\to X$ das zugehörige Vektorbündel. Dann gilt:
\[ \mathbf{P}(\mathcal{E})\cong P(E) \]
\end{enumerate}
\end{proposition}

\begin{definition}
Die \textit{Segre-Klassen} $s_i(E)$ für $i\geq 1-r$ eines Vektorbündels $E\to X$ vom Rang $r = \operatorname{rk}(E)$ auf einer Varietät $X$ sind die Homomorphismen:
\begin{align*}
s_i(E)\cap -: A_k(X)&\to A_{k-i}(X),\\ 
\alpha&\mapsto p_\ast(c_1(\mathcal{O}_E(1))^{(r-1)+i} \cap p^\ast\alpha)
\end{align*}
wobei $r-1$ die relative Dimension des projektiven Vektorbündels $p: P(E)\to X$ bezeichnet. Für eine lokal freie Garbe $\mathcal{E}$ setzen wir:
\[ s_i(\mathcal{E})= s_i(\mathbf{V}(\mathcal{E}^\vee)) \]
Dies impliziert $s_i(E) = s_i(\Gamma(E))$.
\end{definition}

\begin{proposition}
Sei $\mathcal{E} = \Gamma(E)$ ein Vektorbündel auf einer Varietät $X$ und $\alpha\in A_\bullet(X)$.
\begin{enumerate}[(i)]
\item $s_i(\mathcal{E})\cap\alpha = 0$ für alle $1-r\leq i <0$.
\item $s_0(\mathcal{E})\cap\alpha = \alpha$
\item Sei $\mathcal{F} = \Gamma(F)$ ein weiteres Vektorbündel auf $X$. Dann gilt für alle $i,j\geq 0$:
\[ s_i(\mathcal{E})\cap (s_j(\mathcal{F}) \cap \alpha) = s_j(\mathcal{F}) \cap (s_i(\mathcal{E})\cap \alpha) \]
\item Sei $f:Y\to X$ ein eigentlicher Morphismus und $\beta\in A_\bullet(Y)$. Dann gilt die Projektionsformel:
\[ f_\ast(s_i(f^\ast\mathcal{E})\cap\beta) = s_i(\mathcal{E})\cap f_\ast\beta \]
\item Sei $f:Y\to X$ ein flacher Morphismus. Dann gilt:
\[ f^\ast(s_i(\mathcal{E})\cap \alpha) = s_i(f^\ast\mathcal{E})\cap f^\ast\alpha \]
\item Ist $\mathcal{E}$ vom Rang $1$, also ein Geradenbündel. Dann gilt:
\[ s_1(\mathcal{E}) \cap\alpha = -c_1(\mathcal{E})\cap \alpha \]
\end{enumerate}
\end{proposition}

\begin{proof}
Projektive Vektorbündel sind stabil unter Basiswechsel. Wir haben das kartesische Quadrat:
\[ \begin{tikzcd}
P(f^\ast E) \rar["P(f)"]\dar["q"'] & P(E) \dar["p"] \\
Y \rar["f"'] & X
\end{tikzcd} \]
und $P(f)^\ast\mathcal{O}_E(1) = \mathcal{O}_{f^\ast E}(1)$. Es gilt:
\begin{align*}
f_\ast(s_i(f^\ast\mathcal{E})\cap\beta) &= f_\ast q_\ast(c_1(\mathcal{O}_{f^\ast E}(1))^{r-1+i}\cap q^\ast \beta)\\
&= p_\ast P(f)_\ast (c_1(P(f)^\ast\mathcal{O}_E(1))^{r-1+i}\cap q^\ast\beta)\\
&= p_\ast(c_1(\mathcal{O}_E(1))^{r-1+i}\cap P(f)_\ast q^\ast\beta) \\
&\phantom{==} \text{(Projektionsformel für $c_1$)}\\
&= p_\ast(c_1(\mathcal{O}_E(1))^{r-1+i}\cap p^\ast f_\ast\beta)\\
&= s_i(\mathcal{E})\cap f_\ast\beta
\end{align*}
\begin{align*}
f^\ast(s_i(\mathcal{E})\cap\alpha) &= f^\ast p_\ast(c_1(\mathcal{O}_E(1))^{r-1+i}\cap p^\ast\alpha)\\
&= q_\ast P(f)^\ast(c_1(\mathcal{O}_E(1))^{r-1+i}\cap p^\ast \alpha)\\
&= q_\ast(c_1(P(f)^\ast\mathcal{O}_{E}(1))^{r-1+i}\cap P(f)^\ast p^\ast \alpha)\\
&= q_\ast(c_1(\mathcal{O}_{f^\ast E}(1))^{r-1+i}\cap q^\ast f^\ast \alpha)\\
&= s_i(f^\ast\mathcal{E})\cap f^\ast\alpha
\end{align*}
Dies zeigt (iv) und (v). Für (i) und (ii) sei $\alpha = [V]$ ein Primzykel und nach der Projektionsformel für $V\hookrightarrow X$ können wir o.B.d.A. annehmen, dass $V=X$ ganz ist. Dann gilt:
\[s_i(\mathcal{E})\cap [X] \begin{cases}
\in A_{\dim(X)-i}(X) = 0, & \text{wenn $i<0$}\\
= m\cdot [X], & \text{für ein $m\in \mathbb{Z}$ wenn $i = 0$}
\end{cases}
\]
Um $m=1$ zu zeigen, können wir nach (v) auf eine offene Teilmenge einschränken, so dass $P(E)=X\times_k\mathbf{P}^{r-1}_k$ das triviale Bündel ist. Wieder nach (v) für $X\to k$ können wir sogar annehmen, dass $X=\operatorname{Spec}(k)$ und $P(E)=\mathbf{P}_k^{r-1}$. Es gilt:
\[ c_1(\mathcal{O}(1))\cap [\mathbf{P}_k^{r-1}] = [\mathbf{P}_k^{r-2}] \]
Wendet man dies $(r-1)$-mal an, zeigt dies $m=1$. Für (iii) betrachte das kartesische Quadrat:
\[ \begin{tikzcd}
Y \rar["q'"]\dar["p'"']\ar[rd, "f", dashed] & P(E) \dar["p"]\\
P(F)\rar["q"'] & X
\end{tikzcd} \]
Dann gilt:
\begin{align*}
s_i(\mathcal{E})\cap (s_j(\mathcal{F})\cap \alpha) &= p_\ast (c_1(\mathcal{O}_E(1))^{r-1+i} \cap p^\ast(q_\ast(c_1(\mathcal{O}_F(1))^{s-1+j}\cap q^\ast \alpha))\\
&= p_\ast(c_1(\mathcal{O}_E(1))^{r-1+i}\cap q'_\ast(c_1(p'^\ast\mathcal{O}_F (1))^{s-1+j}\cap p'^\ast q^\ast\alpha))\\
&= f_\ast(c_1(q'^\ast\mathcal{O}_E(1))^{r-1+i}\cap c_1(p'^\ast\mathcal{O}_F (1))^{s-1+j}\cap f^\ast\alpha)\\
&\phantom{==} \text{(Projektionsformel)}
\end{align*}
Die Aussage folgt, da $c_1$ kommutativ ist. Für (vi) sei $E\to X$ ein Geradenbündel und $\mathcal{E}$ eine invertierbare Garbe mit $E=\mathbf{V}(\mathcal{E})$ und $P(E)=\mathbf{P}(\mathcal{E})=X$ und $\mathcal{O}_E(1)=\mathcal{E} = \Gamma(E)^\vee$. Dann gilt:
\[ s_1(E)\cap \alpha = c_1(\mathcal{O}_E(1))\cap\alpha = -c_1(\Gamma(E))\cap\alpha = -c_1(E)\cap\alpha\qedhere \]
\end{proof}

\begin{corollary}
Sei $E\to X$ ein Vektorbündel vom Rang $r$. Dann ist
\[ p^\ast: A_k(X)\to A_{k+r-1}(P(E)) \]
injektiv und besitzt den Schnitt:
\[ \alpha\mapsto p_\ast(c_1(\mathcal{O}_E(1))^{r-1}\cap\alpha) \]
\end{corollary}

\begin{proof}
Die Komposition des Schnitts mit $p^\ast$ ist gerade $s_0(E)\cap - = \operatorname{id}$.
\end{proof}

\begin{corollary}[\textit{Splitting-Prinzip}]
Sei $\mathcal{E}$ eine lokal freie Garbe vom Rang $r$ auf einer Varietät $X$. Dann gibt es einen flachen projektiven Morphismus $f:Y\to X$, so dass:
\begin{enumerate}[(i)]
\item Die induzierte Abbildung $f^\ast:A_\bullet(X)\to A_\bullet(Y)$ ist injektiv.
\item Die Garbe $f^\ast\mathcal{E}$ besitzt eine \textit{vollständige Filtration}, d.h. eine Filtration von lokal freien Garben
\[ 0 = \mathcal{E}_0 \subset \mathcal{E}_1 \subset\ldots\subset \mathcal{E}_r = f^\ast\mathcal{E} \]
so dass die Quotienten $\mathcal{E}_i/\mathcal{E}_{i-1}$ vom Rang $1$ sind.
\end{enumerate}
\end{corollary}

\begin{proof}
Per Induktion über $r$. Ist $r=1$, so ist die Aussage trivial. Sei nun $r>1$ und $p: \mathbf{P}(\mathcal{E})\to X$ das projektive Vektorbündel. Nach dem vorherigen Korollar ist $p^\ast$ injektiv. Betrachte die exakte Folge: 
\[0 \longrightarrow \mathcal{K} \longrightarrow p^\ast\mathcal{E}\longrightarrow \mathcal{O}_\mathcal{E}(1) \longrightarrow 0\]
Nach Induktionsvoraussetzung gibt es ein $f': Y\to\mathbf{P}(\mathcal{E})$ mit injektivem $f'^\ast$ und eine vollständige Filtration von $f'^\ast\mathcal{K} \subset f'^\ast p^\ast\mathcal{E}$. Setze $f = p\circ f'$.
\end{proof}

\begin{proposition}
Sei $\mathcal{E}$ eine lokal freie Garbe vom Rang $r$ und $\mathcal{L}$ ein Geradenbündel auf $X$. Dann gilt:
\[ s_i(\mathcal{E}\otimes\mathcal{L}) = \sum_{j=0}^i(-1)^{i-j} \binom{r-1+i}{r-1+j}s_j(\mathcal{E}) c_1(\mathcal{L})^{i-j} \]
\end{proposition}

\begin{definition}
Das \textit{Segre-Polynom} einer lokal freien Garbe $\mathcal{E}$ einer Varietät $X$ ist definiert durch:
\[ s_t(\mathcal{E}) = \sum_{i = 0}^\infty s_i(\mathcal{E})t^i \in\operatorname{End}(A_\bullet(X))[t] \]
Es hat höchstens Grad $\dim(X)$. Die \textit{totale Segre-Klasse} ist:
\[ s(\mathcal{E}) = s_t(\mathcal{E})|_{t = 1} \in\operatorname{End}(A_\bullet(X)) \]
\end{definition}

\begin{remark}
Die Segre-Klassen $s_i(\mathcal{E})$ sind für $i>0$  nilpotent und liegen in einem kommutativen Teilring von $\operatorname{End}(A_\bullet(X))$. Da $s_0(\mathcal{E}) = \operatorname{id}$, folgt:
\[ s_t(\mathcal{E}) \in\operatorname{End}(A_\bullet(X))[t]^\times \]
\end{remark}

\begin{definition}
Sei $E\to X$ ein Vektorbündel auf $X$ und $\mathcal{E} = \Gamma(E)$. Das \textit{Chern-Polynom} von $E$ bzw. $\mathcal{E}$ ist definiert durch:
\[ c_t(\mathcal{E}) = s_t(\mathcal{E})^{-1} \in\operatorname{End}(A_\bullet(X))[t]^\times \]
Die Koeffizienten sind die \textit{Chern-Klassen}:   
\[ c_t(\mathcal{E}) = \sum_{i =0}^\infty c_i(\mathcal{E}) t^i \]
wobei $c_i(\mathcal{E})\cap - : A_k(X)\to A_{k-i}(X)$. Die \textit{totale Chern-Klasse} ist:
\[ c(\mathcal{E}) = c_t(\mathcal{E})|_{t =1}\in\operatorname{End}(A_\bullet(X)) \]
\end{definition}

\begin{remark}
\begin{enumerate}[(i)]
\item Die zwei Definitionen von $c_1$ stimmen überein.
\item Die Chern-Klassen liegen in denselben kommutativen Teilring wie die Segre-Klassen, kommutieren also mit Segre-Klassen.
\item Explizite Formeln für Chern-Klassen sind gegeben durch:
\begin{align*}
c_0 &= 1\\
c_1 &= -s_1\\
c_2 &= s_1^2 - s_2\\
&\vdots\\
c_n &= -s_1c_{n-1}-s_2c_{n-2}-\ldots -s_{n-1}c_1-s_n\\
&\vdots
\end{align*}
\end{enumerate}
\end{remark}

\begin{lemma}
Sei $X$ eine Varietät und $\mathcal{E}$ eine lokal freie Garbe vom Rang $r$ auf $X$. Sei $s\in\Gamma(X,\mathcal{E})$ und $Z=Z(s)\hookrightarrow X$ das geschlossene Unterschema wo $s$ verschwindet, d.h. $Z=\{x\in X\mid\text{Bild von $s$ verschwindet in $\mathcal{E}_x \otimes_{\mathcal{O}_{X,x}} k(x)$} \}$, und $j:U=X\setminus Z\hookrightarrow X$ das offene Komplement. Dann gilt:
\begin{enumerate}[(i)]
\item Für alle $\alpha\in A_\bullet(X)$ gilt $c_r(\mathcal{E})\cap\alpha = \beta$, wobei $\beta$ einen Vertreter besitzt, dessen Träger in $Z$ liegt.
\item $c_r(j^\ast\mathcal{E}) = 0$
\item Sei $Z=\varnothing$ und $\mathcal{E}$ besitze eine vollständige Filtrierung mit Geradenbündeln $\mathcal{L}_1,\ldots,\mathcal{L}_r$ als Quotienten. Dann gilt für alle $\alpha\in A_\bullet(X)$:
\[ \prod_{i=1}^r c_1(\mathcal{L}_i)\cap\alpha = 0 \]
\end{enumerate}
\end{lemma}

\begin{proposition}
Sei $\mathcal{E} = \Gamma(E)$ ein Vektorbündel auf einer Varietät $X$ vom Rang $r$ und sei $\alpha\in A_\bullet(X)$.
\begin{enumerate}[(i)]
\item $c_i(\mathcal{E})\cap \alpha = 0$ für $i > r$.
\item $c_0(\mathcal{E})\cap \alpha = \alpha$
\item Sei $\mathcal{F} = \Gamma(F)$ ein weiteres Vektorbündel auf $X$. Dann gilt für alle $i,j\geq 0$:
\[ c_i(\mathcal{E})\cap (c_j(\mathcal{F}) \cap \alpha) = c_j(\mathcal{F}) \cap (c_i(\mathcal{E})\cap \alpha) \]
\item Sei $f:Y\to X$ ein eigentlicher Morphismus und $\beta\in A_\bullet(Y)$. Dann gilt die Projektionsformel:
\[ f_\ast(c_i(f^\ast\mathcal{E})\cap\beta) = c_i(\mathcal{E})\cap f_\ast\beta \]
\item Sei $f:Y\to X$ ein flacher Morphismus. Dann gilt:
\[ f^\ast(c_i(\mathcal{E})\cap \alpha) = c_i(f^\ast\mathcal{E})\cap f^\ast\alpha \]
\item \textit{(Whitneysche Summenformel)} Für eine kurze exakte Folge 
\[ 0\longrightarrow\mathcal{E}'\longrightarrow\mathcal{E}\longrightarrow \mathcal{E}''\longrightarrow 0 \]
von lokal freien Garben auf $X$ gilt:
\[ c_n(\mathcal{E}) = \sum_{i+j = n}c_i(\mathcal{E}')c_j(\mathcal{E}'') \]
\item Für einen Cartier-Divisor $D$ auf $X$ gilt:
\[ c_1(\mathcal{O}_X(D))\cap [X] = [D] \]
\end{enumerate}
\end{proposition}

\begin{remark}
Chern-Klassen sind eindeutig durch die Eigenschaften (v), (vi) und (vii) bestimmt.
\end{remark}

\begin{proof}
(ii), (iii) und (vii) sind klar. (iv) und (v) folgen aus den Aussagen für die Segre-Klassen, da Chern-Klassen Polynome in den Segre-Klassen sind.

Für (i) sei zunächst $E\to X$ ein Geradenbündel und $\mathcal{E} = \Gamma(E)$. Dann ist $P(E)=X$ und $\mathcal{O}_E(1)=\mathcal{E}^\vee$ und:
\[ s_i(\mathcal{E})\cap\alpha = c_1(\mathcal{O}_E(1))^i\cap\alpha = (-1)^ic_1(\mathcal{E})^i\cap\alpha \]
Somit gilt für die totale Segre-Klasse:
\[ s(\mathcal{E}) = \sum_{i\geq 0}(-1)^ic_1(\mathcal{E})^i = \frac{1}{1+c_1(\mathcal{E})}\in\operatorname{End}(A_\bullet(X)) \]
Dies zeigt $c_i(\mathcal{E}) = 0$ für $i> \operatorname{rk}(\mathcal{E}) = 1$. Für den allgemeinen Fall betrachte das Splitting-Prinzip und (v). Wir können annehmen, dass $\mathcal{E}$ eine vollständige Filtration besitzt mit Quotienten $\mathcal{L}_1,\ldots,\mathcal{L}_r$. Die Whitneysche Summenformel ist äquivalent zu $c(\mathcal{E}')c(\mathcal{E}'') = c(\mathcal{E})\in\operatorname{End}(A_\bullet(X))$. Daher gilt:
\[ (\star) \qquad c(\mathcal{E}) = \prod_{i=1}^r c(\mathcal{L}_i) = \prod_{i=1}^r(1+c_1(\mathcal{L}_i)) \phantom{\qquad(\star)} \]
Dies zeigt (a).

Für die Summenformel können wir nach dem Splitting-Prinzip und (v) annehmen, dass $\mathcal{E}'$ und $\mathcal{E}''$ vollständige Filtrationen besitzen. Diese induzieren eine vollständige Filtration auf $\mathcal{E}$ mit Geradenbündeln $\mathcal{L}_1,\ldots,\mathcal{L}_r$ als Quotienten und zu zeigen ist nun $(\star)$.

Ist $\sigma_i\in\operatorname{End}(A_\bullet(X))$ das $i$-te elementar-symmetrische Polynom in $c_1(\mathcal{L}_1),\ldots,$ $c_1(\mathcal{L}_r)$ und $\sigma_0 =1$. Dann ist $(\star)$ äquivalent zu $c_i(\mathcal{E}) = \sigma_i$. 

Sei $p:P(E)\to X$ das zugehörige projektive Vektorbündel und $\tilde{\sigma}_i$ das $i$-te elementar-symmetrische Polynom in $c_1(p^\ast\mathcal{L}_1),\ldots,c_1(p^\ast\mathcal{L}_r)$ und $\tilde{\sigma}_0 = 1$. Setze $\zeta = c_1(\mathcal{O}_E(1))$. Die natürliche Surjektion $p^\ast\mathcal{E}^\vee\to \mathcal{O}_E(1)$ liefert einen Schnitt $s\in\mathrm{H}^0(P(E), p^\ast\mathcal{E}\otimes \mathcal{O}_E(1))$, das nirgendwo verschwindet. Die Garbe $p^\ast\mathcal{E}\otimes\mathcal{O}_E(1)$ hat eine vollständige Filtrierung mit Quotienten $p^\ast\mathcal{L}_i\otimes\mathcal{O}_E(1)$ und der nachfolgende Satz zeigt:
\[ 0 = \prod_{i=1}^r c_1(p^\ast\mathcal{L}_i\otimes\mathcal{O}_E(1)) = \prod_{i=1}^{r}(\zeta + c_1(p^\ast\mathcal{L}_i)) = \sum_{i=0}^r\tilde{\sigma}_i\zeta^{r-i}\]
Für ein $\alpha\in A_\bullet(X)$ und $\nu\geq 1$ gilt:
\begin{align*}
0 &= p_\ast\Big(\zeta^{\nu-1}\sum_{i=0}^r\zeta^{r-i}\tilde{\sigma}_i\cap p^\ast\alpha\Big)\\
&= p_\ast\Big(\sum_{i=0}^r c_1(\mathcal{O}_E(1))^{r-1+\nu-i}\cap p^\ast(\sigma_i\cap\alpha)\Big)\qquad \text{(Rückzug)}\\
&= (s_\nu(\mathcal{E})\sigma_0 + s_{\nu-1}(\mathcal{E})\sigma_1 + \ldots + s_{\nu-r}(\mathcal{E})\sigma_r)\cap \alpha
\end{align*}
und somit folgt unter Beachtung von $s_j(\mathcal{E}) = 0$ für $j<0$:
\[ s(\mathcal{E})(\sigma_0+\sigma_1+\ldots+\sigma_r) = s_0(\mathcal{E})\sigma_0 = 1 \]
Dies zeigt $c_i(\mathcal{E}) = \sigma_i$.
\end{proof}

\begin{proof}[Beweis von Lemma.]
(i) folgt aus (ii) durch Rückzug:
\[ j^\ast(c_r(\mathcal{E})\cap\alpha) = c_r(j^\ast\mathcal{E})\cap j^\ast\alpha = 0 \]
und der Ausschneidungssequenz $A_\bullet(Z)\to A_\bullet(X) \stackrel{j^\ast}{\to} A_\bullet(U)\to 0$. (ii) folgt aus (iii) mit dem Splitting-Prinzip und der Whitney-Formel $c_r(\mathcal{E}) = \prod_{i=1}^rc_1(\mathcal{L}_i)$ für den höchsten Koeffizienten.

(iii) zeigen wir per Induktion über $r$. Sei $s_r\in\mathrm{H}^0(X, \mathcal{L}_r)$ das Bild von $s$ unter $\mathcal{E}\twoheadrightarrow\mathcal{L}_r$, und sei $i_r: Z_r\hookrightarrow X$ das geschlossene Unterschema, wo $s_r$ verschwindet. Dann kann $\mathcal{L}_r$ durch einen Pseudo-Divisor $(\mathcal{L}_r,Z_r,s_r)$ dargestellt werden und nach Konstruktion gilt:
\[ c_1(\mathcal{L}_r)\cap \alpha = i_{r,\ast}((\mathcal{L}_r, Z_r, s_r)\cdot\alpha) = i_{r,\ast}\beta \in i_{r,\ast}A_\bullet(Z_r)\subset A_\bullet(X) \]
Nach der Induktionsvoraussetzung, angewendet auf $\mathcal{E}' = i_r^\ast \ker(\mathcal{E}\to\mathcal{L}_r)$, mit dem induzierten nicht-verschwindenden Schnitt $s|_{Z_r}\in\mathrm{H}^0(Z_r,\mathcal{E}')\subset \mathrm{H}^0(Z_r, i_r^\ast\mathcal{E})$ und Filtrationsquotienten $i^\ast_r\mathcal{L}_1,\ldots,i_r^\ast\mathcal{L}_{r-1}$, erhalten wir:
\[ \prod_{i=1}^rc_1(\mathcal{L}_i)\cap\alpha = \prod_{i=1}^{r-1} c_1(\mathcal{L}_i)\cap (i_{r,\ast}\beta) = i_{r,\ast}\Big(\prod_{i=1}^{r-1} c_1(i_r^\ast\mathcal{L}_i)\cap\beta\Big) = 0 \]
Der Induktionsanfang $r=1$ folgt aus der Tatsache, dass für den trivialen Geradenbündel $c_1(\mathcal{O}_X) = 0$ gilt.
\end{proof}

\begin{theorem}
Sei $p:E\to X$ ein Vektorbündel vom Rang $r$ und $q:P(E)\to X$ das assoziierte projektive Vektorbündel.
\begin{enumerate}[(i)]
\item Der flache Rückzug $p^\ast: A_d(X)\to A_{d+r}(E)$ ist ein Isomorphismus.
\item Es gibt einen kanonischen Isomorphismus:
\begin{align*}
\vartheta_E: \bigoplus_{i=0}^{r-1}A_{d+i}(X) &\stackrel{\sim}{\longrightarrow} A_{d+r-1}(P(E))\\
A_{d+i}(X)\ni \alpha_i &\longmapsto c_1(\mathcal{O}_E(1))^i\cap q^\ast(\alpha_i)
\end{align*}
\item Sei $\mathcal{E} = \Gamma(E)$. Die Chern-Klassen von $\mathcal{E}$ sind eindeutig durch die folgende Formeln bestimmt:
\[ c_0(\mathcal{E}) = 1,\quad \sum_{i=0}^r c_1(\mathcal{O}_E(1))^i\cap q^\ast(c_{r-i}(\mathcal{E})\cap\alpha) = 0 \]
\end{enumerate}
\end{theorem}

\begin{proof}
Die Surjektivität von $p^\ast$ wurde im zweiten Vortrag bewiesen.
\end{proof}

\begin{definition}
Für ein Vektorbündel $p:E\to X$ vom Rang $r$ mit Nullschnitt $s:X\to E$ definieren wir den \textit{Gysin-Homomorphismus} entlang $s$ als:
\[ s^\ast = (p^\ast)^{-1}: A_{d+r}(E)\to A_d(X) \]
\end{definition}

\end{document}