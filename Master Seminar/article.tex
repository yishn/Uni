\documentclass[10pt,b5paper]{article}
\usepackage[left=2.00cm, right=2.00cm, top=2.00cm, bottom=2.00cm]{geometry}
\usepackage{fontspec}
\usepackage[ngerman]{babel}
\usepackage{amsmath}
\usepackage{amsfonts}
\usepackage{amssymb}
\usepackage{amsthm}
\usepackage{paralist}
\usepackage{tikz}
\usetikzlibrary{cd, babel}
\usepackage{hyperref}
\hypersetup{
	bookmarksopen=true,
	pdfstartview=FitH
}
\linespread{1.1}

\author{von \textsc{yichuan shen}}
\title{Galoissche Kennzeichnung von Funktionenkörper \\ vom Transzendenzgrad $1$ über Zahlkörper}
\begin{document}

\theoremstyle{plain}
\theoremstyle{definition}
\newtheorem{theorem}{Theorem}
\newtheorem{lemma}[theorem]{Lemma}
\newtheorem{proposition}[theorem]{Satz}
\newtheorem{corollary}[theorem]{Korollar}
\theoremstyle{definition}
\newtheorem*{definition}{Definition}
\newtheorem*{example}{Beispiel}
\theoremstyle{remark}
\newtheorem*{remark}{Bemerkung}
\newtheorem{step}{Schritt}

\maketitle

\section{Valuations}

\begin{definition}
Let $K$ be a field and $P$ its prime field. $K$ is called \textit{finitely generated} if $K/P$ is finitely generated. The \textit{(Kronecker) dimension}\index{dimension!Kronecker} $\dim(K)$ of $K$ is defined as:
\[\dim(K) = \begin{cases}
\operatorname{trdeg}(K/P), & \operatorname{char}(K)>0\\
\operatorname{trdeg}(K/P)+1, & \text{otherwise}
\end{cases} \]
\end{definition}

\begin{definition}\label{2.1}
Let $K$ be a field of finite dimension. We denote the set of (equivalence classes) of Krull valuations of $K$ by $\operatorname{Spv}(K)$.

\[|\operatorname{Spv}(K)|=\{v\in\operatorname{Spv}(K)\mid \dim k(v)=0\}\]
\end{definition}

\begin{lemma}\label{rank-dimension-inequality}
Let $v\in\operatorname{Spv}(K)$. Then we have $\operatorname{rrank}(v)+\dim k(v)\leq\dim(K)$. If equality holds and $K$ is finitely generated, then $k(v)$ is finitely generated as well, and $\Gamma_v$ is a finitely generated $\mathbb{Z}$-module.
\end{lemma}

\begin{definition}
We define:
\begin{align*}
\operatorname{Spv}_\text{d}(K) &= \{v\in\operatorname{Spv}(K)\mid \operatorname{rrank}(v)+\dim k(v)=\dim(K) \}
\end{align*}
Valuations in $\operatorname{Spv}_\text{d}(K)$ are called \textit{defectless}\index{valuation!defectless}.
\[\operatorname{Spv}_\text{d}^{(i)}(K)=\{v\in\operatorname{Spv}_\text{d}(K)\mid \operatorname{rrank}(v)=i \} \]
\end{definition}

Let $K/\mathbb{Q}$ be a field of dimension $2$.

\begin{lemma}[\textit{Main Lemma}]\label{2.4}
Let $\ell\neq 2$ be a prime. The following map, induced by the restrictions, is injective:
\[ \mathrm{H}^3(K,\mathbb{Z}/\ell\mathbb{Z})\longrightarrow \prod_{w\in |\operatorname{Spv}(K)| }\mathrm{H}^3(K_w,\mathbb{Z}/\ell\mathbb{Z}) \]
\end{lemma}

\begin{lemma}\label{2.5}
Let $K$ be henselian with respect to $w\in|\operatorname{Spv}(K)|$. Then $k(w)$ is an algebraic extension of a finite field $\mathbb{F}_p$. Let $\ell\neq p$ be a prime and assume $\mu_\ell\subset K$. Then:
\[ \mathrm{H}^3(K, \mathbb{Z}/\ell\mathbb{Z}) = \begin{cases}
\mathbb{Z}/\ell\mathbb{Z}, & \ell^\infty\nmid [k(w):\mathbb{F}_p] \text{ and } \dim_{\mathbb{F}_\ell}(\Gamma_w / \ell\Gamma_w) = 2\\
0, & \text{otherwise}
\end{cases} \]
If $L/K$ is an algebraic extension and $\mathrm{H}^3(K, \mathbb{Z}/\ell\mathbb{Z}) = \mathbb{Z}/\ell\mathbb{Z}$ holds, we have:
\[ \mathrm{H}^3(L,\mathbb{Z}/\ell\mathbb{Z}) = \begin{cases}
\mathbb{Z}/\ell\mathbb{Z}, & \ell^\infty\nmid [L:K]\\
0, & \text{otherwise}
\end{cases} \]
\end{lemma}

\begin{proposition}\label{2.6}
Let $K_0/K$ be an algebraic extension. Suppose there exists a prime number $\ell\neq 2$ with $\ell^\infty\nmid[K_0:K]$ and such that for every algebraic extension $L/K_0$ we have: 
\[ \mathrm{H}^3(L,\mathbb{Z}/\ell\mathbb{Z}) = \begin{cases}
\mathbb{Z}/\ell\mathbb{Z}, & \ell^\infty\nmid [L:K_0]\\
0, & \text{otherwise}
\end{cases} \]
Then there exists a unique henselian $w\in|\operatorname{Spv}(K)|$.
\end{proposition}

\section{Local Correspondence}

Let $K,K'/\mathbb{Q}$ be two finitely generated fields of dimension $2$.

\begin{lemma}\label{3.1}
Let $w\in\operatorname{Spv}_\text{d}^{(2)}(K)$ and let $E=K_w$ be a henselization of $w$. Then $E'$ is a henselization of $K'$ with respect to a unique $w'\in\operatorname{Spv}_\text{d}^{(2)}(K')$. The map $w\mapsto w'$ does not depend on the choice of the henselization.
\end{lemma}

\begin{proof}
Let $\ell_1,\ell_2$ be two different odd prime numbers different from $p=\operatorname{char} k(w)$ and let $E_0 = E(\mu_{\ell_1\ell_2})$. Since $K$ is finitely generated, by Lemma \ref{rank-dimension-inequality}, $k(w)$ and $\Gamma_w$ are finitely generated as well. In particular we have $\ell_i^\infty \nmid [k(w):\mathbb{F}_p]$ and, since $\Gamma_w$ is torsion free, $\dim_{\mathbb{F}_{\ell_i}}(\Gamma_w/{\ell_i}\Gamma_w) = \dim_{\mathbb{Q}}(\Gamma_w\otimes_{\mathbb{Z}}\mathbb{Q})= \operatorname{rrank}(w)=2$ for $i=1,2$.

By Lemma \ref{2.5}, $E_0/E$, and also $E'_0/E'$, satisfy the condition of Proposition \ref{2.6} for both $\ell_1$ and $\ell_2$. Thus there exists a unique henselian valuation $w'_{E'}\in|\operatorname{Spv}(E')|$. Choose a prime $\ell\neq \operatorname{char} k(w'_{E'})$ from $\ell_1,\ell_2$. Since $\mathrm{H}^3(E',\mathbb{Z}/\ell\mathbb{Z})\cong \mathrm{H}^3(E,\mathbb{Z}/\ell\mathbb{Z})\neq 0$, we get from Lemma \ref{2.5}:
\[\operatorname{rrank}(w'_{E'}) = \dim_{\mathbb{Q}}(\Gamma_{w'_{E'}}\otimes_\mathbb{Z}\mathbb{Q}) \geq \dim_{\mathbb{F}_\ell}(\Gamma_{w'_{E'}}/\ell\Gamma_{w'_{E'}})=2=\dim(E')\]
which makes $w'_{E'}\in\operatorname{Spv}^{(2)}_\text{d}(E')$. Now $E'$ contains a henselization $K'_{w'}$ of $w' = w'_{E'}|_{K'}\in\operatorname{Spv}^{(2)}_\text{d}(K')$. By applying the same argument to $\sigma^{-1}$ we find that $E' = K'_{w'}$.

Let $F$ be another henselization of $w$. Then $E, F$ are $K$-isomorphic, hence $E',F'$ are $K'$-isomorphic and the corresponding henselian valuations of $E', F'$ respectively are mapped to each other.
\end{proof}

\begin{lemma}\label{3.2}
$\sigma$ induces a bijection $\sigma:\operatorname{Spv}_\text{d}^{(1)}(K) \to\operatorname{Spv}_\text{d}^{(1)}(K')$. $\sigma$ induces an isomorphism:
\[ G_{k(v)}\cong\operatorname{Gal}(E^\text{t}/E) \to\operatorname{Gal}(E'^{\text{t}}/E')\cong G_{k(v')} \]
which is induced by a unique isomorphism $\varphi_v^\text{s}:k(v')^\text{s}\to k(v)^\text{s}$. In particular $\varphi_v=\varphi_v^\text{s}|_{k(v')}:k(v')\to k(v)$ is an isomorphism.
\end{lemma}

\section{Global Correspondence mod $n$}

Let $k$ and $k'$ be the algebraic closures of $\mathbb{Q}$ in $K$ and $K'$ respectively, i.e. the constant fields of $K$ and $K'$, and let $\overline{k}$ and $\overline{k'}$ be the algebraic closure of $k$ and $k'$ in $\overline{K}$ and $\overline{K'}$ respectively.

Let $X$ and $X'$ be the unique smooth projective curves over $k$ and $k'$ with function field $K$ and $K'$ respectively. We identify the set of closed points $X_0$ of $X$ with the set of valuations $v\in\operatorname{Spv}_\text{d}^{(1)}(K)$ with $\operatorname{char} k(v)=0$ and similarly identify $X'_0$ with the corresponding subset of $\operatorname{Spv}_\text{d}^{(1)}(K')$. 

By Lemma \ref{3.2} $\sigma$ induces a bijection $\sigma: X^{\phantom{'}}_0\to X'_0,\ P\mapsto P'$ and for any $P\in X_0$ an isomorphism $\varphi_P: k(P')\to k(P)$.

\begin{lemma}\label{4.2}
$\sigma$ induces an isomorphism $G_k\to G_{k'}$ such that the following diagram commutes:
\[ \begin{tikzcd}
G_K \ar[r, "\sigma"] \ar[d] & G_{K'}\ar[d]\\
G_k \ar[r, "\sigma"'] & G_{k'}
\end{tikzcd} \]
Moreover, $\sigma:G_k\to G_{k'}$ is induced by a unique isomorphism $\varphi:\overline{k'}\to\overline{k}$.
\end{lemma}

\begin{remark}
Consider $\overline{K}'' = \overline{K}'\otimes_{\overline{k'}}\overline{k}$ via the isomorphism $\varphi: \overline{k'}\to \overline{k}$. If we replace $\sigma: G_K\to G_{K'}$ with the composition
\[ \begin{tikzcd}
G_K \ar[r, "\sigma"] & G_{K'} \ar[r, "\phi"] & G_{K'}
\end{tikzcd} \]
where $\phi$ is induced by the isomorphism $\overline{K''}\to\overline{K'},\ x\otimes y\mapsto \varphi^{-1}(y)\cdot x$, we may assume that $k=k'$, $\overline{k} = \overline{k'}$ and that $\varphi:\overline{k}\to\overline{k}$, $\sigma: G_k\to G_k$ are identities. Now the following diagram commutes:
\[ \begin{tikzcd}
G_K \ar[r]\ar[d, "\sigma"'] & G_k\\
G_{K'}\ar[ru]
\end{tikzcd} \]
\end{remark}

\paragraph{} Let $n\in\mathbb{N}$. The operation of $G_K$ and $G_{K'}$ on $\mu_n$ factors through $G_k$, thus we have $\sigma(g)\xi = g\xi$ for all $\xi\in\mu_n$ and $g\in G_K$. Therefore, for all intermediate fields $L$ of $\overline{K}/K$, $\sigma$ induces an isomorphism:
\[ \varphi_n:L'^\times/n \cong\mathrm{H}^1(L', \mu_n) \to \mathrm{H}^1(L,\mu_n) \cong L^\times/n \]

\begin{lemma}\phantomsection\label{4.3}
$\varphi_n$ induces an isomorphism $\varphi_n: U_{P'}/n\to U_P/n$. The following diagram commutes:
\[ \begin{tikzcd}
U_{P'}/n \rar["\varphi_n"] \dar["r_{P'}/n"'] & U_P/n \dar["r_P/n"]\\
k(P')^\times/n \rar["\varphi_P/n"'] & k(P)^\times/n
\end{tikzcd} \]
\end{lemma}

\section{Proof of Pop's Theorem}

\begin{definition}\label{4.4}
For a finite subset $S\subset X_0$ set:
\[ K^S=\bigcap_{P\in X_0-S}U_P \]
$K'^{S'}$ is defined similarly. For finite subsets $S$ of $X_0$, the corresponding subsets of $X_0'$ will be denoted by $S'$, i.e. $S'=\{\sigma P\mid P\in S \} $.
\end{definition}

\begin{lemma}\phantomsection\label{4.5}
\begin{enumerate}[(i)]
\item There exists a finite subset $S\subset X_0$ such that:
\[ \operatorname{Pic}(X -  S)=0=\operatorname{Pic}(X' -  S') \]
Such a set $S$ is called \textit{admissible}. If $T\subset X_0$ is a finite set of closed points, then we can choose $S$ in $X_0 -  T$.
\item Let $S$ be admissible. For all $n\in\mathbb{N}$ the map $\varphi_n$ induces isomorphisms: \[\varphi_n: K'^{S'}/n\to K^S/n\]
\end{enumerate}
\end{lemma}

\begin{proof}
\begin{enumerate}[(i)]
\item By Mordell-Weil, the Picard group of $X$ is finitely generated. Let $D_1,\ldots, D_r$ be Weil divisors of $X$ such that the associated line bundles generate $\operatorname{Pic}(X)$ and similarly let $D'_1,\ldots,D'_s$ generate $\operatorname{Pic}(X')$. Clearly any finite $S\subset X_0$ such that
\[ \bigcup_{i=1}^r \operatorname{supp}D_i\subset S\quad\text{and}\quad \bigcup_{j=1}^s\operatorname{supp}D'_j\subset S' \]
is admissible. Now let $T\subset X_0$ be finite. By the Weak Approximation Theorem, we can add a principal divisor to the $D_i$ and the $D'_j$ so that their support is in $X_0 -  T$ and $X'_0 -  T'$ respectively.
\item Observe that the following sequence is exact:
\[ \begin{tikzcd}
1 \ar[r] & K^S \ar[r]& K^\times \ar[r, "\operatorname{div}"]& \displaystyle \bigoplus_{P\in X_0 -  S}\mathbb{Z} \ar[r]& \operatorname{Pic}(X -  S) = 0
\end{tikzcd} \]
Since $\bigoplus\mathbb{Z}$ is $\mathbb{Z}$-flat, tensoring with $\mathbb{Z}/n\mathbb{Z}$ yields another exact sequence and we can consider $K^S/n$ as a subgroup of $K^\times/n$. Then:
\[ K^S/n = \bigcap_{P\in X_0 -  S}U_P/n \]
The analogous result holds for $K'^{S'}/n$. Thus the assertion follows from Lemma \ref{4.3} (i).\qedhere
\end{enumerate}
\end{proof}

\begin{lemma}\label{5.4}
Let $S\subset X_0$ be admissible. Then there exists a finite subset $T\subset X_0 -  S$ such that the following map is injective:
\[ r_{S,T}:K^S \to\prod_{P\in T}k(P)^\times,\ x\mapsto(r_P(x))_{P\in T} \]
In this case, $\bigcap_{n\in\mathbb{N}} n\cdot\operatorname{coker}(r_{S,T}) =0$.
\end{lemma}

\begin{theorem}[\textit{Pop}]\label{1.1}
Let $K,K'$ be two finitely generated fields of characteristic $0$ and dimension $2$. Suppose there is an isomorphism $\sigma:G_K\to G_{K'}$ between their absolute Galois groups, then there exists a unique field isomorphism $\varphi:\overline{K'}\to\overline{K}$ such that:
\[\sigma(g)=\varphi^{-1}g\varphi\quad \text{for all }g\in G_K\] 
In particular $\varphi$ induces a field isomorphism $\varphi:K'\to K$.
\end{theorem}

\begin{step}
There exists a unique group isomorphism $\varphi:K'^\times\to K^\times$ such that for any $P\in X_0$ we have $\varphi(U_{P'})=U_P$ where $P'=\sigma(P)$ and the following diagram commutes:
\[ \begin{tikzcd}
U_{P'} \ar[r, "\varphi"]\ar[d, "r_{P'}"'] & U_P \ar[d, "r_P"]\\
k(P')^\times \ar[r, "\varphi_P"'] & k(P)^\times
\end{tikzcd} \]
\end{step}

\begin{proof}
Let $S\subset X_0$ be admissible and $T\subset X_0-S$ such that $r_{S,T}$ and $r_{S',T'}$ are injective. First, we show that there is a unique homomorphism $\varphi^{S,T}$, making the following diagram commutative:
\[ (\star) \qquad \begin{tikzcd}
K'^{S'} \rar["\varphi^{S,T}", dashed] \dar["r_{S',T'}"'] & K^S \dar["r_{S,T}"]\\
\displaystyle \prod_{P\in T}k(\sigma P)^\times \rar["\prod\varphi_P"'] & \displaystyle \prod_{P\in T}k(P)^\times
\end{tikzcd} \phantom{\qquad (\star)} \]
The vertical maps are injective by definition. Thus, it suffices to show that the following composition is trivial:
\[\varrho:K'^{S'}\longrightarrow\prod_{P\in T} k(\sigma P)^\times \longrightarrow \prod_{P\in T} k(P)^\times\longrightarrow\operatorname{coker}(r_{S,T})\] 
Consider the commutative diagram which results from tensoring $(\star)$ with $\mathbb{Z}/n\mathbb{Z}$:
\[ \begin{tikzcd}
K'^{S'}/n \rar[dashed]\dar & K^S/n\dar\\
\displaystyle \prod_{P\in T}k(\sigma P)^\times/n \rar & \displaystyle \prod_{P\in T}k(P)^\times/n
\end{tikzcd} \]
According to Lemma~\ref{4.3} and Lemma~\ref{4.5} (ii) this diagram can now be completed by $\varphi_n: K'^{S'}/n\to K^S/n$. This implies $\varrho/n=0$, i.e. $\operatorname{im}(\varrho)\subset n\operatorname{coker}(r_{S,T})$ for every $n\in\mathbb{N}$. It follows $\varrho = 0$. By symmetry, $\varphi^{S,T}$ is an isomorphism.

$\varphi^{S,T}$ actually independent of $T$, thus we can omit the $T$ from its notation. Now if $T$ ranges over all $S$-admissible subsets of $X_0$, we can replace $T$ by $X_0 -  S$ in $(\star)$. We can combine the isomorphisms $\varphi^S$, where $S$ ranges over the admissible subsets of $X_0$, into an isomorphism $\varphi = \varinjlim \varphi^S: K'^\times \to K^\times$. 
\end{proof}

\begin{step}
Extending $\varphi$ to a map $K'\to K$ by setting $\varphi(0)=0$ yields an isomorphism of fields: 
\[\varphi:K'\to K\]
\end{step}

\begin{proof}
Observe that $\varphi(-1)^2 = \varphi(1) = 1$, and since $\varphi$ is bijective on $K'^\times$, it follows $\varphi(-1) \neq 1$ and hence $\varphi(-1) = -1$. Now, to prove the assertion, it suffices to show $\varphi(x + y) = \varphi(x) + \varphi(y)$ for all $x,y\in K'^\times$.
\begin{itemize}
\item \textit{Case 1:} If $x + y = 0$, we have:
\[ \varphi(x + y) = 0 = \varphi(x) + \varphi(-1)\varphi(x) = \varphi(x) + \varphi(-x) = \varphi(x) + \varphi(y) \]
\item \textit{Case 2:} Let $x + y\neq 0$ and $S\subset X_0$ be an admissible finite subset such that $\varphi(x + y)$, $\varphi(x)$ and $\varphi(y)$ are contained in $K^S$. Looking at the commutative diagram in \mbox{Step 1} we get
\begin{align*}
r_P(\varphi(x+ y)) &= \varphi_P(r_{P'}(x + y)) \\
&= \varphi_P(r_{P'}(x)) + \varphi_P(r_{P'}(y)) = r_P(\varphi(x) + \varphi(y))
\end{align*}
for all $P\in X_0 -  S$. Note that $\varphi_P: k(P')\to k(P)$ is a field isomorphism and that the residue maps $r_{P'}: \mathcal{O}_{P'} \to k(P')$ are ring homomorphisms. Since $K^S\to \prod_{P\in X_0 -  S} k(P)^\times$ is injective, we get $\varphi(x+y) = \varphi(x) + \varphi(y)$.\qedhere
\end{itemize}
\end{proof}

\begin{remark}
For any finite subextension $L/K$ of $\overline{K}/K$ there exists an isomorphism $\varphi_L: L'\to L$. By passing to limits we obtain an isomorphism $\varphi: \overline{K}'\to \overline{K}$.
\end{remark}

\begin{step}
$\varphi$ induces $\sigma:G_K\to G_{K'}$, i.e. $\sigma(g) = \varphi^{-1}\circ g\circ \varphi$ for all $g\in G_K$, and is uniquely determined by this property.
\end{step}

\begin{proof}
Let $\tau:G_K\to G_{K'},\ g\mapsto \varphi^{-1}\circ g\circ \varphi$. Clearly, $\tau$ is a group isomorphism and we have $\tau(G_L)=G_{L'}=\sigma(G_L)$ for every finite subextension $L/K$ of $\overline{K}/K$. By applying Lemma \ref{3.2} on $\tau$, we get bijections $\tau: \operatorname{Spv}_\text{d}^{(1)}(L)\to \operatorname{Spv}_\text{d}^{(1)}(L')$ for any intermediate field $L$ of $\overline{K}/K$.

We have $L_{\tau(v)} = (L_v)' = L_{\sigma(v)}$ for all $v\in\operatorname{Spv}_\text{d}^{(1)}(L)$, i.e. $\sigma$ and $\tau$ induce the same maps $\operatorname{Spv}_\text{d}^{(1)}(L)\to \operatorname{Spv}_\text{d}^{(1)}(L')$. By limit process, $\sigma$ and $\tau$ induce the same map $\operatorname{Spv}_\text{d}^{(1)}(\overline{K})\to \operatorname{Spv}_\text{d}^{(1)}(\overline{K}')$.

Let $g\in G_K$ and $\overline{v}\in\operatorname{Spv}_\text{d}^{(1)}(\overline{K})$. We have:
\[ \sigma(g)\sigma(\overline{v}) = \sigma(g\overline{v}) = \tau(g\overline{v}) = \tau(g)\tau(\overline{v}) = \tau(g) \sigma(\overline{v}) \]
Therefore $\sigma(g)^{-1}\tau(g)$ is contained in $G_{\overline{v}}$ for all $\overline{v}\in\operatorname{Spv}_\text{d}^{(1)}(\overline{K})$. By the Theorem of F.K. Schmidt we have $G_{\overline{v}_1}\cap G_{\overline{v}_2} = 1$ for $\overline{v}_1\neq\overline{v}_2$. Therefore, $\sigma(g) = \tau(g) = \varphi^{-1}g\varphi$ for all $g\in G_K$.

For the uniqueness of $\varphi$ let $\psi: \overline{K}'\to \overline{K}$ be another isomorphism with $\sigma(g) = \psi^{-1} g \psi$. Setting $h = \psi\varphi^{-1}\in G_K$ we observe
\[ h^{-1}gh = \varphi\sigma(g)\varphi^{-1} = \sigma^{-1}\sigma(g) = g \]
for all $g\in G_K$. For $\overline{v}\in\operatorname{Spv}_\text{d}^{(1)}(\overline{K})$ we have therefore $G_{\overline{v}} = h G_{\overline{v}} h^{-1} = G_{h\overline{v}}$, thus $h\in G_{\overline{v}}$ by F.K. Schmidt. Since $\overline{v}$ was arbitrary, we get $h\in G_{\overline{v}}$ for all $\overline{v}$. Again, by the Theorem of F.K. Schmidt, we have $h=1$ and thus $\varphi = \psi$.
\end{proof}

\end{document}