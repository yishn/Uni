\documentclass[10pt,b5paper]{article}
\usepackage[left=2.00cm, right=2.00cm, top=2.00cm, bottom=2.00cm]{geometry}
\usepackage{fontspec}
\usepackage[ngerman]{babel}
\usepackage{amsmath}
\usepackage{amsfonts}
\usepackage{amssymb}
\usepackage{amsthm}
\usepackage{paralist}
\usepackage{tikz-cd}
\linespread{1.1}

\author{von \textsc{yichuan shen}}
\title{Ebene Graphen}
\begin{document}

\theoremstyle{plain}
\theoremstyle{definition}
\newtheorem{theorem}{Theorem}
\newtheorem{lemma}[theorem]{Lemma}
\newtheorem{proposition}[theorem]{Satz}
\newtheorem{corollary}[theorem]{Korollar}
\theoremstyle{definition}
\newtheorem*{definition}{Definition}
\newtheorem*{example}{Beispiel}
\theoremstyle{remark}
\newtheorem*{remark}{Bemerkung}

\maketitle

\section{Maximale Planarität}

\begin{definition}
Ein \textit{ebener Graph} ist ein Paar $G= (V, E)$ endlicher Mengen (Elemente von $V$ heißen \textit{Knoten}, Elemente von $E$ heißen \textit{Kanten}), so dass:
\begin{enumerate}[(i)]
\item $V$ ist Teilmenge von $\mathbb{R}^2$.
\item Jedes Element in $E$ ist ein Polygonzug zwischen zwei Knoten.
\item  Veschiedene Kanten haben verschiedene Menge von Endpunkten.
\item Das Innere einer Kante enthält keine Knote oder einen Punkt einer anderen Kante.
\end{enumerate}
\end{definition}

\begin{remark}
\begin{itemize}
\item  Ein ebener Graph $G$ definiert in natürlicher Weise einen (abstrakten) Graphen, den wir ebenfalls mit $G$ bezeichnen.
\item Die unterliegende Punktmenge eines ebenen Graphens bezeichnen wir auch mit $G$:
\[ G = V \cup\bigcup_{e\in E}e \]
\end{itemize}
\end{remark}

\begin{definition}
\begin{itemize}
\item Sei $G = (V, E)$ ein ebener Graph. Die Zusammenhangskomponenten von $\mathbb{R}^2\setminus G$ heißen \textit{Gebiete} von $G$. 
\item Die Menge aller Gebiete von $G$ bezeichnen wir mit $F(G)$.
\item Sei $f\in F(G)$. Der Rand von $f$ bezeichnen wir mit $G[f]$. Wir können $G[f]$ als Teilgraph von $G$ auffassen.
\end{itemize}
\end{definition}

\begin{definition}
Sei $G$ ein ebener Graph.
\begin{itemize}
\item $G$ heißt \textit{maximal eben}, wenn durch das Hinzufügen einer Kante $G$ kein ebener Graph mehr ist.
\item $G$ heißt \textit{ebener Dreiecksgraph}, wenn jedes seiner Gebiete durch einen $K^3$ berandet ist.
\end{itemize}
\end{definition}

\begin{proposition}
Ein ebener Graph $G$ mit $|G|\geq 3$ ist genau dann maximal eben, wenn er ein ebener Dreiecksgraph ist.
\end{proposition}

\begin{proof}
Sei $G$ ein ebener Dreiecksgraph und $e$ eine zusätzliche Kante. Dann hat $e$ ihr Inneres in einem Gebiet $f$ von $G$ und ihre Endpunkte auf dem Rand von $f$. Per Definition ist $G[f] = K^3$ ein vollständiger Graph, also sind die Endpunkte von $e$ bereits in $G$ benachbart. Da Multikanten in einem ebenen Graphen nicht erlaubt sind, war $G$ schon maximal eben.

Sei nun umgekehrt $G$ ein maximal ebener Graph und $f$ ein Gebiet von $G$. Setze $H=G[f]$ und betrachte den induzierten Untergraphen $G[H]$. Angenommen, es gibt zwei Knoten $x,y$ in $G[H]$, die nicht benachbart sind. Aber dann könnten wir einen Polygonzug zwischen $x$ und $y$ in $f$ konstruieren und diese als ebene Kante zu $G$ hinzufügen, ein Widerspruch zur Maximalität von $G$. Also muss $G[H]$ vollständig sein.

Sei $n = |H|$. Angenommen, $H$ enthält keinen Kreis. Ist $n\geq 3$, so ist $K^3\subseteq G[H]\subseteq G$, d.h. $G$ enthält einen Kreis und daher $G\setminus H \neq\varnothing$. Für $n<3$ gilt auch $G\setminus H\neq\varnothing$, da $|G|\geq 3$. Andererseits ist $H$ ein Wald und hat daher genau einen Gebiet. $f$ ist ein Gebiet von $G[f] = H$, also ist $f$ das einzige Gebiet von $H$, d.h. $f\cup H = \mathbb{R}^2$. Insgesamt erhalten wir $G\setminus H \subseteq f$ und es folgt der Widerspruch:
\[ G\setminus H = G\setminus H \cap f \subseteq G\cap f =\varnothing \]
Also muss $H$ einen Kreis enthalten.

Es bleibt noch zu zeigen, dass $n\leq 3$ ist. Nehmen wir an, dass $n\geq 4$. Sei $C = v_1v_2v_3v_4v_1$ ein Kreis in $G[H]$. Wegen $C\subseteq G$ liegt $f$ in einem Gebiet $c\in F(C)$. Sei $c'\in F(C)$ das andere Gebiet. $v_1$ und $v_3$ liegen auf dem Rand von $f$. Wir können sie mit einem Polygonzug $P$ in $c$ verbinden, das sich mit $G$ nicht schneidet. Daher muss die ebene Kante zwischen $v_2$ und $v_4$ in $c'$ befinden, da sich diese mit $P$ nicht schneiden darf. Das gleiche Argument für $v_2$ und $v_4$ zeigt, dass die ebene Kante zwischen $v_1$ und $v_3$ in $c'$ befinden muss. Dies ein Widerspruch, da eine solche Kante mit der Kante zwischen $v_2$ und $v_4$ schneiden muss.
\end{proof}

\begin{corollary}
Ein ebener Graph $G$ der Ordnung $n\geq 3$ hat höchstens $3n-6$ Kanten.
\end{corollary}

\begin{proof}
Jeder ebene Dreiecksgraph mit $n$ Ecken hat $3n-6$ Kanten.
\end{proof}

\begin{corollary}\label{cor:kuratowski-motivation}
Kein ebener Graph enthält einen $K^5$ oder $K_{3,3}$ als einen topologischen Minor.
\end{corollary}

\begin{proof}
$K^5$ hat $10 > 3\cdot 5 - 6$ Kanten. Für $K_{3,3}$ kann man mithilfe der Euler-Charakteristik für ebene Graphen auch eine widersprüchliche Abschätzung für die Kantenzahl finden, siehe Korollar 3.2.11 in [Diestel: Graphentheorie]. Mit $K^5$ und $K_{3,3}$ können natürlich auch deren Unterteilungen nicht als ebene Graphen auftreten.
\end{proof}

\begin{proposition}
\begin{itemize}
\item Eine \textit{Einbettung in die Ebene} eines (abstrakten) Graphen $G$ ist ein abstrakter Graphenisomorphismus zwischen $G$ und einem ebenen Graphen $H$. 
\item $H$ nennen wir auch eine \textit{Zeichnung} von $G$.
\item Ein Graph $G$ heißt \textit{plättbar}, wenn es eine Einbettung in die Ebene für $G$ gibt.
\end{itemize}

\end{proposition}

\section{Satz von Kuratowski}

Interessanterweise gilt auch die Umkehrung von Korollar~\ref{cor:kuratowski-motivation}:

\begin{theorem}[\textit{Satz von Kuratowski, 1930}]\label{thm:kuratowski}
Die folgenden Aussagen sind für einen Graphen $G$ äquivalent:
\begin{enumerate}[(i)]
\item $G$ ist plättbar.
\item  $G$ enthält weder einen $K^5$ noch einen $K_{3,3}$ als Minor.
\item  $G$ enthält weder einen $K^5$ noch einen $K_{3,3}$ als topologischen Minor.
\end{enumerate}
\end{theorem}

Wir zeigen die Umkehrung zunächst für $3$-zusammenhängende Graphen. Dazu brauchen wir zunächst die folgenden Lemmata:

\begin{lemma}\label{lem:kuratowski-2lemma}
In einem $2$-zusammenhängenden ebenen Graphen ist jedes Gebiet durch einen Kreis berandet.
\end{lemma}

\begin{proof}
Siehe Lemma 3.2.6 in [Diestel: Graphentheorie].
\end{proof}

\begin{lemma}\label{lem:kuratowski-3lemma}
Ist $G$ $3$-zusammenhängend und $|G|>4$, so hat $G$ eine Kante $e$, so dass $G/e$ wieder $3$-zusammenhängend ist.
\end{lemma}

\begin{proof}
Siehe Lemma 2.2.1 in [Diestel: Graphentheorie].
\end{proof}

\begin{proposition}
Ist ein Graph $G$ $3$-zusammenhängend, und enthält $G$ weder einen $K^5$ noch einen $K_{3,3}$ als topologischen Minor, so ist $G$ plättbar.
\end{proposition}

\begin{proof}
Per Induktion nach $|G|$. Für $|G|=4$ ist $G = K^4$ und $G$ plättbar:
\[\begin{tikzcd}
& 1\ar[d, -]\ar[ldd, -]\ar[rdd, -] & K^4\\
& 2\ar[ld, -]\ar[rd, -] & \\
3\ar[rr, -] & & 4
\end{tikzcd}\]
Sei nun $|G|>4$ und die Aussage wahr für kleinere Graphen. Nach Lemma~\ref{lem:kuratowski-3lemma} haben wir eine Kante $xy$, so dass $G/xy$ wieder $3$-zusammenhängend ist. Nun enthält $G/xy$ ebenfalls weder $K^5$ noch einen $K_{3,3}$ als topologischen Minor. Nach Induktionsvoraussetzung ist $G/xy$ plättbar. Sei also $H$ eine Zeichnung von $G/xy$.

Sei $v$ der Knoten in $H$, das die Kante $xy$ repräsentiert. Betrachte das Gebiet $f$ von $H-v$, das den Punkt $v$ enthält und sei $C$ der Rand von $f$. Setze:
\[ X = N(x) \setminus{y},\quad Y = N(y)\setminus{x} \]
Dann gilt $X\cup Y\subseteq N(v)\subseteq C$. Da $H-v$ $2$-zusammenhängend ist, ist $C$ nach Lemma~\ref{lem:kuratowski-2lemma} ein Kreis. Seien $x_1,\ldots, x_k$ die Elemente in $X$ in natürlicher Reihenfolge entlang $C$ und $P_i$ der Verbindungsweg auf $C$ zwischen $x_i$ und $x_{i+1}$, wobei $x_{k+1} = x_1$. Betrachte den folgenden ebenen Graphen:
\[ H' = H - \{vw \mid w\in Y\setminus X\} \]
Wir können $H'$ auch als Zeichnung von $G-y$ deuten, indem wir den Knoten $v$ als $x$ auffassen. Ziel ist es nun, auch $y$ in der Zeichnung unterzubringen. 

Dafür reicht es zu zeigen, dass ein $i$ existiert, so dass $Y\subseteq V(P_i)$. Dann können wir $y$ in dem Gebiet platzieren, der durch $x_i P_i x_{i+1} x x_i$ definiert ist. Angenommen, es existiert kein $i$ mit $Y\subseteq V(P_i)$. Wir unterscheiden drei Fälle:
\begin{enumerate}
\item Fall: Es gibt ein $y'\in Y\setminus X$. Sei etwa $y'\in P_i$ und $y''\in C\setminus P_i$ ein weiterer Nachbar von $y$. Setze $x'=x_i$ und $x''=x_{i+1}$. Dann werden $y'$ und $y''$ durch $x'$ und $x''$ in $C$ getrennt.
\item Fall: Es ist $Y\subseteq X$ und $|Y| \leq 2$, d.h. $y$ hat genau zwei Nachbarn $y'$ und $y''$ auf $C$, die nicht im gleichen $P_i$ liegen. Diese werden durch zwei $x',x''\in X$ in $C$ getrennt.
\item Fall: Es ist $Y\subseteq X$ und $|Y|\geq 3$.
\end{enumerate}
In den ersten beiden Fällen bilden $x,y',y''$ und $y,x',x''$ einen $\mathrm{T}K_{3,3}$ in $G$. Im dritten Fall haben $y$ und $x$ drei gemeinsame Nachbarn auf $C$. Diese bilden zusammen mit $x$ und $y$ einen $\mathrm{T}K^5$ in $G$.
\end{proof}

Um den Beweis von Satz von Kuratowski abzuschließen, muss man noch die folgenden Lemmata beweisen:

\begin{lemma}
Ein Graph enthält genau dann einen $\mathrm{T}K^5$ oder einen $\mathrm{T}K_{3,3}$, wenn er einen $K^5$ oder einen $K_{3,3}$ als Minor enthält.
\end{lemma}

\begin{proof}
Siehe Lemma 3.4.2 in [Diestel: Graphentheorie].
\end{proof}

\begin{lemma}
Ist $G$ ein Graph mit $|G|>4$, der kantenmaximal mit $\mathrm{T}K^5,\mathrm{T}K_{3,3}\not\subseteq G$ ist, so ist $G$ $3$-zusammenhängend.
\end{lemma}

\begin{proof}
Siehe Lemma 3.4.5 in [Diestel: Graphentheorie].
\end{proof}

\section{Algebraisches Plättbarkeitskriterium}

\begin{definition}
Sei $G=(V, E)$ ein Graph.
\begin{itemize}
\item Der \textit{Kantenraum} von $G$ ist definiert als den $\mathbb{F}_2$-Vektorraum 
\[\{\text{Abbildungen } h:E\to\mathbb{F}_2\}\] 
mit komponentenweiser Addition. Wir identifizieren Vektoren darin mit Teilmengen von $E$. Somit ist die Addition von Teilmengen nichts anderes als das Bilden der symmetrischen Differenz der beiden Mengen:
\[ E_1 + E_2 = (E_1 \cup E_2) \setminus (E_1\cap E_2) \]
\item Der \textit{Schnittraum} $\mathcal{C}^\ast(G)$ von $G$ ist definiert als der $\mathbb{F}_2$-Untervektorraum des Kantenraums, der nur aus den Schnittmengen in $G$ besteht, d.h. Mengen der Form $E(V', V'')$ für eine Partition $\{V', V''\}$ in $G$.
\item  Der \textit{Zyklenraum} $\mathcal{C}(G)$ von $G$ ist definiert als der $\mathbb{F}_2$-Untervektorraum des Kantenraums, der von den Kantenmengen von Kreisen in $G$ erzeugt wird. Vektoren in $\mathcal{C}(G)$ kann man als Summe von disjunkten Kreisen in $G$ schreiben.
\item Eine Teilmenge $\mathcal{F}$ des Kantenraums von $G$ heißt \textit{schlicht}, wenn jede Kante in $G$ in höchstens zwei Mengen aus $\mathcal{F}$ liegt.
\end{itemize}
\end{definition}

\begin{lemma}\label{lem:dimensionsformel}
Sei $G$ ein zusammenhängender Graph mit $n$ Knoten und $m$ Kanten. Dann gilt für die Dimension des Zyklenraums:
\[ \dim \mathcal{C}(G) = m-n+1 \]
\end{lemma}

\begin{proof}
Siehe Satz 0.9.6 in [Diestel: Graphentheorie].
\end{proof}

\begin{theorem}[\textit{MacLane 1937}]\label{thm:maclane}
Ein Graph $G$ ist genau dann plättbar, wenn sein Zyklenraum $\mathcal{C}(G)$ eine schlichte Basis besitzt.
\end{theorem}

\begin{proof}
Für $|G|\leq 2$ ist die Aussage trivial. Sei $|G|\geq 3$. Sei $G$ zunächst einmal höchstens $1$-zusammenhängend, d.h. $G$ ist die Vereinigung zweier Untergraphen $G',G''\subset G$ mit $|G'\cap G''|\leq 1$. Ein Kreis in $G$ ist entweder ein Kreis in $G'$ oder ein Kreis in $G''$, also folgt:
\[ \mathcal{C}(G) = \mathcal{C}(G')\oplus\mathcal{C}(G'') \]
Somit hat $\mathcal{C}(G)$ genau dann eine schlichte Basis, wenn $\mathcal{C}(G')$ und $\mathcal{C}(G'')$ eine haben. Ferner ist $G$ genau dann plättbar, wenn $G'$ und $G''$ es sind. Somit folgt die Aussage induktiv. Sei ab jetzt $G$ $2$-zusammenhängend. 

Sei $G$ plättbar und wähle eine Zeichnung. Nach Lemma~\ref{lem:kuratowski-2lemma} sind alle Gebietsränder Kreise, liegen also in $\mathcal{C}(G)$. Wir zeigen, dass die Gebietsränder schon ganz $\mathcal{C}(G)$ erzeugen. Da eine ebene Kante auf dem Rand höchstens zweier Gebiete liegen kann, besitzt $\mathcal{C}(G)$ dann eine schlichte Basis. 

Sei $C\subset G$ ein Kreis und $f$ sein Innengebiet. Jede Kante $e$ liegt auf einem Kreis von $G$. Liegt das Innere von $e$ in $f$, so liegt $e$ auf dem Rand genau zweier Gebiete von $G$, die in $f$ enthalten sind. Liegt $e$ dagegen auf $C$, so liegt $e$ genau auf einem Gebiet, das in $f$ enthalten ist. Somit gilt:
\[ C = \sum_{\substack{f'\in F(G)\\ f'\subseteq f}} E(G[f']) \]

Sei nun umgekehrt $\{C_1,\ldots, C_k\}$ eine schlichte Basis von $\mathcal{C}(G)$. Für jede Kante $e$ besitzt $\mathcal{C}(G-e)$ auch eine schlichte Basis, denn:
\begin{itemize}
\item Liegt $e$ in nur einem Basiszyklus, etwa $C_1$, so ist $\{C_2,\ldots, C_k\}$ eine Basis von $\mathcal{C}(G-e)$.
\item Liegt $e$ in zwei Basiszyklen, etwa $C_1$ und $C_2$, so ist $\{C_1+C_2,\ldots, C_k\}$ eine Basis von $\mathcal{C}(G-e)$.
\end{itemize}
Angenommen, $G$ ist nicht plättbar. Nach dem Satz von Kuratowski enthält $G$ einen $\mathrm{T}K^5$ oder einen $\mathrm{T}K_{3,3}$. Als Teilgraphen  von $G$ hat dann $\mathcal{C}(\mathrm{T}K^5)$ bzw. $\mathcal{C}(\mathrm{T}K_{3,3})$ eine schlichte Basis. Da topologische Minoren keine Kreise hinzufügen bzw. entfernen bleibt der Zyklenraum gleich, d.h. $\mathcal{C}(K^5)$ bzw. $\mathcal{C}(K_{3,3})$ hat eine schlichte Basis. Wir führen nun beide Fälle in den folgenden Lemmata zum Widerspruch.
\end{proof}

\begin{lemma}
$\mathcal{C}(K^5)$ hat keine schlichte Basis.
\end{lemma}

\begin{proof}
Nach der Dimensionsformel Lemma~\ref{lem:dimensionsformel} gilt $\dim\mathcal{C}(K^5) = 6$. Angenommen, $\mathcal{C}(K^5)$ habe eine schlichte Basis $B=\{C_1,\ldots,C_6\}$. Setze:
\[ C_0 = C_1+\ldots + C_6 \]
Keines der $C_0, C_1,\ldots, C_6$ sind leer und enthalten alle mindestens drei Kanten. Da eine Kante in höchstens zwei Zyklen aus $B$ liegt, liegt jede Kante in $C_0$ in nur einem der Basiszyklen in $B$. Daher ist die Menge $\{C_0,C_1,\ldots ,C_6\}$ ebenfalls schlicht, also folgt:
\[ 21 = 7\cdot 3 \leq |C_0|+\ldots + |C_6| \leq 2\cdot \|K^5\| = 20 \qedhere \]
\end{proof}

\begin{lemma}
$\mathcal{C}(K_{3,3})$ hat keine schlichte Basis.
\end{lemma}

\begin{proof}
Nach der Dimensionsformel Lemma~\ref{lem:dimensionsformel} gilt $\dim\mathcal{C}(K_{3,3}) = 4$. Angenommen,  $\mathcal{C}(K_{3,3})$ habe eine schlichte Basis $B=\{C_1,\ldots, C_4\}$. Setze:
\[ C_0 = C_1+\ldots + C_4 \]
Keines der $C_0,C_1,\ldots,C_4$ sind leer und enthalten alle mindestens vier Kanten wegen der Bipartität. Mit dem gleichen Argument wie im vorherigen Lemma ist $\{C_0,C_1,\ldots,C_4\}$ ebenfalls schlicht, also folgt:
\[ 20 = 5\cdot 4 \leq |C_0|+\ldots + |C_4|\leq 2\cdot \|K_{3,3}\| = 18\qedhere \]
\end{proof}

\section{Plättbarkeit \& Dualität}

\begin{definition}
Ein \textit{ebener Multigraph} ist ein Paar $G=(V,E)$ endlicher Mengen (Elemente von $V$ heißen \textit{Knoten}, Elemente von $E$ heißen \textit{Kanten}), so dass:
\begin{enumerate}[(i)]
\item $V$ ist Teilmenge von $\mathbb{R}^2$.
\item Jedes Element in $E$ ist ein Polygonzug zwischen zwei Knoten oder ein Polygon, das genau eine Knote enthält.
\item Das Innere einer Kante enthält keine Knote oder einen Punkt einer anderen Kante.
\end{enumerate}
\end{definition}

\begin{definition}
Sei $G=(V, E)$ ein ebener Multigraph. Wir setzen in jedes ebiet von $G$ einen neuen Knoten und verbinden diese zu einem neuen ebenen Multigraphen $G^\ast$:
\begin{itemize}
\item Für jede Kante $e$ von $G$ verbinden wir die neuen Knoten in den beiden Gebieten, auf deren Rand $e$ liegt, durch eine neue Kante $e^\ast$.
\item Liegt $e$ nur auf dem Rand eines Gebiets, so legen wir an dessen neue Knote eine Schlinge $e^\ast$ durch $e$.
\end{itemize}
Der neue Graph $G^\ast$ heißt \textit{topologisches Dual} zu $G$.
\end{definition}

Wir finden den folgenden einfachen Zusammenhang zwischen einem Graphen und sein topologisches Dual:

\begin{proposition}\label{prop:kombdual-motivation}
Sei $G$ ein zusammenhängender ebener Multigraph und $E\subseteq E(G)$ eine Kantenmenge. $E$ ist genau dann von einem Kreis induziert, wenn $E^\ast = \{e^\ast\mid e\in E\}$ ein minimaler Schnitt in $G^\ast$ ist.
\end{proposition}

\begin{proof}
Siehe Proposition 3.6.1 in [Diestel: Graphentheorie].
\end{proof}

\begin{definition}
Sei $G$ ein abstrakter Multigraph. Ein abstrakter Multigraph $G^\ast$ heißt zu $G$ \textit{kombinatorisch dual}, wenn $E(G^\ast) = E(G)$ und die Minimalschnitte von $G^\ast$ gerade die Kantenmengen von Kreise in $G$ sind.
\end{definition}

\begin{proposition}\label{prop:kombdual-zyklenraum}
Sei $G^\ast$ kombinatorisch dual zu $G$. Dann gilt:
\[ \mathcal{C}^\ast(G^\ast) =\mathcal{C}(G) \]
\end{proposition}

\begin{proof}
$\mathcal{C}^\ast(G^\ast)$ wird erzeugt durch die Minimalschnitte von $G^\ast$, während $\mathcal{C}(G)$ von den Kantenmengen von Kreise in $G$ erzeugt wird.
\end{proof}

\begin{lemma}\label{lem:schnittraum-schlicht}
Sei $G=(V, E)$ ein Graph. Dann wird $\mathcal{C}^\ast(G)$ erzeugt von Schnitten der Form $E(v) = \{vw\mid w\in V\setminus\{v\} \},\ v\in V$.
\end{lemma}

\begin{proof}
Sei $\{V', V''\}$ eine Partition von $G$ und betrachte den Schnitt $E(V', V'')$. Da jede Kante $vw$ in genau $E(v)$ und $E(w)$ liegt, gilt:
\[ E(V', V'') = \sum_{v\in V'} E(v) \qedhere \]
\end{proof}

\begin{corollary}\label{cor:schnittraum-schlicht}
Sei $G$ ein Graph. Dann hat $\mathcal{C}^\ast(G)$ eine schlichte Basis.
\end{corollary}

\begin{theorem}[\textit{Whitney 1993}]
Ein Graph $G$ ist genau dann plättbar, wenn ein zu ihm kombinatorisch dualer Multigraph existiert.
\end{theorem}

\begin{proof}
Eine Richtung folgt aus Satz~\ref{prop:kombdual-motivation}. Sei $G^\ast$ ein kombinatorisches Dual zu $G$. Nach Satz~\ref{prop:kombdual-zyklenraum} gilt $\mathcal{C}^\ast(G^\ast) =\mathcal{C}(G)$. Nach MacLane Theorem~\ref{thm:maclane} reicht es zu zeigen, dass $\mathcal{C}^\ast(G^\ast)$ eine schlichte Basis besitzt. Dies folgt aus Korollar~\ref{cor:schnittraum-schlicht}.
\end{proof}



\end{document}